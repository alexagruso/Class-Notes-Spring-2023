\documentclass[12pt]{article}
\pagenumbering{gobble}
\linespread{1}

\usepackage{amsfonts}
\usepackage{amsmath}
\usepackage{amssymb}
\usepackage{array}
\usepackage{fancyhdr}
\usepackage{textcomp}
\usepackage{pgfplots}
\usepackage[margin=1in,headheight=10mm]{geometry}

\usetikzlibrary{fillbetween}

\pgfplotsset{compat=1.17}

\newcommand{\contradiction}{%
    \ensuremath{{\Rightarrow\mspace{-2mu}\Leftarrow}}%
}

\newcommand{\angleb}[1]{\left\langle#1\right\rangle}
\newcommand{\vertb}[1]{\left\vert#1\right\vert}

\begin{document}
\pagestyle{fancy}
\fancyhead{}
\fancyhead[L]{Alexander Agruso}
\fancyhead[R]{Homework 1}

\normalsize
\section*{Exercises 12.4}
\begin{itemize}
	\item [5.)] $\angleb{\dfrac{1}{2},\dfrac{1}{3},\dfrac{1}{4}}\times\angleb{1,2,-3}=\left(-1-\dfrac{1}{2}\right)i-\left(-\dfrac{3}{2}-\dfrac{1}{4}\right)j+\left(1-\dfrac{1}{3}\right)k=\angleb{-\dfrac{3}{2},\dfrac{7}{4},\dfrac{2}{3}}$\newline
	      $\angleb{\dfrac{1}{2},\dfrac{1}{3},\dfrac{1}{4}}\cdot\angleb{-\dfrac{3}{2},\dfrac{7}{4},\dfrac{2}{3}}=-\dfrac{3}{4}+\dfrac{7}{12}+\dfrac{2}{12}=0$\newline
	      $\angleb{1,2,-3}\cdot\angleb{-\dfrac{3}{2},\dfrac{7}{4},\dfrac{2}{3}}=-\dfrac{3}{2}+\dfrac{14}{4}-\dfrac{6}{3}=0$

	\item [7.)] $\angleb{t,1,\dfrac{1}{t}}\times\angleb{t^2,t^2,1}=\left(1-t\right)i-(t-t)j+(t^3-t^2)k=\angleb{1-t,0,t^3-t^2}$\newline
	      $\angleb{t,1,\dfrac{1}{t}}\cdot\angleb{1-t,0,t^3-t^2}=(t-t^2)+(t^2-t)=0$\newline
	      $\angleb{t^2,t^2,1}\cdot\angleb{1-t,0,t^3-t^2}=(t^2-t^3)+(t^3-t^2)=0$

	\item [17.)] $\angleb{2,-1,3}\times\angleb{4,2,1}=(-1-6)i-(2-12)j+(4+4)k=\angleb{-7,10,8}$\newline
	      $\angleb{4,2,1}\times\angleb{2,-1,3}=-\angleb{2,-1,3}\times\angleb{4,2,1}=\angleb{7,-10,-8}$.

	\item [19.)] $\angleb{3,2,1}\times\angleb{-1,1,0}=\angleb{-1,-1,5};\dfrac{\angleb{-1,-1,5}}{\Vert\angleb{-1,-1,5}\Vert}=\dfrac{1}{3\sqrt{3}}\angleb{-1,-1,5}$\newline
	      Thus $\pm\dfrac{1}{3\sqrt{3}}\angleb{-1,-1,5}$ are orthogonal unit vectors.

	\item [27.)] $\overline{AB}=\angleb{-1+3,3-0}=\angleb{2,3}$; $\overline{BC}=\angleb{5+1,2-3}=\angleb{6,-1}$.\newline
	      $\begin{vmatrix}
			      2 & 3 \\6&-1
		      \end{vmatrix}=-2-18=-20,\vert-20\vert=20$

	\item [29.)] \begin{itemize}
		      \item [a.)] $\overline{PQ}=\angleb{-2-1,1-0,3-1}\angleb{-3,1,2}$\newline
		            $\overline{PR}=\angleb{4-1,2-0,5-1}=\angleb{3,2,4}$\newline
		            $\overline{PQ}\times\overline{PR}=\begin{vmatrix}
				            \overline{i} & \overline{j} & \overline{k} \\
				            -3           & 1            & 2            \\3 & 2 & 4
			            \end{vmatrix}=\overline{i}(4-4)-\overline{j}(-12-6)+\overline{k}(-6-3)=\angleb{0,18,-9}$

		      \item [b.)] $A=\dfrac{\Vert\overline{PQ}\times\overline{PR}\Vert}{2}=\dfrac{\sqrt{0^2+18^2+(-9)^2}}{2}=\dfrac{\sqrt{405}}{2}=\dfrac{9\sqrt{5}}{2}$
	      \end{itemize}

	\item [33.)] The volume of the parallelepiped formed by 3 vectors is equal to their determinant:
	      \[\begin{vmatrix}
			      1 & -1 & 2 \\
			      2 & 1  & 1 \\
			      3 & 2  & 4
		      \end{vmatrix}=1(4-2)+1(8-3)+2(4-3)=2+5+2=9\]
	      Which is the volume of the parallelepiped.

	\item [41.)] $\theta=\tan^{-1}\left(\frac{4}{3}\right)$\newline
	      $\tau=100=Fr\sin\theta=F(0.3)(0.8)\implies F=\frac{100}{0.24}\approx417N$
\end{itemize}
\section*{Exercises 12.5}
\begin{itemize}
	\item [3.)] $P_0(2,2.4,3.5)$, $\overline{v}=\angleb{3,1,-1}$\newline
	      $L=P_0+t\overline{v}=\angleb{2,2.4,3.5}+t\angleb{3,1,-1}=\angleb{2+3t,2.4+t,3.5-t}$

	      Parametric:\newline
	      $x=2+3t,\ y=2.4+t,\ z=3.5-t$

	\item [5.)] $P_0(1,0,6)$, $P=x+3y+z=5\implies x+3y+z-5=(x-5)+3(y-0)+(z-0)=0$\newline
	      $L=P_0+t\hat{n}=\angleb{1,0,6}+t\angleb{1,3,1}=\angleb{1+t,3t,6+t}$

	\item [9.)] $P_0(-8,1,4)$, $\overline{v}=\angleb{3+8,-2-1,4-4}=\angleb{11,-3,0}$\newline
	      $L=\angleb{-8,1,4}+t\angleb{11,-3,0}=\angleb{11t-8,1-3t,4}$

	      Parametric:\newline
	      $x=11t-8,\ y=1-3t,\ z=4$

	      Symmetric:\newline
	      $t=\dfrac{x+8}{11}=-\dfrac{y-1}{3},\ z=4$

	\item [17.)] $P_0(6,-1,9)$, $\overline{v}=\angleb{7-6,6+1,0-9}=\angleb{1,7,-9}$\newline
	      $L=\angleb{6,-1,9}+t\angleb{1,7,-9}=\angleb{6+t,7t-1,9-9t}$, $t\in[0,1]$

	\item [23.)] $P_0(0,0,0),\ \hat{n}=\angleb{1,-2,5}$\newline
	      $P=\hat{n}\cdot(\overline{v}-P_0)=(x-0)-2(y-0)+5(z-0)=0\implies P=x-2y+5z=0$\newline

	\item [27.)] $P_0(1,-1,-1),\ P\vert_{P_0}=5(x-1)-(y+1)-(z+1)=5x-y-z=7$

	\item [33.)] $\overline{AB}=\angleb{3-2,-8-1,6-2}=\angleb{1,-9,4}$\newline
	      $\overline{AC}=\angleb{-2-2,-3-1,1-2}=\angleb{-4,-4,-1}$\newline
	      $\overline{AB}\times\overline{AC}=\begin{vmatrix}
			      \overline{i} & \overline{j} & \overline{k} \\
			      1            & -9           & 4            \\-4 & -4 & -1
		      \end{vmatrix}=i(9+16)-j(-1+16)+k(-4-36)=25i-15j-40k$\newline
	      $P=25(x-2)-15(y-1)-40(z-2)=0\implies5(x-2)-3(y-1)-8(z-2)$\newline
	      $\implies P=5x-3y-8z=-9$

	\item [45.)] $x+2y-z=7\implies2-2t+6t-1-t=1+3t=7\implies t=2$\newline
	      $L|_t=\angleb{2-2t,3t,1+t}=\angleb{-2,6,3}$

	\item [69.)] $\overline{A}=\overline{P_0L_0}=\angleb{1-4,3-1,4+2}=\angleb{-3,2,6}$\newline
	      $\overline{B}=\overline{L_1L_0}=\angleb{1-2,3-1,4-1}=\angleb{-1,2,3}$\newline
	      $A\times B=\begin{vmatrix}
			      \overline{i} & \overline{j} & \overline{k} \\
			      -3           & 2            & 6            \\-1 & 2 & 3
		      \end{vmatrix}=i(6-12)-j(-9+6)+k(-6+2)=-6i+3j-4k$\newline
	      $\Vert A\times B\Vert=\sqrt{6^2+3^2+4^2}=\sqrt{36+9+16}=\sqrt{61}$\newline
	      $\Vert B\Vert=\sqrt{1^2+2^2+3^2}=\sqrt{14}$\newline
	      $d=\dfrac{\sqrt{61}}{\sqrt{14}}=\sqrt{\dfrac{61}{14}}$

	\item [71.)] $P=\angleb{1,1,0}\text{ is e point on the plane}$\newline
	      $\overline{PS}=\angleb{1-1,-2-1,4-0}=\angleb{0,-3,4}$\newline
	      $\hat{n}=\angleb{3,2,6}$\newline
	      $D=\text{comp}_{\hat{n}}\overline{PS}=\dfrac{\overline{PS}\cdot\hat{n}}{\Vert\hat{n}\Vert}=\dfrac{-6+24}{\sqrt{9+4+36}}=\dfrac{18}{7}$
\end{itemize}

\end{document}
