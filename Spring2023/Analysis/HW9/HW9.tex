\documentclass[12pt]{article}
\pagenumbering{gobble}
\linespread{1.25}

\usepackage{amsfonts}
\usepackage{amsmath}
\usepackage{amssymb}
\usepackage{array}
\usepackage{fancyhdr}
\usepackage{mathrsfs}
\usepackage{textcomp}
\usepackage[margin=1in,headheight=1in]{geometry}

\newcommand{\contradiction}{
    \ensuremath{{\Rightarrow\mspace{-2mu}\Leftarrow}}
}

\newcommand{\angleb}[1]{\left\langle#1\right\rangle}
\newcommand{\vertb}[1]{\left\vert#1\right\vert}
\newcommand{\bracks}[1]{\left[#1\right]}
\newcommand{\parns}[1]{\left(#1\right)}

\newcommand{\derv}[2]{\dfrac{d#1}{d#2}}
\newcommand{\e}{\varepsilon}

\begin{document}
\pagestyle{fancy}
\fancyhead{}
\fancyhead[L]{Alexander Agruso}
\fancyhead[R]{MATH 3380 Homework 9}

\normalsize
\begin{itemize}
    % \item [12.)] \begin{itemize}
    %     \item [a.)] Since $0<\lambda<1$, $0<(1-\lambda)<1$ and $(1-\lambda)y < y$.
    % \end{itemize}

    % \item [29.)] \begin{itemize}
    %     \item [a.)] Since $S\neq\varnothing$, $\mathscr{L}\neq\varnothing$. In addition, since $S$ is bounded below, there exists $v\in\mathbb{R}$ such that $v=\inf(S)$, thus $v\geq x$ for all $x\in\mathscr{L}$, thus $v$ is an upper bound of $\mathscr{L}$, thus $\mathscr{L}$ is bounded above. Q.E.D.

    %     \item [b.)] Let $w=\sup(\mathscr{L})$. For the sake of establishing a contradiction, suppose there exists $x\in S$ such that $x<w$, thus $x$ is not an upper bound of $\mathscr{L}$, thus there exists $l\in\mathscr{L}$ such that $l>x$, thus $l$ is not a lower bound of $S$, thus $l\notin\mathscr{L}$\contradiction, thus $x\in S\implies x\geq w$, thus $w$ is a lower bound of $S$. Q.E.D.

    %     \item [c.)] Since $w=\sup(\mathscr{L})$, $w\geq l$ for all $l\in\mathscr{L}$, thus $w\geq l$ for all lower bounds $l$ of $S$, thus $w=\inf(S)$. Q.E.D.
    % \end{itemize}

    % \item [29.)] \begin{itemize}
    %     \item [a.)] Since $S$ is nonempty and bounded below, there exists some lower bound $l$ of $S$, thus $\mathscr{L}\neq\varnothing$.
    % \end{itemize}

    % \item [33.)] Let $x,y\in\mathbb{R}$, and consider $x$:
    %     \[x=x+0=x-y+y\]
    %     And thus by the triangle inequality:
    %     \[\vertb{x-y+y}\leq\vertb{x-y}+\vertb{y}\]
    %     Manipulating:
    %     \[\vert
    %     x-y+y\vert\leq\vertb{x-y}+\vertb{y}\implies\vertb{x}-\vertb{y}\leq\vertb{x-y}\]
    %     \[\implies\vert x\vert-\vert x\vert-\vert y\vert=-\vert y\vert\leq\vert x\vert-\vert y\vert\leq\vert x-y\vert\]
    %     Thus the inequality holds. Q.E.D.***

    % \item [37.)] awd

    % \item [38.)] Let $S$ be a uniformly discrete set,

    \item [40.)] Let $\varepsilon>0$, and $n\geq k$ for some $k\in\mathbb{N}$, then
    \[\vertb{\frac{2n-1}{n}-2}<\varepsilon\]
    Solving for $n$, we can find a sufficiently large value for $k$:
    \[\vertb{\frac{2n-1}{n}-2}=\vertb{\frac{2n-1-2n}{n}}=\vertb{\frac{-1}{n}}=\frac{1}{n}<\varepsilon\]
    \[\implies n>\frac{1}{\varepsilon}\]
    Now, let $k>\dfrac{1}{\varepsilon}$, then
    \[\vertb{\frac{2n-1}{n}-2}=\frac{1}{n}\leq\frac{1}{k}<\frac{1}{1/\varepsilon}=\varepsilon\]
    \[\therefore\ \vertb{\frac{2n-1}{n}-2}<\varepsilon\]
    Thus $x_n\to2$. Q.E.D.

    \item [41.)] Let $\varepsilon>0$, and $n\geq k$ for some $k\in\mathbb{N}$, then
    \[\vertb{\frac{(-1)^n}{n}-0}<\varepsilon\]
    Solving for $n$, we can find a sufficiently large value for $k$:
    \[\vertb{\frac{(-1)^n}{n}-0}=\vertb{\frac{(-1)^n}{n}}=\frac{\vertb{(-1)^n}}{\vertb{n}}=\frac{1}{n}<\varepsilon\]
    \[\implies n>\frac{1}{\varepsilon}\]
    Now, let $k>\dfrac{1}{\varepsilon}$, then
    \[\vertb{\frac{(-1)^n}{n}-0}=\frac{1}{n}\leq\frac{1}{k}<\frac{1}{1/\varepsilon}=\varepsilon\]
    \[\therefore\ \vertb{\frac{(-1)^n}{n}-0}<\varepsilon\]
    Thus $x_n\to0$. Q.E.D.

    \pagebreak
    \item [42.)] Let $\varepsilon>0$, and $n\geq k$ for some $k\in\mathbb{N}$, then
    \[\vertb{\frac{3n+1}{5n-2}-\frac{3}{5}}<\varepsilon\]
    Solving for $n$, we can find a sifficiently large value for $k$:
    \[\vertb{\frac{3n+1}{5n-2}-\frac{3}{5}}=\vertb{\frac{15n+5-15n+6}{25n-10}}=\vertb{\frac{11}{25n-10}}=\frac{\vertb{11}}{\vertb{25n-10}}=\frac{11}{25n-10}<\varepsilon\]
    \[\implies\frac{11}{\varepsilon}<25n-10\implies\frac{11}{25\varepsilon}+\frac{2}{5}<n\]
    Now, let $k>\dfrac{11}{25\varepsilon}+\dfrac{2}{5}$, then
    \[\vertb{\frac{3n+1}{5n-2}-\frac{3}{5}}=\frac{11}{25n-10}\leq\frac{11}{25k-10}<\frac{11}{25\parns{\frac{11}{25\varepsilon}+\frac{2}{5}}-10}=\frac{11}{\frac{11}{\varepsilon}+10-10}=\frac{11}{\frac{11}{\varepsilon}}=\varepsilon\]
    \[\therefore\vertb{\frac{3n+1}{5n-2}-\frac{3}{5}}<\varepsilon\]
    Thus $x_n\to\dfrac{3}{5}$. Q.E.D.

    % \item [45.)] ***

    % \item [46.)] ***

    % \item [47.)] Let $x_n$ be a bounded sequence, and $y_n\to0$.

    % \item [48.)] ***

    % \item [49.)] ***

    % \item [50.)] Let $x_n\to A$ and $c\in\mathbb{R}$. Since $x_n\to A$, then for all $\varepsilon>0$, there exists $k$ such that $n\geq k\implies\vertb{x_n-A}<\varepsilon$. Consider the following equation:


    \item [51.)] For $x_n$ to be bounded, we must find $M\in\mathbb{R}$ such that $M\geq x_n$ for all $n\in\mathbb{N}$. Consider $x_n$. When $n$ is odd, $2(-1)^n+5=2(-1)+5=5-2=3$. When $n$ is even, $2(-1)^n+5=2(1)+5=2+7$, thus $x_n=3$ or $x_n=7$ for all $n\in\mathbb{N}$. Let $M=7$, thus $M\geq x_n$ for all $n\in\mathbb{N}$, thus $x_n$ is bounded. Q.E.D.

    % \item [52.)] ***

    % \item [53.)] ***

    % \item [54.)] ***

    \item [55.)] \begin{itemize}
        \item [1.)] $(-\infty,0]$

        \item [2.)] DNE

        \item [3.)] DNE
    \end{itemize}

    \item [56.)] \begin{itemize}
        \item [4.)] $(0,1)$

        \item [5.)] $x_n=\dfrac{n}{2n}$

        \item [6.)] $x_n=n$

        \item [7.)] $(0,1)$
    \end{itemize}

    % \item [57.)] ***

    % \item [58.)] ***

    % \item [59.)] ***

    % \item [60.)] ***

    \item [61.)] Let $\varepsilon>0$, and $n\geq k$ for some $k\in\mathbb{N}$, then
    \[\vertb{\frac{6n+1}{2n-1}-3}<\varepsilon\]
    Solving for $n$, we can find a sufficiently large value for $k$:
    \[\vertb{\frac{6n+1}{2n-1}-3}=\vertb{\frac{6n+1-6n+3}{2n-1}}=\vertb{\frac{4}{2n-1}}=\frac{\vertb{4}}{\vertb{2n-1}}=\frac{4}{2n-1}<\epsilon\]
    \[\implies \frac{4}{\varepsilon}<2n-1\implies\frac{4}{2\varepsilon}+\frac{1}{2}<n\]
    Now, let $k>\dfrac{4}{2\varepsilon}+\dfrac{1}{2}$, then
    \[\vertb{\frac{6n+1}{2n-1}-3}=\frac{4}{2n-1}\leq\frac{4}{2k-1}<\frac{4}{2\parns{\frac{4}{2\varepsilon}+\frac{1}{2}}-1}=\frac{4}{\parns{\frac{4}{\varepsilon}}+1-1}=\frac{4}{\frac{4}{\varepsilon}}=\varepsilon\]
    \[\therefore\ \vertb{\frac{6n+1}{2n-1}-3}<\varepsilon\]
    Thus $x_n\to3$. Q.E.D.

    % \item [62.)] ***

    % \item [63.)] ***

    % \item [64.)] ***

    \item [65.)] So far, the topic I have had the most trouble grasping is $\varepsilon$-$k$ convergence proofs. I understand the process of constructing the proof, but am still working on understanding the logic.

    % \item [66.)] ***

    % \item [67.)] ***

    % \item [68.)] ***

    % \item [69.)] ***

    % \item [70.)] ***

    % \item [71.)] ***

    \item [72.)] We can show that $x_n=\dfrac{(-1)^n}{n}$ is not monotonic by showing that it is neither nonincreasing nor nondecreasing. First, consider $x_1$ and $x_2$:
    \[x_1=\frac{(-1)^1}{1}=-\frac{1}{1}=-1\]
    \[x_2=\frac{(-1)^2}{1}=\frac{1}{1}=1\]
    Thus $x_1<x_2$, thus $x_n$ is not nonincreasing. Next, consider $x_2$ and $x_3$:
    \[x_2=1\]
    \[x_3=\frac{(-1)^3}{1}=-\frac{1}{1}=-1\]
    Thus $x_2>x_3$, thus $x_n$ is not nondecreasing, thus $x_n$ is not monotonic. Q.E.D.

    \item [73.)] \begin{itemize}
        \item [a.)] First, assume that $x_n$ is a nondecreasing sequence. From this, we know that $x_n\leq x_{n+1}$ for all $n\in\mathbb{N}$. We can manipulate this inequality as follows:
        \[x_n\leq x_{n+1}\implies 0\leq x_{n+1}-x_n=(\partial x)_n\]
        Thus $(\partial x)_n\geq0$ for all $n\in\mathbb{N}$, thus $\partial x$ is a nonnegative sequence.
        
        Next, assume that $\partial x$ is a nonnegative sequence. From this, we know that $(\partial x)_n=x_{n+1}-x_n\geq0$ for all $n\in\mathbb{N}$. We can manipulate this inequality as follows:
        \[x_{n+1}-x_n\geq0\implies x_{n+1}\geq x_n\]
        Thus $x_{n+1}\geq x_n$ for all $n\in\mathbb{N}$, thus $x_n$ is nondecreasing, thus $x_n$ is nondecreasing if and only if $\partial x$ is nonnegative. Q.E.D.

        \item [b.)] This reminds me about a video I watched a while back that was about discrete calculus, and the ``discrete derivative" resembles the $\partial x$ sequence.
    \end{itemize}

    % \item [74.)] ***

    % \item [75.)] ***

    % \item [76.)] ***

    % \item [77.)] ***

    \item [78.)] \begin{itemize}
        \item [a.)] $x_n=n$.

        \item [b.)] $x_n=-\dfrac{1}{n}$.
    \end{itemize}

    % \item [79.)] ***

    % \item [80.)] ***

    % \item [81.)] ***

    % \item [82.)] ***

    % \item [83.)] ***

    % \item [84.)] ***

    % \item [85.)] ***

    % \item [86.)] ***

    % \item [87.)] ***

    % \item [88.)] ***

    % \item [89.)] ***

    % \item [90.)] ***

    % \item [91.)] ***

    % \item [92.)] ***

    % \item [93.)] ***

    % \item [94.)] ***

    % \item [95.)] ***

    % \item [96.)] ***

    % \item [97.)] ***

    \item [98.)] $x_n$ is a cauchy sequence if for all $\varepsilon>0$, there exists $k\in\mathbb{N}$ such that
    \[n,m\geq k\implies\vertb{x_n-x_m}<\varepsilon\].

    % \item [99.)] ***

    % \item [100.)] ***

    % \item [101.)] ***

    % \item [102.)] ***

    % \item [103.)] ***

    % \item [104.)] ***

    % \item [105.)] ***

    % \item [106.)] ***

    % \item [107.)] ***

    % \item [108.)] ***

    % \item [109.)] ***

    % \item [110.)] ***

    % \item [111.)] ***

    % \item [112.)] ***

    % \item [113.)] ***

    % \item [114.)] ***
    
    % \item [115.)] ***
    
    % \item [116.)] ***

    % \item [117.)] ***

    % \item [118.)] ***

    % \item [119.)] ***

    % \item [120.)] ***

\end{itemize}

\end{document}
