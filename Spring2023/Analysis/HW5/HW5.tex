\documentclass[12pt]{article}
\pagenumbering{gobble}
\linespread{1.15}

\usepackage{amsfonts}
\usepackage{amsmath}
\usepackage{amssymb}
\usepackage{array}
\usepackage{fancyhdr}
\usepackage{mathrsfs}
\usepackage{textcomp}
\usepackage[margin=1in,headheight=1in]{geometry}

\newcommand{\contradiction}{
    \ensuremath{{\Rightarrow\mspace{-2mu}\Leftarrow}}
}

\newcommand{\angleb}[1]{\left\langle#1\right\rangle}
\newcommand{\vertb}[1]{\left\vert#1\right\vert}
\newcommand{\bracks}[1]{\left[#1\right]}

\newcommand{\derv}[2]{\dfrac{d#1}{d#2}}

\begin{document}
\pagestyle{fancy}
\fancyhead{}
\fancyhead[L]{Alexander Agruso}
\fancyhead[R]{MATH 3380 Homework 4}

\normalsize
\begin{itemize}
    % \item [12.)] \begin{itemize}
    %     \item [a.)] Since $0<\lambda<1$, $0<(1-\lambda)<1$ and $(1-\lambda)y < y$.
    % \end{itemize}

    \item [21.)] Let $v=\inf(S)$ and $y\in S$ such that $y>v$. For the sake of establishing a contradiction, assume that no element $s\in S$ exists such that $s<y$. Consequently, $y\leq x$ for all $x\in S$, thus $y$ is a lower bound of $S$, but $y>v$, thus $v\neq\inf(S)$\contradiction, thus for all $y\in S$ such that $y>v$, there exists $s\in S$ such that $s<y$. Q.E.D.

    \item [25.)] Let $v\in\mathbb{R}$ be a lower bound of $B$, thus $v\leq x$ for all $x\in B$. Since for all $a\in A$, $a\in B$, $v\leq a$ for all $a\in A$, thus $A$ is bounded below. Now, let $v_b=\inf(B)$ and $v_a=\inf(A)$. Since $v_a\geq v$ for all lower bounds $v$ of $A$, and since $v_b$ is a lower bound of $A$, $v_a\geq v_b$, thus $\inf(A)\geq\inf(B)$. Q.E.D.

    \item [28.)] \begin{itemize}
        \item [a.)] Let $u\in\mathbb{R}$ be an upper bound of $S$, thus $u\geq x$ for all $x\in S$, thus $-u\leq-x$ for all $x\in S$, thus $-u\leq y$ for all $y\in-S$, thus $-u$ is a lower bound of $-S$, thus $-S$ is bounded below. Q.E.D.

        \item [b.)] Let $u=\sup(S)$. Since $u$ is an upper bound of $S$, $-u$ is a lower bound of $-S$. For the sake of establishing a contradiction, suppose there exists $v\in\mathbb{R}$ such that $-u<v$ and $v$ is a lower bound of $-S$, thus $u>-v$. Since $v$ is a lower bound of $-S$, $v\leq y$ for all $y\in-S$, thus $-v\geq-y$ for all $y\in-S$, thus $-v\geq x$ for all $x\in S$, thus $-v$ is an upper bound of $S$, but since $u>-v$, $u\neq\sup(S)$\contradiction, thus $-\sup(S)=-u=\inf(-S)$. Q.E.D.
    \end{itemize}

    \item [29.)] \begin{itemize}
        \item [a.)] Since $S\neq\varnothing$, $\mathscr{L}\neq\varnothing$. In addition, since $S$ is bounded below, there exists $v\in\mathbb{R}$ such that $v=\inf(S)$, thus $v\geq x$ for all $x\in\mathscr{L}$, thus $v$ is an upper bound of $\mathscr{L}$, thus $\mathscr{L}$ is bounded above. Q.E.D.

        \item [b.)] Let $w=\sup(\mathscr{L})$. For the sake of establishing a contradiction, suppose there exists $x\in S$ such that $x<w$, thus $x$ is not an upper bound of $\mathscr{L}$, thus there exists $l\in\mathscr{L}$ such that $l>x$, thus $l$ is not a lower bound of $S$, thus $l\notin\mathscr{L}$\contradiction, thus $x\in S\implies x\geq w$, thus $w$ is a lower bound of $S$. Q.E.D.

        \item [c.)] Since $w=\sup(\mathscr{L})$, $w\geq l$ for all $l\in\mathscr{L}$, thus $w\geq l$ for all lower bounds $l$ of $S$, thus $w=\inf(S)$. Q.E.D.
    \end{itemize}

    \item [30.)] \begin{itemize}
        \item [a.)] For unbounded sets, infinite suprema and infima make sense, as for all $x\in\mathbb{R}$, $-\infty<x<\infty$. In addition, there exists no $y\in\mathbb{R}$ such that $y<-\infty$ or $y>\infty$, thus $\sup(S)=\infty$ and $\inf(S)=-\infty$.

        \item [b.)] If we constrict the empty set to being a subset of $\mathbb{R}$, then we can reason that $\infty$ and $-\infty$ are vacuously upper and lower bounds of the empty set. Since $\infty>-\infty$, $\inf(\varnothing)=\infty$, and $\sup(\varnothing)=-\infty$.
    \end{itemize}

    \pagebreak
    \item [31.)] \begin{itemize}
        \item [a.)] False; let $S=\{x\in\mathbb{Q}:0\leq x<\pi\}$. By definition, all $x\in S$ are rational, but $\sup(S)=\pi$ is irrational.

        \item [b.)] False; let $S=\{x\in\mathbb{R}\backslash\mathbb{Q}:0<x<3\}$. By definition, all $x\in S$ are irrational, but $\sup(S)=3$ is rational.
    \end{itemize}

    \item [33.)] Let $x,y\in\mathbb{R}$, and consider $\vert x\vert$:
        \[\vert x\vert=\vert x-y+y\vert\leq\vert x-y\vert+\vert y\vert\]
        \[\implies\vert x\vert\leq\vert x-y\vert+\vert y\vert\]
        \[\implies\vert x\vert-\vert y\vert\leq\vert x-y\vert\]
        \[\implies(\vert x\vert-\vert y\vert)-(\vert x\vert-\vert y\vert)-\vert x-y\vert\leq\vert x\vert-\vert y\vert\leq\vert x-y\vert\]
        \[\implies0-\vert x-y\vert=-\vert x-y\vert\leq\vert x\vert-\vert y\vert\leq\vert x-y\vert\]
        \[\implies\Big\vert\vert x\vert-\vert y\vert\Big\vert\leq\vert x-y\vert\]
        Thus the inequality holds. Q.E.D.

    % \item [34.)] \begin{itemize}
    %     \item [a.)] False; let $S=(-\infty,0]$, thus $\{\vert x\vert:x\in S\}=[0,\infty)$, which has no upper bound.

    %     \item [b.)] True; let $u=\sup(\{\vert x\vert:x\in S\})$,
    % \end{itemize}

    \item [35.)] Let $S$ be a bounded set, thus there exist upper and lower bounds $u,v\in\mathbb{R}$ of $S$, thus $v\leq x\leq u$ for all $x\in S$. Let $M=\max(\vert u\vert,\vert v\vert)$, thus $M\geq0$, $M\geq u$ and $-M\leq v$, thus for all $x\in S$, $-M<x<M$, thus $\vert x\vert<M$ for some $M\geq0$.

    Now suppose there exists $M\geq0$ such that $\vert x\vert\leq M$ for all $x\in S$, thus $-M\leq x\leq M$, thus $-M\leq x$ and $x\leq M$ for all $x$, thus $-M$ and $M$ are lower and upper bounds of $S$ respectively, thus $S$ is bounded, thus $S$ is bounded if and only if there exists $M\geq0$ such that $\vert x\vert\leq M$ for all $x\in S$. Q.E.D.

    \item [36.)] When learning the various integration techniques taught in calculus 2, I noticed that some of them involved manipulating the $dx$ term. I initially found this confusing, as I though that $\dfrac{d}{dx}$ was a single operator that could not be seperated. I later found out that this was not the case, and that it can often be treated just like a fraction. What helped me to realize this was when I watched a video series that explained calculus in a more intuitive, less purely algebraic way. In this context, $dx$ being a seperate variable simply made sense.

    % \item [37.)] Let $u=\sup(A)$ and $v=\inf(B)$. Since $v\leq b$ for all $b\in B$, and since for all lower bounds $l$ of $B$, $v\geq l$,

    % \item [38.)] Let $u=\sup(S)$, thus $u\geq x$ for all $x\in S$. For the sake of establishing a contradiction, suppose $u\notin S$, then for some $\epsilon>0$, $u=x+\epsilon$ for some $x\in S$. Consider $x+\frac{\epsilon}{2}$. Since $\epsilon>0$, $\frac{\epsilon}{2}>0$,

    \item [39.)] \begin{itemize}
        \item [a.)] A sequence is defined as a function $x(n)$ such that $x:\mathbb{N}\rightarrow\mathbb{R}$.

        \item [b.)] The sequence $\{x_n\}^\infty_{n=1}$ converges to $L\in\mathbb{R}$ if and only if
        \[\forall\epsilon>0:\exists k\in\mathbb{N}:n\geq k\implies\vert x_n-L\vert<\epsilon\]
    \end{itemize}

    % \item [40.)] awd

    % \item [41.)] awd

    % \item [42.)] We can simplify the fraction to $\dfrac{7}{25n-10}$.

    \pagebreak
    \item [43.)] \begin{itemize}
        \item [a.)] $\displaystyle\lim_{n\to\infty}\frac{1}{10n}=10$

        \item [b.)] $\displaystyle\lim_{n_\to\infty}\sin n$ diverges

        \item [c.)] Suppose $x_n\to15$ and $x_n\to-77$. Since $x_n\to15$, $x_n$ gets arbitrarily close to $15$. Also, since $x_n\to-77$, $x_n$ gets arbitrarily close to $-77$. However, as $x_n$ gets closer to $15$, $x_n$ moves farther from $-77$, and vice versa, thus $x_n$ cannot get arbitrarily close to both, thus $x_n$ cannot converge to both.
    \end{itemize}

    % \item [44.)] \begin{itemize}
    %     \item [a.)] Let $\{x_n\}^\infty_{n=1}$ be a sequence such that $x_n\to L$, then by definition for all $n>k$ for some $k\in\mathbb{N}$, $\vert x_n-L\vert<\epsilon$ for some $\epsilon>0$. By the reverse triangle inequality, we know that $\Big\vert\vert x_n\vert-\vert L\vert\Big\vert<\vert x_n-L\vert<\epsilon$. Let $\epsilon_1=\vert x_n-L\vert$, thus $\Big\vert\vert x_n\vert-\vert L\vert\Big\vert<\epsilon_1$ for some $\epsilon_1>0$ given $n>k$, thus $\vert x_n\vert\to\vert L\vert$. Q.E.D.

    %     \item [b.)] Consider the sequence $\{x_n\}^\infty_{n=1}$ such that $x_n=$
    % \end{itemize}

\end{itemize}

\end{document}
