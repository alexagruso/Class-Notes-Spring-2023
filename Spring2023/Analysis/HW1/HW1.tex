\documentclass[12pt]{article}
\pagenumbering{gobble}
\linespread{1.075}

\usepackage{amsfonts}
\usepackage{amsmath}
\usepackage{amssymb}
\usepackage{array}
\usepackage{fancyhdr}
\usepackage{textcomp}
\usepackage[margin=1in,headheight=10mm]{geometry}

\newcommand{\contradiction}{%
    \ensuremath{{\Rightarrow\mspace{-2mu}\Leftarrow}}%
}

\newcommand{\angleb}[1]{\left\langle#1\right\rangle}
\newcommand{\vertb}[1]{\left\vert#1\right\vert}

\begin{document}
\pagestyle{fancy}
\fancyhead{}
\fancyhead[L]{Alexander Agruso}
\fancyhead[R]{MATH 3380 Homework 1}

\normalsize
\begin{itemize}
    \item [1.)] I am currently double majoring in math and computer science. I first started as just a computer science major, but added a math major when I started getting into studying math as a hobby. In my free time I enjoy programming, especially making video games, and playing/listening to music. I was actually introduced to studying math through youtube channels like numberphile.

    \item [2.)] \begin{itemize}
        \item [a.)] $4!=4\times3\times2\times1=12\times2=24$

        \item [b.)] $\binom{8}{5}=\frac{8!}{5!(8-5)!}=\frac{8\times7\times6}{6}=8\times7=56$
    \end{itemize}

    \item [3.)] \begin{itemize}
        \item [a.)] Suppose $k=n+1$, then
        \begin{align*}
            \binom{n}{k-1}+\binom{n}{k}&=\binom{n}{n+1-1}+\binom{n}{n+1}\\
            &=\binom{n}{n}+0=1\\
            &=\binom{n+1}{n+1}\\
            &=\binom{n+1}{k}
        \end{align*}
        Thus $k=n+1$ satisfies the equation. Now suppose $k<0$. According to the definition, $\binom{n}{k-1}=\binom{n}{k}=\binom{n+1}{k}=0$,
        thus
        \begin{equation*}
            \binom{n}{k-1}+\binom{n}{k}=0+0=\binom{n+1}{k}=0
        \end{equation*}
        Thus the equality holds for all $k\leq n+1$. Q.E.D.

        \item [b.)] We have not considered $k>n+1$. However, this case is similar to proving for all $k<0$, as we can leverage the binomial definition to show that all terms equal 0 given $k>n+1$.
    \end{itemize}

    \item [4.)] According to the binomial theorem, we know that
    \begin{equation*}
        (1+x)^n=\sum_{k=0}^{n}\binom{n}{k}x^k
    \end{equation*}
    We can manipulate this as follows:
    \begin{equation*}
        (1+x)^n=\sum_{k=0}^{n}\binom{n}{k}x^k=\binom{n}{0}x^0+\binom{n}{1}x^1+\hdots+\binom{n}{k}x^k
    \end{equation*}
    \begin{equation*}
        (1+1)^n=\binom{n}{0}1^0+\binom{n}{1}1^1+\hdots+\binom{n}{k}1^k=\binom{n}{0}+\binom{n}{1}+\hdots+\binom{n}{k}
    \end{equation*}
    \begin{equation*}
        2^n=\sum_{k=0}^{n}\binom{n}{k}
    \end{equation*}
    Thus the equality holds. Q.E.D.

    \item [5.)] Consider the base case where $n=1$:
    \begin{equation*}
        \sum_{k=1}^{1}(2k-1)=2(1)-1=1=1^2
    \end{equation*}
    Thus the base case holds. Next, suppose $n$ satisfies the equation, and consider $n+1$:
    \begin{align*}
        \sum_{k=1}^{n+1}(2k-1)&=\sum_{k=1}^{n}(2k-1)+2(n+1)-1\\
        &=n^2+2n+2-1\\
        &=n^2+2n+1\\
        &=(n+1)^2
    \end{align*}
    Thus the equality holds. Q.E.D.

    \item [6.)] Consider the base case where $n=1$:
    \begin{equation*}
        \sum_{k=1}^{1}k^3=1^3=1=\frac{1^2(1+1)^2}{4}=\frac{4}{4}=1
    \end{equation*}
    Thus the base case holds. Next, suppose $n$ satisfies the equation, and consider $n+1$:
    \begin{align*}
        \sum_{k=1}^{n+1}k^3&=\sum_{k=1}^{n}k^3+(n+1)^3\\
        &=\frac{n^2(n+1)^2}{4}+\frac{4(n+1)^3}{4}\\
        &=\frac{n^2(n^2+2n+1)+4(n^3+3n^2+3n+1)}{4}\\
        &=\frac{n^4+6n^3+13n^2+12n+4}{4}\\
        (\text{substituting }n+1)&=\frac{(n+1)^2((n+1)+1)^2}{4}\\
        &=\frac{(n+1)^2(n+2)^2}{4}\\
        &=\frac{(n^2+2n+1)(n^2+4n+4)}{4}\\
        &=\frac{(n^4+4n^3+4n^2)+(2n^3+8n^2+8n)+(n^2+4n+4)}{4}\\
        &=\frac{n^4+6n^3+13n^2+12n+4}{4}
    \end{align*}
    Thus the equality holds. Q.E.D.

    \pagebreak
    \item [7.)] \begin{itemize}
        \item [a.)] Let $a=0$ and $b=1$, thus $ax=0x=0$, thus $ax+b=0+1=1=0$ \contradiction, thus the statement is false. Q.E.D.

        \item [b.)] The proof does not restrict $a$ to the nonzero reals. When $a=0$, $-b/a$ is undefined, making the equation invalid.
    \end{itemize}

    \item [8.)] \begin{itemize}
        \item [a.)] Consider $\sqrt[3]{y}$, $(\sqrt[3]{y})^3=y$, thus for all $y$ we can construct $x$ such that $x^3=y$, thus the statement is true. Q.E.D.

        \item [b.)] Suppose you have $x,y\in\mathbb{R}$ such that $x^3=y$. For the statement $\exists x\in\mathbb{R}\break[\forall y\in\mathbb{R}(x^3=y)]$ to be true, $x^3=y+1$ must hold, thus $x^3=x^3+1$ but there are no real solutions $x$ for this equation \contradiction, thus the original statement is false. Q.E.D.
    \end{itemize}

    \item [9.)] \begin{itemize}
        \item [a.)] Consider $n+1$: $n+1\in\mathbb{N}$ and $n+1>n$, thus the statement is true. Q.E.D.

        \item [b.)] Consider $j+1$: $j+1\in\mathbb{N}$, but $j\not\geq j+1$, thus the statement is false. Q.E.D.
    \end{itemize}

    \item [10.)] \begin{itemize}
        \item [a.)] Let $x=0$, $xy=0y=0$ for all real $y$, thus the statement is true. Q.E.D.

        \item [b.)] Let $y=1$, $xy=1x=x$ for all real $x$, thus the statement is true. Q.E.D.
    \end{itemize}
\end{itemize}

\end{document}
