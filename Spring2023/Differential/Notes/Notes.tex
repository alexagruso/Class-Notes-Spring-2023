\documentclass[12pt]{article}
\pagenumbering{gobble}
\linespread{1.15}

\usepackage{amsfonts}
\usepackage{amsmath}
\usepackage{amssymb}
\usepackage{array}
\usepackage{fancyhdr}
\usepackage{textcomp}
\usepackage[margin=1in,headheight=1in]{geometry}

\newcommand{\contradiction}{
    \ensuremath{{\Rightarrow\mspace{-2mu}\Leftarrow}}
}

\newcommand{\angleb}[1]{\left\langle#1\right\rangle}
\newcommand{\vertb}[1]{\left\vert#1\right\vert}
\newcommand{\bracks}[1]{\left[#1\right]}

\newcommand{\derv}[2]{\dfrac{d#1}{d#2}}
\newcommand{\pderv}[2]{\dfrac{\partial#1}{\partial#2}}

\begin{document}
\pagestyle{fancy}
\fancyhead{}
\fancyhead[L]{Alexander Agruso}
\fancyhead[R]{MATH 3323 Homework 6}

\normalsize
\section*{Second Order with Constant Coefficients}
\[ay''+by'+cy=0\implies ar^2+br+c=0\]
\subsection*{Case 3: Equal Roots}
\[r=r_1=r_2\implies r=\frac{-b\pm\sqrt{b^2-4ac}}{2a}\]
\[\therefore\ y_1=e^{rt}\]
\subsubsection*{Reduction of Order}
\[y_2=v(t)y_1=ve^{rt}\]
\[\therefore\ ay_2''+by_2'+cy_2=0\]
\[y_2'=v'e^{rt}+rve^{rt}\]
\[y_2''=v''e^{rt}+2rv'e^{rt}+r^2ve^{rt}\]
\[\therefore\ a(v''e^{rt}+2rv'e^{rt}+r^2ve^{rt})+b(v'e^{rt}+rve^{rt})+cve^{rt}=0\]
\[\implies a(v''+2rv'+r^2v)+b(v'+rv)+cv=0\]
\[\implies av''+v'(2ar+b)+v(ar^2+br+c)+=0\]
\[\implies av''+v'(2ar+b)=av''+v'\left[2a\left(\frac{-b}{2a}+b\right)\right]=0\]
\[\implies av''=v''=0\]
\[v=\int\int0\ dt=c_1t+c_2\]
\[\therefore\ y=vy_1=(c_1t+c_2)e^{rt}=c_1te^{rt}+c_2e^{rt}\]
Which is the general solution.

\section*{Theorem 3.2.1 Existence \& Uniqueness}
Consider the initial value problem
\[y''+p(t)y'+q(t)y=g(t);\ y(t_0)=y_0;\ y'(t_0)=y'_0\]
Where $p,q$, and $g$ are continuous on the open interval $I$ that contains point $t_0$, then there is exactly one solution $y=\phi(t)$ of this problem, and the solution exists on the interval $I$.

\begin{itemize}
    \item [e.g.)] Find the longest interval in which the solution of the initial value problem
          \[(t^2-3t)y''+ty'-(t-3)t=0;\ y(1)=2;\ y'(1)=1\]
          is certain to exist.\newline
          Solution:

          \[y''+\frac{1}{(t-3)}y'+\frac{1}{t}y=0\]
          \[\therefore\ p(t)=\frac{1}{t-3},\ q(t)=\frac{1}{t},\ g(t)=0\]
          $p$ is continuous when $t\neq3$, $q$ is continuous when $t\neq0$, $g$ is always continuous, thus $I=(0,3)$.
\end{itemize}
\subsection*{Principle of Super Position}
Given solutions $y_1$ and $y_2$ to the following different equation
\[y''+p(t)y'+q(t)y=0\]
$c_1y_1+c_2y_2$ is a solution for all $c_1,c_2\in\mathbb{R}$.
\subsection*{Theorem}
$c_1y_1+c_2y_2$ is a general solution
\subsubsection*{Proof}
Apply initial conditions:
\[c_1y_1(t_0)+c_2y_2(t_0)=y_0\]
\[c_1y_1'(t_0)+c_2y_2'(t_0)=y'_0\]
\[w=\begin{vmatrix}
        y_1(t_0) & y_2(t_0) \\y'_1(t_0) & y'_2(t_0)
    \end{vmatrix}=y_1(t_0)y'_2(t_0)-y_2(t_0)y'_1(t_0)\]
\subsection*{Wronskian of solutions $y_1$ and $y_2$}
\[W[y_1,y_2]=w=\begin{vmatrix}
        y_1(t_0) & y_2(t_0) \\ y'_1(t_0) & y'_2(t_0)
    \end{vmatrix}\]
If $w\neq0$ for $y_1$, $y_2$, then they are a fundamental set
\section*{Theorem 3.2.4}
Let $y_1$, $y_2$ be solutions to the following differential equation
\[y''+p(t)y'+q(t)y=0\]
Then $y=c_1y_1+c_2y_2$ is a basis for the solution family of the equation.
\subsection*{Proof}
Let $\phi(t)$ be a solution to the equation, and $t_0$ be a point where $W[y_1,y_2]\neq0$. Let $y_0=\phi(t_0)$, $y'_0=\phi'(t_0)$, thus the equation is satisfied. Q.E.D.

\end{document}
