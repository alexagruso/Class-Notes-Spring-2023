\documentclass[12pt]{article}

\usepackage{amsfonts}
\usepackage{amsmath}
\usepackage{amssymb}
\usepackage{cite}
\usepackage{enumerate}
\usepackage{fancyhdr}
\usepackage[headheight=1in,margin=1.25in]{geometry}
\usepackage[colorlinks=true,linkcolor=blue]{hyperref}
\usepackage{mathtools}
\usepackage{scalerel}
\usepackage{setspace}

\usepackage{tikz}

\newcommand{\N}{\ensuremath{\mathbb{N}}}
\newcommand{\Z}{\ensuremath{\mathbb{Z}}}
\newcommand{\Q}{\ensuremath{\mathbb{Q}}}
\newcommand{\R}{\ensuremath{\mathbb{R}}}
\newcommand{\C}{\ensuremath{\mathbb{C}}}

\newcommand{\e}{\ensuremath{\varepsilon}}
\renewcommand{\d}{\ensuremath{\delta}}

\newcommand{\angleb}[1]{\left\langle#1\right\rangle}
\newcommand{\braceb}[1]{\left\{#1\right\}}
\newcommand{\bracketb}[1]{\left[#1\right]}
\newcommand{\parenb}[1]{\left(#1\right)}
\newcommand{\vertb}[1]{\left\vert#1\right\vert}
\newcommand{\dvertb}[1]{\left\Vert#1\right\Vert}

\DeclarePairedDelimiter\floor{\lfloor}{\rfloor}
\DeclarePairedDelimiter\ceil{\lceil}{\rceil}

\renewcommand{\Re}{\text{Re}}
\renewcommand{\Im}{\text{Im}}

\newcommand{\comp}{\complement}
\newcommand{\sdiff}{\setminus}

\newcommand{\solution}{\textit{Solution: }}
\newcommand{\proof}{\textit{Proof: }}
\newcommand{\partialdone}{\ensuremath{\strut\hfill\blacktriangle}}
\newcommand{\done}{\ensuremath{\strut\hfill\blacksquare}}

\newcommand{\mc}[1]{\ensuremath{\mathcal{#1}}}

\renewcommand{\t}[1]{\text{ #1 }}
\newcommand{\impl}{\ensuremath{\implies}}
\newcommand{\sectionskip}{\vspace{0.15in}}
\newcommand{\tri}{\triangle}
\newcommand{\ovl}[1]{\ensuremath{\overline{#1}}}

\begin{document}

\pagestyle{fancy}
\fancyhead[L]{Alex Agruso}
\fancyhead[C]{Integration Theory}
\fancyhead[R]{Journal}

\setlength{\parindent}{0in}
\setlength{\parskip}{0.1in}
\setstretch{1}

\section*{Chapter 4: Hilbert Spaces}

\textbf{5)}
There exists a function \( f \in L^1(\R^d) \) where \( f \notin L^2(\R^d) \),
and a function \( g \in L^2(\R^d) \) where \( g \in L^1(\R^d) \).

However, if \( f \) is a measurable function supported on a set \( E \) of
finite measure, and if \( f \in L^2(\R^d) \), we have that
\( f \in L^1(\R^d) \) and
\[
	\dvertb{f}_{L^1(\R^d)}
	\leq m(E)^{1/2} \dvertb{f}_{L^2(\R^d)}.
\]
Furthermore, if \( f \) is bounded, say by \( M \), and if
\( f \in L^1(\R^d) \), then \( f \in L^2(\R^d) \) and
\[
	\dvertb{f}_{L^2(\R^d)}
	\leq M^{1/2} \dvertb{f}_{L^1(\R^d)}^{1/2}.
\]

\proof
(1)

(2)

Finally, suppose \( f \) is

\textbf{6)}
The following sets of functions are dense subspaces of \( L^2(\R^d) \):
\begin{enumerate}[(a)]
	\item Simple functions

	\item Continuous functions of compact support
\end{enumerate}

\proof
Fix \( \e > 0 \) and a simple function \( f \) on \( \R^d \).
Using the canonical representation of \( f \), we obtain disjoint measurable
sets \( E_1, E_2, \dots, E_n \) of finite measure and constants
\( a_1, a_2, \dots, a_n \) in \( \R \) where
\[
	f(x) = \sum_{1 \leq k \leq n} a_k\chi_{E_k}(x).
\]
Also denote \( E = \bigcup_{k = 1}^n E_k \).
We will now show that the function \( g \), defined as
\[
	g(x) = f(x) - \d\chi_{E}(x)
\]
where \( 0 < \d < \sqrt{\e/m(E)} \), is a simple function, and that
\( \dvertb{f - g}_{L^2} < \e \).

We first note that
\[
	\d\chi_{E}(x)
	= \sum_{1 \leq k \leq n} \d\chi_{E_k}(x)
\]
since the \( E_k \) are disjoint.
This leads to
\[
	g(x)
	= f(x) - \d\chi_{E}(x)
	= \sum_{1 \leq k \leq n} a_k\chi_{E_k}(x)
	- \sum_{1 \leq k \leq n} \d\chi_{E_k}(x)
	= \sum_{1 \leq k \leq n} \parenb{a_k - \d}\chi_{E_k}(x),
\]
thus showing that \( g \) is a simple function.
Finally, we compute \( \dvertb{f - g}_{L^2} \) as follows:
\[
	\dvertb{f - g}_{L^2}
	= \int \vertb{f(x) - g(x)}^2 \, dx
	= \int \vertb{f(x) - f(x) + \d\chi_{E}(x)}^2 \, dx
	= \int \vertb{\d\chi_{E}(x)}^2 \, dx
\]
\[
	= \int \vertb{\d}^2 \vertb{\chi_E(x)}^2 \, dx
	= \int \delta^2 \chi_E(x) \, dx
	= \int_E \delta^2 \, dx
	< \int_E \frac{\e}{m(E)} \, dx
	= m(E)\frac{\e}{m(E)}
	= \e,
\]
thus proving (a).

We now let \( f \) be a continuous function of compact support.

\end{document}
