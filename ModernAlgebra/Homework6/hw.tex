\documentclass[12pt]{article}

\usepackage{amsfonts}
\usepackage{amsmath}
\usepackage{amssymb}
\usepackage{fancyhdr}
\usepackage[headheight=1in,margin=1in]{geometry}
\usepackage[colorlinks=true,linkcolor=blue]{hyperref}
\usepackage{mathtools}
\usepackage{tikz}

\newcommand{\N}{\mathbb{N}}
\newcommand{\Z}{\mathbb{Z}}
\newcommand{\Q}{\mathbb{Q}}
\newcommand{\R}{\mathbb{R}}
\newcommand{\C}{\mathbb{C}}
\newcommand{\F}{\mathbb{F}}

\renewcommand{\t}[1]{\text{#1}}

\newcommand{\angleb}[1]{\left\langle#1\right\rangle}
\newcommand{\braceb}[1]{\left\{#1\right\}}
\newcommand{\bracketb}[1]{\left[#1\right]}
\newcommand{\parenb}[1]{\left(#1\right)}
\newcommand{\vertb}[1]{\left\vert#1\right\vert}
\newcommand{\ovl}[1]{\overline{#1}}

\newcommand{\normsub}{\trianglelefteq}
\newcommand{\normsup}{\trianglerighteq}

\newcommand{\proof}{\textit{Proof: }}
\newcommand{\done}{\ensuremath{\strut\hfill\blacksquare}}

\begin{document}

\pagestyle{fancy}
\fancyhead[L]{Modern Algebra}
\fancyhead[C]{Alex Agruso}
\fancyhead[R]{Homework 5}

\setlength{\parindent}{0in}
\setlength{\parskip}{0.09in}

\subsection*{4.1 Group Actions and Permutation Representations}

\textbf{1)}
Let \( G \) act on a set \( A \).
If \( a, b \in A \) where \( b = g \cdot a \) for some \( g \in G \), then
\( G_b = gG_ag^{-1} \).
Additionally, if \( G \) acts transitively on \( A \), then its kernel is
\( \bigcap_{g \in G} gG_bg^{-1} \) for some \( b \in A \).

\proof
Ler \( x \in G_b \), then \( x \cdot b = (xg) \cdot a = g \cdot a \), thus
\( (g^{-1}xg) \cdot a = a \) and \( g^{-1}xg \in G_a \).
Since \( x = g(g^{-1}xg)g^{-1} \), we have that \( x \in gG_ag^{-1} \), thus
\( G_b \subseteq gG_ag^{-1} \).
Now, let \( x \in gG_ag^{-1} \), then \( x = gyg^{-1} \) for some
\( y \in G_a \).
Since \( b = g \cdot a \), we have that \( a = g^{-1} \cdot b \), thus
\( y \cdot a = (yg^{-1}) \cdot b = g^{-1} \cdot b \), showing that
\( (gyg^{-1}) \cdot b = b \) and \( x = gyg^{-1} \in G_b \).
From this we have that \( gG_ag^{-1} \subseteq G_b \), thus
\( G_b = gG_ag^{-1} \).

We know that the kernel of a group action is \( \bigcap_{a \in A} G_a \).
Fix \( b \in A \).
Assuming \( G \) acts transitively on \( A \), we have for all \( a \in A \)
that there exists \( g \in G \) where \( b = g \cdot a \),
thus \( G_a = gG_bg^{-1} \) for all \( a \).
We also have that \( a = g^{-1} \cdot b \), thus
\( gG_bg^{-1} = g(g^{-1}G_ag)g^{-1} = G_a \) for any \( g \in G \), thus
\( \bigcap_{a \in A} G_a = \bigcap_{g \in G} gG_bg^{-1} \) and we are done.
\done

\subsection*{4.2 Groups Acting on Themselves by Left Multiplication}

\textbf{2)}
Let \( S_3 \) act on itself by left multiplication and denote
\( \phi : S_3 \to S_6 \) as the left regular representation of this action.
Indexing \( S_3 = \braceb{e, (1\ 2), (1\ 3), (2\ 3), (1\ 2\ 3), (1\ 3\ 2)} \)
as \( \braceb{1, 2, 3, 4, 5, 6} \), we have that \( \phi \) is defined as
follows:
\begin{align*}
	\phi(e)         & = e                  \\
	\phi((1\ 2))    & = (1\ 2)(3\ 5)(4\ 6) \\
	\phi((1\ 3))    & = (1\ 3)(2\ 5)(4\ 6) \\
	\phi((2\ 3))    & = (1\ 4)(2\ 6)(3\ 5) \\
	\phi((1\ 2\ 3)) & = (1\ 5\ 6)(2\ 3\ 4) \\
	\phi((1\ 3\ 2)) & = (1\ 6\ 5)(2\ 4\ 3)
\end{align*}

\textbf{5a)}
Let \( D_8 \) act on the set of left cosets of \( H = \angleb{s} \) by left
multiplication and index the cosets \( \braceb{H, rH, r^2H, r^3H} \) as
\( \braceb{1, 2, 3, 4} \), then the permutation representation of this action
\( \phi : D_8 \to S_4 \) is defined as follows:
\begin{align*}
	\phi(e)    & = e            \\
	\phi(r)    & = (1\ 2\ 3\ 4) \\
	\phi(r^2)  & = (1\ 3)(2\ 4) \\
	\phi(r^3)  & = (1\ 4\ 3\ 2) \\
	\phi(s)    & = (2\ 4)       \\
	\phi(sr)   & = (1\ 4)(2\ 3) \\
	\phi(sr^2) & = (1\ 3)       \\
	\phi(sr^3) & = (1\ 2)(3\ 4)
\end{align*}
Because \( \phi \) is injective, we deduce that \( D_8 \cong \phi(D_8) \), and
thus the action is faithful.

\textbf{8)}
If \( H \) is a subgroup of \( G \) with finite index \( n \), then there
exists a normal subgroup \( K \normsub G \) where \( K \leq H \) and
\( \vertb{G : K} \leq n! \).

\proof
Let \( G \) act on the set of cosets of \( H \) by left multiplication.
This action induces a permutation representation \( \phi : G \to S_n \) which
is a homomorphism.
Using Cayley's theorem, we have that \( G/\ker\phi \) is isomorphic to some
subgroup of \( S_n \), thus \( \vertb{G/\ker\phi} \) divides
\( \vertb{S_n} = n! \), and thus \( \vertb{G : \ker\phi} \leq n! \).
Since \( K = \ker\phi \) is a normal subgroup of \( G \), we are finished.
\done

\subsection*{4.3 Group Actions and Permutation Representations}

\textbf{2a)} The conjugacy classes of \( D_8 \) are \( \braceb{e} \),
\( \braceb{r^2} \), \( \braceb{s, r^2s} \), \( \braceb{rs,r^3s} \), and
\( \braceb{r, r^3} \).

\textbf{2c)} The conjugacy classes of \( A_4 \) are
\[ \braceb{e} \]
\[ \braceb{(1\ 2\ 3), (2\ 4\ 3), (1\ 4\ 2), (1\ 3\ 4)} \]
\[ \braceb{(1\ 3\ 2), (1\ 2\ 4), (1\ 4\ 3), (2\ 3\ 4)} \]
\[ \braceb{(1\ 2)(3\ 4), (1\ 3)(2\ 4), (1\ 4)(2\ 3)} \]

\textbf{5)}
If \( G \) is a group where \( \vertb{G : Z(G)} = n \), then the size of each
conjugacy class in \( G \) is at most \( n \).

\proof
Fix \( a \in G \) where \( a \notin Z(G) \), then the size of the conjugacy
class of \( a \) is \( \vertb{G : C_G(a)} \).
Since \( Z(G) \) is contained in \( C_G(a) \), we have that
\[
	\frac{\vertb{G}}{\vertb{C_G(a)}}
	\leq \frac{\vertb{G}}{\vertb{Z(G)}}
	= n,
\]
showing that \( \vertb{G : C_G(a)} \leq n \).
\done

\textbf{7)}
The partitions of \( 3 \) are \( 1 + 1 + 1 \), \( 1 + 2 \), and \( 3 \), with
respective cycle representatives \( e \), \( (1\ 2) \), and \( (1\ 2\ 3) \).

The partitions of \( 4 \) are \( 1 + 1 + 1 + 1 \), \( 1 + 1 + 2 \),
\( 1 + 3 \), \( 2 + 2 \), and \( 4 \), with respective cycle representatives
\( e \), \( (1\ 2) \), \( (1\ 2\ 3) \), \( (1\ 2)(3\ 4) \), and
\( (1\ 2\ 3\ 4) \).

The partitions of \( 6 \) are \( 1 + 1 + 1 + 1 + 1 + 1 \),
\( 1 + 1 + 1 + 1 + 2 \), \( 1 + 1 + 1 + 3 \), \( 1 + 1 + 4 \), \( 1 + 5 \),
\( 1 + 1 + 2 + 2 \), \( 1 + 2 + 3 \), \( 2 + 2 + 2 \), \( 2 + 4 \),
\( 3 + 3 \), and \( 6 \), with cycle representatives \( e \), \( (1\ 2) \),
\( (1\ 2\ 3) \), \( (1\ 2\ 3\ 4) \), \( (1\ 2\ 3\ 4\ 5) \), \( (1\ 2)(3\ 4) \),
\( (1\ 2)(3\ 4\ 5) \), \( (1\ 2)(3\ 4)(5\ 6) \), \( (1\ 2)(3\ 4\ 5\ 6) \),
\( (1\ 2\ 3)(4\ 5\ 6) \), and \( (1\ 2\ 3\ 4\ 5\ 6) \).

\textbf{10)}
Given \( \sigma = (1\ 2\ 3\ 4\ 5) \), \( \tau_1 = (2\ 3\ 5\ 4) \), and
\( \tau_2 = (2\ 5)(3\ 4) \), we have that
\( \tau_1\sigma\tau_1^{-1} = \sigma^2 \) and
\( \tau_2\sigma\tau_2^{-1} = \sigma^{-1} \).

\( \t{Aut}(I^{sm}) \)


\end{document}
