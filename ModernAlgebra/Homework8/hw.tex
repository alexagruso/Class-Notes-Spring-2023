\documentclass[12pt]{article}

\usepackage{amsfonts}
\usepackage{amsmath}
\usepackage{amssymb}
\usepackage{fancyhdr}
\usepackage[headheight=1in,margin=1.25in]{geometry}
\usepackage[colorlinks=true,linkcolor=blue]{hyperref}
\usepackage{makecell}
\usepackage{mathtools}
\usepackage{setspace}
\usepackage{tikz}

\newcommand{\N}{\mathbb{N}}
\newcommand{\Z}{\mathbb{Z}}
\newcommand{\Q}{\mathbb{Q}}
\newcommand{\R}{\mathbb{R}}
\newcommand{\C}{\mathbb{C}}
\newcommand{\F}{\mathbb{F}}

\renewcommand{\t}[1]{\text{#1}}

\newcommand{\angleb}[1]{\left\langle#1\right\rangle}
\newcommand{\braceb}[1]{\left\{#1\right\}}
\newcommand{\bracketb}[1]{\left[#1\right]}
\newcommand{\parenb}[1]{\left(#1\right)}
\newcommand{\vertb}[1]{\left\vert#1\right\vert}
\newcommand{\ovl}[1]{\overline{#1}}

\newcommand{\normsub}{\trianglelefteq}
\newcommand{\normsup}{\trianglerighteq}

\newcommand{\aut}{\t{Aut}}
\newcommand{\inn}{\t{Inn}}

\newcommand{\proof}{\textit{Proof: }}
\newcommand{\done}{\ensuremath{\strut\hfill\blacksquare}}

\begin{document}

\pagestyle{fancy}
\fancyhead[L]{Modern Algebra}
\fancyhead[C]{Alex Agruso}
\fancyhead[R]{Homework 8}

\setlength{\parindent}{0in}
\setlength{\parskip}{0.15in}
\setstretch{1}

\subsubsection*{4.1 Direct Products}

\textbf{7)} -

\subsubsection*{4.2 The Fundamental Theorem of Finitely Generated Abelian Groups}

\textbf{1)}
\begin{center}
	\begin{tabular}{c|c}
		\( \vertb{G} \)
		& \makecell{\# \textbf{of abelian} \\ \textbf{groups}} \\
		\hline
		\hline
		100   & 4  \\
		576   & 22 \\
		1155  & 1  \\
		42875 & 10 \\
		2704  & 10
	\end{tabular}
\end{center}

\textbf{2)}
\begin{center}
	\begin{tabular}{c|c}
		\( \vertb{G} \) & \textbf{invariant factors} \\
		\hline\hline
		270  & \makecell{
			\(
				2 \cdot 3^3 \cdot 5;
				2 \cdot 3^2 \cdot 5, 3;
				2 \cdot 3 \cdot 5, 3, 3
			\)
		} \\
		\hline\hline
		9801 & \makecell{
			\(
				3^4 \cdot 11^2;
				3^4 \cdot 11, 11;
				3^3 \cdot 11^2, 3;
				3^3 \cdot 11, 3 \cdot 11;
			\) \\ \(
				3^2 \cdot 11^2, 3^2;
				3^2 \cdot 11^2, 3, 3;
				3^2 \cdot 11, 3^2 \cdot 11;
				3^2 \cdot 11, 3 \cdot 11, 3;
			\) \\ \(
				3 \cdot 11^2, 3, 3, 3;
				3 \cdot 11, 3 \cdot 11, 3, 3
			\)
		} \\
		\hline\hline
		320 & \makecell{
			\(
				2^6 \cdot 5;
				2^5 \cdot 5, 2;
				2^4 \cdot 5, 2^2;
				2^4 \cdot 5, 2, 2;
			\) \\ \(
				2^3 \cdot 5, 2^3;
				2^3 \cdot 5, 2^2, 2;
				2^3 \cdot 5, 2, 2, 2;
				2^2 \cdot 5, 2^2, 2^2;
			\) \\ \(
				2^2 \cdot 5, 2^2, 2, 2;
				2^2 \cdot 5, 2, 2, 2, 2;
				2 \cdot 5, 2, 2, 2, 2, 2
			\)
		}
	\end{tabular}
\end{center}

\textbf{3)} (Given in same order as the invariant factors).
\begin{center}
	\begin{tabular}{c|c}
		\( \vertb{G} \) & \textbf{elementary divisors} \\
		\hline\hline
		270  & \makecell{
			\(
				(2, 5, 27);
				(2, 3, 5, 9);
				(2, 3, 5)
			\)
		} \\
		\hline\hline
		9801 & \makecell{
			\(
				(81, 121);
				(11, 11, 81);
				(3, 27, 121);
				(3, 11, 11, 27);
			\) \\ \(
				(9, 9, 121);
				(3, 3, 9, 121);
				(9, 9, 11, 11);
				(3, 3, 9, 11, 11);
			\) \\ \(
				(3, 3, 3, 3, 121);
				(3, 3, 3, 3, 11, 11)
			\)
		} \\
		\hline\hline
		320 & \makecell{
			\(
				(5, 64);
				(2, 5, 32);
				(4, 5, 16);
				(2, 2, 5, 16)
			\) \\ \(
				(5, 8, 8);
				(2, 4, 5, 8);
				(2, 2, 2, 5, 8);
				(4, 4, 4, 5);
			\) \\ \(
				(2, 2, 4, 4, 5);
				(2, 2, 2, 2, 4, 5);
				(2, 2, 2, 2, 2, 2, 5)
			\)
		}
	\end{tabular}
\end{center}


\textbf{4a)}
The only pair of isomorphic groups is \( \Z_9 \times \Z_4 \) and
\( \Z_4 \times \Z_9 \).

\textbf{4b)}
The only pair of isomorphic groups is \( \braceb{2^2, 2 \cdot 3^2} \) and
\( \braceb{2^2 \cdot 3^2, 2} \).

\subsubsection*{4.4 Recognizing Direct Products}

\textbf{5)}
If \( n \geq 5 \), then the commutator subgroup \( S'_n \) of \( S_n \) is
\( A_n \).

\proof
Let \( (a\ b\ c) \) be any 3-cycle in \( S_n \), then
\( (a\ b\ c) = (a\ c)(c\ b)(a\ c)(c\ b) = (a\ c)^{-1}(c\ b)^{-1}(a\ c)(c\ b) \),
and thus is a commutator in \( S_n \).
Since \( A_n \) is generated by the 3-cycles in \( S_n \), we have that
\( A_n \subseteq S'_n \).
Conversely, because \( [S_n : A_n] = 2 \), we have that \( S_n / A_n \) is
cyclic and hence abelian, showing that \( S'_n \subseteq A_n \) and thus
\( S'_n = A_n \).
\done

\textbf{7)}
Fix a prime \( p \) and a non-abelian group \( P \) with order \( p^3 \), then
\( P' = Z(P) \).

\proof
Since \( P \) is a p-group, we have that \( Z(P) \ne \braceb{e} \), and since
\( P \) is non-abelian, \( \vertb{Z(P)} \ne p^3 \).
We also have that \( \vertb{Z(P)} \ne p^2 \), else \( \vertb{P / Z(P)} \)
would have order \( p \) and thus be cyclic, additionally implying that \( P \)
is abelian, a contradiction.
Thus, it must be the case that \( \vertb{Z(P)} = p \), which implies
\( \vertb{P / Z(P)} = p^2 \), showing that \( P / Z(P) \) is abelian.
Since \( P / \braceb{e} \cong P \) is non-abelian, we have that
\( Z(P) \) is the smallest normal subgroup of \( P \) whose quotient is
abelian, thus proving that \( Z(P) = P' \).
\done

\textbf{10)}
If \( G \) is a finite abelian group, then
\( G \cong S_1 \times \cdots \times S_n \), where each \( S_i \) is some Sylow
subgroup.

\proof
Since \( G \) covered by the \( S_i \), we have that
\( G = S_1S_2 \cdots S_n \).
Additionally, \( G \) is abelian, so each \( S_i \) is normal, and since
\( S_i \cap S_j = \braceb{e} \) for \( i \ne j \), we have that
\( S_1S_2 \cdots S_n \cong S_1 \times S_2 \times \cdots \times S_n \).
\done

\end{document}
