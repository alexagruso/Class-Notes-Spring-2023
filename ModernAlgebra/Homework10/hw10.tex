\documentclass[12pt]{article}

\usepackage{amsfonts}
\usepackage{amsmath}
\usepackage{amssymb}
\usepackage{cite}
\usepackage{enumerate}
\usepackage{fancyhdr}
\usepackage[headheight=1in,margin=1.25in]{geometry}
\usepackage[colorlinks=true,linkcolor=blue]{hyperref}
\usepackage{mathtools}
\usepackage{setspace}

\newcommand{\N}{\ensuremath{\mathbb{N}}}
\newcommand{\Z}{\ensuremath{\mathbb{Z}}}
\newcommand{\Q}{\ensuremath{\mathbb{Q}}}
\newcommand{\R}{\ensuremath{\mathbb{R}}}
\newcommand{\C}{\ensuremath{\mathbb{C}}}
\newcommand{\F}{\ensuremath{\mathbb{F}}}

\newcommand{\e}{\ensuremath{\varepsilon}}
\renewcommand{\d}{\ensuremath{\delta}}

\newcommand{\angleb}[1]{\left\langle#1\right\rangle}
\newcommand{\braceb}[1]{\left\{#1\right\}}
\newcommand{\bracketb}[1]{\left[#1\right]}
\newcommand{\parenb}[1]{\left(#1\right)}
\newcommand{\vertb}[1]{\left\vert#1\right\vert}

\newcommand{\comp}{\complement}
\newcommand{\sdiff}{\setminus}

\newcommand{\proof}{\textit{Proof: }}
\newcommand{\done}{\ensuremath{\strut\hfill\blacksquare}}

\newcommand{\mc}[1]{\ensuremath{\mathcal{#1}}}

\renewcommand{\t}[1]{\text{ #1 }}
\newcommand{\sectionskip}{\vspace{0.1in}}

\bibliographystyle{plain}

\begin{document}

\pagestyle{fancy}
\fancyhead[L]{Modern Algebra}
\fancyhead[C]{Alex Agruso}
\fancyhead[R]{Homework 10}

\setlength{\parindent}{0in}
\setlength{\parskip}{0.1in}
\setstretch{1}

\subsection*{8.1 Euclidean Domains}

\textbf{3)}
Let \( R \) be a Euclidean domain with norm \( n \), and \( m \) the minimum
over the norm of all nonzero elements of \( R \).
If \( n(a) = m \) for some nonzero \( a \in R \), then \( a \) is a unit.

\proof
\( R \) is a Euclidean domain, and so there exist unique \( q, r \in R \) where
\( 1 = qa + r \) and where \( r = 0 \) or \( n(r) < n(q) \).
If \( r = 0 \), we are finished, so assume \( r \ne 0 \), then
\( n(r) < n(q) \).
We also have that \( 1 = aq + r \) since \( R \) is commutative, thus
\( n(r) < n(a) \).
This is a contradiction however, since by assumption \( n(a) \) is minimal
over nonzero elements of \( R \).
This shows that \( r = 0 \), completing the proof.
\done

\subsection*{8.2 Principal Ideal Domains}

\textbf{3)}
If \( R \) is a principal ideal domain and \( P \) a prime ideal of \( R \),
then \( R/P \) is also a principal ideal domain.

\proof
If \( P = (0) \), then \( R/P \cong R \) and we are done.
Otherwise, \( P \) is nonzero.
Since every non-zero prime ideal in a principal ideal domain is maximal, we
have that \( R/P \) is a field, and thus a principal ideal domain.
\done

\subsection*{9.2 Polynomial Rings over Fields}

\textbf{1)}
Given a field \( F \), fix \( f \in F[X] \) with degree \( n \geq 1 \).
Additionally, fix \( g \in F[X] \), and denote
\( \bar{g} = g + (f) \in F[X]/(f) \), then there exists a unique polynomial
\( r \in F[X] \) with degree strictly less than \( n \) where
\( \bar{g} = \bar{r} \).

\proof
Since \( F[X] \) is a Euclidean domain with the norm being the degree of the
polynomial, there exist unique polynomials \( q, r \in F[X] \) where
\( g = qf + r \), and where \( r = 0 \) or \( \deg r < \deg q \).
This shows that \( g \in \bar{r} \), but this also implies that
\( r = g + (-q)f \), thus \( r \in \bar{g} \) and \( \bar{g} = \bar{r} \),
proving the existence of such an \( r \).
By the division algorithm \( r \) is unique, thus we are finished.
\done

\sectionskip

\textbf{2)}
Let \( F \) be a finite field of order \( q \) and fix \( f \in F[X] \) with
degree \( n \geq 1 \), then we have that
\( \vertb{F[X]/(f)} = q^n \).

\proof
We established that the elements in \( \vertb{F[X]/f(x)} \) are in bijection
with the polynomials in \( F[X] \) of degree strictly less than \( n \),
i.e. the polynomials \( g \in F[X] \) that take the form
\[
	g(X) = a_0 + a_1X + \cdots + a_{n - 1}X^{n - 1}
\]
for elements \( a_i \in F \).
Since we have \( n \) coefficients and \( q \) options for each coefficient,
it is clear that there are \( q^n \) polynomials of this form.
\done

\textbf{3)}
For a field \( F \) and polynomial \( f \in F[X] \), we have that
\( F[X]/(f) \) is a field if and only if \( f \) is irreducible.

\proof
First assume \( F[X]/(f) \) is a field.
This is equivalent to \( (f) \) being a maximal ideal in \( F[X] \), but
since \( F[X] \) is a principal ideal domain, we have that \( (f) \) is prime,
and in turn \( f \) is prime, and thus irreducible in \( F[X] \).

Conversely, assume \( f \) is irreducible.
Again, because \( F[X] \) is a P.I.D., \( f \) is prime, thus the ideal
\( (f) \) of \( F[X] \) is prime and thus maximal, showing that
\( F[X]/(f) \) is a field.
\done

\subsection*{9.3 Polynomial Rings that are Unique Factorization Domains}

\textbf{3)}
Fix a field \( F \) and denote \( R \) as the set of polynomials
\( f \in F[X] \) that take the form
\[
	f(X) = a_0 + a_2X^2 + \cdots + a_nX^n
\]
with elements \( a_i \in F \).
Then \( R \) is a ring, but not a unique factorization domain.

\proof
Consider the polynomials \( X^2 \) and \( X^3 \) in \( R \).
Since \( X \notin R \), and since 2 and 3 are prime, we have that
\( X^2 \) and \( X^3 \) are irreducible.
Additionally, \( X^2 \) and \( X^3 \) are clearly not associates.
We can see that
\[
	X^6 = X^2 \cdot X^2 \cdot X^2 = X^3 \cdot X^3,
\]
thus we have obtained two factorizations of \( X^6 \) with non-associate
irreducible factors, thus showing that \( R \) is not a U.F.D.
\done

\subsection*{9.4 Irreducibility Criteria}

\textbf{1)}
For \( p \in \F_2[X] \) defined as \( p(X) = X^2 + X + 1 \), we have
that
\[
	p(0) = 0^2 + 0 + 1 = 1^2 + 1 + 1 = 1,
\]
showing that \( p \) has no root in \( \F_2 \), and thus is irreducible.

If we define \( p \in \F_3[X] \) as \( p(X) = X^3 + X + 1 \), then
\[
	p(\bar{1}) = \bar{1}^3 + \bar{1} + \bar{1} = \bar{0},
\]
thus it has a root in \( \F_3 \) and is reducible.

\pagebreak

Next, define \( p \in \F_5[X] \) as \( p(X) = X^4 + 1 \), then if
\( p(X - 1) \) is irreducible, then so is \( p(X) \).
We evaluate \( p(X - 1) \) as
\[
	p(X - 1) = (X - 1)^4 + 1 = X^4 - 4X^3 + 6X^2 - 4X^2 + 2,
\]
and note that \( 2 \mid -4 \), \( 2 \mid 6 \), \( 2 \mid -4 \), and
\( 2 \mid 2 \), but \( 2^2 \mid 2 \), thus \( p \) is irreducible.

\sectionskip

\textbf{2)}
In \( \Z[X] \), we have that \( X^4 - 4X^3 + 6 \) is irreducible by
Eisenstein's criterion, since \( 2 \mid -4 \) and \( 2 \mid 6 \), but
\( 2^2 \nmid 6 \).
Similary for \( X^6 + 30X^5 -15X^3 + 6X - 120 \), we have that
\( 3 \mid 30 \), \( 3 \mid -15 \),\( 3 \mid 6 \), and \( 3 \mid 120 \), but
\( 3^3 \nmid 120 \), thus it is irreducible.
Finally, consider \( p(X) = X^4 + 4X^3 + 6X^2 + 2X + 1 \).
It is clear that if \( p(X - 1) \) is irreducible, then so is \( p(X) \).
Evaluating \( p(X - 1) \), we have that
\[
	p(X - 1) = x^4 - 8x^3 - 2x + 2.
\]
Since \( 2 \mid -8 \), \( 2 \mid -2 \), and \( 2 \mid 2 \), but
\( 2^2 \nmid 2 \), we have by E.C. that \( p \) is irreducible.

\sectionskip

\textbf{5)}
A monic polynomial in \( \F_2[X] \) has no root when the constant coefficient
is \( \bar{1} \) and the number of nonzero coefficients is odd.
Thus, the irreducible monic polynomials of \( \F_2[X] \) with degree at most
3 are given by \( X^3 + X + 1 \) and \( X^3 + X^2 + 1 \).

\sectionskip

\textbf{6)}
Define the polynomial \( p \in \F_3[X] \) as \( p(X) = X^2 + 1 \).
It is clear that \( p \) has no root in \( \F_3 \), and thus it is
irreducible.
Since \( \F_3[X] \) is a principal ideal domain, we have that
\( p \) is prime, thus the ideal \( (p) \) is prime and thus maximal.
This implies that \( \F_3[X]/(X^2 + 1) \) is a field.
Additionally, we established that this field has order
\( 3^2 = 9 \), and thus we have constructed a field of order 9.

\end{document}
