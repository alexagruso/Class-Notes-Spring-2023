\documentclass[12pt]{article}

\usepackage{amsfonts}
\usepackage{amsmath}
\usepackage{amssymb}
\usepackage{fancyhdr}
\usepackage[headheight=1in,margin=1.25in]{geometry}
\usepackage[colorlinks=true,linkcolor=blue]{hyperref}
\usepackage{mathtools}
\usepackage{setspace}
\usepackage{tikz}

\newcommand{\N}{\mathbb{N}}
\newcommand{\Z}{\mathbb{Z}}
\newcommand{\Q}{\mathbb{Q}}
\newcommand{\R}{\mathbb{R}}
\newcommand{\C}{\mathbb{C}}
\newcommand{\F}{\mathbb{F}}

\renewcommand{\t}[1]{\text{#1}}

\newcommand{\angleb}[1]{\left\langle#1\right\rangle}
\newcommand{\braceb}[1]{\left\{#1\right\}}
\newcommand{\bracketb}[1]{\left[#1\right]}
\newcommand{\parenb}[1]{\left(#1\right)}
\newcommand{\vertb}[1]{\left\vert#1\right\vert}
\newcommand{\ovl}[1]{\overline{#1}}

\newcommand{\normsub}{\trianglelefteq}
\newcommand{\normsup}{\trianglerighteq}

\newcommand{\aut}{\t{Aut}}
\newcommand{\inn}{\t{Inn}}

\newcommand{\proof}{\textit{Proof: }}
\newcommand{\done}{\ensuremath{\strut\hfill\blacksquare}}

\begin{document}

\pagestyle{fancy}
\fancyhead[L]{Modern Algebra}
\fancyhead[C]{Alex Agruso}
\fancyhead[R]{Homework 7}

\setlength{\parindent}{0in}
\setlength{\parskip}{0.25in}
\setstretch{1}

Let \( H \) be a subgroup of \( G \) and fix \( g \in G \), then
\( H \cong gHg^{-1} \).

\proof
Letting \( \phi_g \) denote conjugation by \( g \), we have for
\( h_1, h_2 \in H \) that
\[
	\phi(h_1h_2)
	= gh_1h_2g^{-1}
	= gh_1(gg^{-1})h_2g^{-1}
	= \phi(h_1)\phi(h_2),
\]
thus \( \phi_g \) is a homomorphism.
If \( x \in gHg^{-1} \), then \( x = ghg^{-1} \) for some \( h \in H \), and
thus is mapped to by \( h \), showing that \( \phi_g \) is surjective.
Finally, suppose \( \phi(h_1) = \phi(h_2) \), then
\[
	gh_1g^{-1}
	= gh_2g^{-1}
	\implies 
	(g^{-1}g)h_1(g^{-1}g)
	= (g^{-1}g)h_2(g^{-1}g)
	\implies
	h_1
	= h_2,
\]
thus showing that \( \phi_g \) is injective, which proves
\( H \cong gHg^{-1} \).
\done

\subsection*{4.4 Automorphisms}

\textbf{1)}
Given a group \( G \), let \( \sigma \in \aut(G) \) and denote \( \phi_g \) as
conjugation by \( g \in G \), then we have that
\( \sigma\phi_g\sigma^{-1} = \phi_{\sigma(g)} \).
As a corollary, \( \inn(G) \normsub \aut(G) \).

\proof
Fixing \( h \in G \), we see that
\[
	(\sigma\phi_g\sigma^{-1})(h)
	= \sigma(\phi_g(\sigma^{-1}(h)))
	= \sigma(g\sigma^{-1}(h)g^{-1}),
\]
and because \( \sigma \) is an automorphism,
\[
	\sigma(g\sigma^{-1}(h)g^{-1})
	= \sigma(g)\sigma(\sigma^{-1}(h))\sigma(g^{-1})
	= \sigma(g)h\sigma(g)^{-1}
	= \phi_{\sigma(g)}(h),
\]
thus \( \sigma\phi_g\sigma^{-1} = \phi_{\sigma(g)} \).
Since \( \phi_{\sigma(g)} \in \inn(G) \), we have that
\( \sigma\phi_g\sigma^{-1} \in \inn(G) \) for every \( \sigma \in \aut(G) \),
thus \( \inn(G) \normsub \aut(G) \).
\done

\textbf{2)}
If \( G \) is an abelian group of order \( pq \) where \( p \) and \( q \) are
distinct primes, then \( G \) is cyclic.

\proof
Since \( p \) and \( q \) are distinct, we can assume \( p < q \) without loss
of generality.
By Cauchy's theorem, there exist elements \( x, y \in G \) with orders \( p \)
and \( q \) respectively.
Clearly \( \vertb{xy} \ne 1 \), else the order of \( x \) or \( y \) would be
1.
We can also see that \( \vertb{xy} \ne p \), else we would have that
\[
	e = (xy)^p = x^py^p = y^p,
\]
which is impossible since \( \vertb{y} = q > p \).
Finally, \( \vertb{xy} \ne q \), since if we did we would have
\[
	e = (xy)^q = x^qy^q = x^q,
\]
and because \( p \nmid q \), we have that \( q = pn + r \) for some
\( n, r \in \Z \) and \( 0 < r < p \), thus
\[
	e = x^q = x^{pn + r} = (x^p)^nx^r = x^r,
\]
which violates \( \vertb{x} = p > r \).
Thus, our only choice for the order of \( xy \) is \( pq \), which shows that
\( G \) is cyclic.
\done

\textbf{15)}
In \( (\Z/5\Z)^\times \), we have
\( \angleb{\bar2} = \braceb{\bar2, \bar4, \bar3, \bar1} = (\Z/5\Z)^\times \),
thus \( \Z/5\Z \) is cyclic.

In \( (\Z/9\Z)^\times \), we have
\(
	\angleb{\bar2}
	= \braceb{\bar2, \bar4, \bar8, \bar7, \bar5, \bar1}
	= (\Z/9\Z)^\times
\),
thus \( \Z/9\Z \) is cyclic.

\subsection*{4.5 Sylow's Theorem}

\textbf{3)}
If \( G \) is a group and \( p \) a prime that divides the order of \( G \),
then there exists an element \( x \in G \) of order \( p \).

\proof
Fix a group \( G \) with order \( p^\alpha m \), where \( p \) is prime,
\( \alpha \geq 1 \), and \( p \nmid m \).
By Sylow's theorem, there exists a subgroup \( H \) of \( G \) with order
\( p^\alpha \).
Let \( x \in H \) where \( x \ne e \).
Since \( \vertb{x} \) divides \( p^\alpha \), we have that
\( \vertb{x} = p^n \) for some \( 1 \leq n \leq \alpha \).
Consider the element \( y = x^{p^{n - 1}} \).
Because \( p^{n - 1} < p^n \), we have that \( y \ne e \).
We can see that
\[
	y^p
	= \parenb{x^{p^{n - 1}}}^p
	= x^{p^{n - 1}p}
	= x^{p^n}
	= e,
\]
thus \( \vertb{y} \leq p \).
Since no integer \( 2 \leq k < p \) divides \( p^\alpha \), we must have that
\( \vertb{y} = p \), thus proving Cauchy's theorem.
\done

\textbf{7)}
The Sylow 2-subgroups of \( S_4 \) are given by
\[
	H_1
	= \braceb{
		e, (1\ 2), (3\ 4), (1\ 2)(3\ 4), (1\ 3)(2\ 4), (1\ 4)(2\ 3),
		(1\ 3\ 2\ 4), (1\ 4\ 2\ 3)
	}
\]
\[
	H_2
	= \braceb{
		e, (1\ 3), (2\ 4), (1\ 3)(2\ 4), (1\ 4)(2\ 3), (1\ 2)(3\ 4),
		(1\ 4\ 3\ 2), (1\ 2\ 3\ 4)
	}
\]
\[
	H_3
	= \braceb{
		e, (1\ 4), (2\ 3), (1\ 4)(2\ 3), (1\ 2)(3\ 4), (1\ 3)(2\ 4),
		(1\ 2\ 4\ 3), (1\ 3\ 4\ 2)
	}
\]
We have that \( H_1 = (2\ 3)H_2(2\ 3)^{-1} = (2\ 4)H_3(2\ 4)^{-1} \),
\( H_2 = (2\ 3)H_1(2\ 3)^{-1} = (3\ 4)H_3(3\ 4)^{-1} \), and
\( H_3 = (2\ 4)H_1(2\ 4)^{-1} = (3\ 4)H_2(3\ 4)^{-1} \).

\textbf{13)}
If \( G \) is a group with order 56, then there exists a normal Sylow
\( p \)-subgroup where \( p \) is a prime divisor of \( \vertb{G} \).

\proof
The prime factorization of 56 is \( 2^3 \cdot 7 \).
We know that \( n_7 \equiv 1 \pmod{7} \) and that \( n_7 \mid 2^3 \), thus the
possible options for \( n_7 \) are 1 and 8.
If \( n_7 = 8 \), then there are 8 Sylow 7-subgroups, each having 6 elements
of order 7 that are unique to that group, plus the identity element.
This means that there can be only one Sylow 2-subgroup, since such a group
would have 7 non-identity elements whose orders are powers of two.
Since \( 8 \cdot 6 + 7 + 1 = 56 \), we have accounted for all elements in
\( G \) and thus know that \( n_2 = 1 \), proving that \( G \) has a normal
Sylow 2-subgroup.
Otherwise, \( n_7 = 1 \), and \( G \) has a normal Sylow 7-subgroup, completing
the proof.
\done

\textbf{23)}
No group \( G \) of order 462 is simple.

\proof
The prime factorization of 462 is \( 2 \cdot 3 \cdot 7 \cdot 11 \).
We have that \( n_{11} \equiv 1 \pmod{11} \) and that
\( n_{11} \mid 42 \), thus the possible values for
\( n_{11} \) are \( 1, 12, 23, 34, 45, \dots \), but the only one of these
that divides 42 is 1, showing that \( G \) has a normal Sylow 11-subgroup, and
thus is not simple.
\done

\textbf{26)}
If a group \( G \) of order 105 has a normal Sylow 3-subgroup, then it is
abelian.

\proof
-

\end{document}
