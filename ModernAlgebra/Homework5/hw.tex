\documentclass[12pt]{article}

\usepackage{amsfonts}
\usepackage{amsmath}
\usepackage{amssymb}
\usepackage{fancyhdr}
\usepackage[headheight=1in,margin=1in]{geometry}
\usepackage[colorlinks=true,linkcolor=blue]{hyperref}
\usepackage{mathtools}
\usepackage{tikz}

\newcommand{\N}{\mathbb{N}}
\newcommand{\Z}{\mathbb{Z}}
\newcommand{\Q}{\mathbb{Q}}
\newcommand{\R}{\mathbb{R}}
\newcommand{\C}{\mathbb{C}}
\newcommand{\F}{\mathbb{F}}

\renewcommand{\t}[1]{\text{#1}}

\newcommand{\angleb}[1]{\left\langle#1\right\rangle}
\newcommand{\braceb}[1]{\left\{#1\right\}}
\newcommand{\bracketb}[1]{\left[#1\right]}
\newcommand{\parenb}[1]{\left(#1\right)}
\newcommand{\vertb}[1]{\left\vert#1\right\vert}
\newcommand{\ovl}[1]{\overline{#1}}

\newcommand{\normsub}{\trianglelefteq}
\newcommand{\normsup}{\trianglerighteq}

\newcommand{\done}{\ensuremath{\strut\hfill\blacksquare}}

\begin{document}

\pagestyle{fancy}
\fancyhead[L]{Modern Algebra}
\fancyhead[C]{Alex Agruso}
\fancyhead[R]{Homework 5}

\setlength{\parindent}{0in}
\setlength{\parskip}{0.09in}

\subsection*{3.3 The Isomorphism Theorems}

\textbf{1)}
Fix any \( c \in \F_q \) where \( c \ne 0 \), then we have for all
\( M \in \t{SL}_n(\F_q) \) that \( cM \in \t{GL}_n(\F_q) \) and
\( c^{-1}cM \in \t{SL}_n(\F_q) \), which shows a bijection between the elements
in \( \t{SL}_n(\F_q) \) and \( c\cdot\t{SL}_n(\F_q) \).
We also have that if \( c_1,c_2 \in \F_q \) are distinct, then
\( \det(c_1M) \ne \det(c_2M) \), thus since \( \F_q \) has \( q - 1 \) non-zero
choices for \( c \), we have that
\( \vertb{\t{GL}_n(\F_q) / \t{SL}_n(\F_q)} = q - 1 \).
\done

\textbf{3)}
Since \( H \normsub G \), we have that \( N_G(H) = G \), thus any subgroup
\( K \) of \( G \) is a subgroup of \( N_G(H) \), thus by the second
isomorphism theorem, we have that \( KH \leq G \) and \( H \normsub KH \), and
by Lagrange's theorem we have
\( \vertb{G : H} = \vertb{G : KH}\vertb{KH : H} \).
Since \( \vertb{G : H} = p \) is prime, we must have that \( \vertb{G : KH} \)
is \( p \) or 1.
If it is \( p \), then \( \vertb{KH : H} = 1 \) which shows that \( KH = H \),
thus \( K \leq H \).
Otherwise, if \( \vertb{G : KH} = 1 \), then \( \vertb{KH : H} = p \), and
again by the second isomorphism theorem we have \( K/K \cap H \cong KH/H \)
which implies \( \vertb{K : K \cap H} = \vertb{KH : H} = p \), and we are
finished.
\done

\textbf{4)}
Given groups \( A \) and \( B \), fix normal subgroups
\( C \normsub A \) and \( D \normsub B \).
Clearly \( C \times D \leq A \times B \).
Let \( (c,d) \in C \times D \) and \( (a,b) \in A \times B \), then we have
that \( aca^{-1} \in C \) and \( bdb^{-1} \in D \).
Since \( (a,b)^{-1} = (a^{-1},b^{-1}) \), we have that
\(
	(a,b) \circ (c,d) \circ (a^{-1},b^{-1})
	= (aca^{-1},bdb^{-1}) \in C \times D
\),
thus showing that \( C \times D \normsub A \times B \).

Since \( C \times D \) is normal, it is the kernel of some homomorphism
\( \phi : A \times B \to (A \times B)/(C \times D) \).
By the first isomorphism theorem, we have that
\( (A \times B)/\ker\phi \cong \phi(A \times B) \).
For each \( (aC,bD) \in (A/C) \times (B/D) \), we have that
\( \phi((a,b)) = (a,b)(C \times D) = (aC,bD) \), thus \( \phi \) is surjective
and \( \phi(A \times B) = (A/C) \times (B/D) \), showing that
\( (A \times B)/(C \times D) \cong (A/C) \times (B/D) \).
\done

\subsection*{3.4 Composition Series and the H\"older Program}

\textbf{1)}
Let \( G \) be an abelian simple group and \( x \in G \) be a non-identity
element.
We must have that \( \angleb{x} = G \), otherwise it would be a normal subgroup
that is both proper and non-trivial, and thus \( G \) would not be simple.
Now, suppose \( \vertb{G} \) is infinite, then \( \angleb{x^n} \) is a proper
normal subgroup for any \( n \in \Z \), thus \( \vertb{G} \) must be finite.
Finally, if \( n > 1 \) is a proper divisor of \( \vertb{G} \), then
\( \angleb{x^n} \) is a proper normal subgroup, thus \( \vertb{G} \) must
have no proper divisors \( > 1 \), thus \( \vertb{G} \) is prime.
Since \( G \) is also cyclic, we have that \( G \cong \Z/p\Z \).
\done

\textbf{2)}
By the subgroup lattice for \( D_8 \) provided in the text, we have that the
decomposition series for \( D_8 \) are as follows:
\[
	1 \normsub \angleb{s} \normsub \angleb{s,r^2} \normsub D_8 \\
\]
\[
	1 \normsub \angleb{sr^2} \normsub \angleb{s,r^2} \normsub D_8 \\
\]
\[
	1 \normsub \angleb{r^2} \normsub \angleb{s,r^2} \normsub D_8 \\
\]
\[
	1 \normsub \angleb{r^2} \normsub \angleb{r} \normsub D_8 \\
\]
\[
	1 \normsub \angleb{r^2} \normsub \angleb{rs,r^2} \normsub D_8 \\
\]
\[
	1 \normsub \angleb{rs} \normsub \angleb{rs,r^2} \normsub D_8 \\
\]
\[
	1 \normsub \angleb{sr} \normsub \angleb{rs,r^2} \normsub D_8
\]

\textbf{4)}
Fix a finite abelian group \( G \).
If \( \vertb{G} = 1 \), we have that \( \braceb{e} \leq \braceb{e} \), thus the
base case holds.
Now fix a positive integer \( n \) and suppose the proposition holds for all
groups where \( \vertb{G} < n \), and let \( k \) be a divisor of \( n \).
If \( k \) is prime, then by Cauchy's theorem there exists \( x \in G \) with
\( \vertb{x} = k \), and thus \( \vertb{\angleb{x}} = k \).
Otherwise, some prime \( p < k \) divides \( k \), and again we can choose
\( x \in G \) with \( \vertb{\angleb{x}} = p \).
Since \( k/p \) divides \( \vertb{G/\angleb{x}} \), we can use the induction
hypothesis and the fourth isomorphism theorem to obtain a subgroup
\( H/\angleb{x} \leq G/\angleb{x} \) with order \( k/p \) and \( H \leq G \).
By Lagrange's theorem we find that
\( k/p = \vertb{H/\angleb{x}} = \vertb{H}/p \), thus \( \vertb{H} = k \) and we
are finished.
\done

\subsection*{3.5 Transpositions and the Alternating Group}

\textbf{1)} \( \sigma \) is even and \( \tau \) is odd, thus \( \sigma\tau \)
is odd.

\textbf{3)} For any \( \sigma \in S_n \), we can take its decomposition into
transpositions, thus without loss of generality we can take \( \sigma \) to
be a transposition in \( S_n \), say \( (i\ j) \) for integers
\( 1 \leq i < j \leq n \).
Let \( \sigma' \in S_n \) be defined as
\( (i\ i+1)(i+1\ i+2)\cdots(j-1\ j)(j-2\ j-1)\cdots(i\ i+1) \).
We have that \( \sigma'(i) \) brings \( i \) to \( i + 1 \), then \( i + 2 \),
until we get to \( j \), then doesn't touch \( j \) for the rest of the
transpositions, thus \( \sigma'(i) = j \).
Likewise, \( j \) isn't touched until \( (j-1\ j) \), and then cascades through
the rest of the transpositions until it reaches \( i \), thus
\( \sigma'(j) = i \) and \( \sigma = \sigma' \).
We can repeat this process for each transposition in a general element of
\( S_n \), thus we are finished.
\done

\textbf{9)}
We have that \( \braceb{e,(1\ 2)(3\ 4),(1\ 3)(2\ 4), (1\ 4)(2\ 3)} \) is the
subgroup of \( A_4 \) with order 4.
The mapping \( e \mapsto (0,0) \), \( (1\ 2)(3\ 4) \mapsto (0,1) \),
\( (1\ 3)(2\ 4) \mapsto (1,0) \), \( (1\ 4)(2\ 3) \mapsto (1,1) \) is an
isomorphism, so the subgroup is isomorphic to \( V_4 \).

\end{document}
