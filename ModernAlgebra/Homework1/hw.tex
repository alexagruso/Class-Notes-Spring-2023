\documentclass[12pt]{article}

\usepackage{amsfonts}
\usepackage{amsmath}
\usepackage{amssymb}
\usepackage{fancyhdr}
\usepackage[headheight=1in,margin=1in]{geometry}
\usepackage[colorlinks=true,linkcolor=blue]{hyperref}
\usepackage{mathtools}

\newcommand{\N}{\mathbb{N}}
\newcommand{\Z}{\mathbb{Z}}
\newcommand{\Q}{\mathbb{Q}}
\newcommand{\R}{\mathbb{R}}
\newcommand{\C}{\mathbb{C}}

\newcommand{\angleb}[1]{\left\langle#1\right\rangle}
\newcommand{\braceb}[1]{\left\{#1\right\}}
\newcommand{\bracketb}[1]{\left[#1\right]}
\newcommand{\parenb}[1]{\left(#1\right)}
\newcommand{\vertb}[1]{\left\vert#1\right\vert}

\newcommand{\done}{\ensuremath{\strut\hfill\blacksquare}}

\begin{document}

\pagestyle{fancy}
\fancyhead[L]{Modern Algebra}
\fancyhead[C]{Alex Agruso}
\fancyhead[R]{Homework 1}

\setlength{\parindent}{0in}
\setlength{\parskip}{0.1in}

\section*{0.3 \( \Z / n\Z \): The Integers Modulo \( n \)}

\textbf{Proposition 4:}
\(
	(\Z / n\Z)^\times
	= \braceb{\overline{a} \in \Z / n\Z : (a, n) = 1}
\)

\textit{Proof:}
Let \( \overline{a} \in (\Z / n\Z)^\times \), then there exists
\( \overline{b} \in (\Z / n\Z)^\times \) with
\( \overline{a}\overline{b} = \overline{ab} = \overline{1} \).
Consequently, there exists \( k \in \Z \) where \( ab + kn = 1\), and thus by
Bezout's lemma, we have that \( (a, n) = 1 \).

Now, fix \( a \in \Z \) with \( (a, n) = 1 \), then there exist integers
\( b \) and \( k \) where \( ab + nk = 1 \).
Consequently, \( \overline{ab} = \overline{a}\overline{b} = \overline{1} \),
and thus \( \overline{a} \in (\Z / n\Z)^\times \).
\done

\textbf{11)}
Let \( \overline{a}, \overline{b} \in (\Z / n\Z)^\times \).
Since \( a \) and \( b \) are units mod \( n \), we have that
\( (a, n) = (b, n) = 1 \), thus by Bezout's lemma, there exist integers
\( x_1, x_2, y_1 \), and \( y_2 \) where
\[
	ax_1 + ny_1 = bx_2 + ny_2 = 1,
\]
and thus
\[
	(ax_1 + ny_1)(bx_2 + ny_2)
	= ab(x_1x_2) + n(ax_1y_2 + bx_2y_1 + ny_1y_2)
	= (ab)x + (n)y
	= 1
\]
for integers \( x \) and \( y \), thus \( ab \) is a unit mod \( n \) and
\( \overline{ab} \in (\Z / n\Z)^\times \).
\done

\section*{1.1 Basic Axioms and Examples}

\textbf{1a)}
\( a * b = a - b \) is not associative, as
\[
	1 - (2 - 3)
	= 1 - (-1)
	= 2
	\ne (1 - 2) - 3
	= -1 - 3
	= -4
\]
\done

\textbf{1b)}
Let \( a, b, c \in \Z \), then
\begin{align*}
	a * (b * c)
	& = a * (b + c + bc) \\
	& = a + (b + c + bc) + a(b + c + bc) \\
	& = (a + b + ab) + c + (a + b + ab)c \\
	& = (a + b + ab) * c \\
	& = (a * b) * c,
\end{align*}
and thus \( a * b = a + b + ab \) is associative.
\done

\textbf{5)}
Fix \( n > 1 \). For all \( \overline{x} \in \Z / n\Z \), we have that
\( \overline{0}\overline{x} = \overline{0x} = \overline{0} \), and thus
\( \overline{0} \) has no multiplicative inverse in \( \Z / n\Z \).
\done

\textbf{6a)}
Is a group. \textit{\hyperref[proof1]{Proof of Closure}}

\textbf{6b)}
Not a group, isn't closed
\( \parenb{\frac{1}{2} + \frac{1}{6} = \frac{2}{3}} \).

\textbf{6c)}
Not a group, lacks inverses and an identity.

\textbf{6d)}
Not a group, lacks inverses.

\textbf{9)}
Let \( G = \braceb{a + b\sqrt{2} : a, b \in \Q} \), and
\( x_1, x_2, x_3 \in G \) where \( x_1 = a_1 + b_1\sqrt{2} \),
\( x_2 = a_2 + b_2\sqrt{2} \), and \( x_3 = a_3 + b_3\sqrt{2} \)
for \( a_i, b_i \in \Q \).

\textbf{a)}
We have that
\(
	x_1 + x_2
	= a_1 + b_1\sqrt{2} + a_2 + b_2\sqrt{2}
	= a_1 + a_2 + (b_1 + b_2)\sqrt{2}
	= a + b\sqrt{2}
\)
where \( a, b \in \Q \), thus \( x_1 + x_2 \in G \) and closure is satisfied.

We have that \( x_1 + 0 = x_1 \), thus an identity element exists.

Fix \( x_1 \) and let \( y = -a_1 - b_1\sqrt{2} \), then we have that
\(
	x_1 + y
	= a_1 + b_1\sqrt{2} - a_1 - b_1\sqrt{2}
	= a_1 - a_1 + (b_1 - b_1)\sqrt{2}
	= 0,
\)
thus \( y = (x_1)^{-1} = -x_1 \), and inverses are satisfied.

Finally, we can see that
\begin{align*}
	x_1 + (x_2 + x_3)
	& = a_1 + b_1\sqrt{2}
	+ \parenb{a_2 + b_2\sqrt{2} + a_3 + b_3\sqrt{2}} \\
	& = \parenb{a_1 + b_1\sqrt{2} + a_2 + b_2\sqrt{2}}
	+ a_3 + b_3\sqrt{2} \\
	& = (x_1 + x_2) + x_3,
\end{align*}
and thus associativity is satisfied, which proves that \( G \) is a group under
addition.
\done

\textbf{b)}
Let \( G^* \) = \( G \setminus \braceb{0} \), and suppose
\( x_1, x_2, x_3 \in G^* \).

We have that
\begin{align*}
	x_1x_2
	& = (a_1 + b_1\sqrt{2})(a_2 + b_2\sqrt{2}) \\
	& = a_1a_2 + 2b_1b_2 + (a_1b_2 + a_2b_1)\sqrt{2} \\
	& = a + b\sqrt{2},
\end{align*}
thus \( x_1x_2 \in G^* \) and closure is satisfied.

We have that \( 1 \times x_1 = x_1 \), thus an identity exists.

Fix \( x_1 \) and let \( y = \frac{1}{a_1^2 - 2b_1^2}(a_1 - b_1\sqrt{2}) \).
We know that \( a_1^2 - 2b_1^2 \ne 0 \) since if it did we would have that
\( \sqrt{2} \in \Q \).
Consequently,
\begin{align*}
	x_1y
	& = \frac{(a_1 + b_1\sqrt{2})(a_1 - b_1\sqrt{2})}{a_1^2 - 2b_1^2} \\
	& = \frac{a_1^2 - 2b_1^2}{a_1^2 - 2b_1^2} = 1,
\end{align*}
thus \( y = (x_1)^{-1} = 1/x_1 \), and thus inverses are satisfied.

Finally, we have that
\begin{align*}
	x_1(x_2x_3)
	& = (a_1 + b_1\sqrt{2})
	\parenb{(a_2 + b_2\sqrt{2})(a_3 + b_3\sqrt{2})} \\
	& = \parenb{(a_1 + b_1\sqrt{2})(a_2 + b_2\sqrt{2})}
	(a_3 + b_3\sqrt{2}) = (x_1x_2)x_3,
\end{align*}
thus associativity is satisfied, and thus \( G^* \) is a group under
multiplication.
\done

\textbf{11)}
In \( \parenb{\Z / 12\Z, +} \), we have

\(
	\vertb{\overline{1}}
	= \vertb{\overline{5}}
	= \vertb{\overline{7}}
	= \vertb{\overline{11}}
	= 12
\)

\(
	\vertb{\overline{2}}
	= \vertb{\overline{10}}
	= 6
\)

\(
	\vertb{\overline{3}}
	= \vertb{\overline{9}}
	= 4
\)

\(
	\vertb{\overline{4}}
	= \vertb{\overline{8}}
	= 3
\)

\(
	\vertb{\overline{6}}
	= 2
\)

\(
	\vertb{\overline{0}}
	= 1
\)

\textbf{12)}
In \( \parenb{(\Z / 12\Z)^\times, \times} \), we have

\(
	\vertb{\overline{1}} = 1
\)

\(
	\vertb{\overline{-1}}
	= \vertb{\overline{5}}
	= \vertb{\overline{7}}
	= \vertb{\overline{-7}}
	= \vertb{\overline{13}}
	= 2
\)

\textbf{17)}
Fix \( x \in G \) with \( \vertb{x} = n \), then we have that
\( x(x^{n - 1}) = x^{n - 1 + 1} = x^n = e \), and thus
\( x^{-1} = x^{n - 1} \).
\done

\textbf{20)}
Let \( x \in G \) with \( \vertb{x} = n \), then \( x^n = e \), and so
\(
	\parenb{x^n}^{-1}
	= x^{-n}
	= \parenb{x^{-1}}^n
	= e^{-1}
	= e.
\)
Let \( \vertb{x^{-1}} = k \) and, to establish a contradiction, suppose
\( k < n \).
Then we would have that \( \parenb{x^{-1}}^k = e \), and thus
\(
	\big(\parenb{x^{-1}}^k\big)^{-1}
	= \parenb{x^{-k}}^{-1}
	= x^k
	= e^{-1}
	= e,
\)
but this implies \( \vertb{x} \leq k < n \), which is a contradiction, thus we
must have that \( \vertb{x^{-1}} = n \).
\done

\textbf{23)}
Fix \( x \in G \) with finite order \( \vertb{x} = n \) and suppose that
\( n = st \) for positive integers \( s \) and \( t \), then we have that
\( x^n = x^{st} = \parenb{x^s}^t = e \).
To establish a contradiction, suppose that \( \vertb{x^s} = k < t \), then we
have that \( \parenb{x^s}^k = x^{sk} = e \) with \( sk < st = n \), which is a
contradiction.
Thus, \( \vertb{x^s} = t \).
\done

\textbf{25)}
Let \( a, b \in G \).
We have that \( (aa)(bb) = e \), as well as \( (ab)(ab) = e \), and thus
\( aabb = abab \).
Multiplying by \( a \) on the left and \( b \) on the right, we maintain
equality and obtain \( ab = ba \).
\done

\textbf{29)}
Let \( \parenb{A, *_A} \) and \( \parenb{B, *_B} \) be groups.

First, assume that \( A \) and \( B \) are abelian.
Let \( (a_1, b_1), (a_2, b_2) \in A \times B \), then we have that
\[
	(a_1, b_1) * (a_2, b_2)
	= (a_1 *_A a_2, b_1 *_B b_2)
	= (a_2 *_A a_1, b_2 *_B b_1)
	= (a_2, b_2) * (a_1, b_1),
\]
and thus \( A \times B \) is abelian.

Now, assume that \( A \times B \) is abelian and let
\( (a_1, b_1), (a_2, b_2) \in A \times B \), then we have that
\[
	(a_1, b_1) * (a_2, b_2)
	= (a_2, b_2) * (a_1, b_1),
\]
and thus
\[
	(a_1 *_A a_2, b_1 *_B b_2)
	= (a_2 *_A a_1, b_2 *_B b_1),
\]
which implies that \( a_1 *_A a_2 = a_2 *_A a_1 \) and
\( b_1 *_B b_2 = b_2 *_B b_1 \), thus \( A \) and \( B \) are abelian.
\done

\pagebreak
\textbf{31)}
For a group \( G \) with even order, fix \( x \in G \) such that
\( \vertb{x} = n \) is maximal.
Such an \( x \) exists because \( \vertb{x} \leq \vertb{G} \) and
\( \vertb{G} \) is finite.
If \( n \) is odd, then \( G \) has an odd number of distinct elements, and
thus cannot have even order, thus \( n \) must be even.
From this, we have that \( x^n = (x^{n / 2})^2 = e \).
Clearly \( x^{n / 2} \ne e \), and thus \( \vertb{x^{n / 2}} = 2 \), thus an
element of order 2 exists in \( G \).
\done

\textbf{32)}
Let \( x \in G \) with \( \vertb{x} = n \).
To establish a contradiction, suppose there exist integers \( a \) and \( b \)
such that \( 0 \leq a < b < n \) and \( x^a = x^b \).
Then we have that \( x^ax^{-a} = e = x^bx^{-a} = x^{b - a} \), which implies
that \( \vertb{x} \leq b - a < n \), which is a contradiction.
Thus we must have that \( x^a \ne x^b \) when \( 0 \leq a < b < n \), thus the
elements \( e, x, x^2, \dots, x^{n - 1} \) are distinct.
\done

Let \( G \) be a group with \( \vertb{G} = n \).
To establish a contradiction, suppose there exists an element \( x \in G \)
where \( \vertb{x} = k > n \).
This implies that the elements \( e, x, x^2, \dots, x^{k - 1} \) are distinct,
but then \( G \) would have \( k \) distinct elements, and thus
\( \vertb{G} = k > n \), which is a contradiction.
Thus, for all \( x \in G \), we must have that \( \vertb{x} \leq \vertb{G} \).
\done

\section*{1.2 Dihedral Groups}

\textbf{1b)}
In \( D_8 \) we have that

\(
	\vertb{e} = 1
\)

\(
	\vertb{r^2}
	= \vertb{s}
	= \vertb{sr}
	= \vertb{sr^2}
	= \vertb{sr^3}
	= 2
\)

\(
	\vertb{r}
	= \vertb{r^3}
	= 4
\)

\textbf{2)}
Let \( x \in D_{2n} \) and suppose \( x \) is not a power of \( r \), then
\( x = sr^{i} \) for some integer \( i \).
Since \( rs = sr^{-1} \) by the given presentation of \( D_{2n} \), we have
that \( rx = rsr^i = sr^{-1}r^i = sr^{i - 1} = sr^ir^{-1} = xr^{-1} \).
\done

\textbf{3)}
Let \( x \in D_{2n} \) where \( x = sr^{i} \) for some integer \( i \).
By induction we will show that \( x^2 = e \) for all \( i \).
For the base case \( i = 0 \), we have that \( x^2 = (sr^0)(sr^0) = s^2 = e \).
Now, assume the induction hypothesis holds for \( n = i - 1 \).
Since
\(
	x^2
	= (sr^i)(sr^i)
	= sr^{i - 1}(rr^{-1})sr^{i - 1}
	= (sr^{i - 1})(sr^{i - 1})
	= e,
\)
the hypothesis holds for \( n = i \).
Since \( x \ne e \), we have that \( \vertb{x} = 2 \).
\done

\textbf{11)} We can define an equivalence relation on the rigid rotational
symmetries \( G \) of an octahedron where two rotations are equivalent if their
uppermost vertices are the same.
Since an octahedron has six vertices, we can see that there are six equivalence
classes.
Now, fix the uppermost vertex, and consider the rotations in the corresponding
equivalence class.
The only rotations that can be in this equivalence class are rotations about
the vertical axis and whose angle is a multiple of \( 90^\circ \), which
restricts us to four rotations.
Thus, we have six equivalence classes with each class having four elements.
Since each possible rotation of an octahedron is contained in one of the
equivalence classes, we have that \( \vertb{G} = 6 \times 4 = 24 \).
\done

\textbf{15)}
\(
	\Z / n\Z
	= \langle \overline{1} : \underbrace{
	\overline{1} + \overline{1} + \cdots + \overline{1}
	}_{n \ \text{times}} = \overline{0} \rangle.
\)

\section*{Extra Proofs}

\label{proof1}
The set
\(
	G = \braceb{
		\frac{a}{m} \in \Q : a, m \in \Z, (a, m) = 1,
		\ \text{and} \
		m
		\ \text{is odd}
	}
\)
is closed under addition.

\textit{Proof:}
Let \( x, y \in G \) with \( x = \frac{a_2}{m_2} \) and
\( y = \frac{a_2}{m_2} \), then we have
\[
	x + y
	= \frac{a_1}{m_1} + \frac{a_2}{m_2} = \frac{a_1m_2 + a_2m_1}{m_1m_2}.
\]
We can see that \( m_1m_2 \) is odd.
Now let \( d = (a_1m_2 + a_2m_1, m_1m_2) \).
If \( d = 1 \) we are finished, else \( d \) must be odd since
\( 2 \nmid m_1m_2 \).
Consequently, \( m_1m_2 / d \) is odd.
In addition, we have that \( \parenb{(a_1m_2 + a_2m_1) / d, m_1m_2 / d} = 1 \),
thus
\[
	x + y
	= \frac{a_1m_2 + a_2m_1}{m_1m_2}
	= \frac{\frac{a_1m_2 + a_2m_1}{d}}{\frac{m_1m_2}{d}}
	= \frac{a}{m}
\]
for integers \( a \) and \( m \) where \( (a, m) = 1 \) and \( m \) is odd,
thus \( x + y \in G \).
\done

\end{document}
