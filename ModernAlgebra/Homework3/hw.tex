\documentclass[12pt]{article}

\usepackage{amsfonts}
\usepackage{amsmath}
\usepackage{amssymb}
\usepackage{fancyhdr}
\usepackage[headheight=1in,margin=1in]{geometry}
\usepackage[colorlinks=true,linkcolor=blue]{hyperref}
\usepackage{mathtools}

\newcommand{\N}{\mathbb{N}}
\newcommand{\Z}{\mathbb{Z}}
\newcommand{\Q}{\mathbb{Q}}
\newcommand{\R}{\mathbb{R}}
\newcommand{\C}{\mathbb{C}}
\newcommand{\F}{\mathbb{F}}

\newcommand{\angleb}[1]{\left\langle#1\right\rangle}
\newcommand{\braceb}[1]{\left\{#1\right\}}
\newcommand{\bracketb}[1]{\left[#1\right]}
\newcommand{\parenb}[1]{\left(#1\right)}
\newcommand{\vertb}[1]{\left\vert#1\right\vert}

\newcommand{\done}{\ensuremath{\strut\hfill\blacksquare}}

\begin{document}

\pagestyle{fancy}
\fancyhead[L]{Modern Algebra}
\fancyhead[C]{Alex Agruso}
\fancyhead[R]{Homework 3}

\setlength{\parindent}{0in}
\setlength{\parskip}{0.1in}

\subsection*{2.1 Definitions and Examples}

\textbf{2a)}
We have that \( (1\ 2)(1\ 3) = (1\ 3\ 2) \), thus closure isn't satisfied.
We also have no identity element.

\textbf{2b)}
We have that \( (sr^{2})(sr^{2}) = (sr^{2})(r^{-2}s) = s^2 = e \), thus closure
isn't satisfied.
We also have no identity element.

\textbf{6)}
Let \( G \) be an abelian group.
Since \( \vertb{e} = 1 \), we have that the set of torsion elements in \( G \)
is non-empty.
Now, suppose \( x, y \in G \) have finite orders \( m \) and \( n \)
respectively, then \( mn \) is also finite and
\( (xy)^{mn} = x^{mn}y^{mn} = (x^m)^n(y^n)^m = e \), thus
\( \vertb{xy} \leq mn \) and \( xy \) is a torsion element and closure is
satisfied.
Additionally, \( \vertb{x^{-1}} = \vertb{x} = n \), thus \( x^{-1} \) is a
torsion element and inverses exist.
Thus, the torsion elements of \( G \) satisfy the subgroup criterion.
\done

\textbf{9)} Since all matrices in \( \text{SL}_n(F) \) have non-zero
determinant, we have that \( \text{SL}_n(F) \subset \text{GL}_n(F) \).
Let \( A, B \in \text{SL}_n(F) \), thus
\( \det(A) = \det(B) = 1 \).
Since \( \det(AB) = \det(A)\det(B) = 1 \), we have closure.
We also have that \( A \) is invertible, thus there exists \( A^{-1} \) where
\( AA^{-1} = E \), but \( \det(AA^{-1}) = \det(A)\det(A^{-1}) = \det(E) = 1 \),
thus \( \det(A^{-1}) = \det(E) / \det(A) = 1 \), thus
\( A^{-1} \in \text{SL}_n(F) \) and inverses are satisfied.
Consequently, \( \text{SL}_n(F) \leq \text{GL}_n(F) \).
\done

\textbf{10a)} Fix a group \( G \) and subgroups \( H, K \leq G \).
Clearly \( H \cap K \) is non-empty since they both contain \( e \).
Let \( x, y \in H \cap K \), thus \( x, y \in H \) and \( x, y \in K \).
By closure, we have that \( xy \in H \) and \( xy \in K \), thus
\( xy \in H \cap K \).
We also have that \( x^{-1} \in H \) and \( x^{-1} \in K \), thus
\( x^{-1} \in H \cap K \), thus \( H \cap K \) satisfies the subgroup
criterion.
\done

\subsection*{2.2 Centralizers and Normalizers, Stabilizers and Kernels}

\textbf{2)}
Given a group \( G \), we have
\(
	C_G(Z(G))
	= \braceb{
		g \in G : gag^{-1} = a \text{ for all } a \in Z(G)
	}
\),
but since every element in \( G \) commutes with every element in \( Z(G) \),
we have that \( C_G(Z(G)) = G \).
Since \( C_G(A) \subseteq N_G(A) \) for all subsets \( A \subseteq G \), we
have \( G \subseteq N(Z(G)) \), thus \( G = N_G(Z(G)) \).
\done

\textbf{5a)} We have \( (1\ 2\ 3)(1\ 3\ 2) = (1) \), so \( C_G(A) = A \).

\textbf{6a)} Fix a subgroup \( H \leq G \).
Since \( H \) is a group, it suffices to show that \( H \subseteq N_G(H) \).
Fix \( a \in H \), then \( aH = H \), thus
\( aHa^{-1} = (aH)a^{-1} = Ha^{-1} = H \), which shows that \( a \in N_G(H) \),
thus \( H \subseteq N_G(H) \).
\done

\textbf{6b)} Let \( G \) be an abelian group and \( H \leq G \) a subgroup,
then \( H \) is also abelian and \( C_G(H) = G \), thus
\( H \subseteq C_G(H) \).
Since \( H \) is a group, we have \( H \leq C_G(H) \).
\done

\pagebreak
\subsection*{2.3 Cyclic Groups and Cyclic Subgroups}

\textbf{1)} We have that
\(
	\angleb{1}, \angleb{3}, \angleb{5}, \angleb{9}, \angleb{15}
	\leq \Z / 45\Z
\).
If \( m \mid n \), then \( \angleb{m} \leq \angleb{n} \).

\textbf{3)} Since \( 48 = 2^4 \times 3\), any number \( \leq 48 \) without
a \( 2 \) or \( 3 \) in its prime factorization generates \( \Z / 48\Z \).

\textbf{12a} For all \( n \), we have \( (0, 0)^n = (0, 0) \),
\( (1, 1)^n = (0, 0) \text{ or } (1, 1) \),
\( (0, 1)^n = (0, 1) \text{ or } (0, 0) \), and
\( (1, 0)^n = (1, 0) \text{ or } (0, 0) \), thus
no element in \( \Z / 2\Z \times \Z / 2\Z \) generates the group.
\done

\textbf{12b)} Fix \( (a, k) \in \Z / 2\Z \times \Z \), then there exists no
\( n \in \N \) where \( (a, k)^n = (a, 2k) \), so this group is not cyclic.
\done

\textbf{19)} Fix a group \( H \) where \( h \in H \) and suppose
\( \phi : \Z \to H \) is a homomorphism where \( \phi(1) = h \).
Clearly \( \phi(0) = e \).
If \( n > 0 \), then
\( \phi(n) = \phi(1 + 1 + \cdots + 1) = \phi(1)\phi(1)\cdots\phi(1) = h^n \),
and if \( n < 0 \) we have
\( \phi(n) = \phi(-\vertb{n}) = \phi(\vertb{n})^{-1} = (h^n)^{-1} = h^{-n} \).
Consquently, we are forced to define \( \phi \) as \( n \mapsto h^n \), thus
\( \phi \) is unique.
\done

\end{document}
