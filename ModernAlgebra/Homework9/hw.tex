\documentclass[12pt]{article}

\usepackage{amsfonts}
\usepackage{amsmath}
\usepackage{amssymb}
\usepackage{fancyhdr}
\usepackage[headheight=1in,margin=1.25in]{geometry}
\usepackage[colorlinks=true,linkcolor=blue]{hyperref}
\usepackage{makecell}
\usepackage{mathtools}
\usepackage{setspace}
\usepackage{tikz}

\newcommand{\N}{\mathbb{N}}
\newcommand{\Z}{\mathbb{Z}}
\newcommand{\Q}{\mathbb{Q}}
\newcommand{\R}{\mathbb{R}}
\newcommand{\C}{\mathbb{C}}
\newcommand{\F}{\mathbb{F}}

\renewcommand{\t}[1]{\text{#1}}

\newcommand{\angleb}[1]{\left\langle#1\right\rangle}
\newcommand{\braceb}[1]{\left\{#1\right\}}
\newcommand{\bracketb}[1]{\left[#1\right]}
\newcommand{\parenb}[1]{\left(#1\right)}
\newcommand{\vertb}[1]{\left\vert#1\right\vert}
\newcommand{\ovl}[1]{\overline{#1}}

\newcommand{\normsub}{\trianglelefteq}
\newcommand{\normsup}{\trianglerighteq}

\newcommand{\symdiff}{\, \Delta \,}
\newcommand{\sdiff}{\setminus}

\newcommand{\aut}{\t{Aut}}
\newcommand{\inn}{\t{Inn}}

\newcommand{\proof}{\textit{Proof: }}
\newcommand{\done}{\ensuremath{\strut\hfill\blacksquare}}

\newcommand{\mc}[1]{\ensuremath{\mathcal{#1}}}

\begin{document}

\pagestyle{fancy}
\fancyhead[L]{Modern Algebra}
\fancyhead[C]{Alex Agruso}
\fancyhead[R]{Homework 9}

\setlength{\parindent}{0in}
\setlength{\parskip}{0.15in}
\setstretch{1}

\subsection*{7.1 Basic Definitions and Examples}

Unless otherwise specified, \( R \) is a ring with 1.

\textbf{2)}
If \( u \) is a unit in \( R \), then so is \( -u \).

\proof
Since \( u \) is a unit, there exists \( u^{-1} \in R \) where
\( uu^{-1} = 1 \), but \( uu^{-1} = (-u)(-u^{-1}) = 1, \) showing that \( -u \)
is also a unit.

\textbf{5)} \( (a) \) and \( (d) \) are rings. \( (b) \) fails closure, since
\( (1/12) + (1/4) = (1/3) \), and \( (c) \) lacks additive inverses.

\textbf{7)}
Let \( Z(R) \) denote the center of \( R \).
Then \( Z(R) \) is a subring of \( R \).
Additionally, if \( R \) is a division ring, then \( Z(R) \) is a field.

\proof
For all \( a \in R \), we have \( 1 \cdot a = a = a \cdot 1 \), thus
\( 1 \in Z(R) \).
Now let \( r, s \in Z(R) \), then
\[
	a(r + s) = ar + as = ra + sa = (r + s)a,
\]
showing that \( r + s \in Z(R) \).
We also have that
\[
	a(-r) = -(ar) = -(ra) = (-r)a,
\]
so \( -r \in Z(R) \).
Finally,
\[
	a(rs) = (ar)s = (ra)s = r(as) = r(sa) = (rs)a,
\]
thus \( rs \in Z(R) \), and \( Z(R) \) is a subring of \( R \).

If \( R \) is a division ring, then every non-zero element of \( R \), and thus
every non-zero element of \( Z(R) \) is a unit.
Additionally, if \( r \in Z(R) \), then for all \( a \in R \), we have
\[
	r^{-1}a = r^{-1}(ar)r^{-1} = r^{-1}(ra)r^{-1} = ar^{-1},
\]
thus \( r^{-1} \in Z(R) \), showing that \( Z(R) \) is also a division ring.
Finally, since \( Z(R) \) is commutative, it is a field.
\done

\textbf{11)}
Let \( R \) be an integral domain and fix \( x \in R \).
If \( x^2 = 1 \), then \( x = \pm 1 \).

\proof
Suppose \( x^2 = 1 \), then \( x^2 - 1 = (x + 1)(x - 1) = 0 \).
Since \( R \) is an integral domain, one of \( (x + 1) \) and \( (x - 1) \)
must be 0, implying that \( x \) must be \( 1 \) or \( -1 \).
\done

\pagebreak
\textbf{15)}
All boolean rings are commutative.

\proof
Let \( R \) be a boolean ring and fix \( a, b \in R \).
We have that
\[
	aabb = a^2b^2 = ab = (ab)^2 = abab,
\]
and cancelling on both sides obtains \( ab = ba \).
\done

\textbf{21)}
Fix an arbitrary set \( X \), and denote \( \mc{P}(X) \) as its powerset.
Also, for \( A, B \in \mc{P}(X) \), define the operations \( + \) and
\( \times \) as
\[
	A + B = A \symdiff B = (A \sdiff B) \cup (B \sdiff A)
	\hspace{0.1in}
	\t{and}
	\hspace{0.1in}
	A \times B = A \cap B.
\]
Then \( \mc{P}(X) \) forms a commutative boolean ring with 1 under these
operations.

\proof
It is clear that \( \mc{P}(X) \) is closed under both operations.
For any \( A \in \mc{P}(X) \), we have \( A \symdiff \varnothing = A \), so
\( \varnothing \) serves as our additive identity.
Additionally, \( A \symdiff A = \varnothing \), so each element is its own
additive inverse.
It is left as an exercise for the reader that symmetric differences are
associative.
Thus, \( \mc{P}(X) \) forms an additive group under \( + \).

We have that \( A \cap X = A \) for all \( A \), thus \( X \) serves as our
multiplicative identity.
We also have that intersections distribute across symmetric differences,
showing that \( \mc{P}(X) \) forms a ring.
Finally, we have that \( A \cap A = A \) and \( A \cap B = B \cap A \) for
all \( A \) and \( B \), so the ring is boolean and commutative.

\subsection*{7.2 Polynomial Rings, Matrix Rings, and Group Rings}

\textbf{1)}
Let \( p(x) = 2x^3 - 3x^2 + 4x - 5 \) and \( q(x) = 7x^3 + 33x - 4 \) be
polynomials.
If the coefficients exist in \( \Z \), then we have
\[
	p(x) + q(x) = 9x^3 - 3x^2 + 37x - 9,
\]
\[
	p(x)q(x) = 14x^6 - 21x^5 + 94x^4 - 142x^3 - 20x^2 - 181x + 20
\]
In \( \Z/2\Z \), we have
\[
	p(x) + q(x) = x^3 + x^2 + x + 1,
\]
\[
	p(x)q(x) = x^5 + x.
\]
In \( \Z/3\Z \), we have
\[
	p(x) + q(x) = x,
\]
\[
	p(x)q(x) = 2x^6 + x^4 + x^3 + x + 2
\]

\subsection*{7.3 Ring Homomorphisms and Quotient Rings}

\textbf{4)} -

\textbf{6)}
\( \phi : M_2(\Z) \to \Z \) defined as the projection of the 1,1-th entry is
not a homomorphism since the 1,1-th entry of the product of two matrices is not
necessarily equal the product of the respective components.

\( \phi : M_2(\Z) \to \Z \) defined as the trace of the matrix is not a
homomorphism since the trace of the product of two matrices contains all
entries of both matrices rather than just the diagonal entries.

\( \phi : M_2(\Z) \to \Z \) defined as the determinant of the matrix is not a
homomorphism the determinant of a sum is not necessarily equal to the sum of
the determinants.

\textbf{Lemma 1:}
If \( I \) and \( J \) are ideals in \( R \), then \( I \times J \) is an ideal
in \( R \times R \).

\proof
Let \( (a, b) \in R \times R \) and \( (i, j) \in I \times J \), then
\( ai \in I \) and \( bj \in J \), thus
\( (a, b) \cdot (i, j) = (ai, bj) \in I \times J \).
\done

\textbf{8)}
For \( (b) \) and \( (c) \), we can represent them as \( 2\Z \times 2\Z \) and
\( 2\Z \times \braceb{0} \) respectively, and thus by \textbf{Lemma 1} they are
ideals.

For \( (a) \) and \( (d) \), if \( m \) and \( n \) are distinct in \( \Z \),
then \( (m, n) \cdot (a, a) \) and \( (m, n) \cdot (a, -a) \) are not in their
respective sets, and thus they are not ideals.


\end{document}
