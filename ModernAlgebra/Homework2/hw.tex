\documentclass[12pt]{article}

\usepackage{amsfonts}
\usepackage{amsmath}
\usepackage{amssymb}
\usepackage{fancyhdr}
\usepackage[headheight=1in,margin=1in]{geometry}
\usepackage[colorlinks=true,linkcolor=blue]{hyperref}
\usepackage{mathtools}

\newcommand{\N}{\mathbb{N}}
\newcommand{\Z}{\mathbb{Z}}
\newcommand{\Q}{\mathbb{Q}}
\newcommand{\R}{\mathbb{R}}
\newcommand{\C}{\mathbb{C}}

\newcommand{\angleb}[1]{\left\langle#1\right\rangle}
\newcommand{\braceb}[1]{\left\{#1\right\}}
\newcommand{\bracketb}[1]{\left[#1\right]}
\newcommand{\parenb}[1]{\left(#1\right)}
\newcommand{\vertb}[1]{\left\vert#1\right\vert}

\newcommand{\done}{\ensuremath{\strut\hfill\blacksquare}}

\begin{document}

\pagestyle{fancy}
\fancyhead[L]{Modern Algebra}
\fancyhead[C]{Alex Agruso}
\fancyhead[R]{Homework 2}

\setlength{\parindent}{0in}
\setlength{\parskip}{0.1in}

\section*{1.3 Symmetric Groups}

\textbf{2)} We have that
\[
	\sigma = (1 \ 13 \ 5 \ 10)(3 \ 15 \ 8)(4 \ 14 \ 11 \ 7 \ 12 \ 9) \)
\]
and
\[
	\tau = (1 \ 14)(2 \ 9 \ 15 \ 13 \ 4)(3 \ 10)(5 \ 12 \ 7)(8 \ 11) \),
\]
thus
\[
	\sigma\tau
	= (1 \ 13 \ 5 \ 10)(3 \ 15 \ 8)(4 \ 14 \ 11 \ 7 \ 12 \ 9) 
	(1 \ 14)(2 \ 9 \ 15 \ 13 \ 4)(3 \ 10)(5 \ 12 \ 7)(8 \ 11),
\]
\[
	= (1 \ 11 \ 3)(2 \ 4)(5 \ 9 \ 8 \ 7 \ 10 \ 15)(13 \ 14).
\]

\textbf{10)} Let \( \sigma \) be an \( n \)-cycle and assume indices are taken
as their lowest positive residue mod \( n \).
We will show by induction that \( \sigma^i(a_k) = a_{k + i} \).
For \( i = 1 \), we have by definition that \( \sigma(a_k) = a_{k + 1} \).
Now, assume the induction hypothesis \( \sigma^{i - 1}(a_k) = a_{k + i - 1} \),
then
\(
	\sigma^i(a_k)
	= \sigma(\sigma^{i - 1}(a_k))
	= \sigma(a_{k + i - 1})
	= a_{k + i}.
\)
\done

\textbf{13)} Let \( \sigma \in S_n \) and consider its cycle decomposition
\( \sigma = \sigma_1\sigma_2\cdots\sigma_m \).
To establish a contradiction, assume that one of the cycles \( \sigma_i \) has
length \( \geq 3 \),

\textbf{15)}

\textbf{18)}

\section*{1.4 Matrix Groups}

\textbf{7)}

\textbf{8)}

\textbf{9)}

\textbf{10a)}

\textbf{10b)}

\section*{1.6 Homomorphisms and Isomorphisms}

\textbf{3)}

\textbf{5)}

\textbf{6)}

\textbf{9)}

\textbf{11)}

\textbf{14)}

\textbf{20)}

\textbf{22)}

\textbf{25)}

\end{document}
