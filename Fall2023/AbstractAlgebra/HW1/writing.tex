\documentclass[12pt]{article}
\pagenumbering{gobble}
\linespread{1.2}

\usepackage{amsfonts}
\usepackage{amsmath}
\usepackage{amssymb}
\usepackage{array}
\usepackage{fancyhdr}
\usepackage{mathrsfs}
\usepackage{mathtools}
\usepackage{textcomp}
\usepackage[margin=1in,headheight=1in]{geometry}

\newcommand{\contradiction}{
    \ensuremath{{\Rightarrow\mspace{-2mu}\Leftarrow}}
}

\newcommand{\angleb}[1]{\left\langle#1\right\rangle}
\newcommand{\vertb}[1]{\left\vert#1\right\vert}
\newcommand{\bracks}[1]{\left[#1\right]}
\newcommand{\parns}[1]{\left(#1\right)}

\newcommand{\derv}[2]{\dfrac{d#1}{d#2}}
\newcommand{\e}{\varepsilon}
\newcommand{\di}{\,/\,}

\newcommand{\lm}[1]{\displaystyle\lim_{#1}}
\newcommand{\lcm}{\text{lcm}}
\newcommand{\pmd}[1]{\text{ (mod }#1\text{)}}

\begin{document}
\pagestyle{fancy}
\fancyhead{}
\fancyhead[L]{Alex Agruso}
\fancyhead[R]{Topology Writing 1}

\normalsize

\begin{itemize}
    \item [2.)] \begin{itemize}
        \item [a.)] $\gcd(2,10)=2$, $\lcm(2,10)=10$
        \item [b.)] $\gcd(20,8)=4$, $\lcm(20,8)=40$
        \item [c.)] $\gcd(12,40)=4$, $\lcm(12,40)=120$
        \item [d.)] $\gcd(21,50)=1$, $\lcm(21,50)=1050$
        \item [e.)] $\gcd(p^2q^2,pq^3)=pq^2$, $\lcm(p^2q^2,pq^3)=p^2q^3$
    \end{itemize}

    \item [7.)] Let $a,b\in\mathbb{Z}$ and $n\in\mathbb{N}$. Assume $a\equiv b\pmd{n}$, thus $a=b+kn$ for some $k\in\mathbb{Z}$. $a=b+kn\implies a-b=kn\implies n\mid a-b$, thus $a\equiv b\pmd{n}\implies n\mid a-b$.
    
    Next, assume $n\mid a-b$, thus $ln=a-b$ for some $l\in\mathbb{Z}$. $ln=a-b\implies a=b+ln\break\implies a\equiv b\pmd{n}$, thus $n\mid a-b\implies a\equiv b\pmd{n}$,
    
    Thus $a\equiv b\pmd{n}\iff n\mid a-b$. Q.E.D.

    \item [18.)] $8^1\equiv3\pmd{5}$, $8^2\equiv4\pmd{5}$, $8^3\equiv2\pmd{5}$, $8^4\equiv1\pmd{5}$, $8^5\equiv3\pmd{5}$, thus $8^{402}\equiv(8^4)^{100}\times8^2\equiv1^{100}\times4\equiv4\pmd{5}$
    
    \item [58.)] Define a relation $R$ on $\mathbb{R}$ where $a\sim b\implies a-b\in\mathbb{Z}$ for $a,b\in\mathbb{R}$. Since for all $a\in\mathbb{R}$, $a-a=0$, we know that $a\sim a$, thus $R$ is reflexive.
    
    Next, assume $a\sim b$, thus $a-b\in\mathbb{Z}$. Let $n=a-b$; $n=a-b\implies -n=-(a-b)=b-a$. Since $n\in\mathbb{Z}\implies-n\in\mathbb{Z}$, we know that $b\sim a$, thus $R$ is symmetric. ***

    \item [60.)] awd
\end{itemize}

\end{document}
