\documentclass[12pt]{article}
\pagenumbering{gobble}
\linespread{1.15}

\usepackage{amsfonts}
\usepackage{amsmath}
\usepackage{amssymb}
\usepackage{array}
\usepackage{fancyhdr}
\usepackage{textcomp}
\usepackage[margin=1in,headheight=1in]{geometry}

\newcommand{\contradiction}{
    \ensuremath{{\Rightarrow\mspace{-2mu}\Leftarrow}}
}

\newcommand{\angleb}[1]{\left\langle#1\right\rangle}
\newcommand{\vertb}[1]{\left\vert#1\right\vert}
\newcommand{\bracks}[1]{\left[#1\right]}

\newcommand{\derv}[2]{\dfrac{d#1}{d#2}}

\begin{document}
\pagestyle{fancy}
\fancyhead{}
\fancyhead[L]{Alexander Agruso}
\fancyhead[R]{MATH 3323 Homework 6}

\normalsize
\begin{itemize}
    \item [2.)] Find the critical points of $y'$:
    \[y'=y(y-1)(y-2)\implies y=0,1,2\text{ are critical points}\]
    Fill in the sign chart to determine solution stability:
    \begin{center}
        \begin{tabular}{c|c|c|c|c}
            & $y$ & $(y-1)$ & $(y-2)$ & result\\
            \hline
            $y=-\frac{1}{2}$ & $-$ & $-$ & $-$ & $-$\\
            $y=\frac{1}{2}$ & $+$ & $-$ & $-$ & $+$\\
            $y=\frac{3}{2}$ & $+$ & $+$ & $-$ & $-$\\
            $y=\frac{5}{2}$ & $+$ & $+$ & $+$ & $+$\\
        \end{tabular}
    \end{center}
    Thus $y=0,2$ are unstable solutions and $y=1$ is a stable solution.

    \item [5.)] \begin{itemize}
        \item [a.)] For the sake of establishing a contradiction, suppose $y\neq1$ is a critical point of $y'$, thus $k(1-y)^2=0$. We can manipulate the equation as follows:
        \begin{align*}
            k(1-y)^2=0&\implies(1-y)^2=0\\
            &\implies(1-y)=\pm0=0\\
            &\implies1=y
        \end{align*}
        But $y\neq1$\contradiction, thus for $y$ to be a critical point of $y'$, $y=1$. Q.E.D.

        \item [b.)] Fill in the sign chart to determine solution stability:
        \begin{center}
            \begin{tabular}{c|c|c}
                & $k(1-y)^2$ & result\\
                \hline
                $y=\frac{1}{2}$ & $+$ & $+$\\
                $y=\frac{3}{2}$ & $+$ & $+$\\
            \end{tabular}
        \end{center}
        Thus $y=1$ is a semistable solution.

        \item [c.)] Solve for $y$:
        \[\frac{dy}{dt}=k(1-y)^2\implies\frac{dy}{k(1-y)^2}=dt\]
        \[\implies\int\frac{dy}{k(1-y)^2}=\int dt\implies\frac{1}{k(1-y)}=t+C\]
        \[\implies1-y=\frac{1}{k(t+C)}\implies y=1-\frac{1}{k(t+C)}=\frac{k(t+C)-1}{k(t+C)}\]
        Substitute $y(0)=y_0$:
        \[y_0=\frac{k(0+C)-1}{k(0+C)}=\frac{kC-1}{kC}=1-\frac{1}{kC}\implies 1-y_0=\frac{1}{kC}\]
        \[\implies k(1-y_0)=\frac{1}{C}\implies C=\frac{1}{k(1-y_0)}\]
        \[\therefore\ y=1-\frac{1}{k\left(t+\frac{1}{k(1-y_0)}\right)}=1-\frac{1-y_0}{kt(1-y_0)+1}=\frac{kt(1-y_0)+y_0}{kt(1-y_0)+1}\]
        Which is the particular solution.
    \end{itemize}

    \pagebreak
    \item [6.)] Find the critical points of $y'$:
    \[y'=y^2(y^2-1)\implies y=-1,0,1\text{ are critical points}\]
    Fill in the sign chart to determine solution stability:
    \begin{center}
        \begin{tabular}{c|c|c|c}
            & $y^2$ & $(y^2-1)$ & result\\
            \hline
            $y=-\frac{3}{2}$ & $+$ & $+$ & $+$\\
            $y=-\frac{1}{2}$ & $+$ & $-$ & $-$\\
            $y=\frac{1}{2}$ & $+$ & $-$ & $-$\\
            $y=\frac{3}{2}$ & $+$ & $+$ & $+$\\
        \end{tabular}
    \end{center}
    Thus $y=-1$ is a stable solution, $y=0$ is a semistable solution, and $y=1$ is an unstable solution.

    \item [7.)] Find the critical points of $y'$:
    \[y'=y(1-y^2)\implies y=-1,0,1\text{ are critical points}\]
    Fill in the sign chart to determine solution stability:
    \begin{center}
        \begin{tabular}{c|c|c|c}
            & $y$ & $(1-y^2)$ & result\\
            \hline
            $y=-\frac{3}{2}$ & $-$ & $-$ & $+$\\
            $y=-\frac{1}{2}$ & $-$ & $+$ & $-$\\
            $y=\frac{1}{2}$ & $+$ & $+$ & $+$\\
            $y=\frac{3}{2}$ & $+$ & $-$ & $-$\\
        \end{tabular}
    \end{center}
    Thus $y=-1,1$ are stable solutions, and $y=0$ is an unstable solution.

    \item [14.)] $f'(y)$=$\dfrac{d^2y}{dy^2}$, thus $f'(y)$ can be used to determine the concavity of a given solution $\phi(t)$. Suppose $y_1$ is a critical point; when $f'(y_1)<0$, then $\phi(t)=y_1$ is concave with respect to the $y$-axis, thus the solution converges to $y_1$, making it stable. When $f'(y_1)>0$, the solution is convex, thus the solution does not converge to $y_1$, making it unstable.

\end{itemize}

\end{document}
