\documentclass[12pt]{article}
\pagenumbering{gobble}

\usepackage{amsfonts}
\usepackage{amsmath}
\usepackage{amssymb}
\usepackage{array}
\usepackage{fancyhdr}
\usepackage{mathtools}
\usepackage{textcomp}
\usepackage[margin=1in,headheight=1in]{geometry}

\newcommand{\contradiction}{
    \ensuremath{{\Rightarrow\mspace{-2mu}\Leftarrow}}
}

\newcommand{\angleb}[1]{\left\langle#1\right\rangle}
\newcommand{\vertb}[1]{\left\vert#1\right\vert}
\newcommand{\bracks}[1]{\left[#1\right]}
\newcommand{\parns}[1]{\left(#1\right)}

\newcommand{\derv}[2]{\dfrac{d#1}{d#2}}
\newcommand{\pderv}[2]{\dfrac{\partial#1}{\partial#2}}

\begin{document}
\pagestyle{fancy}
\fancyhead{}
\fancyhead[L]{Alexander Agruso}
\fancyhead[R]{MATH 3323 Homework 10}

\normalsize
\begin{itemize}
    \item[1.)] \[W\bracks{e^{2t},e^{-3t/2}}=\begin{vmatrix}
        e^{2t} & e^{-3t/2} \\
        2e^{2t} & -\frac{3}{2}e^{-3t/2}
    \end{vmatrix}=-\frac{3}{2}e^{t/2}-2e^{t/2}=-\frac{7}{2}e^{t/2}\]

    \item[2.)] \[W\bracks{\cos t,\sin t}=\begin{vmatrix}
        \cos t & \sin t \\
        -\sin t & \cos t
    \end{vmatrix}=\cos^2t+\sin^2t=1\]

    \item[3.)] \[W\bracks{e^{-2t},te^{-2t}}=\begin{vmatrix}
        e^{-2t} & te^{-2t} \\
        -2e^{-2t} & e^{-2t}-2te^{-2t}
    \end{vmatrix}=e^{-4t}-2te^{-4t}+2te^{-4t}=e^{-4t}\]

    \item[6.)] Determine $p(t)$, $q(t)$, and $g(t)$:
    \[ty''+3y=t\implies y''+\frac{3}{t}y=1\]
    \[\therefore\ p(t)=0,\ q(t)=\frac{3}{t},\ g(t)=1\]
    Since the only discontinuity point is $t=0$, and since $t_0=1$, the largest interval is $(0,\infty)$.

    \item[7.)] Determine $p(t)$, $q(t)$, and $g(t)$:
    \[t(t-4)y''+3ty'+4y=2\implies y''+\frac{3}{t-4}y'+\frac{4}{t(t-4)}y=\frac{2}{t(t-4)}\]
    \[\therefore\ p(t)=\frac{3}{t-4},\ q(t)=\frac{4}{t(t-4)},\ g(t)=\frac{2}{t(t-4)}\]
    Since $t_0=3$, and there is discontinuity when $t=0$ and $t=4$, the largest interval is $(0,4)$.

    \item[8.)] Determine $p(t)$, $q(t)$, and $g(t)$:
    \[y''+(\cos t)y'+3(\ln\vert t\vert)y=0\]
    \[\therefore\ p(t)=\cos t,\ q(t)=3\ln\vert t\vert,\ g(t)=0\]
    Since $t_0=2$, and since $q(t)$ is only continuous when $t>0$, the largest interval is $(0,\infty)$.

    \item[9.)] Determine $p(x)$, $q(x)$, and $g(x)$:
    \[(x-2)y''+y'+(x-2)(\tan x)y=0\implies y''+\frac{1}{x-2}y'+(\tan x)y=0\]
    \[\therefore\ p(x)=\frac{1}{x-2}\ q(x)=\tan x,\ g(x)=0\]
    Since $q(x)$ is continuous when $\frac{\pi}{2}<x<\frac{3\pi}{2}$, and since $p(x)$ is discontinuous at $x=2$, the largest interval is $\parns{2,\frac{3\pi}{2}}$.
\end{itemize}
\end{document}
