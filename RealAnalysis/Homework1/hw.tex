\documentclass[12pt]{article}

\usepackage{amsfonts}
\usepackage{amsmath}
\usepackage{amssymb}
\usepackage{fancyhdr}
\usepackage[headheight=1in,margin=1in]{geometry}
\usepackage[colorlinks=true,linkcolor=blue]{hyperref}
\usepackage{mathtools}

\newcommand{\N}{\ensuremath{\mathbb{N}}}
\newcommand{\Z}{\ensuremath{\mathbb{Z}}}
\newcommand{\Q}{\ensuremath{\mathbb{Q}}}
\newcommand{\R}{\ensuremath{\mathbb{R}}}
\newcommand{\C}{\ensuremath{\mathbb{C}}}

\newcommand{\e}{\varepsilon}

\newcommand{\angleb}  [1]{\ensuremath{\left\langle#1\right\rangle}}
\newcommand{\braceb}  [1]{\ensuremath{\left\{#1\right\}}}
\newcommand{\bracketb}[1]{\ensuremath{\left[#1\right]}}
\newcommand{\parenb}  [1]{\ensuremath{\left(#1\right)}}
\newcommand{\vertb}   [1]{\ensuremath{\left\vert#1\right\vert}}

\newcommand{\mc}[1]{\ensuremath{\mathcal{#1}}}

\newcommand{\done} {\ensuremath{\strut\hfill\blacksquare}}
\newcommand{\proof}{\textit{Proof: }}

\begin{document}

\pagestyle{fancy}
\fancyhead[L]{Analysis}
\fancyhead[C]{Alex Agruso}
\fancyhead[R]{Homework 1}

\setlength{\parindent}{0in}
\setlength{\parskip}{0.1in}

Let \( \mc{C} \) denote the Cantor set and \( \mc{C}_k \) denote the
\( k^\text{th} \) set in the construction of \( \mc{C} \).

For any interval \( I \), let \( \bar{I} \) denote the closure of that
interval.

\textbf{Example 4)}
Fix a rectangle \( R \subset \R^d \) and let \( M \) be the largest side length
of \( R \).
Since \( R \) contains a constant number of \( d - 1 \) dimensional ``faces'',
say \( n \) of them, and since each face contains at least \( (Mk)^{d - 1} \)
cubes, we have that \( \vertb{\mc{Q}'} \leq (M^{d - 1}n)k^{d - 1} \), and thus
\( \vertb{\mc{Q}'} = O(k^{d - 1}) \).
\done

\textbf{1)}
\( \mc{C} \) is totally disconnected and perfect.

\proof
Fix \( x, y \in \mc{C} \) where \( x \ne y \).
Since \( 1 / 3^k \to 0 \) as \( k \to \infty \), we can fix a nonnegative
integer \( k \) where \( 1 / 3^k < \vertb{x - y} \), thus \( x \) and \( y \)
must lie in disjoint intervals in \( \mc{C}_k \supset \mc{C} \) which
shows that \( \mc{C} \) is totally disconnected.

Next, fix \( x \in \mc{C} \), then \( x \in \mc{C}_k \) for all nonnegative
\( k \), and thus for each \( k \) there exists an integer \( n_k \) where
\( [n_k / 3^k, (n_k + 1) / 3^k] \subset \mc{C}_k \) and
\[
	\frac{n_k}{3^k} \leq x \leq \frac{n_k + 1}{3^k}.
\]
Letting \( k \to \infty \), we have that
\( (n_k + 1) / 3^k - n_k / 3^k = 1 / 3^k \to 0 \), but since
\[
	0
	\leq x - \frac{n_k}{3^k}
	\leq \frac{n_k + 1}{3^k} - \frac{n_k}{3^k}
	= \frac{1}{3^k},
\]
we have that \( x - n_k / 3^k = \vertb{x - n_k / 3^k} \to 0 \).
Since \( n_k / 3^k \in \mc{C} \), we have elements in \( \mc{C} \) that are
arbitrarily close to \( x \), thus \( x \) is not an isolated point and
\( \mc{C} \) is perfect.
\done

\textbf{2a)} We have that \( x \in \mc{C} \) if and only if
\[
	x = \sum_{k = 1}^\infty \frac{a_k}{3^k}
\]
where \( a_k = 0 \) or \( a_k = 2 \) for all \( k \).

\proof
Fix \( x \in [0, 1] \) and suppose \( x \) has a suitable ternary
representation \braceb{a_k}.
If \( a_1 = 0 \), then we have that \( x \in [0, 1 / 3] \), as we have that
\[
	\sum_{n = 2}^\infty \frac{2}{3^n} = \frac{1}{3},
\]
and thus the sum of the remaining digits cannot surpass \( 1 / 3 \).
The same argument shows that if \( a_1 = 2 \) then \( x \in [2 / 3, 1] \), and
thus \( x \in \mc{C}_1 \).
Fix the interval \( I \) in \( \mc{C}_1 \) that contains \( x \).
Since \( \vertb{I} = 1 / 3 \), we can scale it by a factor of 3 to obtain an
interval with length 1 and endpoints 0 and 1.
Likewise, we can drop the \( a_1 \) term of \( x \) and multiply by \( 3 \)
to obtain
\[
	\sum_{n = 1}^\infty \frac{a_{n + 1}}{3^n}
	= \frac{a_2}{3} + \frac{a_3}{3^2} + \cdots.
\]
Applying the same argument as before, we have that \( a_2 = 0 \) or
\( a_2 = 2 \) implies \( x \in \mc{C}_2 \).
Repeating this process infinitely shows that \( x \in \mc{C}_k \) for all
\( k \), and thus \( x \in \mc{C} \).

Now suppose \( x \in [0, 1] \) does not have a ternary representation
comprising only of 0 and 2, and let \( k \) be the smallest integer where
\( a_k = 1 \).
We previously established that \( x' = \sum_{n = 1}^{k - 1} a_n / 3^n\) exists
in \( \mc{C} \) since the rest of the digits in its ternary representation are
0, thus there exists an interval in \( \mc{C}_{k - 1} \) that contains
\( x' \).
Since \( a_k = 1 \), we have that \( x \) is contained in the middle third of
the interval in \( \mc{C}_{k - 1} \), and thus \( x \notin \mc{C}_k \), and
thus is not in \( \mc{C} \).
\done

\textbf{2b)} Show that the Cantor-Lebesgue function \( F \) is well-defined and
continuous.

Not a proof, but if we can limit \( \delta \) less than some
negative power of 3, then given \( x, y \in \mc{C} \) we can ensure that the
first \( n \) digits of \( F(x) \) and \( F(y) \) are the same for arbitrarily
large \( n \), and thus we can ensure that \( \vertb{F(x) - F(y)} \) is
arbitrarily small.

\textbf{2c)} Show that \( F \) is surjective.

\proof
Fix \( y \in [0, 1] \), then it has a binary representation.
Given \( x \in \mc{C} \), we have that by definition \( F(x) \) simply replaces
each 2 in the ternary representation of \( x \) with a \( 1 \) and considers
the number in binary, thus if we replace each 1 in the binary representation of
\( y \) with a 2, we obtain \( x \in \mc{C} \) where \( F(x) = y \), and thus
\( F \) is surjective.
\done

\textbf{12a)} Open discs in \( \R^2 \) are not the union of disjoint open
rectangles.

\proof Special case of \textbf{12b}.
\done

\textbf{12b)} An open connected set \( \Omega \) is the disjoint union of open
rectangles if and only if \( \Omega \) is itself an open rectangle.

\proof
Let \( \Omega \subset \R^2 \) be an open connected set, and fix an open
rectangle \( R \) with
\[
	R = (x_1, x_2) \times (y_1, y_2) \subseteq \Omega.
\]
If \( \Omega = R \), then \( \braceb{R} \) is a collection of disjoint
open rectangles whose union is \( \Omega \), and we are finished.
Now suppose \( \Omega \ne R \), then since \( \Omega \) is connected, there
exists a point \( p \) where
\[
	p = (x, y) \in \Omega \cap \partial R.
\]
Clearly \( p \notin R \).
Also, let \( R' \) be an open rectangle disjoint to \( R \) where 
\[
	R' = (x'_1, x'_2) \times (y'_1, y'_2) \subseteq \Omega,
\]
then one or both of the following is possible:
\[
	x'_2 \leq x_1
	\hspace{0.15in} \text{or} \hspace{0.15in}
	x'_1 \leq x_2,
	\hspace{0.3in}
	y'_2 \leq y_1
	\hspace{0.15in} \text{or} \hspace{0.15in}
	y'_1 \leq y_2.
\]
Assume the former, then we have that \( x \notin (x_2, x'_2) \) since
\( x_1 \leq x \leq x'_1 \), and thus \( p \notin R' \).
Assuming the latter, we find that \( y \notin (y_2, y'_2) \) since
\( y_1 \leq y \leq y'_1 \), thus \( p \notin R' \).
Consequently, any collection of disjoint open rectangles that contains
\( R \subset \Omega \) has no element that contains \( p \in \Omega \), thus
\( \Omega \) cannot be a union of disjoint open rectangles.
\done

\textbf{14a)} Fix \( E \subseteq \R \), then \( J_*(E) = J_*(\bar{E}) \).

\proof
Fix a set \( E \subset \R \) and let \( J_*(E) = M \) for some real
\( M \), then there exists a finite collection of intervals
\( \mc{I} = \braceb{I_1, I_2, \dots, I_n} \) that covers \( E \) and whose
summed length is \( M \).
Without loss of generality, we can assume \mc{I} is disjoint.

If \( E \) is closed we are finished.
Else, let \( p \in \partial E \) where \( p \notin E \), then there exists
\( \e > 0 \) where either \( (p, p + \e) \) or \( (p - \e, p) \) is contained
in \( E \).
Denote this interval \( I \), then for some interval \( I_k \in \mc{I} \) we
have that \( I \subseteq I_k \), thus
\( \bar{I} \subseteq \bar{I_k} \), and thus
\( p \in \bar{I} \subseteq \bar{I_k} \).
Let \( \bar{\mc{I}} \) be the collection of the closures of the intervals in
\mc{I}, then for all \( p \in \bar{E} = \partial E \cup E \), we have that
\( p \) is contained in some interval in \( \bar{\mc{I}} \), and thus
\( \bar{\mc{I}} \) covers \( \bar{E} \).
Since the summed lengths of the intervals in \( \bar{\mc{I}} \) is \( M \),
we have that \( J_*(\bar{E}) \leq M \).

Now, to establish a contradiction, suppose we have a collection of intervals
\( \mc{I}' \) that covers \( \bar{E} \) whose summed length is \( < M \), then
it clearly covers \( E \), but then \( J_*(E) < M \) which is a contradiction.
Thus, we have that \( J_*(\bar{E}) = M \).
\done

\textbf{14b)} There exists a countable subset \( E \subset [0, 1] \) where
\( J_*(E) = 1 \) and \( m_*(E) = 0 \).

\proof
Let \( E = [0, 1] \cap \Q \).
Since \( \Q \) is countable, clearly \( E \) is too.
Thus, we have that \( m_*(E) = \sum_{x \in E} m_*(\braceb{x}) = 0 \), since a
single point is technically a closed cube with volume 0.
Now, to establish a contradiction, suppose \( J_*(E) < 1 \), then we have some
interval \( I = [x, y] \subset [0, 1] \) where \( \vertb{I} > 0 \) and
\( I \cap E = \varnothing \), but this is a contradiction since there exists
a rational number between any two disinct real numbers, thus we must have that
\( J_*(E) = 1 \) since \( [0, 1] \) covers \( E \).
\done

\textbf{15)} Defining the exterior measure using rectangles is equivalent
to defining it using cubes.

\proof
Fix \( E \subseteq \R^d \).
Since any covering of \( E \) with cubes is also a covering with rectangles, we
have that \( m^{\mc{R}}_*(E) \leq m_*(E) \).
Suppose a rectangle can be tiled using uniform cubes, then there must be an
integer number of cubes on each side, and thus the ratio between any two sides
of the rectangle must be rational.
Denote such a rectangle as ``tileable''.
Now, let \braceb{R_j} be a covering of \( E \) using rectangles.
If each \( R_j \) is tileable, then we have a covering of \( E \) with cubes
and we are finished.
Otherwise, fix \( \e > 0 \), then for each non-tileable \( R_j \), we can 
ontain it in a tileable rectangle \( R'_j \) where
\( \vertb{R'_j} - \e / 2^j \leq \vertb{R_j}  \).
This is possible because for each non-rational side length ratio, there exists
an arbitrarily close rational number larger than it, and thus we can extend the
sides so that their ratios are rational.
Note that if \( R_j \) is tileable, then we can set \( R_j = R'_j \).
Thus, we have that
\[
	\sum_{j = 1}^\infty \vertb{R_j}
	\geq \sum_{j = 1}^\infty \parenb{\vertb{R'_j} - \frac{\e}{2^j}}
	= \sum_{j = 1}^\infty \vertb{R'_j} - \e.
\]
Letting \( \e \to 0 \), we have that
\[
	\sum_{j = 1}^\infty \vertb{R'_j}
	\leq \sum_{j = 1}^\infty \vertb{R_j}.
\]
Since any covering of \( E \) with tileable rectangles is also a covering with
cubes, we have that \( m_*(E) \leq m_*^\mc{R}(E) \), and thus
\( m_*(E) = m_*^\mc{R}(E) \).
\done

\end{document}
