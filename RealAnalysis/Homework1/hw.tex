\documentclass[12pt]{article}

\usepackage{amsfonts}
\usepackage{amsmath}
\usepackage{amssymb}
\usepackage{fancyhdr}
\usepackage[headheight=1in,margin=1in]{geometry}
\usepackage[colorlinks=true,linkcolor=blue]{hyperref}
\usepackage{mathtools}

\newcommand{\N}{\mathbb{N}}
\newcommand{\Z}{\mathbb{Z}}
\newcommand{\Q}{\mathbb{Q}}
\newcommand{\R}{\mathbb{R}}
\newcommand{\C}{\mathbb{C}}

\newcommand{\e}{\varepsilon}

\newcommand{\angleb}[1]{\left\langle#1\right\rangle}
\newcommand{\braceb}[1]{\left\{#1\right\}}
\newcommand{\bracketb}[1]{\left[#1\right]}
\newcommand{\parenb}[1]{\left(#1\right)}
\newcommand{\vertb}[1]{\left\vert#1\right\vert}

\newcommand{\mc}[1]{\ensuremath{\mathcal{#1}}}

\newcommand{\done}{\ensuremath{\strut\hfill\blacksquare}}
\newcommand{\proof}{\textit{Proof: }}

\begin{document}

\pagestyle{fancy}
\fancyhead[L]{Analysis}
\fancyhead[C]{Alex Agruso}
\fancyhead[R]{Homework 1}

\setlength{\parindent}{0in}
\setlength{\parskip}{0.1in}

Let \( \mc{C} \) denote the Cantor set and \( \mc{C}_k \) denote the
\( k^\text{th} \) set in the construction of \( \mc{C} \).

\textbf{1)} \( \mc{C} \) is totally disconnected and perfect.

\proof
Fix \( x, y \in \mc{C} \) where \( x \ne y \).
Since \( 1 / 3^k \to 0 \) as \( k \to \infty \), we can fix a nonnegative
integer \( k \) where \( 1 / 3^k < \vertb{x - y} \), thus \( x \) and \( y \)
must lie in disjoint intervals in \( \mc{C}_k \supset \mc{C} \) which
shows that \( \mc{C} \) is totally disconnected.

Next, fix \( x \in \mc{C} \), then \( x \in \mc{C}_k \) for all nonnegative
\( k \), and thus for each \( k \) there exists an integer \( n_k \) where
\( [n_k / 3^k, (n_k + 1) / 3^k] \subset \mc{C}_k \) and
\[
	\frac{n_k}{3^k} \leq x \leq \frac{n_k + 1}{3^k}.
\]
Letting \( k \to \infty \), we have that
\( (n_k + 1) / 3^k - n_k / 3^k = 1 / 3^k \to 0 \), but since
\[
	0
	\leq x - \frac{n_k}{3^k}
	\leq \frac{n_k + 1}{3^k} - \frac{n_k}{3^k}
	= \frac{1}{3^k},
\]
we have that \( x - n_k / 3^k = \vertb{x - n_k / 3^k} \to 0 \).
Since \( n_k / 3^k \in \mc{C} \), we have elements in \( \mc{C} \) that are
arbitrarily close to \( x \), thus \( x \) is not an isolated point and
\( \mc{C} \) is perfect.
\done

\textbf{2a)} We have that \( x \in \mc{C} \) if and only if
\[
	x = \sum_{k = 1}^\infty a_k3^{-k}
\]
where \( a_k = 0 \) or \( a_k = 2 \).

\proof
Fix \( x \in \mc{C} \).

\textbf{12)}

\textbf{14)}

\textbf{15)}

\end{document}
