\documentclass[12pt]{article}

\usepackage{amsfonts}
\usepackage{amsmath}
\usepackage{amssymb}
\usepackage{cite}
\usepackage{enumerate}
\usepackage{fancyhdr}
\usepackage[headheight=1in,margin=1.25in]{geometry}
\usepackage[colorlinks=true,linkcolor=blue]{hyperref}
\usepackage{mathtools}
\usepackage{setspace}

\newcommand{\N}{\ensuremath{\mathbb{N}}}
\newcommand{\Z}{\ensuremath{\mathbb{Z}}}
\newcommand{\Q}{\ensuremath{\mathbb{Q}}}
\newcommand{\R}{\ensuremath{\mathbb{R}}}
\newcommand{\C}{\ensuremath{\mathbb{C}}}

\newcommand{\e}{\ensuremath{\varepsilon}}
\renewcommand{\d}{\ensuremath{\delta}}

\newcommand{\angleb}[1]{\left\langle#1\right\rangle}
\newcommand{\braceb}[1]{\left\{#1\right\}}
\newcommand{\bracketb}[1]{\left[#1\right]}
\newcommand{\parenb}[1]{\left(#1\right)}
\newcommand{\vertb}[1]{\left\vert#1\right\vert}
\newcommand{\dvertb}[1]{\left\Vert#1\right\Vert}
\DeclarePairedDelimiter\floor{\lfloor}{\rfloor}

\newcommand{\comp}{\complement}
\newcommand{\sdiff}{\setminus}

\newcommand{\proof}{\textit{Proof: }}
\newcommand{\partialdone}{\ensuremath{\strut\hfill\blacktriangle}}
\newcommand{\done}{\ensuremath{\strut\hfill\blacksquare}}

\newcommand{\mc}[1]{\ensuremath{\mathcal{#1}}}

\renewcommand{\t}[1]{\text{ #1 }}
\newcommand{\impl}{\ensuremath{\implies}}
\newcommand{\sectionskip}{\vspace{0.15in}}

\bibliographystyle{plain}

\begin{document}

\pagestyle{fancy}
\fancyhead[L]{Analysis}
\fancyhead[C]{Alex Agruso}
\fancyhead[R]{Homework 9}

\setlength{\parindent}{0in}
\setlength{\parskip}{0.1in}
\setstretch{1}

\textbf{9)}
Let \( f : \R^d \to \R \) be a non-negative integrable function.
If \( \alpha > 0 \) and \break
\( E_\alpha = \braceb{x \in \R^d : f(x) > \alpha } \),
then
\[
	m(E_\alpha)
	\leq \frac{1}{\alpha} \int_{E_\alpha} f(x) \, dx.
\]

\proof
We can see that
\[
	m(E_\alpha)
	= \int_{E_\alpha} 1 \, dx.
\]
Additionally, if \( x \in E_\alpha \), then \( f(x) > \alpha \), hence
\( f(x) / \alpha > 1 \).
Thus, by monotonicity, we have that
\[
	\int_{E_\alpha} 1 \, dx
	< \int_{E_\alpha} \frac{f(x)}{\alpha} \, dx
	= \frac{1}{\alpha} \int_{E_\alpha} f(x) \, dx,
\]
which proves the inequality.
\done

\sectionskip

\textbf{17)}
Given a positive convergent series \( \sum_{n \in \N_0} b_n \), define
\( a_n = \sum_{0 \leq k \leq n} b_k \), and let \( f \) be a function on
\( \R^2 \) defined as follows:
\[
	f(x,y) = \begin{cases}
		a_n  & \t{if} x \in [n, n + 1) \t{and} y \in [n, n + 1)     \\
		-a_n & \t{if} x \in [n, n + 1) \t{and} y \in [n + 1, n + 2) \\
		0    & \t{otherwise},
	\end{cases}
\]
where \( n \) is always non-negative.
Then we have that \( f^y \) and \( f_x \) are integrable over \R, and
that \( \int f_x(y) \, dy = 0 \) for all \( x \in \R \), showing that
\( \int \parenb{\int f(x,y) \, dy} dx = 0 \).

However, we also have that \( \int f^y(x) \, dx = a_0 \) for
\( 0 \leq y < 1 \), and \( \int f^y(x) \, dx = a_n - a_{n - 1} \) if
\( n \leq y < n + 1 \) for some \( n \in \N \).
As a result,
\( \int_{(0,\infty)} \int f^y(x) \, dx \, dy = \sum_{k \in \N_0} b_k \).

Finally, we have that \( \iint \vertb{f(x,y)} \, dx \, dy = \infty \).

\proof
Since \( n \) is always non-negative, we have that \( f(x,y) = 0 \) if
either \( x \) or \( y \) is negative.
Fix \( x \geq 0 \) and let \( n = \floor{x} \).
If \( n \leq y < n + 1 \), then \( f_x(y) = a_n \), and if
\( n + 1 \leq y < n + 2 \), then \( f_x(y) = -a_n \).
Otherwise, \( f_x(y) = 0 \).
Thus, we have that
\[
	\int f_x(y) \, dy
	= \int_{[n, n + 2)} f_x(y) \, dy
	= \int_{[n, n + 1)} f_x(y) \, dy + \int_{[n + 1, n + 2)} f_x(y) \, dy
	= a_n - a_n
	= 0,
\]
which shows that \( f_x \) is integrable and \( \int f_x(y) \, dy = 0 \).
From this we also obtain
\[
	\iint f(x,y) \, dy \, dx
	= \iint f_x(y) \, dy \, dx
	= \int 0 \, dx
	= 0.
\]

Fix \( y \geq 0 \) and redefine \( n = \floor{y} \).
If \( y < 1 \), then \( f^y(x) = a_0 \) only when \( 0 \leq x < 1 \),
and is otherwise 0, thus we would have
\[
	\int f^y(x) \, dx
	= \int_{[0,1)} f^y(x) \, dx
	= \int_{[0,1)} a_0 \, dx
	= a_0.
\]
Now let \( y \geq 1 \).
If \( n \leq x < n + 1 \), then \( f^y(x) = a_n \), and if
\( n - 1 \leq x < n \), then \( f^y(x) = -a_{n - 1} \).
This shows that
\[
	\int f^y(x) \, dx
	= \int_{[n, n + 2)} f^y(x) \, dx
	= \int_{[n, n + 1)} f^y(x) \, dx + \int_{[n + 1, n + 2)} f^y(x) \, dx
\]
\[
	= a_n - a_{n - 1} = b_n,
\]
thus \( f^y \) is integrable.
Additionally, since \( [0, \infty) \) is the union of the disjoint intervals
\( [0, 1), [1, 2), [2, 3), \dots \) we have by additivity that
\[
	\int_{(0,\infty)} \int f^y(x) \, dx \, dy
	= \int_{[0,\infty)} \int f^y(x) \, dx \, dy
	= \sum_{k \in \N_0} \int_{[k, k + 1)} \int f^y(x) \, dx \, dy
\]
\[
	= a_0 + \lim_{n \to \infty} \sum_{1 \leq k \leq n}
	\int_{[k, k + 1)} a_{\floor{y}} - a_{\floor{y - 1}} \, dy
	= b_0 + \lim_{n \to \infty} \sum_{1 \leq k \leq n} a_k - a_{k - 1}
	= b_0 + \lim_{n \to \infty} \sum_{1 \leq k \leq n} b_k
\]
\[
	= b_0 + \sum_{k \in \N} b_k
	= \sum_{k \in \N_0} b_k,
\]
thus proving the first part of the theorem.

Finally, fix \( n \leq x < n + 1 \). If \( n \leq y < n + 2 \), then
\( \vertb{f_x(y)} = a_n \), otherwise \( \vertb{f_x(y)} = 0 \).
Additionally, since \( \braceb{b_n} \) is a positive sequence, we have that
\( n_1 < n_2 \) implies \( a_{n_1} < a_{n_2} \), thus using
\textbf{Corollary 10} we find that
\[
	\iint \vertb{f(x,y)} \, dy \, dx
	= \iint \vertb{f_x(y)} \, dy \, dx
	= \int \sum_{k \in \N_0} \int_{[k, k + 2)} a_k \, dy \, dx
	= \int \sum_{k \in \N_0} 2a_k \, dx
\]
\[
	= \sum_{k \in \N_0} \int 2a_k dx.
\]
The integral of a positive constant over \( \R \) is \( \infty \), which
shows that
\[
	\iint \vertb{f(x,y)} \, dy \, dx
	= \sum_{k \in \N_0} \int 2a_k dx
	= \infty,
\]
thus completing the proof.
\done

\end{document}

