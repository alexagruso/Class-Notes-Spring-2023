\documentclass[12pt]{article}

\usepackage{amsfonts}
\usepackage{amsmath}
\usepackage{amssymb}
\usepackage{cite}
\usepackage{fancyhdr}
\usepackage[headheight=1in,margin=1.25in]{geometry}
\usepackage[colorlinks=true,linkcolor=blue]{hyperref}
\usepackage{mathtools}
\usepackage{setspace}

\newcommand{\N}{\ensuremath{\mathbb{N}}}
\newcommand{\Z}{\ensuremath{\mathbb{Z}}}
\newcommand{\Q}{\ensuremath{\mathbb{Q}}}
\newcommand{\R}{\ensuremath{\mathbb{R}}}
\newcommand{\C}{\ensuremath{\mathbb{C}}}

\newcommand{\e}{\ensuremath{\varepsilon}}
\renewcommand{\d}{\ensuremath{\delta}}

\newcommand{\angleb}[1]{\left\langle#1\right\rangle}
\newcommand{\braceb}[1]{\left\{#1\right\}}
\newcommand{\bracketb}[1]{\left[#1\right]}
\newcommand{\parenb}[1]{\left(#1\right)}
\newcommand{\vertb}[1]{\left\vert#1\right\vert}

\newcommand{\comp}{\complement}
\newcommand{\sdiff}{\setminus}

\newcommand{\proof}{\textit{Proof: }}
\newcommand{\done}{\ensuremath{\strut\hfill\blacksquare}}

\newcommand{\mc}[1]{\ensuremath{\mathcal{#1}}}

\renewcommand{\t}[1]{\text{ #1 }}

\bibliographystyle{plain}

\begin{document}

\pagestyle{fancy}
\fancyhead[L]{Analysis}
\fancyhead[C]{Alex Agruso}
\fancyhead[R]{Homework 8}

\setlength{\parindent}{0in}
\setlength{\parskip}{0.1in}
\setstretch{1}

\textbf{6)}
There exists a positive continuous function \( f : \R \to \R \) where
\( f \) is integrable and \( \limsup_{x \to \infty} f(x) = \infty \).
However, if \( f \) is uniformly continuous and integrable, then
\( \lim_{\vertb{x} \to \infty} f(x) = 0 \).

\proof
Let \( g(x) = \sum_{k \in \N} k \chi_{I_k}(x) \), where
\( I_k = [k, k + 1/k^3) \).
Since the \( I_k \) are disjoint, we have that
\[
	\int g(x) \, dx
	= \sum_{k \in \N} k \cdot m(I_k)
	= \sum_{k \in \N} \frac{1}{k^2}
	< \infty,
\]
thus \( g \) is integrable.
For each \( k \), define the function
\( f_k(x) = \min\braceb{h_k(x), h'_k(x)} \chi_{I_k}(x) \), where
\[
	h_k(x) = 2k^4(x - k)
	\hspace{0.1in}
	\t{and}
	\hspace{0.1in}
	h'_k(x) = -2k^4(x - k - 1/k^3).
\]
\( h_k \) and \( h'_k \) are both lines, and thus continuous.
We can see that \( f_k(k) = h_k(k) = 0 \),
Define \( f(x) = \sum_{k \in \N} f_k(x) \).
Since \( f \) is a sum of continuous functions, it is itself continuous.

\textbf{10)}

\textbf{11)}
If \( f \) is a real-valued function integrable on \( \R^d \), and if
\( \int_E f \geq 0 \) for all measurable sets \( E \subseteq \R^d \), then
\( f \geq 0 \) a.e.
As a corollary, if \( \int_E f = 0 \) for all \( E \), then \( f = 0 \) a.e.

\proof
Suppose the proposition is false, then there exists some \( r > 0 \) where the
set \( E = \braceb{f < -r} \) has non-zero measure.
Since \( f \) is integrable and thus measurable, we have that \( E \) is also
measurable.
From this, we see that
\[
	\int_E f(x) \, dx
	< \int_E -r \, dx
	= -r \cdot m(E)
	< 0,
\]
which is a contradiction since \( E \) is measurable.
Thus, we must have that \( m(E) = 0 \) for all \( r > 0 \), showing that
\( f \geq 0 \) a.e.

As a corollary, suppose \( \int_E f = 0 \) for all \( E \), then \( f \geq 0 \)
a.e.
We also have that \( \int_E -f = -\int_E f = 0 \) for all \( E \), thus showing
that \( -f \geq 0 \) a.e., revealing \( f \leq 0 \) a.e.
Thus, \( 0 \leq f \leq 0 \) a.e., which proves that \( f = 0 \) a.e.
\done

\textbf{12)}

\bibliography{\jobname}{}

\end{document}
