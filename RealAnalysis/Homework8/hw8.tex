\documentclass[12pt]{article}

\usepackage{amsfonts}
\usepackage{amsmath}
\usepackage{amssymb}
\usepackage{booktabs}
\usepackage{cite}
\usepackage{enumerate}
\usepackage{fancyhdr}
\usepackage[headheight=1in,margin=1.25in]{geometry}
\usepackage[colorlinks=true,linkcolor=blue]{hyperref}
\usepackage{mathtools}
\usepackage{setspace}

\newcommand{\N}{\ensuremath{\mathbb{N}}}
\newcommand{\Z}{\ensuremath{\mathbb{Z}}}
\newcommand{\Q}{\ensuremath{\mathbb{Q}}}
\newcommand{\R}{\ensuremath{\mathbb{R}}}
\newcommand{\C}{\ensuremath{\mathbb{C}}}

\newcommand{\e}{\ensuremath{\varepsilon}}
\renewcommand{\d}{\ensuremath{\delta}}

\newcommand{\angleb}[1]{\left\langle#1\right\rangle}
\newcommand{\braceb}[1]{\left\{#1\right\}}
\newcommand{\bracketb}[1]{\left[#1\right]}
\newcommand{\parenb}[1]{\left(#1\right)}
\newcommand{\vertb}[1]{\left\vert#1\right\vert}
\newcommand{\dvertb}[1]{\left\Vert#1\right\Vert}

\newcommand{\comp}{\complement}
\newcommand{\sdiff}{\setminus}

\newcommand{\proof}{\textit{Proof: }}
\newcommand{\partialdone}{\ensuremath{\strut\hfill\blacktriangle}}
\newcommand{\done}{\ensuremath{\strut\hfill\blacksquare}}

\newcommand{\mc}[1]{\ensuremath{\mathcal{#1}}}

\renewcommand{\t}[1]{\text{ #1 }}
\newcommand{\impl}{\ensuremath{\implies}}
\newcommand{\sectionskip}{\vspace{0.15in}}

\bibliographystyle{plain}

\begin{document}

\pagestyle{fancy}
\fancyhead[L]{Analysis}
\fancyhead[C]{Alex Agruso}
\fancyhead[R]{Homework 8}

\setlength{\parindent}{0in}
\setlength{\parskip}{0.1in}
\setstretch{1}

\textbf{6)}
There exists a non-negative continuous function \( f : \R \to \R \) that is
integrable, but where \( \limsup_{x \to \infty} f(x) = \infty \).
However, if \( f \) is uniformly continuous and integrable, then
\( \lim_{\vertb{x} \to \infty} f(x) = 0 \).

\proof
Let \( g(x) = \sum_{k \in \N} k \chi_{I_k}(x) \) where
\( I_k = [k, k + 1/k^3) \).
By \textbf{Corollary 10}, we have that
\[
	\int g(x) \, dx
	= \int \sum_{k \in \N} k\chi_{I_k}(x) \, dx
	= \sum_{k \in \N} \int k\chi_{I_k}(x) \, dx
	= \sum_{k \in \N} k \cdot m(I_k)
	= \sum_{k \in \N} \frac{1}{k^2}
	< \infty,
\]
thus \( g \) is integrable.
For each \( k \), define the function
\( g_k(x) = \min\braceb{h_k(x), h'_k(x)} \chi_{I_k}(x) \), where
\[
	h_k(x) = 2k^4(x - k)
	\hspace{0.1in}
	\t{and}
	\hspace{0.1in}
	h'_k(x) = -2k^4(x - k - 1/k^3).
\]
\( h_k \) and \( h'_k \) are both lines, and thus continuous.
Additionally, the minimum of two continuous functions is continuous, thus
\( g_k \) is continuous on \( I_k \).
We also have that \( g_k(k) = g_k(k + 1/k^3) = 0 \), and since \( g_k \) is
clearly zero outside of \( I_k \), we have that \( g_k \) is continuous
everywhere.

We will now show that \( g_k(x) \leq g(x) \) for all \( x \) and \( k \).
If \( k \leq x < k + 1/2k^3 \), then
\[
	h_k(x) < 2k^4(k + 1/2k^3 - k) = 2k^4/2k^3 = k = g(x) < h'_k(x),
\]
and if \( k + 1/2k^3 \leq x < k + 1/k^3 \), we have
\[
	h'_k(x) \leq -2k^4(k + 1/2k^3 - k - 1/k^3) = -2k^4/(-2k^3) = k = g(x)
	\leq h_k(x),
\]
and since \( g(x) = g_k(x) = 0 \) for \( x \) not contained in any \( I_k \),
we have that \( g_k(x) \leq g(x) \) everywhere.

Now, define the function \( f \) as
\[
	f(x) = \sum_{k \in \N} g_k(x)
\]
The \( I_k \) are disjoint, and so if \( x \in [k, k + 1/k^3) \) for some
\( k \), then \( f(x) = g_k(x) \) and \( f(x) = g_m(x) = 0 \) for all
\( m \ne k \).
This combined with the fact that \( g_k(x) \) is continuous everwhere for
all \( k \) shows that \( f \) is continuous.
We also have for all \( x \) that \( f(x) \leq g_k(x) \leq g(x) \) for some
\( k \), and thus by monotonicity \( f \) is integrable.
Finally, since \( f(k + 1/2k^3) = k \) for arbitrary \( k \in \N \), we have
that \( \limsup_{x \to \infty} f(x) = \infty \), thus completing the first part
of the proof.
\partialdone

Next, restrict \( f \) to being uniformly continuous and integrable, and for
\( k \in \N_0 \), define the intervals \( I_k = [k, k + 1] \).
The uniform continuity of \( f \) guarantees a minimum value over all compact
sets, thus we can define \( x_k = \min_{x \in I_k} f(x) \).
Additionally, since the inervals \( I_k \) are almost disjoint, we have that
\[
	\sum_{k \in \N_0} x_k
	= \sum_{k \in \N_0} x_km(I_k)
	= \sum_{k \in \N_0} \int_{I_k} x_k \, dx
	\leq \sum_{k \in \N_0} \int_{I_k} f(x) \, dx
	\leq \int f(x) \, dx
	< \infty,
\]
thus the sum \( \sum_{k \in \N_0} x_n \) must be finite.
Suppose, for the sake of contradiction, that
\( \lim_{x \to \infty} f(x) = L > 0 \), then for any \( 0 < \e < L \), we can
choose \( N \in \N \) where \( x > N \implies \vertb{f(x) - L} < \e \), thus if
\( x > N \), we have that \( f(x) > L - \e \). But we have that
\[
	\infty
	= \sum_{k > N} L - \e
	\leq \sum_{k > N} x_k
	\leq \sum_{k \in \N_0} x_k
	\leq \int f(x) \, dx,
\]
which is a contradiction.
This forces \( L = 0 \) since \( f \) is non-negative, showing that
\( \lim_{x \to \infty} f(x) = 0 \).
Defining \( g(x) = f(-x) \) and using the immediate result, we can see that
\(
0
= \lim_{x \to \infty} g(x)
= \lim_{x \to -\infty} g(-x)
= \lim_{x \to -\infty} f(x)
\),
thus \( \lim_{\vertb{x} \to \infty} f(x) = 0 \) and the proof is complete.
\done

\sectionskip

\textbf{10)}
Fix a non-negative measurable function \( f \) on \( \R^d \), and for
\( k \in \Z \) define the sets \( E_k \) and \( F_k \) as follows:
\[
	E_k = \braceb{x \in \R^d : 2^k < f(x)}
	\hspace{0.1in}
	\t{and}
	\hspace{0.1in}
	F_k = \braceb{x \in \R^d : 2^k < f(x) \leq 2^{k + 1}},
\]
then the following are equivalent:
\begin{itemize}
	\item[(a)] \( f \) is integrable

	\item[(b)] \( \displaystyle \sum_{k \in \Z} 2^k m(E_k) < \infty \)

	\item[(c)] \( \displaystyle \sum_{k \in \Z} 2^k m(F_k) < \infty \)
\end{itemize}

\proof
We will show that (a) \impl (b) \impl (c) \impl (a).

First, suppose \( f \) is integrable.
Using \textbf{Corollary 10} we can see that
\[
	\sum_{k \in \Z} 2^km(E_k)
	= \sum_{k \in \Z} \int_{E_k} 2^k \, dx
	= \sum_{k \in \Z} \int 2^k\chi_{E_k}(x) \, dx
	= \int \sum_{k \in \Z} 2^k\chi_{E_k}(x) \, dx.
\]
We also have that \( \sum_{k \in \Z} 2^k\chi_{E_k}(x) < 2f(x) \), since if
\( 2^n < f(x) \leq 2^{n + 1} \), then
\[
	\sum_{k \in \Z} 2^k\chi_{E_k}(x)
	= \sum_{k \leq n} 2^k
	= \sum_{k \in \N} 2^{-k} + \sum_{0 \leq k \leq n} 2^k
	= 1 + 2^{n + 1} - 1
	= 2^{n + 1}
	< 2f(x),
\]
which shows that
\[
	\int \sum_{k \in \Z} 2^k\chi_{E_k}(x) \, dx
	< \int 2f(x) \, dx
	= 2 \int f(x) \, dx
	< \infty,
\]
thus proving (a) \impl (b).
Next, assume (b).
Because \( F_k \subset E_k \), we have that
\[
	\sum_{k \in \Z} 2^km(F_k)
	< \sum_{k \in \Z} 2^km(E_k)
	< \infty,
\]
showing that (b) \impl (c).
Finally, assume (c).
Since \( f \) is non-negative, we have that
\( \braceb{f \ne 0} = \bigcup_{k \in \Z} F_k \), and since the \( F_k \) are
disjoint, we have
\[
	\int f(x) \, dx
	= \sum_{k \in \N} \int_{F_k} f(x) \, dx
	\leq \sum_{k \in \N} \int_{F_k} 2^{k + 1} \, dx
	= 2 \sum_{k \in \N} 2^km(F_k)
	< \infty,
\]
thus (c) \impl (a), completing the proof.
\done

\textbf{Corollary)}
Fix \( a \in \R \) and define the functions \( f_a \) and \( g_a \) on
\( \R^d \) as
\[
	f_a(x) = \begin{cases}
		\vertb{x}^{-a} & \t{if} \vertb{x} \leq 1 \\
		0              & \t{otherwise}
	\end{cases}
	\hspace{0.1in}
	\t{and}
	\hspace{0.1in}
	g_a(x) = \begin{cases}
		\vertb{x}^{-a} & \t{if} \vertb{x} > 1 \\
		0              & \t{otherwise}
	\end{cases},
\]
then \( f_a \) is integrable if and only if \( 0 < a < d \), and \( g_a \) is
integrable if and only if \( a > d \).

\proof
Define the sets \( F_k \) with \( f(x) = \vertb{x}^{-a} \) as before.
Solving for \( \vertb{x} \), we find that
\[
	F_k = \braceb{x \in \R^d : 2^{-(k + 1)/a} \leq \vertb{x} < 2^{-k/a}}.
\]
Additionally, \( F_k = 2^{-k/a}F \), where
\( F = \braceb{x \in \R^d : 2^{-1/a} \leq \vertb{x} < 1} \), and thus
\( m(F_k) = (2^{-k/a})^dm(F) \).
This combined with the fact that, for \( f_a \), \( k < 0 \) implies
\( F_k = \varnothing \), shows that
\[
	\sum_{k \in \Z} 2^km(F_k)
	= \sum_{k \in \N_0} 2^km(F_k)
	= m(F) \sum_{k \in \N_0} 2^{k(1 - d/a)},
\]
but this series converges if and only if \( 1 - d/a \) is negative, i.e. when
\( 0 < a < d \), thus by the previous theorem we have established that
\( f_a \) is integrable if and only if \( 0 < a < d \).
We also have that, for \( g_a(x) \), \( k \geq 0 \) implies
\( F_k = \varnothing \), thus
\[
	\sum_{k \in \Z} 2^km(F_k)
	= \sum_{k \in \N} \frac{m(F_{-k})}{2^k}
	= m(F) \sum_{k \in \N} 2^{k(d/a - 1)},
\]
which converges if and only if \( d/a - 1 \) is negative, i.e. when \( a > d \),
and again by the previous theorem we have proven the case of \( g_a \).
\done

\sectionskip

\textbf{11)}
If \( f \) is a real-valued function integrable on \( \R^d \), and if
\( \int_E f \geq 0 \) for all measurable sets \( E \subseteq \R^d \), then
\( f \geq 0 \) a.e.

\proof
Suppose the proposition is false, then there exists some \( r > 0 \) where the
set \( E = \braceb{f < -r} \) has non-zero measure.
Since \( f \) is integrable and thus measurable, we have that \( E \) is also
measurable.
From this, we see that
\[
	\int_E f(x) \, dx
	< \int_E -r \, dx
	= -r \cdot m(E)
	< 0,
\]
which is a contradiction since \( E \) is measurable.
Thus, we must have that \( m(E) = 0 \) for all \( r > 0 \), showing that
\( f \geq 0 \) a.e.
\done

\textbf{Corollary)} If \( \int_E f = 0 \) for all measurable sets
\( E \subseteq \R^d \), then \( f = 0 \) a.e.

\proof
By the previous theorem, we have that \( f \geq 0 \) a.e.
We also have that \( \int_E -f = -\int_E f = 0 \) for all \( E \), showing
that \( -f \geq 0 \) a.e., and thus \( f \leq 0 \) a.e.
This proves that \( 0 \leq f \leq 0 \) a.e., showing \( f = 0 \) a.e.
\done

\sectionskip

\textbf{12)}
There exists an integrable function \( f \) on \( \R^d \) and a sequence of
integrable functions \( \braceb{f_n}_{n \in \N} \) on \( \R^d \) where, as
\( n \to \infty \), \( \dvertb{f - f_n} \to 0 \) but \( f_n \to f \) nowhere.

\proof
For \( k, m \in \N \), define the sets \( I_{k,m} \) as
\[
	I_{k,m}
	= \braceb{x \in \R^d : (m - 1)/2^k \leq \vertb{x} < m/2^k}.
\]
We will consider the sequence of functions \( \braceb{f_n}_{n \in \N} \)
defined as \( f_n = \chi_{I_{k,m}} \), where \( f_1 = \chi_{I_{1,1}} \),
and for \( k \) to increase, \( m \) must traverse from \( 1 \) to \( k2^k \).
For example, the sequence starts like this:
\begin{center}
	\begin{tabular}{cccccccc}
		\toprule
		\( n \) & 1                    & 2 & 3 & 4 & 5 & 6 & 7 \\
		\midrule
		\( f_n \)
		        & \( \chi_{I_{1,1}} \)
		        & \( \chi_{I_{1,2}} \)
		        & \( \chi_{I_{2,1}} \)
		        & \( \chi_{I_{2,2}} \)
		        & \( \chi_{I_{2,3}} \)
		        & \( \chi_{I_{2,4}} \)
		        & \( \chi_{I_{2,5}} \)                         \\
		\bottomrule
	\end{tabular}
\end{center}
\begin{center}
	\begin{tabular}{cccccccc}
		\toprule
		\( n \) & 8                    & 9 & 10 & 11 & 12 & 13 & 14 \\
		\midrule
		\( f_n \)
		        & \( \chi_{I_{2,6}} \)
		        & \( \chi_{I_{2,7}} \)
		        & \( \chi_{I_{2,8}} \)
		        & \( \chi_{I_{3,1}} \)
		        & \( \chi_{I_{3,2}} \)
		        & \( \chi_{I_{3,3}} \)
		        & \( \chi_{I_{3,4}} \)                              \\
		\bottomrule
	\end{tabular}
\end{center}
Now, define the set \( I_r = \braceb{x \in \R^d: \vertb{x} < r} \) if
\( r > 0 \) and \( I_r = \varnothing \) if \( r = 0 \).
Fixing \( k \), we can see that
\[
	I_{k,m} = I_{m/2^k} \sdiff I_{(m - 1)/2^k},
\]
and since \( I_{r_1} \subseteq I_{r_2} \) if \( r_1 \leq r_2 \), we have
that
\[
	m(I_{k,m}) = m(I_{m/2^k}) - m(I_{(m - 1)/2^k})
	= m(I_1)\parenb{\frac{m^d}{2^{kd}} - \frac{(m - 1)^d}{2^{kd}}}
\]
\[
	= m(I_1)\parenb{\frac{m^d - (m - 1)^d}{2^{kd}}}.
\]
Note that the derivative of \( x^d - (x - 1)^d \) w.r.t. \( x \) is
\( d(x^{d - 1} - (x - 1)^{d - 1}) \), which is positive for all
\( x > 0 \), thus \( m^d - (m - 1)^d \) is increasing over the positive
integers.
Using this we see that \( m_1 \leq m_2 \) implies that
\( m(I_{k,m_1}) \leq m(I_{k,m_2}) \), and thus
\[
	\int \chi_{I_{k,m_1}}(x) \, dx
	= m(I_{k,m_1})
	\leq m(I_{k,m_2})
	= \int \chi_{I_{k,m_2}}(x) \, dx.
\]
Additionally, we have that
\begin{equation}
	m(I_{k,k2^k})
	= m(I_1)\parenb{\frac{k^d2^{kd} - (k2^k - 1)^d}{2^{kd}}},
\end{equation}
as well as
\[
	k^d2^{kd} - (k2^k - 1)^d
	= -(d - 1)k^{d - 1}2^{k(d - 1)} + \cdots
	+ (-1)^{d - 1}(d - 1)k2^k + (-1)^d
	\leq M2^{k(d - 1)},
\]
where \( M \) is any constant larger than the sum of the binomial
coefficients in the expansion of \( (k2^k - 1)^d \).
Note that \( M \) is defined independently of \( k \), thus
we can see that (1) is bounded by \( M2^{k(d - 1)}/2^{kd} = M/2^k \),
but this tends to 0 as \( k \to \infty \), and thus (1) tends to 0 as well.
Consequently, we have that
\[
	\lim_{n \to \infty} \int f_n(x) \, dx
	= \lim_{n \to \infty} \int \chi_{I_{\eta_k,\eta_m}}(x) \, dx
	\leq \lim_{n \to \infty} \int \chi_{I_{\eta_k,\eta_k2^{\eta_k}}}(x) \, dx
\]
\[
	= \lim_{k \to \infty} m(I_{k,k2^k})
	= 0,
\]
where \( \eta_k \) and \( \eta_m \) are the corresponding values of \( k \)
and \( m \) for each \( n \) taken in the limit.
Thus, setting \( f(x) = 0 \), we have shown that \( \dvertb{f_n - f} \to 0 \)
as \( n \to \infty \) for some function \( f \).

We will now show that \( \lim_{n \to \infty} f_n(x) \) exists nowhere.
Fix \( x \in \R^d \).
Note that for fixed \( k \), the sets \( I_{k,m} \) are disjoint over
\( 1 \leq m \leq k2^k \), and that
\[
	\bigcup_{1 \leq m \leq k2^k} I_{k,m}
	= \braceb{x \in \R^d : \vertb{x} < k}.
\]
Thus, if \( \vertb{x} = r \), then for each integer \( k > r \), there exists
\( m_k \) where \( x \in I_{k,m_k} \).
This means that \( x  \) is contained in infinitely many of the
\( I_{k,m} \), hence \( f_n(x) = 1 \) for infinitely many \( n \in \N \).
For fixed \( k > r \), we have that \( x \) is contained in one
of \( k2^k \) disjoint sets, and thus for each \( k \) we have \( m'_k \)
where \( x \notin I_{k,m'_k} \), thus showing that \( f_n(x) = 0 \) for
infinitely many \( n \).
This is only possible when \( \lim_{n \to \infty} f_n(x) \) does not exist
at \( x \), thus completing the proof.
\done

\end{document}
