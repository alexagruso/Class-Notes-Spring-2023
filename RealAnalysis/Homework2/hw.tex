\documentclass[12pt]{article}

\usepackage{amsfonts}
\usepackage{amsmath}
\usepackage{amssymb}
\usepackage{cite}
\usepackage{fancyhdr}
\usepackage[headheight=1in,margin=1in]{geometry}
\usepackage[colorlinks=true,linkcolor=blue]{hyperref}
\usepackage{mathtools}

\newcommand{\N}{\mathbb{N}}
\newcommand{\Z}{\mathbb{Z}}
\newcommand{\Q}{\mathbb{Q}}
\newcommand{\R}{\mathbb{R}}
\newcommand{\C}{\mathbb{C}}

\newcommand{\e}{\ensuremath{\varepsilon}}

\newcommand{\angleb}[1]{\left\langle#1\right\rangle}
\newcommand{\braceb}[1]{\left\{#1\right\}}
\newcommand{\bracketb}[1]{\left[#1\right]}
\newcommand{\parenb}[1]{\left(#1\right)}
\newcommand{\vertb}[1]{\left\vert#1\right\vert}

\newcommand{\comp}{\complement}
\newcommand{\sdiff}{\setminus}

\newcommand{\proof}{\textit{Proof: }}
\newcommand{\done}{\ensuremath{\strut\hfill\blacksquare}}

\newcommand{\mc}[1]{\ensuremath{\mathcal{#1}}}

\bibliographystyle{plain}

\begin{document}

\pagestyle{fancy}
\fancyhead[L]{Analysis}
\fancyhead[C]{Alex Agruso}
\fancyhead[R]{Homework 2}

\setlength{\parindent}{0in}
\setlength{\parskip}{0.1in}

\textbf{11)}
The set \mc{S} of all \( x \in [0, 1] \) that can be represented without a 4
in its decimal expansion has measure 0.

\proof
Define \( \mc{S}_n \) as containing all \( x \in [0, 1] \)
where the first \( n \) digits of its decimal representation \( \ne 4 \), thus
\( \mc{S} = \lim_{n \to \infty} m(\mc{S}_n) \).
Since \( 0.4 = 0.3\overline{9} \), we have that
\( \mc{S}_1 = [0, 4/10] \cup [5/10, 1] \), thus \( m(\mc{S}_1) = 9/10 \).
At each \( n \), we eliminate a tenth of the numbers, so we have that
\( m(\mc{S}_n) = \frac{9}{10}m(\mc{S}_{n - 1})\), thus
\( m(\mc{S}_n) = (9/10)^n \), and thus
\( m(\mc{S}) = \lim_{n \to \infty}(9/10)^n = 0 \).
\done

\textbf{16)}
Let \( \braceb{E_k}_{k \in \N} \) be a countable collection of measurable sets
where
\[
	M = \sum_{k \in \N} m(E_k) < \infty
\]
and define
\( E = \braceb{x \in \R^d : x \in E_k \text{ for infinitely many } k} \), then
we have that \( E \) is measurable and \( m(E) = 0 \).

\proof
For all \( x \in E \), we have that \( x \in \bigcup_{k > N} E_k \) for
arbitrary \( N \in \N \), thus \( E \subseteq \bigcup_{k > N} E_k \).
Fix \( \e > 0 \), then for each \( E_k \) we can choose an open set \( O_k \)
where \( E_k \subseteq O_k \) and \( m_*(O_k) < m_*(E_k) + \e / 2^k \).
\cite{PW.Difference}
Define the open set \( O = \bigcup_{k > N} O_k \), then \( E \subseteq O \) and
\[
	m_*(O) = m_*\parenb{\bigcup_{k > N} O_k}
	\leq \sum_{k > N} m_*(O_k)
	< \sum_{k > N} \parenb{m_*(E_k) + \frac{\e}{2^k}}
	< \sum_{k > N} m_*(E_k) + \e,
\]
which implies
\[
	m_*(O \setminus E)
	= m_*(O) - m_*(E)
	< \sum_{k > N} m_*(E_k) + \e.
\]
As \( N \to \infty \), we have that \( \sum_{k > N} m_*(E_k) \to 0 \) since
\( M < \infty \), thus \( m_*(O \setminus E) < \e \) and \( E \) is measurable.
Since \( E \subseteq \bigcup_{k > N} E_k \), we have
\( m_*(E) \leq \sum_{k > N} m_*(E_k) \), but \( \sum_{k > N} m_*(E_k) \to 0 \),
so \( m_*(E) \leq 0 \), thus \( m_*(E) = m(E) = 0 \).
\done

\textbf{25)} Fix \( \e > 0 \) and let \( E \subseteq \R^d \), then the
following are equivalent:
\begin{itemize}
	\item[(1)] There exists an open set \( O \) where \( E \subseteq O \) and
		\( m_*(O \setminus E) < \e \)

	\item[(2)] There exists a closed set \( F \) where \( F \subseteq E \) and
		\( m_*(E \setminus F) < \e \)
\end{itemize}

\proof
By \textbf{theorem 3.4}, we have that \( (1) \implies (2) \).

Now assume (2) holds for \( E \) and let \( F \subseteq E \) be a closed set
where \( m_*(E \setminus F) < \e \).
We have that \( F^\comp \sdiff E^\comp \subseteq E \sdiff F \), thus
\( m_*(F^\comp \sdiff E^\comp) \leq m_*(E \sdiff F) < \e \).
Since \( F^\comp \) is open and \( E^\comp \subseteq F^\comp \), we have that
(1) holds for \( E^\comp \), and thus (2) does as well.
But then we can choose \( E = E^\comp \) and use the same argument to show that
(1) holds for \( E \).
Thus \( (2) \implies (1) \) and we are finished.
\done

\pagebreak
\textbf{26)}
Fix measurable sets \( A \) and \( B \) with finite measure and let \( E \) be
a set where \( A \subseteq E \subseteq B \).
If \( m(A) = m(B) \), then \( E \) is measurable.

\proof
Fix \( \e > 0 \).
We can choose an open set \( O \) where \( B \subseteq O \) and
\( m_*(O) - m_*(B) < \e / 2 \) and a closed set \( F \) where
\( F \subseteq A \) and \( m_*(A) - m_*(F) < \e / 2 \), thus we have
\( m_*(O) - m_*(B) + m_*(A) - m_*(F) = m_*(O) - m_*(F) < \e \).
Since \( A \subseteq E \), we have \( m_*(A) \leq m_*(E) \) and thus
\( m_*(F) \leq m_*(E) \).
From this we have \( m_*(O) - m_*(E) \leq m_*(O) - m_*(F) < \e \),
which shows \( m_*(O \setminus E) < \e \) and thus \( E \) is measurable
\done

\bibliography{\jobname}

\end{document}
