\documentclass[12pt]{article}

\usepackage{amsfonts}
\usepackage{amsmath}
\usepackage{amssymb}
\usepackage{cite}
\usepackage{fancyhdr}
\usepackage[headheight=1in,margin=1.25in]{geometry}
\usepackage[colorlinks=true,linkcolor=blue]{hyperref}
\usepackage{mathtools}
\usepackage{setspace}

\newcommand{\N}{\ensuremath{\mathbb{N}}}
\newcommand{\Z}{\ensuremath{\mathbb{Z}}}
\newcommand{\Q}{\ensuremath{\mathbb{Q}}}
\newcommand{\R}{\ensuremath{\mathbb{R}}}
\newcommand{\C}{\ensuremath{\mathbb{C}}}

\newcommand{\e}{\ensuremath{\varepsilon}}
\renewcommand{\d}{\ensuremath{\delta}}

\newcommand{\angleb}[1]{\left\langle#1\right\rangle}
\newcommand{\braceb}[1]{\left\{#1\right\}}
\newcommand{\bracketb}[1]{\left[#1\right]}
\newcommand{\parenb}[1]{\left(#1\right)}
\newcommand{\vertb}[1]{\left\vert#1\right\vert}

\newcommand{\comp}{\complement}
\newcommand{\sdiff}{\setminus}

\newcommand{\proof}{\textit{Proof: }}
\newcommand{\done}{\ensuremath{\strut\hfill\blacksquare}}

\newcommand{\mc}[1]{\ensuremath{\mathcal{#1}}}

\renewcommand{\t}[1]{\text{ #1 }}
\newcommand{\sectionskip}{\vspace{0.15in}}

\begin{document}

\pagestyle{fancy}
\fancyhead[L]{Analysis}
\fancyhead[C]{Alex Agruso}
\fancyhead[R]{Homework 7}

\setlength{\parindent}{0in}
\setlength{\parskip}{0.1in}
\setstretch{1}

\textbf{Lemma 1)}
Let \( f : \R^d \to [0, \infty] \) be a non-negative measurable function and
\( E \) a subset of \( \R^d \) with measure 0, then
\[
	\int_E f(x) \, dx = 0.
\]

\proof
Since \( f \) is non-negative and measurable, there exists an increasing
sequence of simple functions \( \braceb{\phi_k}_{k \in \N} \) where
\( \phi_k \to f \) as \( k \to \infty \).
By definition we can write each \( \phi_k \) as the finite sum
\( \sum_{i = 1}^n a_i\chi_{E_i} \) for constants \( a_i \) and measurable sets
\( E_i \), thus
\[
	\int_E \phi_k(x) \, dx
	= \int \parenb{\sum_{i = 1}^n a_i\chi_{E_i}(x)} \chi_{E}(x) \, dx
	= m(E) \parenb{\sum_{i = 1}^n a_i\chi_{E_i}(x)}
	= 0
\]
for all \( k \in \N \), showing that \( \int_E \phi_k \to 0 \) as
\( k \to \infty \).
By the monotone convergence theorem, we have that
\( \int_E \phi_k \to \int_E f \), thus \( \int_E f = 0 \) and we are finished.
\done

\sectionskip

\textbf{1, 2)}
For a fixed number \( a \), define the functions \( f_a \) and
\( F_a \) on \( \R^d \) as follows:
\[
	f_a = \begin{cases}
		\vertb{x}^{-a} & \t{if} \vertb{x} \leq 1 \\
		0              & \t{otherwise}
	\end{cases}
	\hspace{0.1in}
	\t{and}
	\hspace{0.1in}
	F_a = \frac{1}{1 + \vertb{x}^a},
\]
then \( f_a \) is integrable if and only if \( 0 < a < d \), and \( F_a \) is
integrable if and only if \( a > d \).

\proof
Using additivity and \textbf{Lemma 1}, we have that
\[
	\int f_a(x) \, dx
	= \int_{\R^d \sdiff \braceb{0}} f_a(x) \, dx
	+ \int_{\braceb{0}} f_a(x) \, dx
	= \int_{\R^d \sdiff \braceb{0}} f_a(x) \, dx + 0
	= \int_{\R^d \sdiff \braceb{0}} f_a(x) \, dx,
\]
and so it suffices to show that \( f_a \) is integrable over
\( \R^d \sdiff \braceb{0} \).
The same argument shows that this is also true for \( F_a \).

First, we consider \( f_a \).
For each \( k \in \N_0 \), define the disjoint sets \( E_k \) as
\[
	E_k = \braceb{x \in \R^d : 2^k \leq f_a(x) < 2^{k + 1}},
\]
then solving for \( \vertb{x} \) we see that
\[
	E_k = \braceb{x \in \R^d : 2^{-(k + 1)/a} < \vertb{x} \leq 2^{-k/a}}.
\]
Additionally, if we define the set
\( E = \braceb{x \in \R^d : 2^{-1/a} < \vertb{x} < 1} \), we have that
\( E_k = 2^{-k/a}E \), and thus by relative-dilation invariance we have
\( m(E_k) = (2^{-k/a})^d m(E) \).
Since \( f_a(x) > 0 \) implies that \( f_a(x) \geq 1 \) for all \( x \), we
see that
\[
	\int f_a(x) \, dx
	= \sum_{k \in \N_0} \int_{E_k} f_a(x) \, dx
	< \sum_{k \in \N_0} \int_{E_k} 2^{k + 1} \, dx
	= \sum_{k \in \N_0} 2^{k + 1}m(E_k)
\]
\[
	= 2m(E) \sum_{k \in \N_0} 2^{k(1 - d/a)},
\]
but this series only converges when \( 1 - d/a \) is negative, i.e. when
\( 0 < a < d \).
Thus, our assumption on the value of \( a \) implies that \( f_a \) is
integrable.

Conversely, assume \( f_a \) is integrable.
We just established that
\begin{equation}
	\sum_{k \in \N_0} 2^{k + 1}m(E_k)
\end{equation}
only converges when \( 0 < a < d \), but
\[
	\sum_{k \in \N_0} 2^{k + 1}m(E_k)
	= 2\sum_{k \in \N_0} 2^{k}m(E_k)
	\leq 2\sum_{k \in \N_0} \int_{E_k} f_a(x) \, dx
	= 2\int f_a(x) \, dx,
\]
thus (1) converges, which forces \( 0 < a < d \) and proves the case of
\( f_a \).

We now turn our attention to \( F_a \).
Redefine \( E_k \) and \( E \) as
\[
	E_k = \braceb{x \in \R^d : 2^{-(k + 1)} \leq F_a(x) < 2^{-k}}
	\hspace{0.1in}
	\t{and}
	\hspace{0.1in}
	E = \braceb{x \in \R^d : \vertb{x} \leq 1}.
\]
Since \( 0 < F_a(x) < 1 \) for all \( \vertb{x} > 0 \), we have that
\( E_k = \varnothing \) for all \( k > 0 \).
Additionally, solving for \( \vertb{x} \) we see that
\[
	E_k = \braceb{
		x \in \R^d
		: \sqrt[a]{2^k - 1}
		< \vertb{x}
		\leq \sqrt[a]{2^{k + 1} - 1}
	},
\]
thus
\(
m(E_k)
\leq (\sqrt[a]{2^{k + 1} - 1})^dm(E)
< (\sqrt[a]{2^{k + 1}})^dm(E) \).
This shows that
\[
	\frac{1}{2} \int F_a(x) \, dx
	= \frac{1}{2} \sum_{k \in \N_0} \int_{E_k} F_a(x) \, dx
	< \frac{1}{2} \sum_{k \in \N_0} \frac{m(E_k)}{2^k}
	< m(E)\sum_{k \in \N_0} \frac{2^{d(k + 1)/a}}{2^{k + 1}}
\]
\[
	= m(E)\sum_{k \in \N_0} 2^{(k + 1)(d/a - 1)},
\]
which converges only when \( d/a - 1 \) is negative, and thus when
\( a > d \).
Thus we have shown that \( a > d \) implies that \( F_a \) is integrable.

\pagebreak
\textbf{3)}
Since \( \eta_1(x) = \min\braceb{f(x), \eta(x)} \), we have for fixed
\( x \in \R^d \) that \( \eta_1(x) = f(x) \) or \( \eta_1(x) = \eta(x) \).
By assumption \( g \) is non-negative, and so in the case where
\( \eta_1(x) = \eta(x) \), we have that
\( \eta_2(x) = \eta(x) - \eta_1(x) = \eta(x) - \eta(x) = 0 \leq g(x) \), thus
showing that \( \eta_2(x) \leq g(x) \).
Otherwise, \( \eta_1(x) = f(x) \).
We have \( \eta(x) \leq f(x) + g(x) \) by definition, thus
\( \eta(x) - f(x) \leq g(x) \), which implies
\( \eta_2(x) = \eta(x) - \eta_1(x) = \eta(x) - f(x) \leq g(x) \) and
\( \eta_2(x) \leq g(x) \).
\done

\sectionskip

\textbf{4)}
Let \( \braceb{E_k}_{k \in \N} \) be a collection of measurable sets in
\( \R^d \) where \( \sum_{k \in \N} m(E_k) < \infty \) and define the set
\( E \) as
\[
	E
	= \braceb{x \in \R^d : x \in E_k \t{for infinitely many} k},
\]
then \( m(E) = 0 \).

\proof
If a point \( x \) belongs to infinitely many \( E_k \), then
\( \chi_{E_k}(x) = 1 \) for infinitely many \( k \), and thus
\( \sum_{k \in \N} \chi_{E_k}(x) = \infty \).
Conversely, if \( \sum_{k \in \N} \chi_{E_k}(x) = \infty \), then
\( \chi_{E_k}(x) = 1 \) for infinitely many \( k \), thus \( x \) is contained
in infinitely many \( E_k \).
By assumption, \( \sum_{k \in \N} m(E_k) < \infty \), thus we have
\[
	\sum_{k \in \N} m(E_k)
	= \sum_{k \in \N} \int \chi_{E_k}
	< \infty,
\]
and thus by \textbf{Corollary 1.10} we have shown that
\( \sum_{k \in \N} \chi_{E_k} < \infty \) for almost all \( x \), proving
\( m(E) = 0 \).
\done

\sectionskip

\textbf{5)}
Fix \( \e > 0 \), then the function \( f : \R^d \to \R \), defined as
\[
	f(x)
	= \begin{cases}
		\frac{1}{\vertb{x}^{d + 1}} & x \ne 0       \\
		0                           & \t{otherwise}
	\end{cases}
\]
is integrable over all \( x \in \R^d \) where \( \vertb{x} \geq \e \).
We also have that
\[
	\int_{\vertb{x} \geq \e} f(x) \, dx
	\leq \frac{C}{\e}
\]
for some constant \( C \).

\proof
For integers \( k \geq 0 \), define the collection of disjoint sets
\[
	A_k = \braceb{x \in \R^d : 2^k\e \leq \vertb{x} < 2^{k + 1}\e},
\]
and let
\( g : \R^d \to \R \) be a function defined as
\[
	g(x)
	= \sum_{k \in \N_0} \frac{1}{\parenb{2^{k}\e}^{d + 1}} \chi_{A_k}(x).
\]
Fix \( x \in \R^d \) where \( \vertb{x} \geq \e \), then for some
\( k \in \N_0 \) we have that \( x \in A_k \), thus \( \vertb{x} \geq 2^k\e \).
Since the \( A_k \) are disjoint, we see that
\[
	g(x)
	= \frac{1}{\parenb{2^{k}\e}^{d + 1}} \chi_{A_k}(x)
	= \frac{1}{(2^k\e)^{d + 1}}
	\geq \frac{1}{\vertb{x}^{d + 1}}
	= f(x),
\]
showing that \( f(x) \leq g(x) \).
By monotonicity we have
\( \int_{\vertb{x} \geq \e} f \leq \int_{\vertb{x} \geq \e} g \), and so it
suffices to show that \( \int_{\vertb{x} \geq \e} g < \infty \).

Denote \( A = \braceb{x \in \R^d : 1 \leq \vertb{x} < 2} \), then we have that
\( A_k = 2^k\e A \), and thus by relative-dilation invariance we have
\( m(A_k) = (2^k\e)^d m(A) \).
Since \( g \) is a sum of simple functions, we have by \textbf{Corollary 10}
that
\[
	\int_{\vertb{x} \geq \e} g(x) \, dx
	= \int_{\vertb{x} \geq \e} \sum_{k \in \N_0}
	\frac{1}{\parenb{2^k\e}^{d + 1}} \chi_{A_k}(x) \, dx
	= \sum_{k \in \N_0} \int_{\vertb{x} \geq \e}
	\frac{1}{\parenb{2^k\e}^{d + 1}} \chi_{A_k}(x) \, dx
\]
\[
	= \sum_{k \in \N_0} \frac{m(A_k)}{\parenb{2^k\e}^{d + 1}}
	= m(A) \sum_{k \in \N_0} \frac{
		\parenb{2^k\e}^{d}
	}{
		\parenb{2^k\e}^{d + 1}
	}
	= m(A) \sum_{k \in \N_0} \frac{1}{2^k\e}
	= \frac{2m(A)}{\e},
\]
and since \( m(A) \) is finite, we have shown that
\( \int_{\vertb{x} \geq \e} g = C/\e < \infty \) where \( C = 2m(A) \), thus
completing the proof.
\done

\end{document}
