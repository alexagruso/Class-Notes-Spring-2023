\documentclass[12pt]{article}

\usepackage{amsfonts}
\usepackage{amsmath}
\usepackage{amssymb}
\usepackage{cite}
\usepackage{fancyhdr}
\usepackage[headheight=1in,margin=1.25in]{geometry}
\usepackage[colorlinks=true,linkcolor=blue]{hyperref}
\usepackage{mathtools}
\usepackage{setspace}

\newcommand{\N}{\ensuremath{\mathbb{N}}}
\newcommand{\Z}{\ensuremath{\mathbb{Z}}}
\newcommand{\Q}{\ensuremath{\mathbb{Q}}}
\newcommand{\R}{\ensuremath{\mathbb{R}}}
\newcommand{\C}{\ensuremath{\mathbb{C}}}

\newcommand{\e}{\ensuremath{\varepsilon}}
\renewcommand{\d}{\ensuremath{\delta}}

\newcommand{\angleb}[1]{\left\langle#1\right\rangle}
\newcommand{\braceb}[1]{\left\{#1\right\}}
\newcommand{\bracketb}[1]{\left[#1\right]}
\newcommand{\parenb}[1]{\left(#1\right)}
\newcommand{\vertb}[1]{\left\vert#1\right\vert}

\newcommand{\comp}{\complement}
\newcommand{\sdiff}{\setminus}

\newcommand{\proof}{\textit{Proof: }}
\newcommand{\done}{\ensuremath{\strut\hfill\blacksquare}}

\newcommand{\mc}[1]{\ensuremath{\mathcal{#1}}}

\renewcommand{\t}[1]{\text{ #1 }}

\bibliographystyle{plain}

\begin{document}

\pagestyle{fancy}
\fancyhead[L]{Analysis}
\fancyhead[C]{Alex Agruso}
\fancyhead[R]{Homework 7}

\setlength{\parindent}{0in}
\setlength{\parskip}{0.1in}
\setstretch{1}

\textbf{Lemma 1:}
Let \( f : \R^d \to [0, \infty] \) be a non-negative measurable function and
\( E \subset \R^d \) a set with measure 0.
Then we have that
\[
	\int_E f(x) \, dx = 0.
\]

\proof
Since \( m(E) = 0 \), we have that \( f\chi_E \) is zero almost everywhere,
and thus by \cite{AlmostEverywhere}, we see
\[
	\int_E f(x) \, dx
	= \int f(x)\chi_E(x) \, dx
	= \int \vertb{f(x)\chi_E(x)} \, dx
	= 0.
\] 
\done

\textbf{Lemma 2:}
Fix a measurable function \( f : \R^d \to \R \) and \( \e > 0 \).
Define a collection of disjoint non-negative intervals
\( \braceb{I_k}_{k \in \N_0} \) where \( \sup I_k \leq \inf I_{k + 1} \), and
denote \( n_k := \inf I_k \) and \( n'_k := \sup I_k \).
Also define \( P_k := f^{-1}(I_k) \) and \( P := \bigcup_{k \in \N_0} P_k \).
If \( \braceb{m(P_k)}_{k \in \N_0} \) is a monotone sequence, then we have that
\( \sum_{k \in \N_0} n_k m(P_k) < \infty \) if and only if \( f \) is
integrable on \( P \).

\proof
Assume the former.
Since the \( P_k \) are disjoint, we can see that
\[
	\int_{P} f(x) \, dx
	= \sum_{k \in \N_0} \int_{P_k} f(x) \, dx
	\leq \sum_{k \in \N_0} n_k m(P_k)
	< \infty,
\]
showing that \( f \) is integrable over \( P \).
Conversely, assume \( f \) is integrable over \( P \).
In the case where \( m(P_k) \) is non-decreasing, we have that
\[
	\infty
	> \int_P f(x) \, dx
	\geq \sum_{k \in \N_0} n'_k m(P_k)
	\geq \sum_{k \in \N} n'_k m(P_k)
	\geq \sum_{k \in \N} n_{k - 1} m(P_{k - 1})
	= \sum_{k \in \N_0}
	
	% \sum_{k \in \N_0} (2^k\e)m(E_k)
	% \leq \sum_{k \in \N_0} \int_{E_k} f(x) \, dx
	% = \int_{\vertb{x} \geq \e} f(x) \, dx
	% < \infty,
\]
and thus
\[
	2\sum_{k \in \N_0} (2^k\e)m(E_k)
	= \sum_{k \in \N_0} (2^{k + 1}\e)m(E_k)
	< \infty,
\]
completing the proof
\done

\textbf{1, 2)}
Define the functions \( f_a \) and \( F_a \) on \( \R^d \) as
\[
	f_a := \begin{cases}
		\vertb{x}^{-a} & \vertb{x} \leq 1 \\
		0              & \t{otherwise}
	\end{cases}
	\hspace{0.2in}
	\t{and}
	\hspace{0.2in}
	F_a := \frac{1}{1 + \vertb{x}^a},
\]
then \( f_a \) is integrable if and only if \( 0 < a < d \), and \( f_a \) is
integrable if and only if \( a > d \).

\proof
We have that both \( f_a \) and \( F_a \) are non-negative and measurable,
thus by additivity and \textbf{Lemma 1},
\[
	\int f_a
	= \int_{\R^d \sdiff \braceb{0}} f_a + \int_{\braceb{0}} f_a
	= \int_{\R^d \sdiff \braceb{0}} f_a + 0
	= \int_{\R^d \sdiff \braceb{0}} f_a,
\]
and similarly \( \int F_a = \int_{\R^d \sdiff \braceb{0}} F_a \).
Thus, it suffices to show that these functions are integrable outside any open
ball centered at the origin.

We first consider \( f_a \) and assume \( a \ne 0 \).
Fix \( \e > 0 \), and for integers \( k \geq 0 \), define the collection of sets
\[
	I_k = \braceb{x \in \R^d : 2^k\e \leq f_a(x) < 2^{k + 1}\e}.
\]
Solving for \( \vertb{x} \), we see that
\[
	I_k = \braceb{x : (2^{k + 1}\e)^{-1/a} < \vertb{x} \leq (2^k\e)^{-1/a}},
\]
and denoting \( A = \braceb{x : 1 < \vertb{x} \leq 2^{1/\vertb{a}}} \), we have
by relative-dilation invariance that
\[
	m(I_k) = (2^{k + 1}\e)^{-d/a} m(A),
\]
thus
\[
	\sum_{k \in \N_0} (2^{k + 1}\e) m(I_k)
	= \sum_{k \in \N_0} (2^{k + 1}\e)^{-(d/a) + 1}m(A)
	= \frac{m(A)\e^{1 - (d/a)}}{2} \sum_{k \in \N} (2^k)^{1 - (d/a)},
\]
but the rightmost series only converges when \( 0 < a < d \).
Thus, by \textbf{Lemma 2}, we have that
\( \int_{\vertb{x} \geq \e} f_a(x) \, dx \) is bounded by the sum, and thus is
integrable when \( 0 < a < d \).
Conversely, suppose \( f_a \) is integrable, then again by \textbf{Lemma 2},
\[
	\sum_{k \in \N_0} (2^k\e) m(I_k)
	= \frac{1}{2} \sum_{k \in \N_0} (2^{k + 1}\e) m(I_k)
	= \frac{m(A)\e^{1 - (d/a)}}{4} \sum_{k \in \N} (2^k)^{1 - (d/a)}
	< \infty,
\]
which we previously established to be true only when \( 0 < a < d \), thus the
case for \( f_a \) has been proven.

We now turn our attention to \( F_a \).
Let \( g(x) = x/(1 + x) \) and define \( h(x) = g(x + r) \) where
\( r = \e/(1 - \e) \).
We can see that \( h(0) = \e \) and that \( h(x) \to 1 \) as \( x \to \infty \).
Fix \( 0 < \e < 1 \) and define for each \( k \in \N_0 \) the set
\[
	J_k
	= \braceb{x \in \R^d : h(k)\e \leq F_a(x) < h(k + 1)\e}
\]


\textbf{3)}
Since \( \eta_1(x) = \min\braceb{f(x), \eta(x)} \), we have for fixed
\( x \in \R^d \) that \( \eta_1(x) = f(x) \) or \( \eta_1(x) = \eta(x) \).
By assumption \( g \) is non-negative, and so in the case where
\( \eta_1(x) = \eta(x) \), we have that
\( \eta_2(x) = \eta(x) - \eta_1(x) = \eta(x) - \eta(x) = 0 \leq g(x) \), thus
showing that \( \eta_2(x) \leq g(x) \).
Otherwise, \( \eta_1(x) = f(x) \).
We have \( \eta \leq f + g \) by definition, thus \( \eta - f \leq g \), which
implies \( \eta_2(x) = \eta(x) - \eta_1(x) = \eta(x) - f(x) \leq g(x) \) and
\( \eta_2(x) \leq g(x) \).
\done

\textbf{4)}
Let \( \braceb{E_k}_{k \in \N} \) be a collection of measurable sets in
\( \R^d \) where \( \sum_{k \in \N} m(E_k) < \infty \) and define the set
\( E \) as
\[
	E
	= \braceb{x \in \R^d : x \in E_k \t{for infinitely many} k},
\]
then \( m(E) = 0 \).

\proof
If a point \( x \) belongs to infinitely many \( E_k \), then
\( \chi_{E_k}(x) = 1 \) for infinitely many \( k \), and thus
\( \sum_{k \in \N} \chi_{E_k}(x) = \infty \).
Conversely, if \( \sum_{k \in \N} \chi_{E_k}(x) = \infty \), then
\( \chi_{E_k}(x) = 1 \) for infinitely many \( k \), thus \( x \) is contained
in infinitely many \( E_k \).
By assumption, \( \sum_{k \in \N} m(E_k) < \infty \), thus we have
\[
	\sum_{k \in \N} m(E_k)
	= \sum_{k \in \N} \int \chi_{E_k}
	< \infty,
\]
and thus by \textbf{Corollary 1.10} we have shown that
\( \sum_{k \in \N} \chi_{E_k} < \infty \) for almost all \( x \), proving
\( m(E) = 0 \).
\done

\textbf{5)}
Fix \( \e > 0 \), then the function \( f : \R^d \to \R \), defined as
\[
	f(x)
	= \begin{cases}
		\frac{1}{\vertb{x}^{d + 1}} & x \ne 0 \\
		0                           & \t{otherwise}
	\end{cases}
\]
is integrable over all \( x \in \R^d \) where \( \vertb{x} \geq \e \).
We also have that
\[
	\int_{\vertb{x} \geq \e} f(x) \, dx
	\leq \frac{C}{\e}
\]
for some constant \( C \).

\proof
For integers \( k \geq 0 \), define the collection of disjoint sets
\[
	A_k = \braceb{x \in \R^d : 2^k\e \leq \vertb{x} < 2^{k + 1}\e},
\]
and let
\( g : \R^d \to \R \) be a function defined as
\[
	g(x)
	= \sum_{k \in \N_0} \frac{1}{\parenb{2^{k}\e}^{d + 1}} \chi_{A_k}(x).
\]
If \( \vertb{x} \geq \e \), then \( x \) is contained in \( A_k \) for some
\( k \), thus \( \vertb{x} \geq 2^k\e \) and
\[
	g(x)
	= \frac{1}{\parenb{2^{k}\e}^{d + 1}} \chi_{A_k}(x)
	= \frac{1}{(2^k\e)^{d + 1}}
	\geq \frac{1}{\vertb{x}^{d + 1}}
	= f(x),
\]
showing that \( f(x) \leq g(x) \).
By monotonicity, we have
\( \int_{\vertb{x} \geq \e} f \leq \int_{\vertb{x} \geq \e} g \).
It now suffices to show \( \int_{\vertb{x} \geq \e} g < \infty \).

Denote \( A = \braceb{x \in \R^d : 1 \leq \vertb{x} < 2} \), we have that
\( A_k = 2^k\e A \), and thus by relative-dilation invarance we have
\( m(A_k) = (2^k\e)^d m(A) \).
By \textbf{Corollary 10}, we have
\[
	\int_{\vertb{x} \geq \e} g(x) \, dx
	= \int_{\vertb{x} \geq \e} \sum_{k \in \N_0}
	\frac{1}{\parenb{2^k\e}^{d + 1}} \chi_{A_k}(x) \, dx
	= \sum_{k \in \N_0} \int_{\vertb{x} \geq \e}
	\frac{1}{\parenb{2^k\e}^{d + 1}} \chi_{A_k}(x) \, dx
\]
\[
	= \sum_{k \in \N_0} \frac{m(A_k)}{\parenb{2^k\e}^{d + 1}}
	= m(A) \sum_{k \in \N_0} \frac{
		\parenb{2^k\e}^{d}
	}{
		\parenb{2^k\e}^{d + 1}
	}
	= m(A) \sum_{k \in \N_0} \frac{1}{2^k\e}
	= \frac{2m(A)}{\e},
\]
and since \( m(A) \) is finite, we have shown that
\( \int_{\vertb{x} \geq \e} g = C/\e < \infty \) where \( C = 2m(A) \), thus
completing the proof.
\done

\bibliography{\jobname}{}

\end{document}
