\documentclass[12pt]{article}

\usepackage{amsfonts}
\usepackage{amsmath}
\usepackage{amssymb}
\usepackage{cite}
\usepackage{fancyhdr}
\usepackage[headheight=1in,margin=0.9in]{geometry}
\usepackage[colorlinks=true,linkcolor=blue]{hyperref}
\usepackage{mathtools}

\newcommand{\N}{\ensuremath{\mathbb{N}}}
\newcommand{\Z}{\ensuremath{\mathbb{Z}}}
\newcommand{\Q}{\ensuremath{\mathbb{Q}}}
\newcommand{\R}{\ensuremath{\mathbb{R}}}
\newcommand{\C}{\ensuremath{\mathbb{C}}}

\newcommand{\e}{\ensuremath{\varepsilon}}
\renewcommand{\d}{\ensuremath{\delta}}

\newcommand{\angleb}[1]{\left\langle#1\right\rangle}
\newcommand{\braceb}[1]{\left\{#1\right\}}
\newcommand{\bracketb}[1]{\left[#1\right]}
\newcommand{\parenb}[1]{\left(#1\right)}
\newcommand{\vertb}[1]{\left\vert#1\right\vert}

\newcommand{\comp}{\complement}
\newcommand{\sdiff}{\setminus}

\newcommand{\proof}{\textit{Proof: }}
\newcommand{\done}{\ensuremath{\strut\hfill\blacksquare}}

\newcommand{\mc}[1]{\ensuremath{\mathcal{#1}}}

\renewcommand{\t}[1]{\text{ #1 }}

\bibliographystyle{plain}

\begin{document}

\pagestyle{fancy}
\fancyhead[L]{Analysis}
\fancyhead[C]{Alex Agruso}
\fancyhead[R]{Homework 5}

\setlength{\parindent}{0in}
\setlength{\parskip}{0.1in}

Denote \( \N_0 \) as the set \( \braceb{n \in \Z : n \geq 0 } \).

\textbf{28)}
Fix \( E \subseteq \R \) where \( m_*(E) > 0 \), then for all
\( 0 < \alpha < 1 \) there exists an open interval \( I \) such that
\( m_*(E \cap I) \geq \alpha m_*(I) \)

\proof
% Fix \( 0 < \alpha < 1 \), 
Fix \( 0 < \alpha < 1 \) and \( 0 < \e < \alpha \).
If \( \braceb{I_n}_{n \in \N} \) is a covering of \( E \) using closed
intervals, we can expand each interval by less than \( \e/2^n \) to obtain a
covering \( \braceb{I'_n}_{n \in \N} \) where
\[
	\sum_{n \in \N} m_*(I'_n) < \sum_{n \in \N} m_*(I_n) + \e,
\]
thus we can let \( \braceb{I_n}_{n \in \N} \) be a covering of \( E \) where
\[
	\sum_{n \in \N} m_*(I_n) < m_*(E) + \e/2\alpha.
\]
Because each \( I_n \) is measurable, we can choose open
intervals \( O_n \) such that \( I_n \subset O_n \) and
\break
\( m_*(O_n \sdiff I_n) < \e/2^{n + 1}\alpha \), thus
\( m_*(O_n) < m_*(I_n) + \e/2^{n + 1}\alpha \).
From this, we can see that
\[
	\sum_{n \in \N} m_*(O_n)
	< \sum_{n \in \N} m_*(I_n) + \e/2^{n + 1}\alpha
	< m_*(E) + \e/2\alpha + \e/2\alpha
	= m_*(E) + \e/\alpha.
\]
To establish a contradiction, suppose we have that
\( m_*(E \cap I) < \alpha m_*(I) \) for all open intervals \( I \), thus
\( m_*(E \cap O_n) < \alpha m_*(O_n) \) for all \( n \).
Since \( \braceb{O_n}_{n \in \N} \) covers \( E \), we have that
\(
	E
	\subseteq \bigcup_{n \in \N} E \cap O_n
	\break
	\subseteq \bigcup_{n \in \N} O_n
\),
thus by our assumption we have
\[
	m_*(E)
	< \sum_{n \in \N} \alpha m_*(O_n)
	< \alpha(m_*(E) + \e/\alpha)
	= \alpha m_*(E) + \e.
\]
Since \( \e \) is arbitrarily small, we find that \( m_*(E) < \alpha m_*(E) \),
and thus \( \alpha > 1 \), but this is a contradiction.
Thus, there must exist an open interval \( I \) where
\( m_*(E \cap I) \geq \alpha m_*(I) \), and we are finished.
\done

\textbf{37)}
Given a continuous function \( f : \R \to \R \), define
\( \Gamma = \braceb{(x,f(x)) : x \in \R} \subset \R^2 \), then we have that
\( m(\Gamma) = 0 \).

\proof
For \( n \in \Z \), let \( I_n = [n,n + 1] \), then
\( \R = \bigcup_{n \in \Z} I_n \).
Fix \( \e > 0 \) and denote \( \e_n \) as \( \e/2^{\vertb{n} + 4} \).
Since \( f \) is continuous on \( \R \), we have that it is uniformly
continuous on all \( I_n \), thus for all \( n \) we can choose
\( 0 < \d_n < 1 \) and \( k_n \in \N \) where \( 1/(k_n + 1) < \d_n < 1/k_n \),
and where \( \vertb{x - y} < \d_n \implies \vertb{f(x) - f(y)} < \e_n \) for
all \( x,y \in I_n \).
We have that \( 1 < (k_n + 1)\d_n < 2 \), thus for all \( n \) there exists a
collection of \( k_n + 1 \) intervals \( I_{n,m} = [x_{n,m}, x_{n,m+1}] \) for
\( 1 \leq m \leq k_n + 1 \) where \( \vertb{I_{n,m}} < \d_n \) and
\( I_n = \bigcup_{1 \leq m \leq k_n + 1} I_{n,m} \).
For any \( (x,f(x)) \in \Gamma \), we have some interval where
\( x \in I_{n} \) and thus \( x \) is in some \( I_{n,m} \).
Since \( \vertb{x - x_{n,m}} < \d_n \), we have
\( \vertb{f(x) - f(x_{n,m})} < \e_n \),
thus the image \( f(I_{n,m}) \) is a
subset of \( [f(x_{n,m}) - \e_n, f(x_{n,m}) + \e_n] \)
and \( (x,f(x)) \in I_{n,m} \times f(I_{n,m}) \).
Denoting
\( \Gamma_n = \bigcup_{1 \leq m \leq k_n + 1} I_{m,n} \times f(I_{m,n}) \),
we find that
\[
	m(\Gamma_n)
	\leq \sum_{1 \leq m \leq k_n + 1} m(I_{n,m} \times f(I_{m,n}))
	< \sum_{1 \leq m \leq k_n + 1} \d_n \cdot 2\e_n
	= (\e/2^{\vertb{n} + 3})(k_n + 1)\d_n
	< \e/2^{\vertb{n} + 2}.
\]
Since \( \Gamma \subseteq \bigcup_{n \in \Z} \Gamma_n \), we have
\[
	m(\Gamma)
	\leq \sum_{n \in \Z} m(\Gamma_n)
	\leq 2\sum_{n \in \N_0} \e/2^{n + 2}
	= 2\frac{\e}{2}
	= \e,
\]
and thus \( m(\Gamma) = 0 \).
\done

% \bibliography{\jobname}{}

\end{document}
