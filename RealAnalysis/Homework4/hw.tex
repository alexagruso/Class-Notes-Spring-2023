\documentclass[12pt]{article}

\usepackage{amsfonts}
\usepackage{amsmath}
\usepackage{amssymb}
\usepackage{cite}
\usepackage{fancyhdr}
\usepackage[headheight=1in,margin=0.9in]{geometry}
\usepackage[colorlinks=true,linkcolor=blue]{hyperref}
\usepackage{mathtools}

\newcommand{\N}{\ensuremath{\mathbb{N}}}
\newcommand{\Z}{\ensuremath{\mathbb{Z}}}
\newcommand{\Q}{\ensuremath{\mathbb{Q}}}
\newcommand{\R}{\ensuremath{\mathbb{R}}}
\newcommand{\C}{\ensuremath{\mathbb{C}}}

\newcommand{\e}{\ensuremath{\varepsilon}}
\renewcommand{\d}{\ensuremath{\delta}}

\newcommand{\angleb}[1]{\left\langle#1\right\rangle}
\newcommand{\braceb}[1]{\left\{#1\right\}}
\newcommand{\bracketb}[1]{\left[#1\right]}
\newcommand{\parenb}[1]{\left(#1\right)}
\newcommand{\vertb}[1]{\left\vert#1\right\vert}

\newcommand{\comp}{\complement}
\newcommand{\sdiff}{\setminus}

\newcommand{\proof}{\textit{Proof: }}
\newcommand{\done}{\ensuremath{\strut\hfill\blacksquare}}

\newcommand{\mc}[1]{\ensuremath{\mathcal{#1}}}

\renewcommand{\t}[1]{\text{ #1 }}

\bibliographystyle{plain}

\begin{document}

\pagestyle{fancy}
\fancyhead[L]{Analysis}
\fancyhead[C]{Alex Agruso}
\fancyhead[R]{Homework 4}

\setlength{\parindent}{0in}
\setlength{\parskip}{0.1in}

\textbf{Property 5.ii)}
If \( f \) and \( g \) are finite valued measurable functions, then \( f + g \)
is measurable.

\proof
We will show that
\[
	\braceb{f + g > a}
	= M
\]
where
\[
	M = \bigcup_{r \in \Q} \braceb{f > a - r} \cap \braceb{g > r}
\]
and conclude that \( \braceb{f + g > a} \) is a countable union of measurable
sets, and thus measurable.
It is clear that \( M \subseteq \braceb{f + g > a} \) since if \( x \in M \),
then \( f(x) > a - r \) and \( g(x) > r \) for some \( r \in \Q \), thus
implying that \( f(x) + g(x) > a \).
Conversely, suppose \( x \in \braceb{f + g > a} \), then
\( f(x) + g(x) > a \) and \( g(x) > a - f(x) \).
We can choose \( r \in \Q \) where \( a - f(x) < r < g(x) \), thus obtaining
\( g(x) > r \) and \( f(x) - a > -r \implies f(x) > a - r \), which shows
that \( x \in M \), thus \( \braceb{f + g > a} = M \) and we are finished.
\done

\textbf{18)}
Every measurable function is the limit a.e. of a sequence of
continuous functions.

\proof
Without loss of generality, we let \( f \) be non-negative, since
\( f = f^+ - f^- \).
We must also assume that \( f \) is finite a.e.
Let \( Q_n \subseteq \R^d \) denote the closed cube with side length \( n \),
and
define the function \( f_n : \R^d \to \R \) as follows:
\[
	f_n(x)
	= \begin{cases}
		f(x) & x \in Q_n \t{and} f(x) \leq n \\
		n    & x \in Q_n \t{and} f(x) > n \\
		0    & x \notin Q_n
	\end{cases}
\]
We previously established that \( f_n \to f \) everywhere \( f \) is finite.
Since \( g(x) = \min\braceb{n,f(x)} \) is measurable for all \( n \)
\cite{MinMeasurable},
and since \( f_n(x) = \min\braceb{n,f(x)}\chi_{[0,1]} \), we have that \( f_n \)
is measurable.
Fix \( \e > 0 \), then since \( f_n \) is finite valued for all \( n \), we can
use \textbf{theorem 4.5} to obtain closed sets \( F_n \subseteq Q_n \) with
\( m(Q_n \sdiff F_n) < \e \) and where the restriction \( f_n |_{F_n} \) is
continuous.
By the \textbf{Tietze extension theorem}
\cite{TietzeExtension}, there exists an extension
\( g_n \) of \( f_n |_{F_n} \) that is continuous on \( \R^d \), and since
\( g_n(x) = f_n(x) \) if \( x \in F_n \), we have that \( g_n \to f \)
everywhere it is finite.
Finally, as \( \e \to 0 \), \( m(Q_n \sdiff F_n) \to 0 \), thus the set of
points where \( f_n \) doesn't converge to \( f \) has measure 0.
\done

\textbf{22)}
There exists no everywhere continuous function \( f : \R \to \R \) where
\( f(x) = \chi_{[0,1]}(x) \) a.e.

\proof
Fix a function \( f : \R \to \R \) with \( E \) denoting the set of points
\( x \in \R \) where \( f(x) \ne \chi_{[0,1]}(x) \), and assume \( m(E) = 0 \).
To establish a contradiction, suppose \( f \) is continuous on \R.

First we consider the case where \( 0 \notin E \), then we have that
\( f(0) = 1 \).
Fix \( 0 < \e < 1 \), then since \( f \) is continuous at \( x = 0 \) we can
choose \( \d > 0 \) where
\( x \in (-\d,\d) \implies f(x) \in (1 - \e, 1 + \e) \).
If \( -\d < x < 0 \), then we have that
\( f(x) > 1 - \e > 0 = \chi_{[0,1]}(x) \), thus if
\( x \in (-\d,0) \) we have \( f(x) \ne \chi_{[0,1]}(x) \).
This implies that \( m(E) \geq m((-\d,0)) > 0 \) which is a contradiction.

Next, consider the case where \( 0 \in E \), then \( f(0) \ne 1 \).
Fix \( 0 < \e < \vertb{1 - f(0)} \) and choose \( \d > 0 \)
where \( x \in (-\d,\d) \implies f(x) \in (f(0) - \e, f(0) + \e) \).
Let \( 0 < x < \d \).
If \( f(0) > 1 \), then \( f(x) > f(0) - \e > 1 = \chi_{[0,1]}(x) \), and if
\( f(0) < 1 \), then \( f(x) < f(0) + \e < 1 = \chi_{[0,1]}(x) \).
Thus, if \( x \in (0, \d) \), then \( f(x) \ne \chi_{[0,1]}(x) \), which again
implies that \( m(E) \geq m((0,\d)) > 0 \), thus obtaining a contradiction.

Since each case leads to a contradiction, we know that \( f \) cannot be
continuous everywhere, and we are finished.
\done

\bibliography{\jobname}{}

\end{document}
