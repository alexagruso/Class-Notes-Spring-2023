\documentclass[12pt]{article}

\usepackage{amsfonts}
\usepackage{amsmath}
\usepackage{amssymb}
\usepackage{cite}
\usepackage{fancyhdr}
\usepackage[headheight=1in,margin=0.9in]{geometry}
\usepackage[colorlinks=true,linkcolor=blue]{hyperref}
\usepackage{mathtools}

\newcommand{\N}{\ensuremath{\mathbb{N}}}
\newcommand{\Z}{\ensuremath{\mathbb{Z}}}
\newcommand{\Q}{\ensuremath{\mathbb{Q}}}
\newcommand{\R}{\ensuremath{\mathbb{R}}}
\newcommand{\C}{\ensuremath{\mathbb{C}}}

\newcommand{\e}{\ensuremath{\varepsilon}}
\renewcommand{\d}{\ensuremath{\delta}}

\newcommand{\angleb}[1]{\left\langle#1\right\rangle}
\newcommand{\braceb}[1]{\left\{#1\right\}}
\newcommand{\bracketb}[1]{\left[#1\right]}
\newcommand{\parenb}[1]{\left(#1\right)}
\newcommand{\vertb}[1]{\left\vert#1\right\vert}

\newcommand{\comp}{\complement}
\newcommand{\sdiff}{\setminus}

\newcommand{\proof}{\textit{Proof: }}
\newcommand{\done}{\ensuremath{\strut\hfill\blacksquare}}

\newcommand{\mc}[1]{\ensuremath{\mathcal{#1}}}

\renewcommand{\t}[1]{\text{ #1 }}

\bibliographystyle{plain}

\begin{document}

\pagestyle{fancy}
\fancyhead[L]{Analysis}
\fancyhead[C]{Alex Agruso}
\fancyhead[R]{Homework 6}

\setlength{\parindent}{0in}
\setlength{\parskip}{0.1in}

If \( F \) is a collection of sets, denote \( \bigcup F \) and \( \bigcap F \)
as \( \bigcup_{E \in F} E \) and \( \bigcap_{E \in F} E \), respectively.

\textbf{1)}
Given sets \( F_1, \dots, F_n \), there exists a collection of
\( N = 2^n - 1 \) disjoint sets \( F^*_1, \dots, F^*_N \) where
\( \bigcup_{k = 1}^n F_k = \bigcup_{k = 1}^{N} F^*_k \).

\proof
Denote \( F = \bigcup_{k = 1}^n F_k \) and let \( F^* \) be the collection of
sets of the form \( F^*_1 \cap \dots \cap F^*_n \) where \( F^*_i \) is either
\( F_i \) or \( F_i^\comp \) and excluding the set
\( F_1^\comp \cap \dots \cap F_n^\comp \).
Note that \( \vertb{F^*} = 2^n - 1 \).
Let \( A \) and \( B \) be distinct sets in \( F^* \), then for some
\( 1 \leq i \leq n \), and after possibly switching \( A \) and \( B \), we
have that \( A \subseteq F_i \) and \( B \subseteq F_i^\comp \), but then
\( A \cap B \subseteq F_i \cap F_i^\comp = \varnothing \), thus \( A \) and
\( B \) are disjoint and \( F^* \) is a collection of disjoint sets.

Fix \( x \in F \), then we can take \( I \) and \( J \) to be sets where
\( I = \braceb{F_i : x \in F_i} \) and
\( J = \braceb{F_i^\comp : x \notin F_i} \), given \( 1 \leq i \leq n \);
we can see that \( x \in \bigcap I \cap \bigcap J \).
Since \( x \in F_i \) and \( x \notin F_i \) are mutually exclusive conditions
where exactly one must be true, we have that each \( F_i \) is contained in
exactly one of \( I \) or \( J \), thus \( I \) and \( J \) are disjoint and
\( \vertb{I} + \vertb{J} = n \).
We must also have that \( I \) is non-empty, since \( x \in F \) implies
\( x \) is contained in some \( F_i \).
Thus, \( \bigcap I \cap \bigcap J \) takes the form of a set in
\( F^* \), showing that \( x \) is contained in some set in \( F^* \) and
\( F \subseteq \bigcup F^* \).
Now, suppose \( x \) is contained in some set in \( F^* \).
By the construction of \( F^* \), we have some \( i \) where \( x \in F_i \),
thus \( x \in F \), showing that \( \bigcup F^* \subseteq F \) and
\( F = \bigcup F^* \).
Since the sets in \( F^* \) are disjoint and \( \vertb{F^*} = 2^n - 1 \), we
are finished.
\done

\end{document}
