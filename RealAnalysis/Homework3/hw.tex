\documentclass[12pt]{article}

\usepackage{amsfonts}
\usepackage{amsmath}
\usepackage{amssymb}
\usepackage{cite}
\usepackage{fancyhdr}
\usepackage[headheight=1in,margin=1in]{geometry}
\usepackage[colorlinks=true,linkcolor=blue]{hyperref}
\usepackage{mathtools}

\newcommand{\N}{\mathbb{N}}
\newcommand{\Z}{\mathbb{Z}}
\newcommand{\Q}{\mathbb{Q}}
\newcommand{\R}{\mathbb{R}}
\newcommand{\C}{\mathbb{C}}

\newcommand{\e}{\ensuremath{\varepsilon}}
\renewcommand{\d}{\ensuremath{\delta}}

\newcommand{\angleb}[1]{\left\langle#1\right\rangle}
\newcommand{\braceb}[1]{\left\{#1\right\}}
\newcommand{\bracketb}[1]{\left[#1\right]}
\newcommand{\parenb}[1]{\left(#1\right)}
\newcommand{\vertb}[1]{\left\vert#1\right\vert}

\newcommand{\comp}{\complement}
\newcommand{\sdiff}{\setminus}

\newcommand{\proof}{\textit{Proof: }}
\newcommand{\done}{\ensuremath{\strut\hfill\blacksquare}}

\newcommand{\mc}[1]{\ensuremath{\mathcal{#1}}}

\bibliographystyle{plain}

\begin{document}

\pagestyle{fancy}
\fancyhead[L]{Analysis}
\fancyhead[C]{Alex Agruso}
\fancyhead[R]{Homework 3}

\setlength{\parindent}{0in}
\setlength{\parskip}{0.1in}

Let \( E \subseteq \R^d \), \( r \in \R \), and
\( \d = (\d_1,\d_2,\dots,\d_d) \in \R^d \), then we define the following sets:

\begin{itemize}
	\item[]
		\(
			rE
			= \braceb{(rx_1,rx_2,\dots,rx_d) : (x_1,x_2,\dots,x_d) \in E}
		\)
	\item[]
		\(
			\d E
			= \braceb{
				(\d_1x_1,\d_2x_2,\dots,\d_dx_d)
				: (x_1,x_2,\dots,x_d) \in E
			}
		\)
\end{itemize}

\textbf{6)} Given an open ball \( B \subset \R^d \) centered at \( x \)
with radius \( r \), we have that \( m(B) = r^dm(B_1) \), where \( B_1 \) is
the unit open ball centered at the origin.

\proof
Let \( x \in B_1 \), then denoting \( 0 \) as the origin we have that
\( d(x,0) < 1 \), and since \( rd(x,0) = d(rx,0) < r \), we see that \( rB_1 \)
is equal to the open ball of radius \( r \) centered at the origin.
From here it is clear that \( B = rB_1 + x \).
By translation invariance, we can see that \( m(rB_1 + x) = m(rB_1) \), and by
the dilation property, we have that \( m(rB_1) = r^dm(B_1) \), thus
\( m(B) = m(rB_1 + x) = r^dm(B_1) \).
\done

\textbf{Lemma 1: } Fix a subset \( A \subset \R \) and \( r > 0 \), then
\( \ell = \inf(A) \) if and only if \( r\ell = \inf(rA) \).

\proof
We have that \( \ell \) is a lower bound of \( A \), thus
\( \ell \leq a \) for all \( a \in A \).
Since \( r > 0 \), we also have that \( r\ell \leq ra \) for all
\( ra \in rA \).
Now, let \( k \) be a lower bound of \( A \), then \( k \leq \ell \) by
definition, thus \( rk \leq r\ell \).
Since \( rk \) is a lower bound of \( rA \), we have that
\( r\ell = \inf(rA) \).

The converse is obtained by taking \( A' = rA \) and \( r' = 1/r \).
\done

\textbf{7)} Let \( \delta = (\delta_1, \delta_2, \dots, \delta_d) \in \R^d \)
with \( \delta_i > 0 \) and fix a measurable set \( E \subseteq \R^d \), then
\( \delta E \) is measurable and
\( m(\delta E) = \delta_1\delta_2 \cdots \delta_d m(E) \).

\proof
Fix a closed cube
\( Q = [n_1,n'_1] \times \cdots \times [n_d,n'_d] \subset \R^d \) and
let \( x = (x_1,\dots,x_d) \in Q \), then \( n_i \leq x_i \leq n'_i \) for
\( 1 \leq i \leq d \).
Since \( \d_i > 0 \), we have that \( \d_in_i \leq \d_ix_i \leq \d_in'_i \),
thus
\(
	\d x
	\in [\d_1n_1,\d_1n'_1] \times \cdots \times [\d_dn_d,\d_dn'_d]
	= \d Q
\),
and thus
\[
	\vertb{\d Q}
	= \prod_{i = 1}^d \vertb{\d_in_i - \d_in'_i}
	= \prod_{i = 1}^d \vertb{\d_i}\vertb{n_i - n'_i}
	= \d_1\cdots\d_d \prod_{i = 1}^d \vertb{n_i - n'_i}
	= \d_1\cdots\d_d \vertb{Q}.
\]
% A corollary of this is that if \( x \in \d^{-1}Q \), then
% \( \d x \in Q \), where \( \d^{-1} = (\d_1^{-1},\dots,\d_d^{-1}) \).
For any collection \mc{Q} of closed cubes, denote \( \delta\mc{Q} \) as the set
\( \braceb{\d Q : Q \in \mc{Q}} \) and denote \( \d^{-1} \) as
\( (\d_1^{-1},\dots,\d_d^{-1}) \in \R^d \).
If \mc{Q} covers \( E \), then we have for all \( x \in E \) that \( x \) is
contained in some cube \( Q \in \mc{Q} \), thus \( \delta x \) is contained in
\( \d Q \) and \( \d\mc{Q} \) covers \( \d E \).
Similarly, since \( E = \d^{-1}\d E \), if \( \mc{Q}' \) covers \( \d E \) then
\( \d^{-1}\mc{Q}' \) covers
\( E \).
Thus by \textbf{Lemma 1} we have
\[
	\d_1\cdots\d_d m_*(E)
	= \d_1\cdots\d_d \inf\sum_{Q \in \mc{Q}} \vertb{Q}
	= \inf\sum_{Q \in \mc{Q}} \d_1\cdots\d_d\vertb{Q}
	= \inf\sum_{Q \in \mc{Q}} \vertb{\d Q}
	\geq m_*(\d E)
\]
and
\[
	m_*(\d E)
	= \inf\sum_{Q \in \mc{Q}'} \vertb{Q}
	= \inf\sum_{Q \in \mc{Q}'} \d_1\cdots\d_d\vertb{\d^{-1}Q}
	= \d_1\cdots\d_d \inf\sum_{Q \in \mc{Q}'} \vertb{\d^{-1}Q}
	\geq \d_1\cdots\d_dm_*(E),
\]
thus we have that \( m_*(\d E) = \d_1\cdots\d_dm_*(E) \) for all
\( E \subseteq \R^d \).
Fix \( \e > 0 \) and an open set \( O \) where \( E \subseteq O \) and
\( m_*(O \sdiff E) < \e/(\d_1\cdots\d_d) \).
Note that \( \d E \subseteq \d O \) and that \( \d O \) is open.
~\cite{DilationOfOpenSet}
We have that
\(
	m_*(\d O \sdiff \d E)
	= m_*(\d(O \sdiff E))
	= \d_1\cdots\d_dm_*(O \sdiff E)
	< \e
\),
thus \( \d E \) is measurable.
\done

\bibliography{\jobname}{}

\end{document}
