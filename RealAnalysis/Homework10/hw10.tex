\documentclass[12pt]{article}

\usepackage{amsfonts}
\usepackage{amsmath}
\usepackage{amssymb}
\usepackage{cite}
\usepackage{enumerate}
\usepackage{fancyhdr}
\usepackage[headheight=1in,margin=1.25in]{geometry}
\usepackage[colorlinks=true,linkcolor=blue]{hyperref}
\usepackage{mathtools}
\usepackage{setspace}

\newcommand{\N}{\ensuremath{\mathbb{N}}}
\newcommand{\Z}{\ensuremath{\mathbb{Z}}}
\newcommand{\Q}{\ensuremath{\mathbb{Q}}}
\newcommand{\R}{\ensuremath{\mathbb{R}}}
\newcommand{\C}{\ensuremath{\mathbb{C}}}

\newcommand{\e}{\ensuremath{\varepsilon}}
\renewcommand{\d}{\ensuremath{\delta}}

\newcommand{\angleb}[1]{\left\langle#1\right\rangle}
\newcommand{\braceb}[1]{\left\{#1\right\}}
\newcommand{\bracketb}[1]{\left[#1\right]}
\newcommand{\parenb}[1]{\left(#1\right)}
\newcommand{\vertb}[1]{\left\vert#1\right\vert}
\newcommand{\dvertb}[1]{\left\Vert#1\right\Vert}
\DeclarePairedDelimiter\floor{\lfloor}{\rfloor}

\newcommand{\comp}{\complement}
\newcommand{\sdiff}{\setminus}

\newcommand{\proof}{\textit{Proof: }}
\newcommand{\partialdone}{\ensuremath{\strut\hfill\blacktriangle}}
\newcommand{\done}{\ensuremath{\strut\hfill\blacksquare}}

\newcommand{\mc}[1]{\ensuremath{\mathcal{#1}}}

\renewcommand{\t}[1]{\text{ #1 }}
\newcommand{\impl}{\ensuremath{\implies}}
\newcommand{\sectionskip}{\vspace{0.15in}}
\newcommand{\tri}{\triangle}

\bibliographystyle{plain}

\begin{document}

\pagestyle{fancy}
\fancyhead[L]{Analysis}
\fancyhead[C]{Alex Agruso}
\fancyhead[R]{Homework 10}

\setlength{\parindent}{0in}
\setlength{\parskip}{0.1in}
\setstretch{1}

For a function \( f \), denote \( T_f \) as the total variation function on
\( f \).
Additionally, denote \( P_f \) and \( N_f \) as the positive and negative
variation functions on \( f \).

\textbf{Lemma 1)}
Let \( f : [a,b] \to \R \) be differentiable, then
\[
	T_f(a,b) = \int_a^b \vertb{f'(x)} \, dx.
\]

\proof
Since \( f \) is differentiable on \( [a,b] \), it is also continuous on
\( [a.b] \), and thus the fundamental theorem of calculus holds for \( f \).
Thus, for a fixed partition \( t_0, \dots, t_n \) of \( [a,b] \), we have that
\[
	\sum_{1 \leq k \leq n} \vertb{f(t_k) - f(t_{k - 1})}
	= \sum_{1 \leq k \leq n} \vertb{\int_{t_{k - 1}}^{t_k} f'(x) \, dx}
	\leq \sum_{1 \leq k \leq n} \int_{t_{k - 1}}^{t_k} \vertb{f'(x)} \, dx
\]
\[
	= \int_a^b \vertb{f'(x)} \, dx.
\]
Using this we prove that
\[
	T_f(a,b)
	= \sup \sum_{1 \leq k \leq n} \vertb{f(t_k) - f(t_{k - 1})}
	\leq \sup \int_a^b \vertb{f'(x)} \, dx
	= \int_a^b \vertb{f'(x)} \, dx,
\]
where the supremum is taken over all partitions of \( [a,b] \).

Next, note that \( f \) has bounded variation on \( [a,b] \) since it is
differentiable.
Thus, \( f \) is the difference of two bounded increasing functions, say
\( f = g - h \).
\textbf{Theorem 3.3} allows us to choose \( g(x) \) and \( h(x) \) as
\( P_f(a,x) + f(a) \) and \( N_f(a,x) \), respectively.
We know that \( f' \) exists for all \( x \in [a,b] \), hence
\( f' = g' - h' \) does too, implying that \( g \) and \( h \) are
differentiable, and thus continuous.
Additionally, \( g \) and \( h \) are increasing functions, thus \( g' \)
and \( h' \) are non-negative.
This allows us to show that
\[
	\int_a^b \vertb{f'(x)} \, dx
	= \int_a^b \vertb{g'(x) - h'(x)} \, dx
	\leq \int_a^b \vertb{g'(x)} \, dx + \int_a^b \vertb{-h'(x)} \, dx
\]
\[
	= \int_a^b g'(x) \, dx + \int_a^b h'(x) \, dx
	= g(b) - g(a) + h(b) - h(a)
\]
\[
	= P_f(a,b) + f(a) - P_f(a,a) - f(a) + N_f(a,b) - N_f(a,a)
	= P_f(a,b) + N_f(a,b)
	= T_f(a,b),
\]
since the variation over an interval of length 0 is always 0.
Thus, we have shown that the integral is bounded by the total

\textbf{11)}
Fix \( a, b > 0 \) and define the function \( f \) on \( [0,1] \) as follows:
\[
	f(x) = \begin{cases}
		x^a\sin\parenb{x^{-b}} & \t{if} 0 < x \leq 1 \\
		0                      & \t{otherwise.}
	\end{cases}
\]
We have that \( f \) is of bounded variation on \( [0,1] \) if and only if
\( a > b \).
Additionally, if \( a = b \), we have that for all \( 0 < \alpha < 1 \) there
exists a function \( g \) that satisfies the Lipschitz condition for
\( \alpha \), i.e. there exists a constant \( A \) where
\[
	\vertb{g(x) - g(y)} \leq A\vertb{x - y}^\alpha
\]
for all \( x, y \in [0,1] \), but has unbounded variation.

\proof
We compute the derivative of \( f \) on \( (0,1] \) as
\[
	f'(x) = ax^{a - 1}\sin(x^{-b}) - bx^{a - b - 1}\cos(x^{-b}),
\]
and by \textbf{Lemma 1} we have that
\[
	T_f(0,1) = \int_0^1 \vertb{f'(x)} \, dx
	= \int_0^1 \vertb{ax^{a - 1}\sin(x^{-b}) - bx^{a - b - 1}\cos(x^{-b})} \, dx
\]
\[
	\leq
	\int_0^1 \vertb{ax^{a - 1}\sin(x^{-b})}
	- \int_0^1 \vertb{bx^{a - b - 1}\cos(x^{-b})} \, dx.
\]
\( \sin(x) \) is bounded by \( 1 \), and \( ax^{a - 1} \) is non-negative if
\( x \) is non-negative, thus the left side is bounded by
\[
	\int_0^1 \vertb{ax^{a - 1}} \, dx
	= \int_0^1 ax^{a - 1} \, dx
	= 1 - 0 = 1,
\]
and thus is finite.
Additionally, we have that the right side is finite if and only if
\( a > b \).
\cite{Integrable}
Thus, \( a > b \) implies that the integral of \( \vertb{f'(x)} \) is finite,
thus \( T_f(0,1) \) is finite as well, and \( f \) has bounded variation on
\( [0,1] \).

Conversely, assume that \( f \) has bounded variation, then, using the reverse
triangle inequality, we find
\[
	T_f(0,1)
	= \int_a^b \vertb{f'(x)} \, dx
	= \int_0^1 \vertb{ax^{a - 1}\sin(x^{-b}) - bx^{a - b - 1}\cos(x^{-b})} \, dx
\]
\[
	\geq \int_0^1
	\vertb{
		\vertb{ax^{a - 1}\sin(x^{-b})} - \vertb{bx^{a - b - 1}\cos(x^{-b})}
	} \, dx,
\]
which, by \cite{Integrable} and the previous observations, is only finite when
\( a > b \), thus completing the proof.
\done

\sectionskip

\textbf{14a)}
If \( f \) is a continuous function on some closed interval \( [a,b] \),
then the upper right Dini derivative \( D^+(f)(x) \) is a measurable function
over \( [a,b] \).

\proof
We have that
\[
	D^+(f)(x)
	= \limsup_{h \to 0^+} \frac{f(x + h) - f(x)}{h}
	= \lim_{h \to 0^+} \sup_{k \in (0,h)} \frac{f(x + k) - f(x)}{k}.
\]
Additionally, since \( f \) is continuous, we have for all
\( x \in [a,b] \) and \( \e > 0 \) that there exists a rational number
\( r \) where \( \vertb{f(x) - f(r)} < \e \).
Thus, it suffices to take the supremum over all rational numbers in
\( (0,h) \).
Since \( \Q \) is countable, we can, for fixed \( h \), index the
rational numbers in \( (0,h) \) over \( \N \); let
\( \braceb{r_n}_{n \in \N} \) be such an indexing.
Also denote \( E_h = (0,h) \cap \Q \).
Continuous functions are continuous under shifting, subtraction, and division,
and thus the function \( f_n \) defined for \( n \in \N \) as
\[
	f_n(x) = \frac{f(x + r_n) - f(x)}{r_n}
\]
is continuous, and thus measurable.
Since we also have
\begin{equation}
	\sup_{k \in (0,h)} \frac{f(x + k) - f(x)}{k}
	= \sup_{k \in E_h} \frac{f(x + k) - f(x)}{k}
	= \sup_{k \in \N} f_k(x),
\end{equation}
we see that the leftmost supremum is equal to the supremum of a collection of
measurable functions, and is thus measurable.
We complete the proof by noting that
\[
	D^+(f)(x)
	= \lim_{h \to 0^+} \sup_{k \in (0,h)} \frac{f(x + k) - f(x)}{k}
	= \lim_{n \to \infty} \sup_{k \in (0,1/n)} \frac{f(x + k) - f(x)}{k}
\]
\[
	= \lim_{n \to \infty} g_n(x),
\]
where \( g_n \) is the function described in (1) with \( h = 1/n \).
This shows that \( D^+(f)(x) \) is the pointwise limit of a sequence of
measurable functions, and thus measurable.
\done

\textbf{Proposition)}
Fix a function \( f \), and assume that \( D^+(g)(x) \leq D_-(g)(x) \)
a.e., where \( g(x) = -f(-x) \).
Then we have that \( D^-(f)(x) \leq D_+(f)(x) \) a.e.

\proof
The limit inferior of a sequence is always bounded by the limit superior,
thus \( D_+(f)(x) \leq D^+(f)(x) \) and \( D_-(f)(x) \leq D^-(f)(x) \).
Combining this with our assumption on \( g \), we find that
\( D_+(g)(x) \leq D^-(g)(x) \) a.e.
We compute \( D_+(g)(x) \) as follows:
\[
	D_+(g)(x)
	= \liminf_{h \to 0^+} \frac{g(x + h) - g(x)}{h}
	= \liminf_{h \to 0^+} -\frac{f(-x - h) - f(-x)}{h}
\]
\[
	= \liminf_{h \to 0^-} -\frac{f(x + h) - f(x)}{h}
	= \limsup_{h \to 0^-} \frac{f(x + h) - f(x)}{h}
	= D^-(f)(x),
\]
and similary for \( D^-(g)(x) \):
\[
	D^-(g)(x)
	= \limsup_{h \to 0^-} \frac{g(x + h) - g(x)}{h}
	= \limsup_{h \to 0^-} -\frac{f(-x - h) - f(-x)}{h}
\]
\[
	= \limsup_{h \to 0^+} -\frac{f(x + h) - f(x)}{h}
	= \liminf_{h \to 0^+} \frac{f(x + h) - f(x)}{h}
	= D_+(f)(x),
\]
thus combined with our previous inequality for \( g \), we find that
\( D^-(f)(x) \leq D_+(f)(x) \) a.e.
\done

\bibliography{\jobname}{}

\end{document}

