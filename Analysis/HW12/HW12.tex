\documentclass[12pt]{article}
\pagenumbering{gobble}
\linespread{1.5}

\usepackage{amsfonts}
\usepackage{amsmath}
\usepackage{amssymb}
\usepackage{array}
\usepackage{fancyhdr}
\usepackage{mathrsfs}
\usepackage{textcomp}
\usepackage[margin=1in,headheight=1in]{geometry}

\newcommand{\contradiction}{
    \ensuremath{{\Rightarrow\mspace{-2mu}\Leftarrow}}
}

\newcommand{\angleb}[1]{\left\langle#1\right\rangle}
\newcommand{\vertb}[1]{\left\vert#1\right\vert}
\newcommand{\bracks}[1]{\left[#1\right]}
\newcommand{\parns}[1]{\left(#1\right)}

\newcommand{\derv}[2]{\dfrac{d#1}{d#2}}
\newcommand{\e}{\varepsilon}
\newcommand{\di}{\,/\,}

\newcommand{\lm}[1]{\displaystyle\lim_{#1}}

\begin{document}
\pagestyle{fancy}
\fancyhead{}
\fancyhead[L]{Alexander Agruso}
\fancyhead[R]{MATH 3380 Homework 12}

\normalsize
\begin{itemize}
    % \item [12.)] \begin{itemize}
    %     \item [a.)] Since $0<\lambda<1$, $0<(1-\lambda)<1$ and $(1-\lambda)y < y$.
    % \end{itemize}

    % \item [29.)] \begin{itemize}
    %     \item [a.)] Since $S$ is bounded below, $x\in S\implies x\geq M$ for some $M\in\mathbb{R}$. Let $l<M$, thus $l<M\leq x$, thus $l<x$ for all $x\in S$, thus $l$ is a lower bound of $S$, thus $\mathscr{L}\ne\varnothing$. Next, let $x\in S$, thus $x\geq l$ for all lower bounds $l$ of $S$, thus $x$ is an upper bound of $\mathscr{L}$, thus $\mathscr{L}$ is bounded above. Q.E.D.

    %     \item [b.)] Let $w=\sup(\mathscr{L})$, thus $w\geq l$ for all $l\in\mathscr{L}$.
    % \end{itemize}

    % \item [33.)] Let $x,y\in\mathbb{R}$, and consider $x$:
    %     \[x=x+0=x-y+y\]
    %     And thus by the triangle inequality:
    %     \[\vertb{x-y+y}\leq\vertb{x-y}+\vertb{y}\]
    %     Manipulating:
    %     \[\vert
    %     x-y+y\vert\leq\vertb{x-y}+\vertb{y}\implies\vertb{x}-\vertb{y}\leq\vertb{x-y}\]
    %     \[\implies\vert x\vert-\vert x\vert-\vert y\vert=-\vert y\vert\leq\vert x\vert-\vert y\vert\leq\vert x-y\vert\]
    %     Thus the inequality holds. Q.E.D.***

    % \item [37.)] ***

    % \item [38.)] Let $u$ be an upper bound of $S$, thus $u\geq x$ for all $x\in S$.

    % \item [45.)] ***

    % \item [47.)] Let $y_n\to0$, thus for all $\e_1>0$, there exists $k\in\mathbb{N}$ where
    % \[n\geq k\implies\vertb{y_n-0}<\e_1\]
    % Now suppose $x_n$ is a bounded sequence, and $x_ny_n\to0$, then for all $\e>0$,
    % \[n\geq k\implies\vertb{x_ny_n-0}<\e\]
    % Manipulating the inequality:
    % \[\vertb{x_ny_n-0}=\vertb{x_n(y_n-0)}=\vertb{x_n}\vertb{y_n-0}<\e\implies\vertb{y_n-0}<\frac{\e}{\vertb{x_n}}\]
    % Let $\e_1=\dfrac{\e}{\vertb{x_n}}$, then
    % \[\vertb{y_n-0}<\frac{\e}{\vertb{x_n}}\implies\vertb{x_n}\vertb{y_n-0}<\e\implies\vertb{x_ny_n-0}<\e\]
    % Thus $n\geq k\implies\vertb{x_ny_n-0}<\e$, thus $x_ny_n\to0$. Q.E.D.

    % \item [48.)] ***

    % \item [49.)] ***

    % \item [50.)] Since $x_n\to A$, then for all $\e_1>0$, there exists $k\in\mathbb{N}$ where
    % \[n\geq k\implies\vertb{x_n-A}<\e_1\]
    % Now suppose $cx_n\to cA$, then for all $\e>0$,
    % \[n\geq k\implies\vertb{cx_n-cA}<\e\]
    % Manipulating the inequality:
    % \[\vertb{cx_n-cA}<\vertb{c(x_n-A)}=\vertb{c}\vertb{x_n-A}<\e\]
    % \[\implies\vertb{x_n-A}<\frac{\e}{\vertb c}\]
    % Let $\e_1=\dfrac{\e}{\vertb c}$, then
    % \[\vertb{x_n-A}<\frac{\e}{\vertb c}\implies\vertb{c}\vertb{x_n-A}<\e\implies\vertb{cx_n-cA}<\e\]
    % Thus $n\geq k\implies\vertb{cx_n-cA}<\e$, thus $cx_n\to cA$. Q.E.D.

    % \item [52.)] Let $M,N\in\mathbb{R}$ where $M\geq x_n$ and $N\geq y_n$ for all $n\in\mathbb{N}$. Consider $z_n=x_n+y_n$. $x_n+y_n\leq M+N$, thus $z_n\leq M+N$ for all $n\in\mathbb{N}$, thus $z_n$ is bounded. Q.E.D.

    % \item [57.)] ***

    % \item [58.)] ***

    % \item [59.)] ***

    % \item [60.)] ***

    % \item [62.)] Assume $x_n\to A$, thus for all $\varepsilon>0$, there exists $k$ such that
    % \[n\geq k\implies\vertb{x_n-A}<\varepsilon\]
    % Since $x_n<M$, we know that
    % \[\vertb{x_n-A}<\vertb{M-A}\]
    % ***

    % \item [63.)] Suppose $\inf(S)\in S$, then when $x_n=\inf(S)$, $x_n\to\inf(S)$. Now, suppose $\inf(S)\notin S$.

    % \item [64.)] ***

    \item [76.)]
    \[
    \begin{tabular}{c|c|c|c|c|c|c|c}
        $i_n$ & & & & & & & \\
        \hline
        $j_n$ & & & X & & X & X \\
        \hline
        $k_n$ & X & & & & & & \\
        \hline
        $l_n$ & & & X & & & & X \\
    \end{tabular}
    \]

    % \item [86.)] \begin{itemize}
    %     \item [a.)] True; Let $n_k=k$, thus $x_k=x_{n_k}$, thus $x\preceq x$.

    %     \item [b.)] False; Let $x_n=(-1)^n$ and $y_n=(-1)^{n+1}$. 

    %     \item [c.)] True; Let ***
    % \end{itemize}

    % \item [87.)] ***

    % \item [88.)] ***

    % \item [89.)] ***

    % \item [93.)] Since $x_n$ is bounded, $x_n\leq M$ for some $M\in\mathbb{R}$. Consider $\overline{x}_n$: ***

    % \item [94.)] ***

    % \item [95.)] ***

    \item [96.)] Let $S\subseteq\mathbb{N}$, and consider the cases of $S$:

    Case $S=\mathbb N$: Let $x_n=1$. Since $1\geq 1$, the friends of $x_n$ are $\mathbb{N}$, and thus $S$.

    Case $S\subset\mathbb{N}$: Let the sequence $x_n$ be defined as follows:
    \[x_n=\begin{cases}
        1 & n\in S \\
        -\frac{1}{n} & n\notin S \\
    \end{cases}\]
    Since $1>-\dfrac{1}{n}$ for all $n\in\mathbb{N}$, all $n$ for which $x_n=1$ are friends of $x_n$. In addition, for all $m,n\in\mathbb{N}$:
    \[m>n\implies\dfrac{1}{n}>\dfrac{1}{m}\implies-\dfrac{1}{n}<-\frac{1}{m}\]
    Thus all $n$ for which $x_n=-\dfrac{1}{n}$ cannot be friends of $x_n$, thus $x_n=1$ for all friends $n$ of $x_n$, thusly all $n\in S$ are friends, thus $S$ is the set of all friends of $x_n$, thus for all $S\subseteq\mathbb{N}$, there exists a sequence such that $S$ is the set of friends of that sequence. Q.E.D.

    \item [97.)] Let $x_n\to L$, then by theorem 19, $y_k\to L$ for all subsequences $y_k$ of $x_n$. Similarly, since

    \item [99.)] Let $x_n=n-(-1)^nn$. $x_n$ is unbounded, but $y_k=x_{2k}=2k-(-1)^{2k}2k=2k-2k=0$, thus $y_k\preceq x_n$ and $y_k\to0$.

    \item [100.)] Every cauchy sequence is convergent according to theorem 23, and no convergent sequence can be unbounded.

    \item [103.)] Let $x_n=\dfrac{1}{n^2}$. Since $x_n\to0$, $x_n$ is convergent and thus cauchy. Q.E.D.

    \item [105.)] Since $x_n$ and $y_n$ are cauchy, there exist $A,B\in\mathbb{R}$ where $x_n\to A$ and $y_n\to B$. Since $y_n\ne0$ for all $n\in\mathbb{N}$, $B\ne0$. Let $z_n=x_n\di y_n$. According to theorem 14,\break $z_n=x_n\di y_n\implies z_n\to A\di B$, thus $z_n$ is convergent and thus cauchy. Q.E.D.

    % \item [106.)] ***

    % \item [107.)] ***

    % \item [108.)] ***

    % \item [109.)] ***

    % \item [110.)] ***

    % \item [111.)] ***

    % \item [112.)] ***

    % \item [118.)] ***

    % \item [119.)] ***

    % \item [120.)] ***

    % \item [121.)] ***

    % \item [122.)] ***

    % \item [123.)] ***

    \item [127.)] For $\lm{x\to5}x^2=25$, then given $\e>0$, there must exist $\delta>0$ where
    \[\vertb{x-5}<\delta\implies\vertb{x^2-25}<\e\]
    Suppose $\vertb{x-5}<1$, then $\vertb{x+5}<11$, thus 
    \[\vertb{x^2-25}=\vertb{x-5}\vertb{x+5}<11\vertb{x-5}<11\delta\]
    \[11\delta=\e\implies\delta=\frac{\e}{11}\]
    Let $\delta<\min\parns{1,\dfrac{\e}{11}}$:
    \[\vertb{x-5}<\delta\implies\vertb{x-5}<\dfrac{\e}{11}\implies11\vertb{x-5}<\e\implies\vertb{x-5}\vertb{x+5}<11\vertb{x-5}<\e\]
    \[\implies\vertb{x-5}\vertb{x+5}=\vertb{x^2-25}<\e\]
    Thus $\lm{x\to5}x^2=25$. Q.E.D.

    \item [128.)] For $\lm{x\to\frac{1}{2}}\dfrac{1}{x}=2$, then given $\e>0$, there must exist $\delta>0$ where
    \[\vertb{x-\frac{1}{2}}<\delta\implies\vertb{\frac{1}{x}-2}<\e\]
    Suppose $\vertb{x-\dfrac{1}{2}}<\dfrac{1}{4}$:
    \[\vertb{x-\frac{1}{2}}<\frac{1}{4}\implies-\frac{1}{4}<x-\frac{1}{2}<\frac{1}{4}\implies\frac{1}{4}<x<\frac{3}{4}\implies\frac{4}{3}<\frac{1}{\vertb{x}}<4\]
    \[\implies\frac{\vertb{x-\frac{1}{2}}}{\vertb{x}}<4\vertb{x-\frac{1}{2}}<4\delta\]
    \[4\delta=\e\implies\delta=\frac{\e}{4}\]
    Let $\delta<\min\parns{\dfrac{1}{4},\dfrac{\e}{4}}$:
    \[\vertb{x-\frac{1}{2}}<\delta\implies\vertb{x-\frac{1}{2}}<\frac{\e}{4}\implies4\vertb{x-\frac{1}{2}}<\e\implies\frac{\vertb{x-\frac{1}{2}}}{\vertb{x}}<4\vertb{x-\frac{1}{2}}<\e\]
    \[\frac{\vertb{x-\frac{1}{2}}}{\vertb{x}}=\vertb{\frac{x-\frac{1}{2}}{x}}=1-\]

    % \item [129.)] For $\displaystyle\lim_{x\to0}f(x)$ to exist, then for all $\e>0$, there must exist $\delta>0$ where
    % \[\vertb{x-0}<\delta\implies\vertb{f(x)-f(0)}<\e\]

    % \item [130.)] Since $\lm{x\to c}f(x)=\lm{x\to c}h(x)=L$, there exist 

    % \item [132.)] When a function $f$ is continuous at $c$, you can visualize that 

    % \item [133.)] ***

    % \item [134.)] ***

    % \item [135.)] ***

    % \item [136.)] ***

    % \item [137.)] ***

    % \item [138.)] ***

\end{itemize}

\end{document}
