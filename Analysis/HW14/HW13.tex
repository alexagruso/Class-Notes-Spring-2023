\documentclass[12pt]{article}
\pagenumbering{gobble}
\linespread{1.5}

\usepackage{amsfonts}
\usepackage{amsmath}
\usepackage{amssymb}
\usepackage{array}
\usepackage{fancyhdr}
\usepackage{mathrsfs}
\usepackage{mathtools}
\usepackage{textcomp}
\usepackage[margin=1in,headheight=1in]{geometry}

\newcommand{\contradiction}{
    \ensuremath{{\Rightarrow\mspace{-2mu}\Leftarrow}}
}

\newcommand{\angleb}[1]{\left\langle#1\right\rangle}
\newcommand{\vertb}[1]{\left\vert#1\right\vert}
\newcommand{\bracks}[1]{\left[#1\right]}
\newcommand{\parns}[1]{\left(#1\right)}

\newcommand{\derv}[2]{\dfrac{d#1}{d#2}}
\newcommand{\e}{\varepsilon}
\newcommand{\di}{\,/\,}

\newcommand{\lm}[1]{\displaystyle\lim_{#1}}

\begin{document}
\pagestyle{fancy}
\fancyhead{}
\fancyhead[L]{Alexander Agruso}
\fancyhead[R]{MATH 3380 Homework 14}

\normalsize
\begin{itemize}
    \item [88.)] We can use induction to show that $x_n>0$ for all $n\in\mathbb{N}$. Since $x_1=2>0$, the base case holds. For the induction step, assume $x_n>0$:
    \[x_n>0\implies\frac{1}{x_n}>0\text{ and }\frac{x_n}{2}>0\]
    \[\implies\frac{1}{x_n}+\frac{x_n}{2}=x_{n+1}>0\]
    Thus $x_n>0\implies x_{n+1}>0$, thus $x_n$ is bounded below by $0$.

    \item [107.)] \begin{itemize}
        \item [a.)] awd

        \item [b.)] awd

        \item [c.)] Consider $y_n$ when $n=2^{k}-1$ for some $k\in\mathbb{N}$:
        \[y_n=1+\parns{\frac{1}{2}+\frac{1}{2}}+\parns{\frac{1}{4}+\frac{1}{4}+\frac{1}{4}+\frac{1}{4}}+\cdots+\parns{\frac{1}{2^k}+\cdots+\frac{1}{2^k}}\]
        We can see that the terms in each set of parentheses sum to 1, thus $y_n=1+1+\cdots+1$. As $n\to\infty$, this sum diverges, thus $y_n$ is not bounded, convergent, nor cauchy. However, $y_n$ is nondecreasing, and thus monotone.
    \end{itemize}

    \item [140.)] Let $f:\mathbb R\to\mathbb R$ be defined as follows:
    \[f(x)=\begin{cases}
        \dfrac{1}{x-1}+1 & x<0 \\
        \dfrac{1}{x+1}-1 & x>0 \\
        0 & x=0 \\
    \end{cases}\]
    Since $-1<f(x)<1$ for all $x\in\mathbb R$, $f(x)$ is bounded, but since there are horizontal asymptotes at $y=-1$ and $y=1$, $f(x)$ has no maximal nor minimal value.

    \item [152.)] $a$
\end{itemize}

\end{document}
