\documentclass[12pt]{article}
\pagenumbering{gobble}
\linespread{1.1}

\usepackage{amsfonts}
\usepackage{amsmath}
\usepackage{amssymb}
\usepackage{array}
\usepackage{fancyhdr}
\usepackage{mathrsfs}
\usepackage{textcomp}
\usepackage[margin=1in,headheight=1in]{geometry}

\newcommand{\contradiction}{
    \ensuremath{{\Rightarrow\mspace{-2mu}\Leftarrow}}
}

\newcommand{\angleb}[1]{\left\langle#1\right\rangle}
\newcommand{\vertb}[1]{\left\vert#1\right\vert}
\newcommand{\bracks}[1]{\left[#1\right]}
\newcommand{\parns}[1]{\left(#1\right)}

\newcommand{\derv}[2]{\dfrac{d#1}{d#2}}
\newcommand{\e}{\varepsilon}

\newcommand{\lm}[1]{\displaystyle\lim_{#1}}

\begin{document}
\pagestyle{fancy}
\fancyhead{}
\fancyhead[L]{Alexander Agruso}
\fancyhead[R]{MATH 3380 Homework 10}

\normalsize
\begin{itemize}
    % \item [12.)] \begin{itemize}
    %     \item [a.)] Since $0<\lambda<1$, $0<(1-\lambda)<1$ and $(1-\lambda)y < y$.
    % \end{itemize}

    \item [29.)] \begin{itemize}
        \item [a.)] Since $S$ is bounded below, $x\in S\implies x\geq M$ for some $M\in\mathbb{R}$. Let $l<M$, thus $l<M\leq x$, thus $l<x$ for all $x\in S$, thus $l$ is a lower bound of $S$, thus $\mathscr{L}\ne\varnothing$. Next, let $x\in S$, thus $x\geq l$ for all lower bounds $l$ of $S$, thus $x$ is an upper bound of $\mathscr{L}$, thus $\mathscr{L}$ is bounded above. Q.E.D.

        \item [b.)] Let $w=\sup(\mathscr{L})$, thus $w\geq l$ for all $l\in\mathscr{L}$.
    \end{itemize}

    % \item [33.)] Let $x,y\in\mathbb{R}$, and consider $x$:
    %     \[x=x+0=x-y+y\]
    %     And thus by the triangle inequality:
    %     \[\vertb{x-y+y}\leq\vertb{x-y}+\vertb{y}\]
    %     Manipulating:
    %     \[\vert
    %     x-y+y\vert\leq\vertb{x-y}+\vertb{y}\implies\vertb{x}-\vertb{y}\leq\vertb{x-y}\]
    %     \[\implies\vert x\vert-\vert x\vert-\vert y\vert=-\vert y\vert\leq\vert x\vert-\vert y\vert\leq\vert x-y\vert\]
    %     Thus the inequality holds. Q.E.D.***

    % \item [37.)] ***

    \item [38.)] Let $u$ be an upper bound of $S$, thus $u\geq x$ for all $x\in S$.ssd

    % \item [45.)] ***

    \item [46.)] Suppose $2^{-n}\to0$, then for all $\e>0$, there exists $k\in\mathbb{N}$ where
    \[\vertb{2^{-n}-0}<\e\]
    Manipulating the inequality:
    \[\vertb{2^{-n}-0}=\vertb{2^{-n}}=2^{-n}<\e\implies\frac{1}{\e}<2^n\]
    Thus
    \[\vertb{2^{-n}-0}=\vertb{\frac{1}{2^n}}<\vertb{\frac{1}{\frac{1}{\e}}}=\vertb{\e}=\e\]
    Thus $n\geq k\implies\vertb{2^{-n}-0}<\e$, thus $2^{-n}\to0$. Q.E.D.

    \item [47.)] Let $y_n\to0$, thus for all $\e_1>0$, there exists $k\in\mathbb{N}$ where
    \[n\geq k\implies\vertb{y_n-0}<\e_1\]
    Now suppose $x_n$ is a bounded sequence, and $x_ny_n\to0$, then for all $\e>0$,
    \[n\geq k\implies\vertb{x_ny_n-0}<\e\]
    Manipulating the inequality:
    \[\vertb{x_ny_n-0}=\vertb{x_ny_n}=\vertb{x_n}\vertb{y_n-0}<\e\implies\vertb{y_n-0}<\frac{\e}{\vertb{x_n}}\]
    Let $\e_1=\dfrac{\e}{\vertb{x_n}}$, then
    \[\vertb{y_n-0}<\frac{\e}{\vertb{x_n}}\implies\vertb{x_n}\vertb{y_n-0}<\e\implies\vertb{x_ny_n-0}<\e\]
    Thus $n\geq k\implies\vertb{x_ny_n-0}<\e$, thus $x_ny_n\to0$. Q.E.D.

    % \item [48.)] ***

    % \item [49.)] ***

    \item [50.)] Since $x_n\to A$, then for all $\e_1>0$, there exists $k\in\mathbb{N}$ where
    \[n\geq k\implies\vertb{x_n-A}<\e_1\]
    Now suppose $cx_n\to cA$, then for all $\e>0$,
    \[n\geq k\implies\vertb{cx_n-cA}<\e\]
    Manipulating the inequality:
    \[\vertb{cx_n-cA}<\vertb{c(x_n-A)}=\vertb{c}\vertb{x_n-A}<\e\]
    \[\implies\vertb{x_n-A}<\frac{\e}{\vertb c}\]
    Let $\e_1=\dfrac{\e}{\vertb c}$, then
    \[\vertb{x_n-A}<\frac{\e}{\vertb c}\implies\vertb{c}\vertb{x_n-A}<\e\implies\vertb{cx_n-cA}<\e\]
    Thus $n\geq k\implies\vertb{cx_n-cA}<\e$, thus $cx_n\to cA$. Q.E.D.

    \item [52.)] Let $M,N\in\mathbb{R}$ where $M\geq x_n$ and $N\geq y_n$ for all $n\in\mathbb{N}$. Consider $z_n=x_n+y_n$. $x_n+y_n\leq M+N$, thus $z_n\leq M+N$ for all $n\in\mathbb{N}$, thus $z_n$ is bounded. Q.E.D.

    \item [53.)] \begin{itemize}
        \item [a.)] False; many sequences diverge.

        \item [b.)] True; Since $x_n$ diverges, then we know that $x^2_n=x_nx_n$ diverges too.
    \end{itemize}

    \item [54.)] \begin{itemize}
        \item [a.)] $(-1)^n$

        \item [b.)] DNE; A sequence cannot converge to a value while having terms that are arbitrarily far from that value.
    \end{itemize}

    % \item [57.)] ***

    % \item [58.)] ***

    % \item [59.)] ***

    % \item [60.)] ***

    \item [62.)] Assume $x_n\to A$, thus for all $\varepsilon>0$, there exists $k$ such that
    \[n\geq k\implies\vertb{x_n-A}<\varepsilon\]
    Since $x_n<M$, we know that
    \[\vertb{x_n-A}<\vertb{M-A}\]
    ***

    \item [63.)] Since $S$ is nonempty and bounded below, there exists some $v\in\mathbb{R}$ where $v=\inf(S)$. Let $\{x_n\}^\infty_{n=1}$ be a sequence where $x_n=v$, thus $x_n\to\inf(S)$, thus there exists a sequence $x_n$ where $x_n\to\inf(S)$. Q.E.D.

    % \item [64.)] ***

    \item [65.)] So far, the topic I have had the most trouble grasping is $\varepsilon$-$k$ convergence proofs. I understand the process of constructing the proof, but am still working on understanding the logic.

    % \item [66.)] ***

    % \item [67.)] ***

    % \item [68.)] ***

    % \item [69.)] ***

    % \item [70.)] ***

    % \item [71.)] ***

    % \item [74.)] ***

    % \item [75.)] ***

    % \item [76.)] ***

    \item [77.)] Let $x_n$ and $y_n$ be sequences where $x_n=y_n=n$. Since $n$ is stricly increasing, so are $x_n$ and $y_n$. However $x_n-y_n=n-n=0$, and $0$ is not strictly increasing, thus $x_n-y_n$ is not strictly increasing.

    \item [79.)] A subsequence of $x_n$ is a sequence $y_k$ such that $y_k=x_{n_k}$ for all $k\in\mathbb{N}$, where $n_k$ is a strictly increasing sequence of natural numbers.

    \item [80.)] You can view subsequences as a more abstract form of function composition. You have $x_n$, which is a function $f:\mathbb{N}\to\mathbb{R}$, and you have $n_k$, which is a function $g:\mathbb{N}\to\mathbb{N}$. When you let $y_k=x_{n_k}$, this is equivalent to $y_k=f(g(k))$. Since the codomain of $g$ and domain of $f$ are the same, namely $\mathbb{N}$, we know this function composition is valid.

    \item [81.)] Let $y_k=x_{2k}$, thus $y_k=(-1)^{2k}=1$ for all $k\in\mathbb{N}$, thus $y_k\to1$, thus $y_k$ is convergent.

    \item [82.)] \begin{itemize}
        \item [a.)] DNE, if $y_k\preceq x_n$ and $x_n\to L$, then $y_k\to L$ as per theorem 19.

        \item [b.)] $n-(-1)^nn$, $x_n$ diverges, but $y_k=x_{2k}=2k-(-1)^{2k}2k=2k-2k=0,\,\therefore y_n\to0$.

    \end{itemize}

    \item [83.)] Setting $x_n=y_k$ and solving for $n$, we can find a suitable $n_k$:
    \[2n-1=8k+1\implies2n=8k+2\implies n=4k+1\]
    Thus when $n_k=4k+1$, $x_{n_k}=2(4k+1)-1=8k+2-1=8k+1=y_k$, thus $y_k\preceq x_n$. Q.E.D.

    \item [84.)] Setting $x_n=y_k$ and solving for $n$, we can find a suitable $n_k$:
    \[2n-1=8k^2+24k+17\implies2n=8k^2+24k+18\implies n=4k^2+12k+9\]
    Thus when $n_k=4k^2+12k+9$, $x_{n_k}=2(4k^2+12k+9)-1=8k^2+24k+17=y_k$, thus $y_k\preceq x_n$. Q.E.D.

    \item [85.)] \begin{itemize}
        \item [a.)] ***

        \item [b.)] ***
    \end{itemize}

    \item [86.)] \begin{itemize}
        \item [a.)] True; Let $n_k=k$, thus $x_k=x_{n_k}$, thus $x\preceq x$.

        \item [b.)] False; Let $x_n=(-1)^n$ and $y_n=(-1)^{n+1}$. 

        \item [c.)] True; Let ***
    \end{itemize}

    % \item [87.)] ***

    % \item [88.)] ***

    % \item [89.)] ***

    \item [90.)] Since $x_n$ is bounded, $x_n\leq M$ for some $M\in\mathbb{R}$. In addition, since $y\preceq x$, for all $k\in\mathbb{N}$, there exists $n\in\mathbb{N}$ such that $y_k=x_n$, thus $y_k=x_n\leq M$, thus $y_k\leq M$ for all $k$, thus $y_k$ is bounded. Q.E.D.

    \item [91.)] \begin{itemize}
        \item [a.)] $S=\left\{\dfrac{1}{n}:n\in\mathbb{N}\right\}$ are the friends of $x_n$.

        \item [b.)] $S=\{1\}$ is the friend of $y_n$.
    \end{itemize}

    \item [92.)] $S=\{n\in\mathbb{N}:21\leq n\leq56\}$ are the friends of $z_n$.

    \item [93.)] Since $x_n$ is bounded, $x_n\leq M$ for some $M\in\mathbb{R}$. Consider $\overline{x}_n$: ***

    % \item [94.)] ***

    % \item [95.)] ***

    % \item [96.)] ***

    % \item [97.)] ***

    % \item [99.)] ***

    % \item [100.)] ***

    % \item [101.)] ***

    % \item [102.)] ***

    % \item [103.)] ***

    % \item [104.)] ***

    % \item [105.)] ***

    % \item [106.)] ***

    % \item [107.)] ***

    % \item [108.)] ***

    % \item [109.)] ***

    % \item [110.)] ***

    % \item [111.)] ***

    % \item [112.)] ***

    \item [113.)] $z\in\mathbb{R}$ is a cluster point of $S$ if for all $\varepsilon>0$ there exists $x\in S$ such that $0<\vertb{x-z}<\varepsilon$.

    \item [114.)] Let $S=\{y\in\mathbb{R}:\vertb{x-y}<r\}$,
    \[\vertb{x-y}<r\implies-r<x-y<r\implies -r-x<-y<r-x\]
    \[\implies x-r<y<x+r\implies S=(x-r,x+r)\]
    Thus $\{y\in\mathbb{R}:\vertb{x-y}<r\}=(x-r,x+r)$ Q.E.D.
    
    \item [115.)] $\mathbb{Z}$ has no cluster points.
    
    \item [116.)] 1 is a cluster point of $[0,1)$.

    \item [117.)] For $0$ to be a cluster point of $[1,2]$, then for all $\varepsilon>0$ there exists $a\in[1,2]$ where $\vertb{a-0}<\varepsilon$. Let $\varepsilon=\dfrac{1}{2}$, then
    \[\vertb{a-0}=\vertb{a}<\frac{1}{2}\implies -\frac{1}{2}<a<\frac{1}{2}\]
    But since $-\dfrac{1}{2}<a<\dfrac{1}{2}$, $a\notin[1,2]$, thus $a\in[1,2]$ does not exist for 0, thus $0$ is not a cluster point of $[1,2]$. Q.E.D.

    % \item [118.)] ***

    % \item [119.)] ***

    % \item [120.)] ***

    % \item [121.)] ***

    % \item [122.)] ***

    % \item [123.)] ***

    \item [124.)] Let $f:E\to\mathbb{R}$, $c\in E'$, and $L\in\mathbb{R}$, then $f(x)\to L$ as $x\to c$, or $\displaystyle\lim_{x\to c}f(x)=L$ if for all $\varepsilon>0$, there exists $\delta>0$ such that $\vertb{x-c}<\delta\implies\vertb{f(x)-L}<\varepsilon$.

    \item [125.)] Suppose $\displaystyle\lim_{x\to c}f(x)=a$, then for all $\varepsilon>0$, there exists $\delta>0$ where
    \[\vertb{x-c}<\delta\implies\vertb{a-a}<\varepsilon\]
    Since $\vertb{a-a}=\vertb{0}=0<\varepsilon$, $\vertb{a-a}<\varepsilon$, thus $\displaystyle\lim_{x\to c}f(x)=a$. Q.E.D.

    \item [126.)] Suppose $\lm{x\to2}3x+1=7$, then for all $\varepsilon>0$, there exists $\delta>0$ where
    \[\vertb{x-2}<\delta\implies\vertb{3x+1-7}<\e\]
    We can manipulate the inequality to find a sufficient value for $\delta$:
    \[\vertb{3x+1-7}=\vertb{3x-6}=\vertb{3(x-2)}=3\vertb{x-2}<\e\]
    \[\implies\vertb{x-2}<\frac{\e}{3}\]
    Let $\delta=\dfrac{\e}{3}$, then we know that $\vertb{x-2}<\delta$. Manipulating the inequality:
    \[\vertb{x-2}<\dfrac{\e}{3}\implies3\vertb{x-2}=\vertb{3(x-2)}=\vertb{3x-6}=\vertb{3x+1-7}<\e\]
    Thus $\vertb{x-2}<\delta\implies\vertb{3x+1-7}<\e$, thus $\lm{x\to2}3x+1=7$. Q.E.D.

    \item [127.)] Suppose $\lm{x\to5}x^2=25$, then for all $\e>0$, there exists $\delta>0$ where
    \[\vertb{x-5}<\delta\implies\vertb{x^2-25}<\e\]
    We can manipulate the inequality to find a sufficient value for $\delta$:
    \[\vertb{x^2-25}=\vertb{(x-5)(x+5)}=\vertb{x-5}\vertb{x+5}<\e\]
    \[\implies\vertb{x-5}<\frac{\e}{\vertb{x+5}}\]
    Let $\delta=\dfrac{\e}{\vertb{x+5}}$, then we know that $\vertb{x-5}<\delta$. Manipulating the inequality:
    \[\vertb{x-5}<\frac{\varepsilon}{\vertb{x+5}}\implies\vertb{x-5}\vertb{x+5}=\vertb{(x-5)(x+5)}=\vertb{x^2-25}<\e\]
    Thus $\vertb{x-5}<\delta\implies\vertb{x^2-25}<\e$, thus $\lm{x\to5}x^2=25$. Q.E.D.

    \item [128.)] Suppose $\lm{x\to\frac{1}{2}}\dfrac{1}{x}=2$, then for all $\e>0$ there exists $\delta>0$ where
    \[\vertb{x-\frac{1}{2}}<\delta\implies\vertb{\frac{1}{x}-2}<\e\]
    We can manipulate the inequality to find a sufficient value for $\delta$:
    \[\vertb{\frac{1}{x}-2}=***\]

    % \item [129.)] ***

    % \item [130.)] ***

\end{itemize}

\end{document}
