\documentclass[12pt]{book}

\usepackage{amsfonts}
\usepackage{amssymb}

\usepackage{fancyhdr}
\usepackage[Glenn]{fncychap}

\usepackage[headheight=1in,margin=1.25in]{geometry}

\usepackage{mathtools}

\usepackage{pgfplots}
\pgfplotsset{compat=1.12}

\usepackage{tcolorbox}
\usepackage{tikz}

\newcommand{\N}{\mathbb{N}}
\newcommand{\Z}{\mathbb{Z}}
\newcommand{\Q}{\mathbb{Q}}
\newcommand{\R}{\mathbb{R}}

\newcommand{\claim}{\textit{Claim:} }
\newcommand{\theorem}{\textit{Theorem:} }
\newcommand{\proof}{\textit{Proof:} }
\newcommand{\done}{\ensuremath{\strut\hfill\blacksquare}}

\newcommand{\proofbox}[2]{\begin{tcolorbox}[
    colback=blue!5!white,
    colframe=blue!50!black,
    adjusted title=\textbf{#1},
    before skip=0.25in,
    after skip=0.25in]
    #2
\end{tcolorbox}}

\begin{document}
    \pagestyle{fancy}

    \chapter{Geometry and Arithmetic}
    \thispagestyle{empty}
    \fancyhead[EL]{\textbf{\thepage}}
    \fancyhead[ER]{\textbf{Section \thesection}}
    \fancyhead[OL]{\textbf{Chapter \thechapter}}
    \fancyhead[OR]{\textbf{\thepage}}
    \fancyfoot{}
    
    \section{Rational Points on Conics}

    We start with the rational numbers, $\Q$.
    A point $(x,y)$ on the plane is a \textit{rational point} if both 
    $x,y\in\Q$.
    A \textit{rational line} is a line in the plane whose coefficients
    are rational, i.e.
    \[ax+by+c=0\]
    where $a,b,c\in\Q$.
    It is easy to show, and left as an exercise to the reader,
    that given two rational points, the line between them is rational.
    It is also easy to show that given non-parallel rational lines,
    their intersection point is rational.

    Moving beyond lines, this book aims to study the rational points on
    curves in the plane, specifically cubic curves.
    As an introduction however, we will briefly study conics.
    Let the equation
    \[ax^2+bxy+cy^2+dx+ey+f=0\]
    define a conic.
    We say this is a \textit{rational conic} if each coefficient
    $a,b,\dots,f$ is rational.
    
    Consider the intersection points of a rational line with a rational conic.
    Are these points also rational?
    In general, the answer is no.
    The reason for this is that in the process of solving for the 
    $x$-coordinates of the intersection points, we end up with a quadratic 
    equation.
    Assuming a rational coefficients for our line and conic, we obtain
    a rational quadratic equation, but a such an equation might still
    have irrational roots, e.g. $x^2-2=0$, so the intersection points
    may not be rational. We show this more rigorously with the following example.
    
    \pagebreak
    Consider the line $L:y=1-x$ and the conic $C:y=x^2-2$.
    Setting the two equations to be equal, we can solve for $x$ and find
    the $x$-coordinates of the intersection points:
    \[1-x=x^2-2\implies x^2+x-3=0\]
    Using the quadratic equation, we find 

    But can we find lines and conics that $\textit{do}$ have rational intersection points? It turns out we can. Better yet, if we find one intersection point to be a rational, then it must be that the other point is also rational. We give an example where the intersection points are rational, and then prove our claim of duplicate rational points.

\end{document}