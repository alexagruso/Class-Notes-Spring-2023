\documentclass[11pt]{article}
\pagenumbering{gobble}

\usepackage{amsfonts}
\usepackage{amsmath}
\usepackage{amssymb}
\usepackage{array}
\usepackage{fancyhdr}
\usepackage{mathrsfs}
\usepackage{mathtools}
\usepackage{setspace}
\usepackage{textcomp}
\usepackage[margin=1in,headheight=1in]{geometry}

\newcommand{\contradiction}{
    \ensuremath{{\Rightarrow\mspace{-2mu}\Leftarrow}}
}

% bracket commands
\newcommand{\angleb}[1]{\left\langle#1\right\rangle} % <>
\newcommand{\vertb}[1]{\left\vert#1\right\vert}      % ||
\newcommand{\bracks}[1]{\left[#1\right]}             % []
\newcommand{\braces}[1]{\left\{#1\right\}}           % {}
\newcommand{\parens}[1]{\left(#1\right)}             % ()

% set aliases
\newcommand{\N}{\mathbb{N}}
\newcommand{\Z}{\mathbb{Z}}
\newcommand{\Q}{\mathbb{Q}}
\newcommand{\R}{\mathbb{R}}

\newcommand{\derv}[2]{\dfrac{d#1}{d#2}}
\newcommand{\e}{\varepsilon}
\newcommand{\di}{\,/\,}

\newcommand{\lm}[1]{\displaystyle\lim_{#1}}

\begin{document}
\pagestyle{fancy}
\fancyhead{}
\fancyhead[L]{Alex Agruso}
\fancyhead[R]{Abstract Algebra Homework 4}

\normalsize

\section*{Chapter 5}
\begin{itemize}
    \begin{spacing}{1}
    \item [1.)] \begin{itemize}
        \item [a.)] $\alpha^{-1}=\begin{bmatrix}
            1 & 2 & 3 & 4 & 5 & 6 \\
            2 & 1 & 3 & 5 & 4 & 6
        \end{bmatrix}$

        \item [b.)] $\beta\alpha=\begin{bmatrix}
            1 & 2 & 3 & 4 & 5 & 6 \\
            1 & 6 & 2 & 3 & 4 & 5
        \end{bmatrix}$

        \item [c.)] $\alpha\beta=\begin{bmatrix}
            1 & 2 & 3 & 4 & 5 & 6 \\
            6 & 2 & 1 & 5 & 3 & 4
        \end{bmatrix}$
    \end{itemize}

    \item [3.)] \begin{itemize}
        \item [a.)] $(1235)(413)=(15)(234)$

        \item [b.)] $(13256)(23)(45612)=(124635)$

        \item [c.)] $(12)(13)(23)(142)=(1423)$
    \end{itemize}

    \item [5.)] The order of a permutation is given by the least common multiple of the lengths of its disjoint cycles.
    \begin{itemize}
        \item [a.)] $\vertb{(124)(357)}=\text{lcm}(3,3)=3$

        \item [a.)] $\vertb{(124)(3567)}=\text{lcm}(3,4)=12$

        \item [c.)] $\vertb{(124)(35)}=\text{lcm}(3,2)=6$

        \item [d.)] $\vertb{(124)(357869)}=\text{lcm}(3,6)=6$

        \item [e.)] $\vertb{(1235)(24567)}=\text{lcm}(4,5)=20$

        \item [f.)] $\vertb{(345)(245)}=\text{lcm}(3,3)=3$
    \end{itemize}

    \end{spacing}
\end{itemize}

\section*{Chapter 6}
\begin{itemize}
    \item [1.)] I am assuming that, in this case, the even integers include negatives, as the non-negative even integers do not form a group under addition. Let $f(n)=2n$. First we will show that $f$ is injective. Let $a,b\in\Z$ where $f(a)=f(b)$, thus $2a=2b$, thus $a=b$, thus $f$ is injective. Next, we will show that $f$ is surjective. Let $a$ be an even integer, thus $a=2n$ for some $n\in\Z$, thus $a/2$ is an integer. From this we can see that $f(a/2)=2(a/2)=a$, thus $f$ is surjective, and thus bijective. Finally, given $a,b\in\Z$, we can see that $f(a+b)=2(a+b)=2a+2b=f(a)+f(b)$, thus $f$ is an isomorphism from the integers to the even integers. $\blacksquare$
   
    \item [6.)] Let $G$, $H$, and $K$ be groups with operation $*$. Let $f:G\to G$ where $f(g)=g$. Let $a,b\in G$, thus $f(a*b)=a*b=f(a)*f(b)$. In addition, $f$ is trivially bijective, thus $f$ is an isomorphism from $G$ to $G$, thus $G\cong G$, thus group isomorphism is reflexive.

    Next, let $G\cong H$, thus there exists a bijective $f:G\to H$ where $f(a*b)=f(a)*f(b)$ for all $a,b\in G$. Since $f$ is bijective, there exists an inverse function $f^{-1}$ that is also bijective. Let $a,b\in G$, thus $f(a),f(b)\in H$. We can see that $f^{-1}(f(a)*f(b))=f^{-1}(f(a*b))=a*b=f^{-1}(f(a))*f^{-1}(f(b))$, thus $f^{-1}$ is an isomorphism from $H$ to $G$, thus  $H\cong G$, thus group isomorphism is symmetric.

    Finally, let $G\cong H$ and $H\cong K$, thus there exist isomorphisms $f:G\to H$ and $g:H\to K$. Let $h=g\circ f$ and let $a,b\in G$, then $h(a*b)=g(f(a*b))=g(f(a)*f(b))=g(f(a))*g(f(b))=h(a)*h(b)$, thus $G\cong K$, thus group isomorphism is transitive.

    It follows that group isomorphism is an equivalence relation. $\blacksquare$
\end{itemize}

\end{document}
