\documentclass[11pt]{article}

\usepackage{amsfonts}
\usepackage{amsmath}
\usepackage{amssymb}
\usepackage{array}
\usepackage{fancyhdr}
\usepackage{mathrsfs}
\usepackage{mathtools}
\usepackage{setspace}
\usepackage[skip=0.2in]{parskip}
\usepackage{textcomp}
\usepackage[margin=1in,headheight=1in]{geometry}

\newcommand{\contradiction}{
    \ensuremath{{\Rightarrow\mspace{-2mu}\Leftarrow}}
}

\newcommand{\done}{
    \ensuremath{\strut\hfill\blacksquare}
}

% bracket commands
\newcommand{\angleb}[1]{\left\langle#1\right\rangle} % <>
\newcommand{\vertb}[1]{\left\vert#1\right\vert}      % ||
\newcommand{\bracks}[1]{\left[#1\right]}             % []
\newcommand{\braces}[1]{\left\{#1\right\}}           % {}
\newcommand{\parens}[1]{\left(#1\right)}             % ()

% set aliases
\newcommand{\N}{\mathbb{N}}
\newcommand{\Z}{\mathbb{Z}}
\newcommand{\Q}{\mathbb{Q}}
\newcommand{\R}{\mathbb{R}}

\newcommand{\derv}[2]{\dfrac{d#1}{d#2}}
\newcommand{\e}{\varepsilon}
\newcommand{\di}{\,/\,}

\newcommand{\lm}[1]{\displaystyle\lim_{#1}}

\begin{document}

\pagenumbering{gobble}

\pagestyle{fancy}
\fancyhead{}
\fancyhead[L]{Alex Agruso}
\fancyhead[R]{Abstract Algebra Homework 12}

\normalsize

\section*{Chapter 20}

\begin{itemize}
    \item [1.)] The elements in \( \Q\parens{\sqrt[3]{5}}\) all take the form
    \( a + b\parens{\sqrt[3]{5}} \), where \( a \) and \( b \) are rational
    numbers.

    \item [3.)] We can factor \( x^3 - 1 \) as \( (x - 1)(x^2 + x + 1) \).
    Since \( x - 1 \) has a root in \( \Q \), it suffices to construct a
    splitting field for \( x^2 + x + 1 \). We find the roots of
    \( x^2 + x + 1 \) to be \( -(1/2) \pm i\parens{\sqrt3/2}\), thus the
    extension field \( \Q\parens{i\sqrt3} \) contains all the roots of
    \( x^2 + x + 1 \), and thus all the roots of \( x^3 - 1 \), thus it is a
    splitting field of \( x^3 - 1 \) over \( \Q \).
    \done

    \item [4.)] We can factor \( x^4 + 1 \) as \( (x^2 + i)(x^2 - i) \), thus
    the roots of \( x^4 - 1 \) are \( \pm \sqrt{i} \) and \( \pm \sqrt{-i} \),
    thus the extension field \( \Q\parens{\sqrt{i}, \sqrt{-i}}\) is a
    splitting field of \( x^4 + 1 \) over \( \Q \).
    \done

    \item [11.)] The elements in \( \Q\parens{\pi} \) all take the form
    \( a + b\pi \), where \( a \) and \( b \) are rational numbers. Note that
    since \( \pi \) is transcendental, \( \Q(\pi) \) is not a splitting field
    for any polynomial in \( \Q[x] \).

    \item [35.)] For a root of \( f(x) = x^{p^n} - x \) to be multiple, then
    that root would also be a root of the derivative
    \( f'(x) = p^nx^{p^n} - 1 \). Since this polynomial exists in a field with
    characteristic \( p \ne 0 \), we know that \( p^n = 0 \), so
    \( f'(x) = 0x^{0} - 1 = -1 \), thus \( f'(x) \) has no roots, and thus no
    common roots with \( f(x) \), thus the roots of \( f(x) \) are distinct.
    \done
\end{itemize}

\end{document}
