\documentclass[11pt]{article}
\pagenumbering{gobble}

\usepackage{amsfonts}
\usepackage{amsmath}
\usepackage{amssymb}
\usepackage{array}
\usepackage{fancyhdr}
\usepackage{mathrsfs}
\usepackage{mathtools}
\usepackage{setspace}
\usepackage{textcomp}
\usepackage[margin=1in,headheight=1in]{geometry}

\newcommand{\contradiction}{
    \ensuremath{{\Rightarrow\mspace{-2mu}\Leftarrow}}
}

% bracket commands
\newcommand{\angleb}[1]{\left\langle#1\right\rangle} % <>
\newcommand{\vertb}[1]{\left\vert#1\right\vert}      % ||
\newcommand{\bracks}[1]{\left[#1\right]}             % []
\newcommand{\braces}[1]{\left\{#1\right\}}           % {}
\newcommand{\parens}[1]{\left(#1\right)}             % ()

% set aliases
\newcommand{\N}{\mathbb{N}}
\newcommand{\Z}{\mathbb{Z}}
\newcommand{\Q}{\mathbb{Q}}
\newcommand{\R}{\mathbb{R}}

\newcommand{\derv}[2]{\dfrac{d#1}{d#2}}
\newcommand{\e}{\varepsilon}
\newcommand{\di}{\,/\,}

\newcommand{\lm}[1]{\displaystyle\lim_{#1}}

\begin{document}
\pagestyle{fancy}
\fancyhead{}
\fancyhead[L]{Alex Agruso}
\fancyhead[R]{Abstract Algebra Homework 5}

\normalsize

\section*{Chapter 7}
\begin{itemize}
    \item [1.)] Since $H=\braces{n\in\Z:3\mid n}$, equipped with addition, we can find the left cosets of $H$ in $\Z$. Consider $n\in\Z$:
    \[n\equiv0\pmod{3}\implies n+H=\braces{0+0,0\pm3,0\pm6,\cdots}=\braces{0,\pm3,\pm6,\cdots}=H\]
    \[n\equiv1\pmod{3}\implies n+H=\braces{1+0,1\pm3,1\pm6,\cdots}=\braces{\cdots,-5,-2,1,4,7,\cdots}\]
    \[n\equiv2\pmod{3}\implies n+H=\braces{2+0,2\pm3,2\pm6,\cdots}=\braces{\cdots,-4,-1,2,5,8,\cdots}\]
    Thus the left cosets of $H$ in $\Z$ are $H$, $1+H$, and $2+H$.

    \item [5.)] Since $U(30)=\braces{1,7,11,13,17,19,23,29}$, the left cosets of $H=\braces{1,11}$ in $U(30)$ are given as follows:
    \[1H=\braces{1,11}=H\]
    \[7H=\braces{7,17}\]
    \[11H=\braces{11,1}=H\]
    \[13H=\braces{13,23}\]
    \[17H=\braces{17,7}=7H\]
    \[19H=\braces{19,29}\]
    \[23H=\braces{23,13}=13H\]
    \[29H=\braces{29,19}=19H\]
    Thus the left cosets of $H$ in $U(30)$ are $H$, $7H$, $13H$, and $19H$.

    \item [6.)] Since $\vertb{a}=15$, $H=\angleb{a^5}=\braces{a^5,a^{10},e}$. From this, we can find the left cosets of $H$ in $\angleb{a}$:
    \[eH=\braces{a^5,a^{10},e}=H\]
    \[aH=\braces{a^6,a^{11},a}\]
    \[a^2H=\braces{a^7,a^{12},a^2}\]
    \[a^3H=\braces{a^8,a^{13},a^3}\]
    \[a^4H=\braces{a^9,a^{14},a^4}\]
    \[a^5H=\braces{a^{10},e,a^5}=H\]
    \[\vdots\]
    Thus the left cosets of $H$ in $\angleb{a}$ are $H$, $aH$, $a^2H$, $a^3H$, and $a^4H$.

    \item [15.)] By Lagrange's theorem, if $H$ is a subgroup of $G$, then $\vertb{H}\mid60$, thus the possible orders of $H$ are the divisors of $60$:
    \[\braces{1,2,3,4,5,6,10,12,15,20,30,60}\]

    \item [20.)] Consider $U(n)$ for $n>2$. We can see that $(n-1)^2\equiv n^2-2n+1\equiv0-0+1\equiv1\pmod{n}$, thus $\vertb{n-1}=2$, thus by Lagrange's theorem, $2$ divides $\vertb{U(n)}$, thus $\vertb{U(n)}$ is even. $\blacksquare$
\end{itemize}

\end{document}
