\documentclass[12pt]{article}
\pagenumbering{gobble}
\linespread{1.25}

\usepackage{amsfonts}
\usepackage{amsmath}
\usepackage{amssymb}
\usepackage{array}
\usepackage{fancyhdr}
\usepackage{mathrsfs}
\usepackage{mathtools}
\usepackage{textcomp}
\usepackage[margin=1in,headheight=1in]{geometry}

\newcommand{\contradiction}{
    \ensuremath{{\Rightarrow\mspace{-2mu}\Leftarrow}}
}

% bracket commands
\newcommand{\angleb}[1]{\left\langle#1\right\rangle} % <>
\newcommand{\vertb}[1]{\left\vert#1\right\vert}      % ||
\newcommand{\bracks}[1]{\left[#1\right]}             % []
\newcommand{\braces}[1]{\left\{#1\right\}}           % {}
\newcommand{\parens}[1]{\left(#1\right)}             % ()

% set aliases
\newcommand{\N}{\mathbb{N}}
\newcommand{\Z}{\mathbb{Z}}
\newcommand{\Q}{\mathbb{Q}}
\newcommand{\R}{\mathbb{R}}

\newcommand{\derv}[2]{\dfrac{d#1}{d#2}}
\newcommand{\e}{\varepsilon}
\newcommand{\di}{\,/\,}

\newcommand{\lm}[1]{\displaystyle\lim_{#1}}

\begin{document}
\pagestyle{fancy}
\fancyhead{}
\fancyhead[L]{Alex Agruso}
\fancyhead[R]{Abstract Algebra Homework 3}

\normalsize

\section*{Chapter 4}
\begin{itemize}
    \item [1.)] Consider the following sets under addition:
    \begin{itemize}
        \item [1.] $\Z_6=\braces{0,1,2,3,4,5}$\newline
        For all $a\in\Z_6$, there exist $n_1,n_2\in\Z$ where $a=1^{n_1}=5^{n_2}$.

        \item [2.] $\Z_8=\braces{0,1,2,3,4,5,6,7}$\newline
        For all $a\in\Z_8$, there exist $n_1,n_2,n_3,n_4\in\Z$ where $a=1^{n_1}=3^{n_2}=5^{n_3}=7^{n_4}$.

        \item [3.] $\Z_{20}=\braces{0,1,\dots,19,20}$\newline
        For all $a\in\Z_{20}$, there exist $n_1,\dots,n_8\in\Z$ where $a=u_i^{n_i}$. Given $u_i\in U_{20}$.
    \end{itemize}

    \item [3.)] In $\Z_{30}$, $\angleb{20}=\braces{20,10,0}$, and $\angleb{10}=\braces{10,20,0}$. Suppose $a\in\Z_{30}$ where $\vertb{a}=30$, thus $a^{40}=a^{30}a^{10}=a^{10}$ and $a^{60}=\parens{a^{30}}^2=e$, thus $\angleb{a^{20}}=\braces{a^{20},a^{10},e}$ and $\angleb{a^{10}}=\braces{a^{10},a^{20},e}$.

    \item [9.)] For all $k\in\N$ where $k\mid20$, $\angleb{k}$ is a subgroup of $\Z_{20}$ under addition, thus the subgroups of $\Z_{20}$ are $\angleb{1}$, $\angleb{2}$, $\angleb{4}$, $\angleb{5}$, $\angleb{10}$, and $\angleb{20}$ with generators $1$, $2$, $4$, $5$, $10$, and $20$ respectively. Given a group $G=\angleb{a}$ where $\vertb{a}=20$, the subgroups of $G$ are given by $\angleb{a^{kn}}$ where $k\mid20$ and $n\in\N$, thus there are $6$ subgroups of $G$. Each of these subgroups has a generator of $a^k$.

    \item [22.)] Let $G$ be a group with order $3$ where $G=\braces{e,x,y}$, and consider $x\in G$. If $x^2=e$, then $x^3=ex=x$, thus $x$ would not be a generator. If $x^2=x$, then $x^3=x^2=e$, thus $x$ would not be a generator. Finally, if $x^2=y$, then $x^3=xy$. If $xy=x^2$, then $y=x$, thus $x^3\ne xy$. If $x^3=x$, then $x^2=e$, thus $x^2\ne e$. Thus we can conclude that $x^3=e$, thus $x^1=x$, $x^2=y$, and $x^3=e$, thus $G=\angleb{x}$, thus $G$ is cyclic. $\blacksquare$

    \item [31.)] Let $G$ be a finite group and $N=\vertb{G}$. Since the order of any $a\in G$ divides $N$, then given $\vertb{a}=k$, $a^N=a^{kn}=\parens{a^k}^n=e^n=e$, thus there exists a suitable $N\in\N$. $\blacksquare$
\end{itemize}

\end{document}
