\documentclass[11pt]{article}
\linespread{1.5}
\pagenumbering{gobble}

\usepackage{amsfonts}
\usepackage{amsmath}
\usepackage{amssymb}
\usepackage{array}
\usepackage{fancyhdr}
\usepackage{mathrsfs}
\usepackage{mathtools}
\usepackage{setspace}
\usepackage{textcomp}
\usepackage[margin=1in,headheight=1in]{geometry}

\newcommand{\contradiction}{
    \ensuremath{{\Rightarrow\mspace{-2mu}\Leftarrow}}
}

% bracket commands
\newcommand{\angleb}[1]{\left\langle#1\right\rangle} % <>
\newcommand{\vertb}[1]{\left\vert#1\right\vert}      % ||
\newcommand{\bracks}[1]{\left[#1\right]}             % []
\newcommand{\braces}[1]{\left\{#1\right\}}           % {}
\newcommand{\parens}[1]{\left(#1\right)}             % ()

% set aliases
\newcommand{\N}{\mathbb{N}}
\newcommand{\Z}{\mathbb{Z}}
\newcommand{\Q}{\mathbb{Q}}
\newcommand{\R}{\mathbb{R}}

\newcommand{\derv}[2]{\dfrac{d#1}{d#2}}
\newcommand{\e}{\varepsilon}
\newcommand{\di}{\,/\,}

\newcommand{\lm}[1]{\displaystyle\lim_{#1}}

\begin{document}
\pagestyle{fancy}
\fancyhead{}
\fancyhead[L]{Alex Agruso}
\fancyhead[R]{Abstract Algebra Homework 6}

\normalsize

\section*{Chapter 8}
\begin{itemize}
    \item [3.)] Let $G$ and $H$ be groups with respective identity elements $e_G$ and $e_H$. Let $f:G\to G\bigoplus\braces{e_H}$ where $g\mapsto(g,e_H)$. We can show that $f$ is bijective. Let $g_1,g_2\in G$ where $f(g_1)=f(g_2)$, thus $(g_1,e_H)=(g_2,e_H)$, thus $g_1=g_2$, thus $f$ is injective. Next, for all $(g,e_H)\in G\bigoplus\braces{e_H}$, $f(g)=(g,e_H)$, thus $f$ is surjective, and thus bijective. Finally, let $g_1,g_2\in G$, thus $f(g_1g_2)=(g_1g_2,e_H)=(g_1,e_H)(g_2,e_H)=f(g_1)f(g_2)$, thus $f$ is an isomorphism from $G$ to $G\bigoplus\braces{e_H}$, thus $G\cong G\bigoplus\braces{e_H}$. A similar argument shows that $h\mapsto(h,e_G)$ is an isomorphism from $H$ to $H\bigoplus\braces{e_G}$, thus $H\cong H\bigoplus\braces{e_G}$. $\blacksquare$

    \item [6.)] Consider $(1,1)\in\Z_8\bigoplus\Z_2$. We can see that the order of this element is $\text{lcm}(\vertb{1},\vertb{1})=\text{lcm}(7,1)=7$. However, it is clear that there are no elements with order $7$ in $\Z_4\bigoplus\Z_4$, thus $\Z_8\bigoplus \Z_2\not\cong\Z_4\bigoplus\Z_4$. $\blacksquare$

    \item [14.)] Given $D_n$, we know that the order of the rotation subgroup $R$ is $n$, and the order of the reflection subgroup $S$ is $2$. Because both of these numbers divide $2n$, which is the order of $D_n$, we know that $R\bigoplus S\not\cong D_n$. $\blacksquare$

    \item [20.)] Since $4\mid12$ and $9\mid18$, we can find a subgroup of $\Z_{12}\bigoplus\Z_{18}$ that is isomorphic to $\Z_9\bigoplus\Z_4$. We can see that $\angleb{3}$ is a subgroup of $\Z_{12}$ with order $4$, and also that $\angleb{2}$ is a subgroup of $\Z_{18}$ with order $9$, thus $\angleb{2}\bigoplus\angleb{3}\cong\Z_9\bigoplus\Z_4$. $\blacksquare$

    \item [55.)] Consider $(a,b)\in\Z_m\bigoplus\Z_n$. Since $\vertb{(a,b)}=\text{lcm}(\vertb{a},\vertb{b})$, and since $\vertb{a}\mid m$ and $\vertb{b}\mid n$, we know that $\text{lcm}(\vertb{a},\vertb{b})\mid\text{lcm}(m,n)$, thus $\vertb{(a,b)}\mid\text{lcm}(m,n)$. $\blacksquare$
\end{itemize}

\end{document}
