\documentclass[11pt]{article}
\linespread{1.25}
\pagenumbering{gobble}

\usepackage{amsfonts}
\usepackage{amsmath}
\usepackage{amssymb}
\usepackage{array}
\usepackage{fancyhdr}
\usepackage{mathrsfs}
\usepackage{mathtools}
\usepackage{setspace}
\usepackage[skip=0.2in]{parskip}
\usepackage{textcomp}
\usepackage[margin=1in,headheight=1in]{geometry}

\newcommand{\contradiction}{
    \ensuremath{{\Rightarrow\mspace{-2mu}\Leftarrow}}
}

\newcommand{\done}{
    \ensuremath{\strut\hfill\blacksquare}
}

% bracket commands
\newcommand{\angleb}[1]{\left\langle#1\right\rangle} % <>
\newcommand{\vertb}[1]{\left\vert#1\right\vert}      % ||
\newcommand{\bracks}[1]{\left[#1\right]}             % []
\newcommand{\braces}[1]{\left\{#1\right\}}           % {}
\newcommand{\parens}[1]{\left(#1\right)}             % ()

% set aliases
\newcommand{\N}{\mathbb{N}}
\newcommand{\Z}{\mathbb{Z}}
\newcommand{\Q}{\mathbb{Q}}
\newcommand{\R}{\mathbb{R}}

\newcommand{\derv}[2]{\dfrac{d#1}{d#2}}
\newcommand{\e}{\varepsilon}
\newcommand{\di}{\,/\,}

\newcommand{\lm}[1]{\displaystyle\lim_{#1}}

\begin{document}
\pagestyle{fancy}
\fancyhead{}
\fancyhead[L]{Alex Agruso}
\fancyhead[R]{Abstract Algebra Homework 10}

\normalsize

\section*{Chapter 14}

\begin{itemize}
    \item [1.)] Let $a \in R$. It is clear that $\angleb{a}$ is non-empty, as
    $a=1a$, thus $a \in \angleb{a}$.
    We also know that $ra \in R$ for all $r \in R$, so
    $\angleb{a} \subseteq R$.
    Finally, since $R$ is commutative, we know that $ra=ar$, thus
    $ar \in \angleb{a}$, so $\angleb{a}$ is an ideal.
    \done

    \item [7.)] Let $a \in R$. We know that $aR \subseteq R$, and that it is
    non-empty.
    In addition, since $R$ is commutative, we have $ar=ra$, so we know that
    $ra \in aR$, thus $aR$ is an ideal.
    \done

    Given $R = \braces{2n:n \in \Z}$, we know that
    $4R=\braces{4r:r \in R}=\braces{8n:n \in \Z}$.

    \item [28.)] Let $a\in R$ where $a\ne0$.
    Since $R$ is an ideal, we know that for all $k\in R$, there exists
    $b\in R$ where $ab=k$.
    Since $1\in R$, we know that there exists $b$ where $ab=1$, thus $R$ has
    inverses, and is thus a field.
    \done
\end{itemize}

\section*{Chapter 15}

\begin{itemize}
    \item [5.)] We have $\phi(2)+\phi(3)=30+45=75\equiv5\pmod{10}$, but
    $\phi(2+3)=\phi(5)=\phi(0)=0\pmod{10}$, so $\phi$ does not preserve
    addition.
    \done

    \item [20.)] Let $\phi:R_1 \to R_2$ be a ring homomorphism, $a\in R_1$
    where $a$ is idempotent, and $b\in R_2$ where $\phi(a)=b$.
    Since $a^2=a$, we have that
    $b=\phi(a)=\phi(a*a)=\phi(a)\phi(a)=b*b=b^2$, thus $b=b^2$, thus $b$
    is idempotent in $R_2$, thus ring homomorphisms carry idempotent elements
    to other idempotent elements.
    \done
    
\end{itemize}

\end{document}
