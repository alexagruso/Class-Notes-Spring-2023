\documentclass[12pt]{article}
\pagenumbering{gobble}
\linespread{1.25}

\usepackage{amsfonts}
\usepackage{amsmath}
\usepackage{amssymb}
\usepackage{array}
\usepackage{fancyhdr}
\usepackage{mathrsfs}
\usepackage{mathtools}
\usepackage{textcomp}
\usepackage[margin=1in,headheight=1in]{geometry}

\newcommand{\contradiction}{
    \ensuremath{{\Rightarrow\mspace{-2mu}\Leftarrow}}
}

% bracket commands
\newcommand{\angleb}[1]{\left\langle#1\right\rangle} % <>
\newcommand{\vertb}[1]{\left\vert#1\right\vert}      % ||
\newcommand{\bracks}[1]{\left[#1\right]}             % []
\newcommand{\braces}[1]{\left\{#1\right\}}           % {}
\newcommand{\parens}[1]{\left(#1\right)}             % ()

% set aliases
\newcommand{\N}{\mathbb{N}}
\newcommand{\Z}{\mathbb{Z}}
\newcommand{\Q}{\mathbb{Q}}
\newcommand{\R}{\mathbb{R}}

\newcommand{\derv}[2]{\dfrac{d#1}{d#2}}
\newcommand{\e}{\varepsilon}
\newcommand{\di}{\,/\,}

\newcommand{\lm}[1]{\displaystyle\lim_{#1}}

\begin{document}
\pagestyle{fancy}
\fancyhead{}
\fancyhead[L]{Alex Agruso}
\fancyhead[R]{Abstract Algebra Homework 2}

\normalsize

\section*{Chapter 1}
\begin{itemize}
    \item [5.)] Consider the elements in $D_n$. Given an $n$-gon, rotating it counter-clockwise by $2\pi/n$ gives us another symmetry of the $n$-gon. Repeatedly applying this rotation we can find more symmetries of the $n$-gon. However, we can see that if we rotate the $n$-gon $n$ times, we well rotate it by $2\pi n/n=2\pi=360^\circ$, thus rotating the $n$-gon $n$ times does not affect it, thus we can consider this the identity element. Now, consider the reflections in $D_n$. It is clear to see that each vertex in the $n$-gon has its own axis of reflectional symmetry. However, each reflection can be reach by composing rotations and the vertical reflection, thus we can consider the vertical reflection to be the only reflection in $D_n$. From this, 

    \item [15.)] Consider the symmetries of a rectangle. We can rotate it, but only $180^\circ$ and $360^\circ$ rotations will produce symmetries of the rectangle. In addition, there is no diagonal symmetry, thus the only reflections we can perform are a horizontal and vertical reflection. From this, we can conclude that the symmetries of the rectangle can be described with the identity element $e$, the counter-clockwise $180^\circ$ rotation $r$, and a vertical reflection $s$. By composing these elements, we can reach every symmetry of the rectangle.

\end{itemize}

\section*{Chapter 2}
\begin{itemize}
    \item [7.)] \begin{itemize}
        \item [1.] Let $a$ and $b$ be odd integers, thus there exist $m,n\in\Z$ where $a=2m+1$ and $b=2n+1$. We can see that $a+b=2m+1+2n+1=2m+2n+2=2(m+n+1)$, which is even, thus the odd integers are not closed under addition, and thus are not a group under addition.

        \item [2.] For the sake of establishing a contradiction, let $a$ and $b$ be odd integers where $a+b=a$, thus for some $m,n\in\Z$:
        \[2m+1+2n+1=2m+2n+2=2m+1\]
        \[\implies2n+1=0\implies2n=-1\implies n=-\frac{1}{2}\]
        Thus $n\notin\Z$.\contradiction Thus no such $a,b$ exist in the odd integers, thus the odd integers lack an identity element under addition, and thus are not a group under addition.
    \end{itemize}

    \item [14.)] \begin{itemize}
        \item [1.] $(ab)^3=(ab)(ab)(ab)=ababab$

        \item [2.] $(ab^{-2}c)^{-2}=(ab^{-2}c)^{-1}(ab^{-2}c)^{-1}=(c^{-1}b^2a^{-1})(c^{-1}b^2a^{-1})=c^{-1}b^2a^{-1}c^{-1}b^2a^{-1}$
    \end{itemize}

    \item [33.)] \[
        \begin{tabular}{l|lllll}
            & $e$ & $a$ & $b$ & $c$ & $d$ \\
            \hline
            $e$ & $e$ & $a$ & $b$ & $c$ & $d$ \\
            $a$ & $a$ & $b$ & $c$ & $d$ & $e$ \\
            $b$ & $b$ & $c$ & $d$ & $e$ & $a$ \\
            $c$ & $c$ & $d$ & $e$ & $a$ & $b$ \\
            $d$ & $d$ & $e$ & $a$ & $b$ & $c$
        \end{tabular}
        \]

\end{itemize}

\section*{Chapter 3}
\begin{itemize}
    \item [1.)] awd

    \item [4.)] awd
    
    \item [7.)] awd

    \item [13.)] awd

    \item [14.)] awd
\end{itemize}

\end{document}
