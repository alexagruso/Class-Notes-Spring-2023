\documentclass[12pt]{article}
\pagenumbering{gobble}
\linespread{1.4}

\usepackage{amsfonts}
\usepackage{amsmath}
\usepackage{amssymb}
\usepackage{array}
\usepackage{fancyhdr}
\usepackage{mathrsfs}
\usepackage{mathtools}
\usepackage{textcomp}
\usepackage[margin=1in,headheight=1in]{geometry}

\newcommand{\contradiction}{
    \ensuremath{{\Rightarrow\mspace{-2mu}\Leftarrow}}
}

% bracket commands
\newcommand{\angleb}[1]{\left\langle#1\right\rangle} % <>
\newcommand{\vertb}[1]{\left\vert#1\right\vert}      % ||
\newcommand{\bracks}[1]{\left[#1\right]}             % []
\newcommand{\braces}[1]{\left\{#1\right\}}           % {}
\newcommand{\parens}[1]{\left(#1\right)}             % ()

% set aliases
\newcommand{\N}{\mathbb{N}}
\newcommand{\Z}{\mathbb{Z}}
\newcommand{\Q}{\mathbb{Q}}
\newcommand{\R}{\mathbb{R}}

\newcommand{\derv}[2]{\dfrac{d#1}{d#2}}
\newcommand{\e}{\varepsilon}
\newcommand{\di}{\,/\,}

\newcommand{\lm}[1]{\displaystyle\lim_{#1}}

\begin{document}
\pagestyle{fancy}
\fancyhead{}
\fancyhead[L]{Alex Agruso}
\fancyhead[R]{Abstract Algebra Homework 2}

\normalsize

\section*{Chapter 1}
\begin{itemize}
    \item [5.)] In $D_n$ for $n\ge3$, let $e$ be the identity element, $r$ be a counter-clockwise rotation by $(360/n)^\circ$, and $s$ be a reflection across the vertical axis. These elements of $D_n$ form a generating set of $D_n$, as every symmetry in $D_n$ is represented by one of the following group elements:
    \[\braces{e,r,r^2,\dots,r^{n-1},s,sr,sr^2,\dots,sr^{n-1}}\]
    From this set we can conclude that $D_n$ has $2n$ elements, or $\vertb{D_n}=2n$.

    \item [15.)] Considering the symmetries of a nonsquare rectangle, we find the identity element $e$, the $180^\circ$ counter-clockwise rotation $r$, and the reflection across the vertical axis $s$. These elements form the symmetries of the rectangle with the following set:
    \[\braces{e,r,s,sr}\]
    Using this set we can construct the cayley table for the symmetries of the rectangle as follows:
    \[
        \begin{tabular}{c|cccc}
            & $e$ & $r$ & $s$ & $sr$ \\
            \hline
            $e$ & $e$ & $r$ & $s$ & $sr$ \\
            $r$ & $r$ & $e$ & $sr$ & $s$ \\
            $s$ & $s$ & $sr$ & $e$ & $r$ \\
            $sr$ & $sr$ & $s$ & $r$ & $e$
        \end{tabular}
    \]

\end{itemize}

\section*{Chapter 2}
\begin{itemize}
    \item [7.)] \begin{itemize}
        \item [1.] Let $a$ and $b$ be odd integers, thus there exist $m,n\in\Z$ where $a=2m+1$ and $b=2n+1$. We can see that $a+b=2m+1+2n+1=2m+2n+2=2(m+n+1)$, which is even, thus the odd integers are not closed under addition, and thus are not a group under addition.

        \item [2.] For the sake of establishing a contradiction, let $a$ and $b$ be odd integers where $a+b=a$, thus for some $m,n\in\Z$:
        \[2m+1+2n+1=2m+2n+2=2m+1\]
        \[\implies2n+1=0\implies2n=-1\implies n=-\frac{1}{2}\]
        Thus $n\notin\Z$.\contradiction Thus no such $a,b$ exist in the odd integers, thus the odd integers lack an identity element under addition, and thus are not a group under addition.
    \end{itemize}

    \item [14.)] \begin{itemize}
        \item [1.] $(ab)^3=(ab)(ab)(ab)=ababab$

        \item [2.] $(ab^{-2}c)^{-2}=(ab^{-2}c)^{-1}(ab^{-2}c)^{-1}=(c^{-1}b^2a^{-1})(c^{-1}b^2a^{-1})=c^{-1}b^2a^{-1}c^{-1}b^2a^{-1}$
    \end{itemize}

    \item [33.)] \[
        \begin{tabular}{l|lllll}
            & $e$ & $a$ & $b$ & $c$ & $d$ \\
            \hline
            $e$ & $e$ & $a$ & $b$ & $c$ & $d$ \\
            $a$ & $a$ & $b$ & $c$ & $d$ & $e$ \\
            $b$ & $b$ & $c$ & $d$ & $e$ & $a$ \\
            $c$ & $c$ & $d$ & $e$ & $a$ & $b$ \\
            $d$ & $d$ & $e$ & $a$ & $b$ & $c$
        \end{tabular}
        \]

\end{itemize}

\section*{Chapter 3}
\begin{itemize}
    \item [1.)] \begin{itemize}
        \item [a.] $\vertb{\Z_{12}}=\vertb{\braces{[0],[1],[2],\dots,[11]}}=12$\newline
        $\vertb{[0]}=1,\vertb{[1]}=12,\vertb{[2]}=6,\vertb{[3]}=4,\vertb{[4]}=3,\vertb{[5]}=12,\vertb{[6]}=2,\vertb{[7]}=12\break\vertb{[8]}=3,\vertb{[9]}=4,\vertb{[10]}=6,\vertb{[11]}=12$

        \item [b.] $\vertb{U(10)}=\vertb{\braces{1,3,7,9}}=4$\newline
        $\vertb{1}=1,\vertb{3}=4,\vertb{7}=4,\vertb{9}=2$

        \item [c.] $\vertb{U(12)}=\vertb{\braces{1,5,7,11}}=4$\newline
        $\vertb{1}=1,\vertb{5}=2,\vertb{7}=2,\vertb{11}=2$

        \item [d.] $\vertb{U(20)}=\vertb{\braces{1,3,7,9,11,13,17,19}}=8$\newline
        $\vertb{1}=1,\vertb{3}=4,\vertb{7}=4,\vertb{9}=2,\vertb{11}=2,\vertb{13}=4,\vertb{17}=4,\vertb{19}=2$

        \item [e.] $\vertb{D_4}=\vertb{\braces{e,r,r^2,r^3,s,sr,sr^2,sr^3}}=8$\newline
        $\vertb{e}=1,\vertb{r}=4,\vertb{r^2}=2,\vertb{r^3}=4,\vertb{s}=2,\vertb{sr}=2,\vertb{sr^2}=2,\vertb{sr^3}=2$
    \end{itemize}
    One might notice that the order of any element of a group never exceeds the order of the group itself.

    \pagebreak
    \item [4.)] Consider a group $(G,*)$ and let $a,a^{-1}\in G$ where $\vertb{a}=n$ for $n\in\N$. We can see that $a^n=e\implies\parens{a^{-1}}^n*a^n=\parens{a^{-1}}^n=a^{-n}*a^n=a^{-n}=e$, thus $\parens{a^{-1}}^n=e$. Now consider $m\in\N$ where $m<n$, then $a^m\ne e\implies a^{-m}*a^m\ne a^{-m}\implies a^{-m}\ne e$. From this we can conclude that $\vertb{a}=\vertb{a^{-1}}$. $\blacksquare$

    \item [7.)] Consider a group $(G,*)$ and let $a,b,c\in G$ where $\vertb{a}=6$ and $\vertb{b}=7$. First, we can see that $\parens{a^4*c^{-2}*b^4}^{-1}=b^{-4}*c^2*a^{-4}$. Since $\vertb{a}=6$, we can see that $a^{-4}=e*a^{-4}=a^6*a^{-4}=a^2$, and since $\vertb{b}=7$, we can see that $b^{-4}=e*b^{-4}=b^7*b^{-4}=b^3$. From this we can conclude that $\parens{a^4*c^{-2}*b^4}^{-1}=b^3*c^2*a^2$. $\blacksquare$

    \item [13.)] Consider a group $(G,*)$ and let $a,x,x^{-1}\in G$ where $\vertb{a}=n$ for $n\in\N$. We can see the following:
    \[\parens{xax^{-1}}^n=\underbrace{\parens{xax^{-1}}\cdot\parens{xax^{-1}}\cdot\ldots\cdot\parens{xax^{-1}}}_\text{$n$ times}\]
    \[=xa\parens{x^{-1}x}a\parens{x^{-1}x}a\cdot\ldots\cdot a\parens{x^{-1}x}ax^{-1}=xa^nx^{-1}=xex^{-1}=xx^{-1}=e\]
    Thus $\parens{xax^{-1}}^n=e$. Now consider $m\in\N$ where $m<n$, thus $a^m\ne e$. Knowing this, we can see the following:
    \[\parens{xax^{-1}}^m=\underbrace{\parens{xax^{-1}}\cdot\parens{xax^{-1}}\cdot\ldots\cdot\parens{xax^{-1}}}_\text{$m$ times}\]
    \[=xa\parens{x^{-1}x}a\parens{x^{-1}x}a\cdot\ldots\cdot a\parens{x^{-1}x}ax^{-1}=xa^mx^{-1}\ne xex^{-1}=xx^{-1}=e\]
    \[\implies\parens{xax^{-1}}^m\ne e\]
    From this we can conclude that $\vertb{a}=\vertb{xax^{-1}}$. $\blacksquare$

    \item [14.)] Consider a group $(G,*)$ and let $a\in G$ be the only element in $G$ with order $2$, thus $a^2=e$. Let $b,x,x^{-1}\in G$ where $b=xax^{-1}$, and consider $b^2$:
    \[b^2=\parens{xax^{-1}}\parens{xax^{-1}}=xa\parens{x^{-1}x}ax^{-1}=xa^2x^{-1}=xex^{-1}=xx^{-1}=e\]
    If $a$ is the only element in $G$ with order $2$, then $b=a$, thus $xax^{-1}=a\implies xax^{-1}x=xa=ax$, thus $a\in Z(G)$. $\blacksquare$
\end{itemize}

\end{document}
