\documentclass[11pt]{article}
\linespread{1.25}
\pagenumbering{gobble}

\usepackage{amsfonts}
\usepackage{amsmath}
\usepackage{amssymb}
\usepackage{array}
\usepackage{fancyhdr}
\usepackage{mathrsfs}
\usepackage{mathtools}
\usepackage{setspace}
\usepackage[skip=0.2in]{parskip}
\usepackage{textcomp}
\usepackage[margin=1in,headheight=1in]{geometry}

\newcommand{\contradiction}{
    \ensuremath{{\Rightarrow\mspace{-2mu}\Leftarrow}}
}

\newcommand{\done}{
    \ensuremath{\strut\hfill\blacksquare}
}

% bracket commands
\newcommand{\angleb}[1]{\left\langle#1\right\rangle} % <>
\newcommand{\vertb}[1]{\left\vert#1\right\vert}      % ||
\newcommand{\bracks}[1]{\left[#1\right]}             % []
\newcommand{\braces}[1]{\left\{#1\right\}}           % {}
\newcommand{\parens}[1]{\left(#1\right)}             % ()

% set aliases
\newcommand{\N}{\mathbb{N}}
\newcommand{\Z}{\mathbb{Z}}
\newcommand{\Q}{\mathbb{Q}}
\newcommand{\R}{\mathbb{R}}

\newcommand{\derv}[2]{\dfrac{d#1}{d#2}}
\newcommand{\e}{\varepsilon}
\newcommand{\di}{\,/\,}

\newcommand{\lm}[1]{\displaystyle\lim_{#1}}

\begin{document}
\pagestyle{fancy}
\fancyhead{}
\fancyhead[L]{Alex Agruso}
\fancyhead[R]{Abstract Algebra Homework 9}

\normalsize

\section*{Chapter 12}
\begin{itemize}
    \item [12.)] Let $a,b,c\in R$ where $a$ is a unit. Suppose $b\mid c$, then $c=bd$ for some $d\in R$. We know that $d=d(aa^{-1})=a(da^{-1})$, thus $c=ba(da^{-1})$, and thus $ab\mid c$. Next, suppose $ab\mid c$, then $c=abd$ for some $d\in R$. We previous established that $d=ada^{-1}$, thus $c=abd=ab(ada^{-1})=b(a^2da^{-1}=b(ad)$, thus $b\mid c$. \done

\item [16.)] Let $n,a\in R$. We can see that $n*(-a)=n*((-1)*a)=(n*(-1))*a=(-n)*a=((-1)*n)*a=(-1)*(n*a)=-(n*a)$. \done
\end{itemize}

\section*{Chapter 13}
\begin{itemize}
    \item [3.)] Let $R$ be a commutative ring with cancellation, and let $a,b\in R$ where $a\ne0$ and $ab=0$. Let $c\in R$ where $c\ne0$, then $c(ab)=c(0)=0$, thus $c(ab)=a(cb)=0$. For the sake of establishing a contradiction, suppose $b\ne0$, then 

    \item [4.)] The zero-divisors of $\Z_{20}$ are $2,4,5,6,8,10,12,14,15,16$ and $18$, as 
    \[2*10\equiv4*15\equiv5*8\equiv6*10\equiv12*5\equiv16*10\equiv0\pmod{20}\]
    We can see that the zero-divisors are all elements of $\Z_{20}$ that are not units. \done

    \item [5.)] Let $a\in\Z_n$ where $a\ne0$, and suppose $a\notin U_n$, then $\gcd(a,n)=d$ for some $d\geq2$. Since $d\mid n$, let $k=n/d$. We know that $k\ne0$ because $a\ne0$, thus $gcd(a,n)\ne0$. Finally, since $d\mid a$, we know that $a=dl$ for some $l\in\Z$, thus $ak=dln/d=ln\equiv0\pmod{n}$, thus since $a,k\ne0$, we know that $a$ is a zero-divisor. \done
\end{itemize}

\end{document}
