\documentclass[11pt]{article}

\usepackage{amsfonts}
\usepackage{amsmath}
\usepackage{amssymb}
\usepackage{array}
\usepackage{fancyhdr}
\usepackage{mathrsfs}
\usepackage{mathtools}
\usepackage{setspace}
\usepackage[skip=0.2in]{parskip}
\usepackage{textcomp}
\usepackage[margin=1in,headheight=1in]{geometry}

\newcommand{\contradiction}{
    \ensuremath{{\Rightarrow\mspace{-2mu}\Leftarrow}}
}

\newcommand{\done}{
    \ensuremath{\strut\hfill\blacksquare}
}

% bracket commands
\newcommand{\angleb}[1]{\left\langle#1\right\rangle} % <>
\newcommand{\vertb}[1]{\left\vert#1\right\vert}      % ||
\newcommand{\bracks}[1]{\left[#1\right]}             % []
\newcommand{\braces}[1]{\left\{#1\right\}}           % {}
\newcommand{\parens}[1]{\left(#1\right)}             % ()

% set aliases
\newcommand{\N}{\mathbb{N}}
\newcommand{\Z}{\mathbb{Z}}
\newcommand{\Q}{\mathbb{Q}}
\newcommand{\R}{\mathbb{R}}

\newcommand{\derv}[2]{\dfrac{d#1}{d#2}}
\newcommand{\e}{\varepsilon}
\newcommand{\di}{\,/\,}

\newcommand{\lm}[1]{\displaystyle\lim_{#1}}

\begin{document}

\pagenumbering{gobble}

\pagestyle{fancy}
\fancyhead{}
\fancyhead[L]{Alex Agruso}
\fancyhead[R]{Abstract Algebra Homework 11}

\normalsize

\section*{Chapter 19}

\begin{itemize}
    \item [7.)] Suppose $a_1u+a_2(u+v)+a_3(u+v+w)=0$, thus we have that
    \[(a_1+a_2+a_3)u+(a_2+a_3)v+a_3w=0\]
    Since $u,v$, and $w$ are linearly independent, we have $a_1+a_2+a_3=0$,
    $a_2+a_3=0$, and $a_3=0$, so $a_2+a_3=a_2=0$, so $a_1+a_2+a_3=a_1=0$, thus
    $a_1=a_2=a_3=0$, thus $u$, $v$, and $w$ are linearly independent.
    \done

    \item [8.)] Suppose $a_1v_1+a_2v_2+\cdots+a_nv_n=0$. Since the vectors are
    linearly dependent, we can set $a_i\ne0$ for some $i$, so
    $a_1v_1+\cdots+a_{i-1}v_{i-1}+a_{i+1}v_{i+1}+\cdots+a_nv_n=-a_iv_i$.
    Dividing out by $-a_i$, we obtain $v_i$ as a linear combination of the
    other vectors. 
    \done

    \item [9.)] 

    \item [10.)] 

    \item [22.)] Each vector has $n$ elements and each element has $p$ choices,
    so the total number of elements is $p^n$.
    \done
\end{itemize}

\end{document}
