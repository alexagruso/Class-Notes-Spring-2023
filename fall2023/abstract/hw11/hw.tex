\documentclass[11pt]{article}

\usepackage{amsfonts}
\usepackage{amsmath}
\usepackage{amssymb}
\usepackage{array}
\usepackage{fancyhdr}
\usepackage{mathrsfs}
\usepackage{mathtools}
\usepackage{setspace}
\usepackage[skip=0.2in]{parskip}
\usepackage{textcomp}
\usepackage[margin=1in,headheight=1in]{geometry}

\newcommand{\contradiction}{
    \ensuremath{{\Rightarrow\mspace{-2mu}\Leftarrow}}
}

\newcommand{\done}{
    \ensuremath{\strut\hfill\blacksquare}
}

% bracket commands
\newcommand{\angleb}[1]{\left\langle#1\right\rangle} % <>
\newcommand{\vertb}[1]{\left\vert#1\right\vert}      % ||
\newcommand{\bracks}[1]{\left[#1\right]}             % []
\newcommand{\braces}[1]{\left\{#1\right\}}           % {}
\newcommand{\parens}[1]{\left(#1\right)}             % ()

% set aliases
\newcommand{\N}{\mathbb{N}}
\newcommand{\Z}{\mathbb{Z}}
\newcommand{\Q}{\mathbb{Q}}
\newcommand{\R}{\mathbb{R}}

\newcommand{\derv}[2]{\dfrac{d#1}{d#2}}
\newcommand{\e}{\varepsilon}
\newcommand{\di}{\,/\,}

\newcommand{\lm}[1]{\displaystyle\lim_{#1}}

\begin{document}

\pagenumbering{gobble}

\pagestyle{fancy}
\fancyhead{}
\fancyhead[L]{Alex Agruso}
\fancyhead[R]{Abstract Algebra Homework 11}

\normalsize

\section*{Chapter 19}

\begin{itemize}
    \item [7.)] Suppose $a_1u+a_2(u+v)+a_3(u+v+w)=0$, thus we have that
    \[(a_1+a_2+a_3)u+(a_2+a_3)v+a_3w=0\]
    Since $u,v$, and $w$ are linearly independent, we have $a_1+a_2+a_3=0$,
    $a_2+a_3=0$, and $a_3=0$, so $a_2+a_3=a_2=0$, so $a_1+a_2+a_3=a_1=0$, thus
    $a_1=a_2=a_3=0$, thus $u$, $v$, and $w$ are linearly independent.
    \done

    \item [8.)] Suppose $a_1v_1+a_2v_2+\cdots+a_nv_n=0$. Since the vectors are
    linearly dependent, we can set $a_i\ne0$ for some $i$, so
    $a_1v_1+\cdots+a_{i-1}v_{i-1}+a_{i+1}v_{i+1}+\cdots+a_nv_n=-a_iv_i$.
    Dividing out by $-a_i$, we obtain $v_i$ as a linear combination of the
    other vectors. 
    \done

    \item [9.)] If $\braces{v_1,\,\cdots\,v_n}$ is linearly independent, we are
    done. Otherwise, there exists $v_i$ that is a linear combination of the
    other, so we can remove it from the set to obtain another spanning set.
    Repeat this until the set is linearly independent, at which point we have
    obtained a basis.
    \done

    \item [10.)] If the dimension of $V$ is $n$, we are done. Otherwise, set
    $w_i$ to be linearly dependent to all vectors in the set and add it to the
    set. This is possible if the dimension of $V$ is at least $n+i$. Repeat
    this process until we have $n+m$ vectors, all of which are linearly
    independent, thus we have obtained a basis for $V$.
    \done

    \item [22.)] Each vector has $n$ entries and each entry has $p$ choices,
    so the total number of elements in the vector space is $p^n$.
    \done
\end{itemize}

\end{document}
