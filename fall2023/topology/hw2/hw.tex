\documentclass[12pt]{article}
\pagenumbering{gobble}
\linespread{1.25}

\usepackage{amsfonts}
\usepackage{amsmath}
\usepackage{amssymb}
\usepackage{array}
\usepackage{fancyhdr}
\usepackage{mathrsfs}
\usepackage{mathtools}
\usepackage{textcomp}
\usepackage[margin=1in,headheight=1in]{geometry}

\newcommand{\contradiction}{
    \ensuremath{{\Rightarrow\mspace{-2mu}\Leftarrow}}
}

% bracket commands
\newcommand{\angleb}[1]{\left\langle#1\right\rangle} % <>
\newcommand{\vertb}[1]{\left\vert#1\right\vert}      % ||
\newcommand{\bracks}[1]{\left[#1\right]}             % []
\newcommand{\braces}[1]{\left\{#1\right\}}           % {}
\newcommand{\parens}[1]{\left(#1\right)}             % ()

% set aliases
\newcommand{\N}{\mathbb{N}}
\newcommand{\Z}{\mathbb{Z}}
\newcommand{\Q}{\mathbb{Q}}
\newcommand{\R}{\mathbb{R}}

\newcommand{\derv}[2]{\dfrac{d#1}{d#2}}
\newcommand{\e}{\varepsilon}
\newcommand{\di}{\,/\,}

\newcommand{\lm}[1]{\displaystyle\lim_{#1}}

\begin{document}
\pagestyle{fancy}
\fancyhead{}
\fancyhead[L]{Alex Agruso}
\fancyhead[R]{Topology Homework 2}

\normalsize

\section*{Definitions}
\begin{itemize}
    \item [1.)] Given a set $P$, a partial order on $P$ is a relation $\leq$ on $P$ that satisfies the following conditions:
    \begin{itemize}
        \item [1.] For all $p\in P$, $p\leq p$. (Reflexivity)
        \item [2.] For all $p,q\in P$, $p\leq q\land q\leq p\implies p=q$. (Antisymmetry)
        \item [3.] For all $p,q,r\in P$, $p\leq q\land q\leq r\implies p\leq r$. (Transitivity)
    \end{itemize}

    \item [2.)] Given a relation $R$ on a set $P$, $R$ is transitive if for all $p,q,r\in P$, $p\sim q\land q\sim r\implies p\sim r$.
    
    \item [3.)] The union of a collection of sets $\braces{S_\alpha}_{\alpha\in\mathcal{A}}$ is defined as the set $T$ where
    \[T=\braces{s:\exists\alpha\in \mathcal{A}\text{ where }s\in S_\alpha}\]

    \item [4.)] The intersection of a collection of sets $\braces{S_\alpha}_{\alpha\in\mathcal{A}}$ is defined as the set $T$ where
    \[T=\braces{s:s\in S_\alpha\text{ for all }\alpha\in\mathcal{A}}\]
\end{itemize}

\section*{Proofs}
\begin{itemize}
    \item [a.)] Since $\mathcal{P}\parens{\bracks{2}}=\braces{\braces{0},\braces{1},\braces{2},\braces{0,1},\braces{0,2},\braces{1,2},\braces{0,1,2},\varnothing}$, we can see that the only open subsets of $\bracks2$ are $\braces{2},\braces{1,2},\braces{0,1,2}$, and $\varnothing$.

    \item [b.)] Consider the elements of $\mathcal{P}(\braces{a,b})$. Given an open subset $S\subset\mathcal{P}(\braces{a,b})$, consider the implications of each element existing in $S$. Since $\varnothing\subset\braces{a},\varnothing\subset\braces{b}$, and $\varnothing\subset\braces{a,b}$, we can see that $\varnothing\in S\implies S=\braces{\varnothing,\braces{a},\braces{b},\braces{a,b}}$. Next, since $\braces{a}\subset\braces{a,b}$ and $\braces{b}\subset\braces{a,b}$, we can see that $\braces{a}\in S\lor\braces{b}\in S\implies\braces{a,b}\in S$. Finally, $\braces{a,b}\in S$ does not necessarily implicate the existance of any other element in $S$. From this, we find every open subset of $\mathcal{P}(\braces{a,b})$ to be $\braces{\braces{a,b}},\braces{\braces{a},\braces{a,b}},\braces{\braces{b},\braces{a,b}},\braces{\braces{a},\braces{b},\braces{a,b}},\break\braces{\varnothing,\braces{a},\braces{b},\braces{a,b}}$, and $\varnothing$.

    \item [c.)] Let $S$ be defined as follows:
    \[S=\bigcup_{\alpha\in\mathcal{A}}U_\alpha=\braces{x:\exists\alpha\in\mathcal{A}\text{ where }x\in U_\alpha}\]
    Let $p,q\in P$ where $p\leq q$ and $p\in S$. We can see that $p\in S\implies\exists\alpha\in\mathcal{A}$ where $p\in U_\alpha$. Since $U_\alpha$ is an open subset of $P$, and since $p\leq q$, we know that $q\in U_\alpha$, thus $q\in S$, thus $p\in S\land p\leq q\implies q\in S$, thus by definition, $S$ is an open subset of $P$. $\blacksquare$

    \item [d.)] Let $S$ be defined as follows:
    \[S=\bigcap_{\alpha\in\mathcal{A}}U_\alpha=\braces{x:x\in U_\alpha\text{ for all }\alpha\in\mathcal{A}}\]
    Let $p,q\in P$ where $p\leq q$ and $p\in S$. We can see that $p\in S\implies p\in U_\alpha$ for all $\alpha\in\mathcal{A}$. Since for all $\alpha$, $U_\alpha$ is an open subset of $P$, and since $p\leq q$, we know that $q\in U_\alpha$ for all $\alpha$, thus $q\in S$, thus $p\in S\land p\leq q\implies q\in S$, thus by definition, $S$ is an open subset of $P$. $\blacksquare$

    \item [e.)] 

\end{itemize}

\end{document}
