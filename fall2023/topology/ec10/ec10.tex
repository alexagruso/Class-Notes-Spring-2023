\documentclass[12pt]{article}
\pagenumbering{gobble}

\linespread{1.2}

\usepackage{amsfonts}
\usepackage{amsmath}
\usepackage{amssymb}
\usepackage{array}
\usepackage{fancyhdr}
\usepackage{mathrsfs}
\usepackage{mathtools}
\usepackage{parskip}
\usepackage{textcomp}
\usepackage[margin=1in,headheight=1in]{geometry}

\newcommand{\done}{
    \ensuremath{\strut\hfill\blacksquare}
}

\newcommand{\contradiction}{
    \ensuremath{{\Rightarrow\mspace{-2mu}\Leftarrow}}
}

\setlength{\parskip}{0.2in}

% bracket commands
\newcommand{\angleb}[1]{\left\langle#1\right\rangle} % <>
\newcommand{\vertb}[1]{\left\vert#1\right\vert}      % ||
\newcommand{\bracks}[1]{\left[#1\right]}             % []
\newcommand{\braces}[1]{\left\{#1\right\}}           % {}
\newcommand{\parens}[1]{\left(#1\right)}             % ()

% set aliases
\newcommand{\N}{\mathbb{N}}
\newcommand{\Z}{\mathbb{Z}}
\newcommand{\Q}{\mathbb{Q}}
\newcommand{\R}{\mathbb{R}}

\newcommand{\derv}[2]{\dfrac{d#1}{d#2}}
\newcommand{\e}{\varepsilon}
\newcommand{\di}{\,/\,}

\newcommand{\lm}[1]{\displaystyle\lim_{#1}}

\begin{document}
\pagestyle{fancy}
\fancyhead{}
\fancyhead[L]{Alex Agruso}
\fancyhead[R]{Topology Extra Credit 10}

\normalsize

A quick lemma that will be useful for this proof:
\begin{equation}\tag{$*$}
    (A\times B)\cap(C\times D)=(A\cap C)\times(B\cap D)\,.
\end{equation}
\textit{Proof:} Let $A,B,C,D$ be sets, and let $(x,y)\in(A\times B)\cap(C\times D)$, then $(x,y)\in A\times B$ and $(x,y)\in C\times D$, thus $x\in A$ and $x\in C$, and $y\in B$ and $y\in D$, thus $x\in A\cap C$ and $y\in B\cap D$, thus $(x,y)\in(A\cap C)\times(B\cap D)$. These statements are biconditional, so the converse is also proven. \done

Let $\mathcal{T}$ and $\mathcal{T}'$ be the topologies defined in the problem. If $\mathcal{T}$ and $\mathcal{T}'$ are equal, then a subset $U\subseteq A\times B$ is open in $\mathcal{T}$ if and only if it is open in $\mathcal{T}'$. Let $U$ be open in $\mathcal{T}$, then there exist collections $\braces{U_\alpha}_{\alpha\in\mathcal{A}}$ and $\braces{V_\alpha}_{\alpha\in\mathcal{A}}$ of open sets in $A$ and $B$ respectively where
\[U=\bigcup_{\alpha\in\mathcal{A}}U_\alpha\times V_\alpha\,.\]
Since for all $\alpha\in\mathcal{A}$, $U_\alpha$ and $V_\alpha$ are open in $A$ and $B$ respectively, we know that for each $\alpha$, there exist open sets $U'_\alpha\subseteq X$ and $V'_\alpha\subseteq Y$where $U_\alpha=A\cap U'_\alpha$ and $V_\alpha=B\cap V'_\alpha$. From this, it is clear that
\[U=\bigcup_{\alpha\in\mathcal{A}}(A\cap U'_\alpha)\times(B\cap V'_\alpha)\,,\]
thus by $(*)$, we have
\[U=\bigcup_{\alpha\in\mathcal{A}}(A\times B)\cap(U'_\alpha\times V'_\alpha)=(A\times B)\cap\bigcup_{\alpha\in\mathcal{A}}U'_\alpha\times V'_\alpha\,.\]
Since $\braces{U'_{\alpha}}_{\alpha\in\mathcal{A}}$ and $\braces{V'_{\alpha}}_{\alpha\in\mathcal{A}}$ are collections of open sets in $X$ and $Y$ respectively, we know that $\displaystyle\bigcup U'_\alpha\times V'_\alpha$ is open in $X\times Y$, thus by definition, $U$ is open in the subspace topology on $A\times B$, and thus $U$ is open in $\mathcal{T}'$.

Conversely, let $U\subseteq A\times B$ be open in $\mathcal{T}'$, then there exist collections of open sets $\braces{U_\alpha}_{\alpha\in\mathcal{A}}$ and $\braces{V_\alpha}_{\alpha\in\mathcal{A}}$ in $X$ and $Y$ respectively where
\[U=(A\times B)\cap\bigcup_{\alpha\in\mathcal{A}}U_\alpha\times V_\alpha\,,\]
thus by $(*)$,
\[U=\bigcup_{\alpha\in\mathcal{A}}(A\times B)\cap(U_\alpha\times V_\alpha)=\bigcup_{\alpha\in\mathcal{A}}(A\cap U_{\alpha})\times(B\cap V_\alpha)\,.\]
We know that for each $\alpha$, $A\cap U_\alpha$ and $B\cap V_\alpha$ are open in $A$ and $B$ respectively. Finally, let $\braces{U'_\alpha}_{\alpha\in\mathcal{A}}$ and $\braces{V'_\alpha}_{\alpha\in\mathcal{A}}$ be collections of sets where $U'_{\alpha}=A\cap U_\alpha$ and $V'_\alpha=A\cap V_\alpha$. It is clear that
\[U=\bigcup_{\alpha\in\mathcal{A}}U'_\alpha\times V'_\alpha\,,\]
and since for all $\alpha$, $U'_\alpha$ and $V'_\alpha$ are open in $A$ and $B$ respectively, we know that $U$ is open in the product topology on $A\times B$, thus $U$ is open in $\mathcal{T}$, and thus $\mathcal{T}=\mathcal{T}'$. \done

\end{document}