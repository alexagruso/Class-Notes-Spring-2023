\documentclass[12pt]{article}
\pagenumbering{gobble}

\linespread{1.2}

\usepackage{amsfonts}
\usepackage{amsmath}
\usepackage{amssymb}
\usepackage{array}
\usepackage{fancyhdr}
\usepackage{mathrsfs}
\usepackage{mathtools}
\usepackage{parskip}
\usepackage{textcomp}
\usepackage[margin=1in,headheight=1in]{geometry}

\newcommand{\done}{
    \ensuremath{\strut\hfill\blacksquare}
}

\newcommand{\contradiction}{
    \ensuremath{{\Rightarrow\mspace{-2mu}\Leftarrow}}
}

\setlength{\parskip}{0.2in}

% bracket commands
\newcommand{\angleb}[1]{\left\langle#1\right\rangle} % <>
\newcommand{\vertb}[1]{\left\vert#1\right\vert}      % ||
\newcommand{\bracks}[1]{\left[#1\right]}             % []
\newcommand{\braces}[1]{\left\{#1\right\}}           % {}
\newcommand{\parens}[1]{\left(#1\right)}             % ()

% set aliases
\newcommand{\N}{\mathbb{N}}
\newcommand{\Z}{\mathbb{Z}}
\newcommand{\Q}{\mathbb{Q}}
\newcommand{\R}{\mathbb{R}}

\newcommand{\derv}[2]{\dfrac{d#1}{d#2}}
\newcommand{\e}{\varepsilon}
\newcommand{\di}{\,/\,}

\newcommand{\lm}[1]{\displaystyle\lim_{#1}}

\begin{document}
\pagestyle{fancy}
\fancyhead{}
\fancyhead[L]{Alex Agruso}
\fancyhead[R]{Topology Homework 11}

\normalsize

\section*{Definitions}
\begin{itemize}
    \item [1.)] A \textit{metric} on a set $X$ is a function
    $d:X \times X \to \R$ that satisfies the following properties for all $x$,
    $x'$, and $x''$ in $X$:
    \begin{itemize}
        \item [i.)] $x = x' \iff d(x,x) = 0 \hfill \textit{Non-degeneracy}$

        \item [ii.)] $d(x,x')=d(x',x) \hfill \textit{Symmetry}$

        \item [iii.)] $d(x,x'') \leq d(x,x') + d(x',x'') \hfill
            \textit{Transitivity}$
    \end{itemize}

    \item [2.)] Given a metric $d$ on a set $X$, some $r \in \R^+$, and some
    $x\in X$, the \textit{open ball of radius $r$ centered at $x$} is defined
    as
    \[ \text{Ball}(x,r) = \braces{x' \in X : d(x,x') < r} . \]

    \item [3.)] Given a metric $d$ on a set $X$, the \textit{metric topology}
    on $X$ is defined as
    \[
        \mathcal{T}_d
        = \braces{
            U \in \mathcal{P}(X)
            : U = \bigcup_{\alpha \in \mathcal{A}} U_\alpha
        }
        ,
    \]
    where $\braces{U_\alpha}_{\alpha \in \mathcal{A}}$ is a collection of open
    balls in $X$.

    \item [4.)] A \textit{metric space} is a set $X$ equipped with a metric
    $d$.

    \item [5.)] The \textit{taxi cab metric} on $\R^n$ is defined as
    \[ d_{taxi}(x,y) = \sum_{k=1}^n \vertb{x_k - y_k} \]

    \item [6.)] The \textit{$l^\infty$ metric} on $\R^n$ is defined as
    \[ d_{l^\infty}(x,y) = \max_{k = 1, \dots, n} \vertb{x_k - y_k} \]

    \item [7.)] The \textit{standard metric} on $\R^n$ is defined as
    \[ d_{std}(x,y) = \sqrt{\sum_{k=1}^n (x_k - y_k)^2} \]

    \item [8.)] The \textit{discrete metric} on a set $X$ is defined as
    \[
        d_{discrete}(x,x') = \begin{cases}
            0 & \text{if}\ x = x' \\
            1 & \text{if}\ x \ne x'
        \end{cases}
    \]

    \item [9.)] An \textit{isometry} between two metric spaces $X$ and $Y$ is
    a bijection $f:X \to Y$ where for all $x$ and $x'$ in $X$, we have
    $d_X(x, x') = d_Y(f(x), f(x'))$.

    \item [10.)] A \textit{homeomorphism} between two topological spaces $X$
    and $Y$ is a function $f:X \to Y$ that is bijective and bicontinuous.

    \item [11.)] A topological space $X$ is \textit{path-connected} if for all
    $x$ and $x'$ in $X$, there exists a continuous function
    $\gamma : [0,1] \to X$ where $\gamma(0) = x$ and $\gamma(1) = x'$.

    \item [12.)] A topological space $X$ is \textit{connected} if when a
    subset $U$ of $X$ is both open and closed in $X$, we have that $U = X$ or
    $U = \varnothing$.

    \item [13.)] Define an equivalence relation $\sim$ on a topological space
    $X$ defined as $x \sim x'$ if and only if $x$ is path-connected to $x'$,
    then $\pi_0(X)$ is the quotient space $X/{\sim}$.

    \item [14.)] The \textit{invariance of domain} theorem states that $\R^m$
    is homeomorphic to $\R^n$ if and only if $m = n$.
\end{itemize}

\section*{Proofs}
    \begin{itemize}
        \item [a.)] The hypothesis for the proposition include that $X$ and
        $Y$ are topological spaces, that $X$ is connected, and that $X$ and
        $Y$ are homeomorphic.
        To my understanding, each proof uses pretty much every meaningful
        aspect of our hypotheses.
        The fact that open sets are referenced implies the existence of
        topological space, and both the bijectivity and continuity of the
        homeomorphism are used in the proofs.

        \item [b.)] After probing the problem for a while, I do not see any
        clear way to strengthen the proposition given our assumptions.

        \pagebreak
        \item [c.)] \textit{Claim:} If $X \cong Y$ and $X$ is connected, then
        $Y$ is connected.

        \textit{Proof:} By contrapositive, assume $Y$ is not connected, then
        there exist non-empty and disjoint open sets $V$ and $V'$ in $Y$ where
        $V \cup V' = Y$.
        Let $f:X \to Y$ be a homeomorphism. Since $f$ is a bijection, we know
        that $U=f^{-1}(V)$ and $U'=f^{-1}(V')$ are both non-empty and
        disjoint, and since $f$ is continuous, they are both also open in $X$.
        Finally, we have $U \cup U' = f^{-1}(V) \cup f^{-1}(V')
            = f^{-1}(V \cup V') = f^{-1}(Y) = X$, thus we have found non-empty
        and disjoint open sets $U$ and $U'$ whose union is $X$, thus $X$ is not
        connected.

        \item [d.)] I originally tried removing the assumption that $f$ was a
        bijection, assuming it was either injective or surjective but not
        both, but I was never able to reach a successful proof.
        I did believe that continuity was firmly required for connectedness to
        be preserved, so I did not explore proofs that did not assume
        continuity.
        
    \end{itemize}
\end{document}