\documentclass[12pt]{article}
\pagenumbering{gobble}
\linespread{1.2}

\usepackage{amsfonts}
\usepackage{amsmath}
\usepackage{amssymb}
\usepackage{array}
\usepackage{fancyhdr}
\usepackage{mathrsfs}
\usepackage{mathtools}
\usepackage{textcomp}
\usepackage[margin=1in,headheight=1in]{geometry}

\newcommand{\contradiction}{
    \ensuremath{{\Rightarrow\mspace{-2mu}\Leftarrow}}
}

\newcommand{\angleb}[1]{\left\langle#1\right\rangle}
\newcommand{\vertb}[1]{\left\vert#1\right\vert}
\newcommand{\bracks}[1]{\left[#1\right]}
\newcommand{\parns}[1]{\left(#1\right)}

\newcommand{\derv}[2]{\dfrac{d#1}{d#2}}
\newcommand{\e}{\varepsilon}
\newcommand{\di}{\,/\,}

\newcommand{\lm}[1]{\displaystyle\lim_{#1}}

\begin{document}
\pagestyle{fancy}
\fancyhead{}
\fancyhead[L]{Alex Agruso}
\fancyhead[R]{Topology Writing 1}

\normalsize

When I took number theory, we never got the chance to cover quadratic reciprocity. I vaguely knew what the concept was but didn't have a full understanding. Despite this, I was interested in learning more about it. My current understanding of quadratic reciprocity is that it has something to do with the integers mod $n$ or mod $p$, and an identity involving them.

After looking up what quadratic reciprocity was, I was given the following identity from wolfram.com:

Let $p$ and $q$ be distinct odd primes, and define the legendre symbol as
\[\parns{\frac{p}{q}}=\begin{cases}
    1 & \text{if }x^2\equiv p\text{ (mod}\,q)\text{ for some }x\in\mathbb{Z}_p\\
    -1 & \text{otherwise}
\end{cases}\]

Then the following identity is true:

\[\parns{\frac{p}{q}}\parns{\frac{q}{p}}=(-1)^{\frac{p-1}{2}\frac{q-1}{2}}\]

I had forgotten that this identity uses the Legendre symbol. I understand the definition of the Legendre symbol, but I don't see the motivation for it, other than it being related to primes. My best guess is something to do with generators mod $p$, as I know my number theory class talked a lot about them. Despite this gap in my understanding, I do generally have a better grasp of what quadratic reciprocity means after looking it up.

\end{document}
