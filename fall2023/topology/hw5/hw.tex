\documentclass[11pt]{article}
\pagenumbering{gobble}
\linespread{1.3}

\usepackage{amsfonts}
\usepackage{amsmath}
\usepackage{amssymb}
\usepackage{array}
\usepackage{fancyhdr}
\usepackage{mathrsfs}
\usepackage{mathtools}
\usepackage{textcomp}
\usepackage[margin=1in,headheight=1in]{geometry}

\newcommand{\contradiction}{
    \ensuremath{{\Rightarrow\mspace{-2mu}\Leftarrow}}
}

\setlength{\parskip}{0.2in}

% bracket commands
\newcommand{\angleb}[1]{\left\langle#1\right\rangle} % <>
\newcommand{\vertb}[1]{\left\vert#1\right\vert}      % ||
\newcommand{\bracks}[1]{\left[#1\right]}             % []
\newcommand{\braces}[1]{\left\{#1\right\}}           % {}
\newcommand{\parens}[1]{\left(#1\right)}             % ()

% set aliases
\newcommand{\N}{\mathbb{N}}
\newcommand{\Z}{\mathbb{Z}}
\newcommand{\Q}{\mathbb{Q}}
\newcommand{\R}{\mathbb{R}}

\newcommand{\derv}[2]{\dfrac{d#1}{d#2}}
\newcommand{\e}{\varepsilon}
\newcommand{\di}{\,/\,}

\newcommand{\lm}[1]{\displaystyle\lim_{#1}}

\begin{document}
\pagestyle{fancy}
\fancyhead{}
\fancyhead[L]{Alex Agruso}
\fancyhead[R]{Topology Homework 5}

\normalsize

\section*{Definitions}
\begin{itemize}
    \item [1.)] Given a set $X$, a \textit{topology} on $X$ is a collection of sets $\mathcal{T}\subset\mathcal{P}(X)$ such that
    \begin{itemize}
        \item [i.)] $\varnothing,X\in\mathcal{T}$
        
        \item [ii.)] Given a collection $\braces{U_\alpha}_{\alpha\in\mathcal{A}}$ of sets in $\mathcal{T}$, $\displaystyle\bigcup_{\alpha\in\mathcal{A}}U_\alpha\in{T}$

        \item [iii.)] Given a finite collection $\braces{U_\alpha}_{\alpha\in\mathcal{A}}$ of sets in $\mathcal{T}$, $\displaystyle\bigcap_{\alpha\in\mathcal{A}}U_\alpha\in\mathcal{T}$
    \end{itemize}

    \item [2.)] Given a set $X$ and a topology $\mathcal{T}$ on $X$, $(X,\mathcal{T})$ is a \textit{topological space}.

    \item [3.)] Given a topological space $(X,\mathcal{T})$, an \textit{open subset} of $X$ is a subset $U\subset X$ where $U\in\mathcal{T}$.

    \item [4.)] Given topological spaces $(X,\mathcal{T}_X)$ and $(Y,\mathcal{T}_Y)$ the function $f:X\to Y$ is \textit{continuous} if for all $U\in\mathcal{T}_Y$, $f^{-1}(U)\in\mathcal{T}_X$.

    \item [5.)] Given a set $X$, the collection $\braces{U_\alpha}_{\alpha\in\mathcal{A}}$ of subsets of $X$ is a \textit{cover} of $X$ if $\displaystyle\bigcup_{\alpha\in\mathcal{A}}U_\alpha=X$

    \item [6.)] Given a cover $\braces{U_\alpha}_{\alpha\in\mathcal{A}}$ of $X$, $\braces{U_\alpha}_{\alpha\in\mathcal{A}}$ is an \textit{open cover} if $U_\alpha$ is open in $X$ for all $\alpha\in\mathcal{A}$.

    \item [7.)] Given a cover $\braces{U_\alpha}_{\alpha\in\mathcal{A}}$ of $X$, a \textit{finite subcover} of $\braces{U_\alpha}_{\alpha\in\mathcal{A}}$ is a cover $\braces{U_\beta}_{\beta\in\mathcal{B}}$ of $X$ where $\mathcal{B}\subset\mathcal{A}$ and $\mathcal{B}$ is finite.

    \item [8.)] Given a topological space $(X,\mathcal{T})$, it is \textit{compact} if for all open covers $\braces{U_\alpha}_{\alpha\in\mathcal{A}}$ of $X$, there exists a finite subcover $\braces{U_\beta}_{\beta\in\mathcal{B}}$ of $\braces{U_\alpha}_{\alpha\in\mathcal{A}}$.
\end{itemize}

\section*{Proof}
Assume $f$ is continuous and let $V\subset Y$ be open. Fix $y\in V$, thus there exists $\e>0$ where $\text{Ball}(y,\e)\subset V$. Since Ball$(y,\e)$ is open, $f^{-1}(\text{Ball}(y,\e))$ is open, thus there exists $\delta>0$ where $\text{Ball}(x,\delta)\subset f^{-1}(\text{Ball}(y,\e))$, thus for all $x'\in\text{Ball}(x,\delta)$ we know that $x'\in f^{-1}(\text{Ball}(y,\e))$, and thus $f(x')\in\text{Ball}(y,\e)$.

\noindent
Assuming (b), fix $\e>0$, thus for some $\delta>0$, $x'\in\text{Ball}(x,\delta)\implies f(x')\in\text{Ball}(y,\e)$ for\break arbitrary $x\in X$ and $y\in Y$. We can see that $f(x')\in\text{Ball}(y,\e)\implies x'\in f^{-1}(\text{Ball}(y,\e))$, thus $x'\in\text{Ball}(x,\delta)\implies x'\in f^{-1}(\text{Ball}(y,\e))$, thus $\text{Ball}(x,\delta)\subset f^{-1}(\text{Ball}(y,\e))$. Let $V\subset Y$ be open where $\text{Ball}(y,\e)\subset V$, thus $f^{-1}(\text{Ball}(y,\e)\subset f^{-1}(V))$, thus $\text{Ball}(x,\delta)\subset f^{-1}(V)$. Since $x$ is arbitrary, we know that $\text{Ball}(x,\delta)\subset f^{-1}(V)$ for all $x\in f^{-1}(V)$, thus $f^{-1}(V)$ is open.

\noindent
It follows that (a) and (b) are equivalent. $\blacksquare$

\end{document}
