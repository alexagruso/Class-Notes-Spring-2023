\documentclass[12pt]{article}
\pagenumbering{gobble}
\linespread{1.1}

\usepackage{amsfonts}
\usepackage{amsmath}
\usepackage{amssymb}
\usepackage{array}
\usepackage{fancyhdr}
\usepackage{mathrsfs}
\usepackage{mathtools}
\usepackage{parskip}
\usepackage{textcomp}
\usepackage[margin=1in,headheight=1in]{geometry}

\usepackage{tikz}
\usetikzlibrary{animations}
\usetikzlibrary{arrows.meta}
\usetikzlibrary{calc}
\usetikzlibrary{shapes.geometric}

\graphicspath{ {./images/} }

\newcommand{\contradiction}{
    \ensuremath{{\Rightarrow\mspace{-2mu}\Leftarrow}}
}

% bracket commands
\newcommand{\angleb}[1]{\left\langle#1\right\rangle} % <>
\newcommand{\vertb}[1]{\left\vert#1\right\vert}      % ||
\newcommand{\bracks}[1]{\left[#1\right]}             % []
\newcommand{\braces}[1]{\left\{#1\right\}}           % {}
\newcommand{\parens}[1]{\left(#1\right)}             % ()

% set aliases
\newcommand{\N}{\mathbb{N}}
\newcommand{\Z}{\mathbb{Z}}
\newcommand{\Q}{\mathbb{Q}}
\newcommand{\R}{\mathbb{R}}

\newcommand{\derv}[2]{\dfrac{d#1}{d#2}}
\newcommand{\e}{\varepsilon}
\newcommand{\di}{\,/\,}

\newcommand{\lm}[1]{\displaystyle\lim_{#1}}

\begin{document}
\pagestyle{fancy}
\fancyhead{}
\fancyhead[L]{Alex Agruso}
\fancyhead[R]{Topology Writing 11}

\normalsize

So far, the concept that has given me the most trouble in this class has been
the product topology and product spaces.
I feel like I understand the definition well enough, but when it comes to
extracting properties of open sets in a product space, I feel that I tend to
make assumptions based on my intuition of these spaces that turn out to be
false.
This in turn leads to me doubting my own intuition of how these spaces behave.
Because of this, I am aiming to establish a better intuition of product spaces
to accompany my understanding of their definition.

The two examples I chose to work were the past two extra credit problems, as
they were both challenging problems that involved product spaces.
On submission, extra credit problem 9 had an error.
I falsely assumed that
\[W_\alpha = U_\alpha \times V_\alpha\]
where $W_\alpha$ was an open set in some product topology, and where
$U_\alpha$ and $V_\alpha$ were unions of the respective collections of opens
set that comprised $W_\alpha$.
It turns out this equality does not hold, as the relationships between the
individual sets in the collections are destroyed when the union is taken, so
certain elements that did not exist in any pair of sets before might exist in
the union.
As a result, my proof was invalid, but this new insight into a property of
sets in product topologies has led to a more accurate intuition of how they
behave.
As for extra credit problem 10, I didn't make any errors, but it did serve as
a good exercise in juggling different types of topologies, including the
product topology, at the same time.

Overall, this exposure to working with product spaces has strengthened not
only my understanding of the definition, but has also corrected some errors in
my previous intuition.
The error I made in problem 9 was effective in highlighting a gap in my
intuition that I was then able to fix, and problem 10 served as a good
exercise to strengthen my knowledge of how the product topology interacts with
other topologies.

\end{document}
