\documentclass[12pt]{article}
\pagenumbering{gobble}
\linespread{1.25}

\usepackage{amsfonts}
\usepackage{amsmath}
\usepackage{amssymb}
\usepackage{array}
\usepackage{fancyhdr}
\usepackage{mathrsfs}
\usepackage{mathtools}
\usepackage{textcomp}
\usepackage[margin=1in,headheight=1in]{geometry}

\newcommand{\contradiction}{
    \ensuremath{{\Rightarrow\mspace{-2mu}\Leftarrow}}
}

% bracket commands
\newcommand{\angleb}[1]{\left\langle#1\right\rangle} % <>
\newcommand{\vertb}[1]{\left\vert#1\right\vert}      % ||
\newcommand{\bracks}[1]{\left[#1\right]}             % []
\newcommand{\braces}[1]{\left\{#1\right\}}           % {}
\newcommand{\parens}[1]{\left(#1\right)}             % ()

% set aliases
\newcommand{\N}{\mathbb{N}}
\newcommand{\Z}{\mathbb{Z}}
\newcommand{\Q}{\mathbb{Q}}
\newcommand{\R}{\mathbb{R}}

\newcommand{\derv}[2]{\dfrac{d#1}{d#2}}
\newcommand{\e}{\varepsilon}
\newcommand{\di}{\,/\,}

\newcommand{\lm}[1]{\displaystyle\lim_{#1}}

\begin{document}
\pagestyle{fancy}
\fancyhead{}
\fancyhead[L]{Alex Agruso}
\fancyhead[R]{Topology Writing 5}

\normalsize

So far, an idea that has been covered but that I do not fully understand is the concept of the subspace topology. I understand the definition of the subspace topology on a subset of $X$, and I can grasp the proof behind why such a set is always a topology, but the one thing I lack is an intuitive understanding of \textit{why} it is a topology. Something about the nature of the definition prevents me from visualizing it, and because of this I feel less confident when applying this idea to problems.

After spending some time reviewing the subspace topology, my understanding for it has definitely strengthened, and for the time being I feel comfortable with the concept. While studying the concept, I sketched a few examples of subsets of $\R^2$, considering both open and closed subsets. From these sketches I realized that the openness and closedness of the subset does not affect how you create the subspace topology. This seems obvious, as that property of the subset is not used in the definition of the subspace topology, but somehow the two concepts were linked in my mind, causing confusion. I chalk this up to the general definition of openness also being related to topological spaces. Going forward, I will attempt to be more diligent in my recollection of definitions, especially if there are multiple that are very similar or related to one other.


\end{document}
