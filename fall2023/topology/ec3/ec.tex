\documentclass[11pt]{article}
\pagenumbering{gobble}
\linespread{1.25}

\usepackage{amsfonts}
\usepackage{amsmath}
\usepackage{amssymb}
\usepackage{array}
\usepackage{fancyhdr}
\usepackage{mathrsfs}
\usepackage{mathtools}
\usepackage{textcomp}
\usepackage[margin=1in,headheight=1in]{geometry}

\newcommand{\contradiction}{
    \ensuremath{{\Rightarrow\mspace{-2mu}\Leftarrow}}
}

% bracket commands
\newcommand{\angleb}[1]{\left\langle#1\right\rangle} % <>
\newcommand{\vertb}[1]{\left\vert#1\right\vert}      % ||
\newcommand{\bracks}[1]{\left[#1\right]}             % []
\newcommand{\braces}[1]{\left\{#1\right\}}           % {}
\newcommand{\parens}[1]{\left(#1\right)}             % ()

% set aliases
\newcommand{\N}{\mathbb{N}}
\newcommand{\Z}{\mathbb{Z}}
\newcommand{\Q}{\mathbb{Q}}
\newcommand{\R}{\mathbb{R}}

\newcommand{\derv}[2]{\dfrac{d#1}{d#2}}
\newcommand{\e}{\varepsilon}
\newcommand{\di}{\,/\,}

\newcommand{\lm}[1]{\displaystyle\lim_{#1}}

\begin{document}
\pagestyle{fancy}
\fancyhead{}
\fancyhead[L]{Alex Agruso}
\fancyhead[R]{Topology Extra Credit 3}

\normalsize

\begin{itemize}
    \item [a.)] Let $P$ and $Q$ be posets, and $f:P\times Q\to P$ and $g:P\times Q\to Q$ be defined as follows:
    \[f(p,q)=p,\ g(p,q)=q\]
    Let $(p_1,q_1),(p_2,q_2)\in P\times Q$ where $(p_1,q_1)\leq(p_2,q_2)$, thus $p_1\leq p_2$ and $q_1\leq q_2$. From this we can see that
    \[f(p_1,q_1)=p_1,\ f(p_2,q_2)=p_2\]
    thus
    \[(p_1,q_1)\leq (p_2,q_2)\implies p_1\leq p_2\implies f(p_1,q_1)\leq f(p_2,q_2)\]
    thus $f$ is a map of posets. Similary, we can see that
    \[g(p_1,q_1)=q_1,\ g(p_2,q_2)=q_2\]
    thus
    \[(p_1,q_1)\leq (p_2,q_2)\implies q_1\leq q_2\implies g(p_1,q_1)\leq g(p_2,q_2)\]
    thus $g$ is a map of posets, thus the projection maps $f$ and $g$ are both maps of posets. $\blacksquare$
    
    \item [b.)] Let $W$ be a poset, and consider an arbitrary poset map $h:W\to P\times Q$. Given $w_1,w_2\in W$ where $w_1\leq w_2$, we know that $h(w_1)\leq h(w_2)$, thus there exist $p_1,p_2\in P$ and $q_1,q_2\in Q$ where $h(w_1)=(p_1,q_1)$, $h(w_2)=(p_2,q_2)$, and $(p_1,q_1)\leq(p_2,q_2)$. Since the projection maps $f$ and $g$, defined above, are maps of posets, we know that $(p_1,q_1)\leq(p_2,q_2)\implies f(p_1,q_1)\leq f(p_2,q_2)$ and $g(p_1,q_1)\leq g(p_2,q_2)$. From this we can see that $w_1\leq w_2\implies h(w_1)\leq h(w_2)\implies f(h(w_1))\leq f(h(w_2))$ and $g(h(w_1))\leq g(h(w_2))$, thus the composition of any poset map with the projection maps $f$ and $g$ is a poset map. $\blacksquare$

    \item [c.)] Given the projection maps $f$ and $g$ as defined above, define $j$ as follows:
    \[j:\text{hom}(W,P\times Q)\to\text{hom}(W,P)\times\text{hom}(W,Q)\]
    \[\text{where }j(h)=(f(h),g(h))\text{ given }h:W\to P\times Q\]
    From this we can see that composition by the projection maps allows us to define a function $j$ between the two sets.

    \pagebreak
    \item [d.)] First we will show that $j$ is injective. Let $h_1,h_2\in\text{hom}(W,P\times Q)$, and assume that given $w\in W$, we have $j(h_1(w))=j(h_2(w))$, thus $(f(h_1(w)),g(h_1(w)))=(f(h_2(w)),g(h_2(w)))$, and thus $f(h_1(w))=f(h_2(w))$ and $g(h_1(w))=g(h_2(w))$. We know that $h_1(w)=(p_1,q_1)$ and $h_2(w)=(p_2,q_2)$ for some $p_1,p_2\in P$ and $q_1,q_2\in Q$, thus $f(p_1,q_1)=f(p_2,q_2)$ and $g(p_1,q_1)=g(p_2,q_2)$, thus $p_1=p_2$ and $q_1=q_2$, thus $h_1(w)=h_2(w)$ for all $w\in W$, thus given $h_1,h_2\in\text{hom}(W,P\times Q)$, $j(h_1(w))=j(h_2(w))\implies h_1(w)=h_2(w)$ for all $w\in W$, thus $j$ is injective.

    Next we will show that $j$ is surjective. Given $(f(h),g(h))\in\text{hom}(W,P)\times\text{hom}(W,Q)$ for some $h\in\text{hom}(W,P\times Q)$, we must find $h'\in\text{hom}(W,P\times Q)$ where $j(h')=(f(h),g(h))$. Let $h'(w)=h(w)$ for all $w\in W$, then $j(h')=(f(h'(w)),g(h'(w)))=(f(h(w)),g(h(w)))$, thus for all $(f(h),g(h))\in\text{hom}(W,P)\times\text{hom}(W,Q)$, there exists $h'\in\text{hom}(W,P\times Q)$ where $j(h')=(f(h),g(h))$, thus $j$ is surjective.

    Since $j$ is both injective and surjective, it is bijective. $\blacksquare$
\end{itemize}

\end{document}
