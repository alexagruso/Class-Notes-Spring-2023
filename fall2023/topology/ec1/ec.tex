\documentclass[11pt]{article}
\pagenumbering{gobble}
\linespread{1.25}

\usepackage{amsfonts}
\usepackage{amsmath}
\usepackage{amssymb}
\usepackage{array}
\usepackage{fancyhdr}
\usepackage{mathrsfs}
\usepackage{mathtools}
\usepackage{textcomp}
\usepackage[margin=1in,headheight=1in]{geometry}

\newcommand{\contradiction}{
    \ensuremath{{\Rightarrow\mspace{-2mu}\Leftarrow}}
}

% bracket commands
\newcommand{\angleb}[1]{\left\langle#1\right\rangle} % <>
\newcommand{\vertb}[1]{\left\vert#1\right\vert}      % ||
\newcommand{\bracks}[1]{\left[#1\right]}             % []
\newcommand{\braces}[1]{\left\{#1\right\}}           % {}
\newcommand{\parens}[1]{\left(#1\right)}             % ()

% set aliases
\newcommand{\N}{\mathbb{N}}
\newcommand{\Z}{\mathbb{Z}}
\newcommand{\Q}{\mathbb{Q}}
\newcommand{\R}{\mathbb{R}}

\newcommand{\derv}[2]{\dfrac{d#1}{d#2}}
\newcommand{\e}{\varepsilon}
\newcommand{\di}{\,/\,}

\newcommand{\lm}[1]{\displaystyle\lim_{#1}}

\begin{document}
\pagestyle{fancy}
\fancyhead{}
\fancyhead[L]{Alex Agruso}
\fancyhead[R]{Topology Extra Credit 1}

\normalsize

In $\R^3$, we can define a tetrahedron as a polyhedron having four equilateral triangle faces, each of which is adjacent to the other three. First we will show that the boundary of $\Delta^3$ consists of four equilateral triangles. After, we will show each triangle is adjacent.

Given the following definition of $\Delta^3$:
\[\Delta^3=\braces{(x_1,x_2,x_3,x_4)\in\mathbb{R}^4:\sum_{i=1}^4x_i=1\text{ and }x_i\geq0}\]
let $j\in\braces{1,2,3,4}$ and $\mathcal{F}_j$ be defined as follows:
\[\mathcal{F}_j=\braces{x\in\Delta^3:i=j\implies x_i=0}\]
If we can show bijections $f:\mathcal{F}_j\to\Delta^2$ for all $j$, then we know that $\mathcal{F}_j\subset\Delta^3$ is an equilateral triangle, and since there are 4 distinct $\mathcal{F}_j$, we can conclude that the boundary of $\Delta^3$ consists of 4 equilateral triangles.

Consider the mapping $f:\mathcal{F}_1\to\Delta^2$ where $(0,x_2,x_3,x_4)\mapsto(x_2,x_3,x_4)$. We can first show that $f$ is injective. Let $a,b\in\mathcal{F}_1$ where $a=(0,a_1,a_2,a_3)$ and $b=(0,b_1,b_2,b_3)$, and where $f(a)=f(b)$. Since $f(a)=(a_1,a_2,a_3)$ and $f(b)=(b_1,b_2,b_3)$, we can conclude that $(a_1,a_2,a_3)=(b_1,b_2,b_3)$, thus $f$ is injective. Next we can show that $f$ is surjective. Let $a\in\Delta^2$ where $a=(a_1,a_2,a_3)$ and $b\in\mathcal{F}_1$ where $b=(0,a_1,a_2,a_3)$, then $f(b)=(a_1,a_2,a_3)=a$, thus for all $a\in\Delta^2$, there exists $b\in\mathcal{F}_1$ where $f(b)=a$, thus $f$ is a surjective, and thus $f:\mathcal{F}_1\to\Delta^2$ is a bijection. Without loss of generality, we can conclude that this holds for all $j$.

To show that any two triangles $\mathcal{F}_{j},\mathcal{F}_{k}$ for $j\ne k$ are adjacent, we can show that their intersection is a line. Consider $\mathcal{F}_1$ and $\mathcal{F}_2$:
\[\mathcal{F}_1=\braces{(x_1,x_2,x_3,x_4)\in\Delta^3:x_1=0}\]
\[\mathcal{F}_2=\braces{(x_1,x_2,x_3,x_4)\in\Delta^3:x_2=0}\]
and then $\mathcal{F}_1\cap \mathcal{F}_2$:
\[\mathcal{F}_1\cap \mathcal{F}_2=\braces{a:a\in \mathcal{F}_1\land a\in \mathcal{F}_2}=\braces{(x_1,x_2,x_3,x_4)\in\Delta^3:x_1=x_2=0}\]
Since $(0,0,x_3,x_4)\in\Delta^3$, we know that $x_3+x_4=1$, which is the equation of a line, thus we know that $\mathcal{F}_1\cap \mathcal{F}_2$ are adjacent, and without loss of generality, $\mathcal{F}_j$ and $\mathcal{F}_k$ are adjacent for $j\ne k$.

Since we have shown that the boundary of $\Delta^3$ consists of four equilateral triangles, and that each triangle $\mathcal{F}_j\subset\Delta^3$ is adjacent to the others, we can conclude that according to our definition of a tetrahedron, $\Delta^3$ is a tetrahedron. $\blacksquare$

\end{document}
