\documentclass[12pt]{article}
\pagenumbering{gobble}

\linespread{1.25}

\usepackage{amsfonts}
\usepackage{amsmath}
\usepackage{amssymb}
\usepackage{array}
\usepackage{fancyhdr}
\usepackage{mathrsfs}
\usepackage{mathtools}
\usepackage{parskip}
\usepackage{textcomp}
\usepackage[margin=1in,headheight=1in]{geometry}

\newcommand{\done}{
    \ensuremath{\strut\hfill\blacksquare}
}

\newcommand{\contradiction}{
    \ensuremath{{\Rightarrow\mspace{-2mu}\Leftarrow}}
}

\setlength{\parskip}{0.2in}

% bracket commands
\newcommand{\angleb}[1]{\left\langle#1\right\rangle} % <>
\newcommand{\vertb}[1]{\left\vert#1\right\vert}      % ||
\newcommand{\bracks}[1]{\left[#1\right]}             % []
\newcommand{\braces}[1]{\left\{#1\right\}}           % {}
\newcommand{\parens}[1]{\left(#1\right)}             % ()

% set aliases
\newcommand{\N}{\mathbb{N}}
\newcommand{\Z}{\mathbb{Z}}
\newcommand{\Q}{\mathbb{Q}}
\newcommand{\R}{\mathbb{R}}

\newcommand{\derv}[2]{\dfrac{d#1}{d#2}}
\newcommand{\e}{\varepsilon}
\newcommand{\di}{\,/\,}

\newcommand{\lm}[1]{\displaystyle\lim_{#1}}

\begin{document}
\pagestyle{fancy}
\fancyhead{}
\fancyhead[L]{Alex Agruso}
\fancyhead[R]{Topology Homework 8}

\normalsize

\section*{Definitions}
\begin{itemize}
    \item [1.)] Given sets $X$ and $Y$, the \textit{direct product} or \textit{cartesian product} $X\times Y$ is defined as
    \[X\times Y=\braces{(x,y):x\in X\land y\in Y}\]

    \item [2.)] Given topological spaces $X$ and $Y$, the \textit{product topology} $\mathcal{T}_p$ on $X\times Y$ is defined as
    \[\mathcal{T}_p=\braces{U\subseteq X\times Y:U=\bigcup_{\alpha\in\mathcal{A}}U_\alpha\times V_\alpha}\]
    given $\braces{U_\alpha}_{\alpha\in\mathcal{A}}$ and $\braces{V_\alpha}_{\alpha\in\mathcal{A}}$ are collections of open sets in $X$ and $Y$ respectively.

    \item [3.)] Given topological spaces $X$ and $Y$, the product topology on $X\times Y$ satisfies the following:
    \begin{itemize}
        \item [i.)] The projection maps $p_X:X\times Y\to X$ and $p_Y:X\times Y\to Y$ are continuous.

        \item [ii.)] Given a topological space $W$ and a pair of continuous functions $g_X:W\to X$ and $g_Y:W\to Y$, there exists a unique continuous function $g:W\to X\times Y$ where $g_X=p_X\circ g$ and $g_Y=p_Y\circ g$.
    \end{itemize}

    \item [4.)] Given a collection of sets $\braces{X_\alpha}_{\alpha\in\mathcal{A}}$, the \textit{direct product} $\displaystyle\prod_{\alpha\in\mathcal{A}}X_\alpha$ is defined as
    \[\prod_{\alpha\in\mathcal{A}}X_\alpha=\braces{\braces{x_\alpha}_{\alpha\in\mathcal{A}}:x_\alpha\in X_\alpha}\]

    \item [5.)] Given a collection of topological spaces $\braces{X_\alpha}_{\alpha\in\mathcal{A}}$, the product topology on $\displaystyle\prod_{\alpha\in\mathcal{A}}X_\alpha$ satisfies the following:
    \begin{itemize}
        \item [i.)] For all $\alpha\in\mathcal{A}$, the projection map $p_\alpha:\prod_{\alpha\in\mathcal{A}}X_\alpha\to X_\alpha$ is continuous.

        \item [ii.)] Given a topological space $W$ and a collection of continuous functions $\braces{g_\alpha}_{\alpha\in\mathcal{A}}$ where $g_\alpha:W\to X_\alpha$, there exists a unique continuous function $g:W\to\prod_{\alpha\in\mathcal{A}}X_\alpha$ where $g_\alpha=p_\alpha\circ g$.
    \end{itemize}

    \item [6.)] A topological space $X$ is \textit{Hausdorff} if and only if given distinct $x,x'\in X$, there exist open sets $U,V$ in $X$ where $x\in U$, $x'\in V$, and $U\cap V=\varnothing$.
\end{itemize}

\pagebreak
\section*{Proofs}
\begin{itemize}
    \item [a.)] A topological space $X$ is Hausdorff if and only if its diagonal is closed under the product topology.

    \item [b.)] Because this conjecture holds for every example we covered in class, I am choosing to pursue a proof of it.

    \item [c.)] Let $X$ be a topological space, and suppose it is Hausdorff. Let $D^\complement$ denote the complement of the diagonal of $X$, thus $D^\complement=\braces{(x,x')\in X\times X:x\ne x'}$. Since $X$ is Hausdorff, we know that for all $(x,x')\in D^\complement$, there exist open sets $U,V$ in $X$ where $x\in U$, $x'\in V$, and $U\cap V=\varnothing$, thus $(x,x')\in U\times V$. In addition, since $U\cap V=\varnothing$, $(x,x')\in U\times V\implies x\ne x'$. For each point $\alpha\in D^\complement$, assign open sets $U_\alpha$ and $V_\alpha$ that satisfy the Hausdorff property, thus $\alpha\in U_\alpha\times V_\alpha$, thus $D^\complement\subseteq\displaystyle\bigcup U_\alpha\times V_\alpha$. Finally, if $\alpha\in U_\alpha\times V_\alpha$, then $\alpha=(x,x')$ where $x\ne x'$, thus $\alpha\in D^\complement$, thus $\displaystyle\bigcup U_\alpha\times V_\alpha\subseteq D^\complement$, thus $D^\complement=\displaystyle\bigcup U_\alpha\times V_\alpha$, thus $D^\complement$ is open, and thus the diagonal of $X$ is closed.

    Next, suppose the diagonal of $X$ is closed, then $D^\complement$ is open, thus there exist collections of open sets $\braces{U_\alpha}_{\alpha\in\mathcal{A}}$ and $\braces{V_\alpha}_{\alpha\in\mathcal{A}}$ where $D^\complement=\displaystyle\bigcup U_\alpha\times V_\alpha$. It is clear that for all $\alpha\in\mathcal{A}$, $U_\alpha\times V_\alpha\subseteq D^\complement$, thus $(x,x')\in U_\alpha\times V_\alpha\implies x\ne x'$, thus $U_\alpha\cap V_\alpha=\varnothing$. Finally, since $(x,x')\in U_\alpha\times V_\alpha$, we know that $x\in U_\alpha$ and $x'\in V_\alpha$, thus for all distinct $x,x'\in X$, the Hausdorff property is satisfied, thus $X$ is Hausdorff.

    Thus we can conclude that $X$ is Hausdorff if and only if the diagonal of $X$ is closed. $\newline\strut\hfill\blacksquare$

    \item [d.)] This property stuck out to me when we were looking at examples, so not much reworking of ideas was needed before settling on trying to prove it.
\end{itemize}

\end{document}