\documentclass[12pt]{article}
\pagenumbering{gobble}
\linespread{1.3}

\usepackage{amsfonts}
\usepackage{amsmath}
\usepackage{amssymb}
\usepackage{array}
\usepackage{fancyhdr}
\usepackage{mathrsfs}
\usepackage{mathtools}
\usepackage{textcomp}
\usepackage[margin=1in,headheight=1in]{geometry}

\newcommand{\contradiction}{
    \ensuremath{{\Rightarrow\mspace{-2mu}\Leftarrow}}
}

\setlength{\parskip}{0.2in}

% bracket commands
\newcommand{\angleb}[1]{\left\langle#1\right\rangle} % <>
\newcommand{\vertb}[1]{\left\vert#1\right\vert}      % ||
\newcommand{\bracks}[1]{\left[#1\right]}             % []
\newcommand{\braces}[1]{\left\{#1\right\}}           % {}
\newcommand{\parens}[1]{\left(#1\right)}             % ()

% set aliases
\newcommand{\N}{\mathbb{N}}
\newcommand{\Z}{\mathbb{Z}}
\newcommand{\Q}{\mathbb{Q}}
\newcommand{\R}{\mathbb{R}}

\newcommand{\derv}[2]{\dfrac{d#1}{d#2}}
\newcommand{\e}{\varepsilon}
\newcommand{\di}{\,/\,}

\newcommand{\lm}[1]{\displaystyle\lim_{#1}}

\begin{document}
\pagestyle{fancy}
\fancyhead{}
\fancyhead[L]{Alex Agruso}
\fancyhead[R]{Topology Homework 6}

\normalsize

\section*{Definitions}
\begin{itemize}
    \item [1.)] Given a relation $\sim$ on $X$, it is an \textit{equivalence relation} if
    \begin{itemize}
        \item [i.] For all $a\in X$, $a\sim a$.

        \item [ii.] For all $a,b\in X$, $a\sim b\implies b\sim a$.

        \item [iii.] For all $a,b,c\in X$, $a\sim b\land b\sim c\implies a\sim c$.
    \end{itemize}

    \item [2.)] Given an equivalence relation $\sim$ on $X$ and $x\in X$, the \textit{equivalence class} of $x$ is defined as $\bracks{x}=\braces{y\in X:x\sim y}$.

    \item [3.)] Given an equivalence relation $\sim$ on $X$, $X/{\sim}$ is defined as $X/{\sim}=\braces{[x]:x\in X}$

    \item [4.)] Given an equivalence relation $\sim$ on $X$, the \textit{quotient map} $p$ is defined as $p:X\to X/{\sim}$ where $x\mapsto\bracks{x}$.

    \item [5.)] Given a topological space $X$ and an equivalence relation $\sim$ on $X$, the \textit{quotient topology} $\mathcal{T}_q$ on $X/{\sim}$ is defined as $\mathcal{T}_q=\braces{V\subset X/{\sim}:p^{-1}(V)\text{ is open in }X}$
\end{itemize}

\section*{Proofs}
\begin{itemize}
    \item [a.)] Suppose $e^{i\theta}=1$, then $\cos\theta+i\sin\theta=1$, thus $\sin\theta=0$ and $\cos\theta=1$, thus $\theta=0$.

    \item [b.)] Suppose $e^{i\theta}=i$, then $\cos\theta+i\sin\theta=i$, thus $\cos\theta=0$ and $\sin\theta=1$, thus $\theta=\pi/2$.

    \item [c.)] Suppose $e^{i\theta}=1/2+i\parens{\sqrt3/2}$, then $\cos\theta+i\sin\theta=1/2+i\parens{\sqrt3/2}$, thus $\cos\theta=1/2$ and $\sin\theta=\parens{\sqrt3/2}$, thus $\theta=\pi/3$.

    \item [d.)] Since $e^{i\theta}$ is simply a rotation by $\theta$, we see that $e^{i\theta_1}=e^{i\theta_2}$ if $\theta_1$ and $\theta_2$ are equivalent angles, that is $\theta_1-\theta_2=2\pi n$ for some $n\in\Z$.
    
    \item [e.)] The unit circle, i.e. $S^1$.

    \pagebreak
    \item [f.)] \begin{itemize}
        \item [i.)] Let $\sim$ be a relation on $\R$ where $a\sim b\iff a-b=2\pi n$ for some $n\in\Z$. Since $a-a=0$, we know that $a\sim a$ for all $a\in\R$. Next, assume $a\sim b$, then $a-b=2\pi n$, thus $b-a=2\pi(-n)$, thus $a~b\implies b~a$. Finally, let $a\sim b$ and $b\sim c$, thus $a-b=2\pi m$ and $b-c=2\pi n$ for $m,n\in\Z$, then $a-c=a-b+b-c=2\pi m-2\pi n=2\pi(m-n)$, thus $a\sim c$, thus $\sim$ is an equivalence relation. $\blacksquare$

        \item [ii.)] Let $f:\R/{\sim}\to S^1$ be defined as $f([\theta])=(\cos\theta,\sin\theta)$. Since $\cos^2\theta+\sin^2\theta=1$ for all $\theta\in\R$, we know that the image of $f$ is $S^1$. Let $[\theta_1],[\theta_2]\in\R/{\sim}$ where $f([\theta_1])=f([\theta_2])$, then $\sin\theta_1=\sin\theta_2$, thus $\theta_1\sim\theta_2$, thus $[\theta_1]=[\theta_2]$, thus $f$ is injective. Next, let $(a,b)\in S^1$. Choose $\theta=\cos^{-1}(a)=\sin^{-1}(b)$. This is possible because $(a,b)\in S^1$. We can see that $f([\theta])=(\cos(\cos^{-1}(a)),\sin(\sin^{-1}(b)))=(a,b)$, thus $f$ is surjective, and thus bijective. $\blacksquare$
    \end{itemize}
\end{itemize}

\end{document}