\documentclass[12pt]{article}
\pagenumbering{gobble}

\usepackage{amsfonts}
\usepackage{amsmath}
\usepackage{amssymb}
\usepackage{array}
\usepackage{fancyhdr}
\usepackage{mathrsfs}
\usepackage{mathtools}
\usepackage{parskip}
\usepackage{textcomp}
\usepackage[margin=1in,headheight=1in]{geometry}

\newcommand{\done}{
    \ensuremath{\strut\hfill\blacksquare}
}

\newcommand{\contra}{
    \Rightarrow\Leftarrow
}

% bracket commands
\newcommand{\angleb}[1]{\left\langle#1\right\rangle} % <>
\newcommand{\vertb}[1]{\left\vert#1\right\vert}      % ||
\newcommand{\bracks}[1]{\left[#1\right]}             % []
\newcommand{\braces}[1]{\left\{#1\right\}}           % {}
\newcommand{\parens}[1]{\left(#1\right)}             % ()

% set aliases
\newcommand{\N}{\mathbb{N}}
\newcommand{\Z}{\mathbb{Z}}
\newcommand{\Q}{\mathbb{Q}}
\newcommand{\R}{\mathbb{R}}

\newcommand{\derv}[2]{\dfrac{d#1}{d#2}}
\newcommand{\e}{\varepsilon}
\newcommand{\di}{\,/\,}

\newcommand{\lm}[1]{\displaystyle\lim_{#1}}

\begin{document}

\linespread{1.2}

\setlength{\parskip}{0.2in}

\pagestyle{fancy}
\fancyhead[L]{Alex Agruso}
\fancyhead[R]{Topology Homework 12}

\normalsize

\section*{Definitions}
\begin{itemize}
    \item [1.)] Given two points $x$ and $x'$ in a topological space $X$, a
    \textit{continuous path} in $X$ from $x$ to $x'$ is a continuous function
    $\gamma : [0,1] \to X$ where $\gamma(0) = x$ and $\gamma(1) = x'$.
    ‡{ú}
    \item [2.)] A topological space $X$ is \textit{path-connected} if for all
    $x,x' \in X$, there exists a continuous path between them.

    \item [3.)] Define the equivalence relation $\sim$ on a topological space
    $X$ as follows:
    \[ x \sim x' \iff x\ \text{and}\ x'\ \text{are path-connected} \, . \]
    We define $\pi_0(X)$ as the quotient space $X/{\sim}$.

    \item [4.)] The \textit{invariance of domain} theorem states that
    $\R^m \cong \R^n \iff m = n$.

    \item [5.)] A topological space $X$ is \textit{connected} if for any set in
    $X$ that is both open and closed, that set is either $X$ or $\varnothing$.
\end{itemize}

\section*{Proof}

We know that $\R^n \cup \braces{*}$ is compact. In addition, since
$\R^{n + 1}$ is Hausdorff and $S^n \subset \R^{n + 1}$, we know that $S^n$ is
Hausdorff. We also know that the stereographic projection
$p : S^n \backslash \braces{0, \dots, 0, 1} \to \R^n$ is bijective, so
consider its inverse $p^{-1}$, and define a function
$f : \R^{n} \cup \braces{*} \to S^n$ as follows:
\[
    f(x) = \begin{cases}
        (0, \dots, 0, 1) \in S^n & x = * \\
        p^{-1}(x) & \text{otherwise}
    \end{cases}
\]
We can readily see that $f$ is bijective. Let $U$ be an open set in $S^n$, thus
$U = S^n \cap V$ where $V$ is open in $\R^n$, thus
$f^{-1}(U) = f^{-1}(S^n \cap V) = f^{-1}(S^n) \cap f^{-1}(V)$. Since
$V \subseteq \R^n$, we know that $* \notin V$, so $f^{-1}(V) = p(V)$, and since
$p$ is a homeomorphism, we know that $V' = p(V)$ is open in $\R^n$, thus
$f^{-1}(S^n) \cap f^{-1}(V) = \parens{\R^n \cup \braces{*}} \cap V'
    = (\R^n \cap V') \cup (V' \cap \braces{*}) = \R^n \cap V' = V'$, which is
open in $\R^n$, thus $f$ is continuous, and thus by theorem 15.5.1, a
homeomorphism.
\done

\end{document}