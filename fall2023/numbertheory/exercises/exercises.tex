\documentclass[12pt]{article}

\usepackage{mathtools}
\usepackage{amssymb}
\usepackage{fancyhdr}
\usepackage[headheight=1in,margin=1in]{geometry}

\newcommand{\claim}{\textit{Claim: }}
\newcommand{\proof}{\textit{Proof: }}
\newcommand{\done}{
    \ensuremath{\strut\hfill\blacksquare}
}

\newcommand{\sep}[1]{\ \ \ \text{#1}\ \ \ }

\newcommand{\braces}[1]{\left\{#1\right\}}
\newcommand{\parens}[1]{\left(#1\right)}

\begin{document}
    \pagestyle{fancy}
    \fancyhead[L]{Analytic Number Theory}
    \fancyhead[R]{Exercises}

    \section*{Chapter 1}

    For all problems in this chapter, lowercase latin letters $a,b,\dots,y,z$
    represent integers.
    
    \begin{itemize}
        \item [1.)] \claim If $(a,b)=1$, and $c\mid a$ and $d\mid b$, then
        $(c,d)=1$.
        
        \proof Since $(a,b)=1$, there exist $x$ and $y$ where $ax+by=1$.
        We also know that $c\mid a $ and $b\mid d$, so there exists $m$ and
        $n$ where $a=cm$ and $b=dn$, thus $ax+by=c(mx)+d(ny)=1$, thus
        $(c,d)=1$.
        \done

        \item [2.)] \claim If $(a,b)=(a,c)=1$, then $(a,bc)=1$.

        \proof Since $(a,b)=(a,c)=1$, there exist $x_1$, $x_2$, $y_1$, and
        $y_2$ where $ax_1+by_1=1$ and $ax_2+cy_2=1$.
        We can see that
        \[
            1 = (ax_1+by_1)(ax_2+cy_2)
            = (a^2x_1x_2+abx_2y_1+acx_1y_2+bcy_1y_2)
        \]\[
            = a(ax_1x_2+bx_2y_1+cx_1y_2)+bc(y_1y_2)
            \, ,
        \]
        so $(a,bc)=1$.
        \done

        \item [3.)] \claim If $(a,b) = 1$, then $(a^n,b^k) = 1$ for all $n$ and $k$.
        
        \proof We can take the prime factorizations of $a$ and $b$:
        \[
            a
            = \prod p_i^{x_i}
            \sep{and}
            b
            = \prod p_i^{y_i}
            \, ,
        \]
        where $p_i$ are the primes, and $x_i$ and $y_i$ are integers that
        depend on $p_i$.
        Further, the prime factorizations for $a^n$ and $b^k$ are
        \[
            a^n
            = \parens{\prod p_i^{x_i}}^n
            = \prod p_i^{x_i^n}
        \]
        and
        \[
            b^k
            = \parens{\prod p_i^{y_i}}^k
            = \prod p_i^{y_i^k}
        \]
        Since $(a,b)=1$, we know that $\min \braces{x_i,y_i}=0$ for all $i$,
        thus $\min \braces{x_i^n,y_i^k}=0$, thus $(a^n,b^k)=1$.
        \done

        \item [4.)] \claim If $(a,b) = 1$, then $(a + b, a - b)$ is either 1
        or 2.

        \proof Let $d = (a + b,a - b)$, then $d \mid a + b$ and
        $d \mid a - b$, thus $a + b = dm$ and $a - b = dn$ for some $m$ and
        $n$.
        From

        \item [5.)] ***

        \item [6.)] ***

        \item [7.)] ***

        \item [8.)] ***

        \item [9.)] ***

        \item [10.)] ***

        \item [11.)] ***

        \item [12.)] ***

        \item [13.)] ***

        \item [14.)] ***

        \item [15.)] ***

        \item [16.)] ***

        \item [17.)] ***

        \item [18.)] ***

        \item [19.)] ***

        \item [20.)] ***

        \item [21.)] ***

        \item [22.)] ***

        \item [23.)] ***

        \item [24.)] ***

        \item [25.)] ***

        \item [26.)] ***

        \item [27.)] ***

        \item [28.)] ***

        \item [29.)] ***

        \item [30.)] ***
    \end{itemize}
\end{document}