\documentclass[11pt]{article}
\pagenumbering{gobble}
\linespread{1.3}

\usepackage{amsfonts}
\usepackage{amsmath}
\usepackage{amssymb}
\usepackage{array}
\usepackage{fancyhdr}
\usepackage{mathrsfs}
\usepackage{mathtools}
\usepackage{textcomp}
\usepackage[margin=1in,headheight=1in]{geometry}

\newcommand{\contradiction}{
    \ensuremath{{\Rightarrow\mspace{-2mu}\Leftarrow}}
}

% bracket commands
\newcommand{\angleb}[1]{\left\langle#1\right\rangle} % <>
\newcommand{\vertb}[1]{\left\vert#1\right\vert}      % ||
\newcommand{\bracks}[1]{\left[#1\right]}             % []
\newcommand{\braces}[1]{\left\{#1\right\}}           % {}
\newcommand{\parens}[1]{\left(#1\right)}             % ()

% set aliases
\newcommand{\N}{\mathbb{N}}
\newcommand{\Z}{\mathbb{Z}}
\newcommand{\Q}{\mathbb{Q}}
\newcommand{\R}{\mathbb{R}}

\newcommand{\derv}[2]{\dfrac{d#1}{d#2}}
\newcommand{\e}{\varepsilon}
\newcommand{\di}{\,/\,}

\newcommand{\lm}[1]{\displaystyle\lim_{#1}}

\begin{document}
\pagestyle{fancy}
\fancyhead{}
\fancyhead[L]{Alex Agruso}
\fancyhead[R]{Analysis Homework 3}

\normalsize

\begin{itemize}
    \item [7.6.8.)] Define an interval $[a,b]$, and let $f$ be continuous on $[a,b]$ and differentiable on $(a,b)$. In addition, suppose $f$ satisfies the Lipschitz condition, thus for all $x,y\in[a,b]$, there exists $M\in\R$ where
    \[\vertb{f(x)-f(y)}\leq M\vertb{x-y}\]
    thus
    \[\vertb{\frac{f(x)-f(y)}{x-y}}\]
    By the mean value theorem, for all $c\in(a,b)$, we can find $x,y\in[a,b]$ where
    \[f'(c)=\frac{f(x)-f(y)}{x-y}\]
    thus
    \[\vertb{f'(c)}=\vertb{\frac{f(x)-f(y)}{x-y}}\leq M\]
    thus $f'$ is bounded on $[a,b]$. $\blacksquare$

    \item [7.6.20.)] Let $f$ and $g$ be continuous on $[a,b]$ and differentiable on $(a,b)$, then according to the Cauchy mean value theorem, there exists $c\in(a,b)$ where
    \[g'(c)(f(b)-f(a))=f'(c)(g(b)-g(a))\]
    Consider the following determinant:
    \[\begin{vmatrix*}
        f(a) & g(a) & 1 \\
        f(b) & g(b) & 1 \\
        f'(c) & g'(c) & 0
    \end{vmatrix*}=-f(a)g'(c)+g(a)f'(c)+f(b)g'(c)-f'(c)g(b)=0\]
    \[\implies g'(c)\parens{f(b)-f(a)}=f'(c)(g(b)-g(a))\]
    Thus the conclusion of Cauchy's mean value theorem aligns with the given determinant form. $\blacksquare$

    \item [7.7.1.)] Let $f(x)=(1-x)e^x$ and consider $f'$:
    \[f'=\derv{}{x}(1-x)e^x=-xe^x\]
    Since $f'(0)=0$, $0$ is an inflection point for $f$. Consider $f'$ for $x\in(0,1)$:
    \[0<x<1\implies0<xe^x<e^x\implies0>-xe^x>-e^x\]
    Thus $f'(x)<0$ given $x\in(0,1)$, thus $f$ is strictly decreasing on $(0,1)$, and since $f(0)=1$, we know that $f(x)=(1-x)e^x\leq 1$ for $x\in[0,1)$. Next, consider $f'$ for $x<0$:
    \[x<0\implies xe^x<0\]
    Thus $f'(x)<0$ given $x<0$, thus $f$ is strictly decreasing on $(-\infty,0)$, thus $f(x)=(1-x)e^x\leq 1$ for $x<0$. Since $f(x)\leq1$ for all $x<1$, we know that the ratio between $e^x$ and $\frac{1}{1-x}$ is less than or equal to $1$ for all $x<1$, thus the inequality $e^x\leq\frac{1}{1-x}$ holds. $\blacksquare$
\end{itemize}

\end{document}
