\documentclass[11pt]{article}
\pagenumbering{gobble}
\linespread{1.3}

\usepackage{amsfonts}
\usepackage{amsmath}
\usepackage{amssymb}
\usepackage{array}
\usepackage{fancyhdr}
\usepackage{mathrsfs}
\usepackage{mathtools}
\usepackage{textcomp}
\usepackage[margin=1in,headheight=1in]{geometry}

\newcommand{\contradiction}{
    \ensuremath{{\Rightarrow\mspace{-2mu}\Leftarrow}}
}

% bracket commands
\newcommand{\angleb}[1]{\left\langle#1\right\rangle} % <>
\newcommand{\vertb}[1]{\left\vert#1\right\vert}      % ||
\newcommand{\bracks}[1]{\left[#1\right]}             % []
\newcommand{\braces}[1]{\left\{#1\right\}}           % {}
\newcommand{\parens}[1]{\left(#1\right)}             % ()

% set aliases
\newcommand{\N}{\mathbb{N}}
\newcommand{\Z}{\mathbb{Z}}
\newcommand{\Q}{\mathbb{Q}}
\newcommand{\R}{\mathbb{R}}

\newcommand{\derv}[2]{\dfrac{d#1}{d#2}}
\newcommand{\e}{\varepsilon}
\newcommand{\di}{\,/\,}

\newcommand{\lm}[1]{\displaystyle\lim_{#1}}

\begin{document}
\pagestyle{fancy}
\fancyhead{}
\fancyhead[L]{Alex Agruso}
\fancyhead[R]{Analysis Homework 5}

\normalsize

\begin{itemize}
    \item [7.11.1.)] \begin{itemize}
        \item [a.)] For $f$ to be continuous at $x=0$, we must have $f(0)=\lm{x\to0}f(x)$, thus
        \[f(0)=\lm{x\to0}\frac{3^x-2^x}{x}=\lm{x\to0}\frac{3^x\ln(3)-2^x\ln(2)}{1}=3^0\ln(3)-2^0\ln(2)=\ln(3)-\ln(2)\]
        Thus defining $f(0)=\ln(3)-\ln(2)$ makes $f$ continuous at $x=0$. $\blacksquare$

        \item [b.)] Now we determine if $f'(0)$ exists given our definition of $f(0)$:
        \[f'(0)=\lm{x\to0}\frac{f(x)-f(0)}{x-0}=\lm{x\to0}\frac{3^x-2^x-\ln(3)+\ln(2)}{x}=\frac{\ln(2)-\ln(3)}{0}\implies\text{diverges}\]
        Thus $f'(0)$ does not exist given our definition of $f(0)$. $\blacksquare$

        \item [c.)] We can try computing $f'(0)$ using typical derivative rules as follows:
        \[f'(x)=\derv{}{x}\braces{\frac{3^x-2^x}{x}}=\frac{x\parens{3^x\ln(3)-2^x\ln(2)}-3^x+2^x}{x^2}\]
        Thus
        \[\lm{x\to0}f'(x)=\lm{x\to0}\frac{x\parens{3^x\ln(3)-2^x\ln(2)}-3^x+2^x}{x^2}\]
        \[=\lm{x\to0}\frac{3^x\ln(3)+3^xx\ln(3)^2-2^x\ln(2)-2^xx\ln(2)^2-3^x\ln(3)+2^x\ln(2)}{2x}\]
        \[=\lm{x\to0}\frac{3^xx\ln(3)^2-2^xx\ln(2)^2}{2x}=\lm{x\to0}\frac{3^x\ln(3)^2+3^xx\ln(3)^3-2^x\ln(2)-2^xx\ln(2)^3}{2}\]
        \[=\frac{1}{2}\parens{3^0\ln(3)^2+3^0(0)\ln(3)^3-2^0\ln(2)-2^0(0)\ln(2)^3}=\frac{1}{2}\parens{\ln(3)^2-\ln(2)^2}\]
        Thus by using the typical rules of calculus and taking a limit, we can reason that\break$f'(0)=\frac{1}{2}\parens{\ln(3)^2-\ln(2)^2}$. $\blacksquare$
    \end{itemize}

    \item [7.11.5.)] Consider $\lm{x\to0}f(x)/g(x)$:
    \[\lm{x\to0}\frac{f(x)}{g(x)}=\lm{x\to0}\frac{x^2\sin\parens{x^{-1}}}{x}=\lm{x\to0}x\sin\parens{x^{-1}}\]
    As $x\to0$, $\sin\parens{x^{-1}}$ does not converge, but since it is multiplied by $x$, it vanishes as $x\to0$, thus $\lm{x\to0}f(x)/g(x)=0$. Now, consider $\lm{x\to0}f'(x)/g'(x)$:
    \[f'(x)=\derv{}{x}\braces{x^2\sin\parens{x^{-1}}}=2x\sin\parens{x^{-1}}-\cos\parens{x^{-1}}\]
    \[g'(x)=\derv{}{x}\braces{x}=1\]
    Thus
    \[\lm{x\to0}\frac{f'(x)}{g'(x)}=\lm{x\to0}\frac{2x\sin\parens{x^{-1}}-\cos\parens{x^{-1}}}{1}=\lm{x\to0}2x\sin\parens{x^{-1}}-\lm{x\to0}\cos\parens{x^{-1}}\]
    \[=-\lm{x\to0}\cos\parens{x^{-1}}\implies\text{diverges}\]
    Thus while $\lm{x\to0}f(x)/g(x)$ exists, $\lm{x\to0}f'(x)/g'(x)$ does not. $\blacksquare$
\end{itemize}

\end{document}
