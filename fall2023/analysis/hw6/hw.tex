\documentclass[11pt]{article}
\pagenumbering{gobble}
\linespread{1.3}

\usepackage{amsfonts}
\usepackage{amsmath}
\usepackage{amssymb}
\usepackage{array}
\usepackage{fancyhdr}
\usepackage{mathrsfs}
\usepackage{mathtools}
\usepackage{textcomp}
\usepackage[margin=1in,headheight=1in]{geometry}

\newcommand{\contradiction}{
    \ensuremath{{\Rightarrow\mspace{-2mu}\Leftarrow}}
}

% bracket commands
\newcommand{\angleb}[1]{\left\langle#1\right\rangle} % <>
\newcommand{\vertb}[1]{\left\vert#1\right\vert}      % ||
\newcommand{\bracks}[1]{\left[#1\right]}             % []
\newcommand{\braces}[1]{\left\{#1\right\}}           % {}
\newcommand{\parens}[1]{\left(#1\right)}             % ()

% set aliases
\newcommand{\N}{\mathbb{N}}
\newcommand{\Z}{\mathbb{Z}}
\newcommand{\Q}{\mathbb{Q}}
\newcommand{\R}{\mathbb{R}}

\newcommand{\derv}[2]{\dfrac{d#1}{d#2}}
\newcommand{\e}{\varepsilon}
\newcommand{\di}{\,/\,}

\newcommand{\lm}[1]{\displaystyle\lim_{#1}}

\begin{document}
\pagestyle{fancy}
\fancyhead{}
\fancyhead[L]{Alex Agruso}
\fancyhead[R]{Analysis Homework 6}

\normalsize

\begin{itemize}
    \item [7.12.5)] Given $f(x)=\ln(1+x)$, we know that for $n>0$,
    \[f^{(n)}(x)=(-1)^{n-1}\frac{(n-1)!}{(1+x)^n}\]
    thus
    \[f^{(n)}(0)=(-1)^{n-1}(n-1)!\]
    From this, we can construct $P_n$ and $R_n$ at $c=0$ as follows:
    \[f(x)=P_n+R_n=\sum_{k=0}^n\braces{\frac{f^{(k)}(c)}{k!}(x-c)^k}+\frac{f^{(n+1)}(z)}{(n+1)!}(x-c)^{n+1}\]
    \[=\frac{\ln(1)}{0!}x^0+\sum_{k=1}^n\braces{\frac{(-1)^{k-1}(k-1)!}{k!}x^k}+(-1)^{n}\frac{n!}{(n+1)!(1+z)^{n+1}}x^{n+1}\]
    \[=\sum_{k=1}^n\braces{\frac{(-1)^{k-1}}{k}x^k}+\frac{(-1)^n}{n+1}\parens{\frac{x}{1+z}}^{n+1}\]
    \[=x-\frac{1}{2}x^2+\frac{1}{3}x^3+\dots+\frac{(-1)^{n-1}}{n}x^n+\frac{(-1)^n}{n+1}\parens{\frac{x}{1+z}}^{n+1}\]
    where $z\in(c,x)$, thus we obtain the desired equation for $f(x)$.

    \item [7.12.6)] \begin{itemize}
        \item [a.)] Since $f(0)=0$, and since $0$ is infinitely differentiable, we know $f^{(n)}(0)=0$ for all $n$. Now, consider the derivative of $f(x)=e^{-\frac{1}{x^2}}$ when $x\ne0$:
        \[f'(x)=\frac{2}{x^3}e^{-\frac{1}{x^2}}\]
        Thus we know that $e^{-\frac{1}{x^2}}$ is differentiable, and that its derivative is of the form $p(1/x)e^{-\frac{1}{x^2}}$ where $p$ is some polynomial in $1/x$. It is trivial to show that $p$ is differentiable. Next, suppose
        \[f^{(n)}(x)=p(1/x)e^{-\frac{1}{x^2}}\]
        for some polynomial $p$ in $1/x$, then we can find $f^{(n+1)}(x)$ as follows:
        \[f^{(n+1)}(x)=\derv{}{x}p(1/x)e^{-\frac{1}{x^2}}=p'(1/x)e^{-\frac{1}{x^2}}+\frac{2}{x^3}p(1/x)e^{-\frac{1}{x^2}}\]
        \[=e^{-\frac{1}{x^2}}\parens{p'(1/x)+\frac{2}{x^3}p(1/x)}=q(1/x)e^{-\frac{1}{x^2}}\]
        for a similar polynomial $q$. We have shown that $f^{(n)}$ is differentiable, and since $f^{(n)}$ and $f^{(n+1)}$ take the same form, we know that $f^{(n+1)}$ is also differentiable, thus by induction, $e^{-\frac{1}{x^2}}$ is infinitely differentiable.
        $\newline\strut\hfill\blacksquare$

        \item [b.)] We can construct $P_n$ at $c=0$ as follows:
        \[P_n(x)=\sum_{k=0}^n\frac{f^{(k)}(0)}{k!}x^k=\sum_{k=0}^n\frac{0}{k!}x^k=0\]

        \item [c.)] Using Lagrange's form, we can construct $R_n$ as follows:
        \[R_n\]
        Because we previously determined that $P_n=0$ for all $n$,
    \end{itemize}
\end{itemize}

\end{document}
