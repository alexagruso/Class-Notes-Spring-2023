\documentclass[11pt]{article}
\pagenumbering{gobble}
\linespread{1.25}

\usepackage{amsfonts}
\usepackage{amsmath}
\usepackage{amssymb}
\usepackage{array}
\usepackage{fancyhdr}
\usepackage{mathrsfs}
\usepackage{mathtools}
\usepackage{parskip}
\usepackage{textcomp}
\usepackage[margin=1in,headheight=1in]{geometry}

\newcommand{\done}{\ensuremath{
    \strut\hfill\blacksquare
}}

\newcommand{\contradiction}{
    \ensuremath{{\Rightarrow\mspace{-2mu}\Leftarrow}}
}

% bracket commands
\newcommand{\angleb}[1]{\left\langle#1\right\rangle} % <>
\newcommand{\vertb}[1]{\left\vert#1\right\vert}      % ||
\newcommand{\bracks}[1]{\left[#1\right]}             % []
\newcommand{\braces}[1]{\left\{#1\right\}}           % {}
\newcommand{\parens}[1]{\left(#1\right)}             % ()

% set aliases
\newcommand{\N}{\mathbb{N}}
\newcommand{\Z}{\mathbb{Z}}
\newcommand{\Q}{\mathbb{Q}}
\newcommand{\R}{\mathbb{R}}

\newcommand{\derv}[2]{\dfrac{d#1}{d#2}}
\newcommand{\e}{\varepsilon}
\newcommand{\di}{\,/\,}

\newcommand{\lm}[1]{\displaystyle\lim_{#1}}

\setlength{\parskip}{0.15in}

\begin{document}
\pagestyle{fancy}
\fancyhead{}
\fancyhead[L]{Alex Agruso}
\fancyhead[R]{Analysis Homework 8}

\normalsize

\begin{itemize}
    \item [9.2.1] We can take the limit as $n\to\infty$ of $f_n(x)$:
    \[\lm{n\to\infty}f_n(x)=\lm{n\to\infty}\frac{x^n}{1+x^n}=\lm{n\to\infty}\frac{x^n}{1+x^n}\frac{x^{-n}}{x^{-n}}=\lm{n\to\infty}\frac{1}{x^{-n}+1}=\frac{1}{0+1}=1\ ,\]
    thus this sequence of functions converges pointwise to $1$ for all $x$. \done

    \item [9.2.2] We can see that following:
    \[L:=\lm{n\to\infty}n\parens{\sqrt[n]{x}-1}=\lm{n\to\infty}\frac{1}{n^{-1}}\parens{\sqrt[n]{x}-1}=\lm{n\to\infty}\frac{\sqrt[n]{x}-1}{n^{-1}}=\frac{0}{0}\ ,\]
    thus we can apply L'H\capitalcircumflex{o}pital's rule:
    \[\derv{}{n}\braces{n^{-1}}=-n^{-2}\]
    \[\derv{}{n}\braces{x^{\frac{1}{n}}-1}=\derv{}{n}\braces{e^{(1/n)\log x}}=(-n^{-2}\log x)(x^{\frac{1}{n}})=-\frac{x^{\frac{1}{n}}\log x}{n^2}\]
    Thus
    \[L=\lm{n\to\infty}\frac{1}{n^{-2}}\cdot\frac{x^{\frac{1}{n}}\log x}{n^2}=\lm{n\to\infty}x^{\frac{1}{n}}\log x\]
    Since $1/n\to0$ as $n\to\infty$, we know $L=x^0\log x=\log x$, thus we obtain our desired equality. For the next part, define $f_n(x)=n\parens{\sqrt[n]{x}-1}$ for all $n\in\N$ and $f(x)=\log x$. Assuming the sequence uniformly converges, we can make the following observations:

    If $f_n(x)$ is continuous over $(0,\infty)$ for all $n$, then we could conclude that $f(x)$ is continuous over $(0,\infty)$ as a result.

    If $f_n(x)$ is differentiable over $(0,\infty)$ for all $n$, then we would know that $f(x)$ is differentiable over $(0,\infty)$, and that its derivative would be given by
    \[f'(x)=\sum_{k=1}^\infty f'_k(x)\]

    Finally, if $f_n(x)$ is integrable over $[0,\infty)$ for all $n$, then we know that $f(x)$ is as well. We could also evaluate $\int_1^2\log x\,dx$ as
    \[\int_1^2\log x\,dx=\sum_{k=1}^\infty\int_1^2f_k(x)\,dx\ .\]
    Note that
    \[\int_1^2f_k(x)\,dx=k\int_1^2\sqrt[k]{x}-1\,dx=k\bracks{\frac{k}{k+1}x^{\frac{k+1}{k}}-x}_1^2=\frac{k}{k+1}\parens{\sqrt[k]{2^{k+1}}-1}-1\]
    But the sum diverges, so it cannot be equal to the integral, so we must conclude that the sequence of functions $f_n$ does not converge uniformly. \done

\end{itemize}

\end{document}
