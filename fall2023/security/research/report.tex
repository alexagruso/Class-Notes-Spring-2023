\documentclass[letterpaper,twocolumn,12pt]{article}

% to be able to draw some self-contained figs
\usepackage{tikz}
\usepackage{amsmath}
\usepackage{amsfonts}

% bibtex
\usepackage{filecontents}
\usepackage{url}

% styles
\usepackage{style}

% aliases
\newcommand{\N}{\mathbb{N}}
\newcommand{\parens}[1]{\left( #1 \right)}

\begin{filecontents}{\jobname.bib}

\end{filecontents}

\begin{document}
    \date{}

    \title{ An Exploration of Methods of Cracking \\
        Encryption Schemes }

    \author{
        { \rm Alexander Agruso } \\
        { \rm Texas State University }
        \and
        { \rm Brandon Howell } \\
        { \rm Texas State University }
    }

    \maketitle

    \begin{abstract}
        In this paper, we explore RSA and Elliptic Curve Cryptography, giving a
        mathematical overview of how these schemes function.
        Additionally, we analyze and compare various methods that exist to
        crack these encryption schemes.
        Finally, we discuss future threats to these encryption schemes
        and the ongoing efforts to mitigate them.
    \end{abstract}

    \section*{Introduction}

    In the age of digital communication, it is more important than ever that we
    safeguard our information online.
    With so much of our sensitive information being sent over the internet, we
    must find ways to ensure that unauthorized parties aren't able to view this
    information.
    The most common way of achieving this goal is through an encryption scheme.
    
    One of the earliest known encryption schemes dates back to the ancient
    Greeks, where Polybius invented the ``Polybius Square'', a primitive
    encryption method that encoded letters in a five by five square.
    Another well known encryption method from antiquity is the Julius Cipher,
    which shifts the letters in the alphabet so that the corresponding letters
    in a given sentence are scrambled in such a way as to make the text
    unreadable unless deciphered.
    ~\cite{JD:History}

    While these schemes may have worked for people in the past, they are
    rendered essentially useless by the ubiquity of powerful computers in our
    modern age.
    As a result, much more sophisticated encryption schemes have been developed
    that can resist computational efforts to crack them. In this paper, we will
    focus on two widely used public-key encryption schemes: RSA and Elliptic
    Curve Encryption.
    These methods use advanced concepts in math, specifically number theory, to
    ensure that even the most powerful computers are unable to decipher a
    message unless they possess the private key, with which deciphering a
    message becomes a significantly easier task.
    
    Both of these schemes, and all public-key encryption schemes in general,
    rely on intractable problems, which are problems whose solutions are
    difficult, if not virtually impossible to find, but are nonetheless easily
    verifiable once found.
    So, if one wants to crack these encryption schemes, one must find solutions
    to these intractable problems.
    This paper explores various methods of finding solutions to the intractable
    problems at the core of RSA and Elliptic Curve Encryption, as well as
    possible improvements that can be made upon existing methods.

    \section*{RSA Encryption}

    The RSA encryption scheme, invented by Ron Rivest, Adi Shamir, and Leonard
    Adleman in 1977, is a widely used public-key encryption system that relies
    on the difficulty of factoring certain large integers.
    ~\cite{RSARivest}
    We provide a brief
    explanation of how public-private key pairs are generated, and how they
    are used to encrypt and decrypt messages.
    
    The first step in creating a public-private key pair is to randomly choose
    two large prime numbers \( p \) and \( q \), then calculate their product
    \( n = pq \).
    ~\cite{RSARivest}
    Next, the value of \( \lambda(n) \) is calculated, where
    \( \lambda \) is Carmichael's totient function.
    We define \( \lambda(n) \) as the smallest integer \( m \) where
    \( a^m \equiv 1 \pmod{n} \) for all integers \( a \) that are relatively
    prime to \( n \).
    ~\cite{WA:Carmichael}
    It should be noted that \( \lambda(n) \) is very difficult to calculate if
    only given \( n \).
    However, if one has the prime factorization of \( n \), in this case
    \( pq \), the calculation becomes considerably easier, with it reducing
    to calculating the least common multiple of \( p - 1 \) and \( q - 1\).
    We then choose an integer \( e \) such that \( 2 < e < \lambda(n) \) and
    \( e \) and \( \lambda(n) \) are relatively prime.
    Finally, we find the
    multiplicative inverse \( d \) of \( e \pmod{\lambda(n)} \).
    The public key consists of the modulus \( n \) and the exponent \( e \),
    while the private key consists of the exponent \( d \).

    Having calculated our public-private key pair, we are now ready to start
    encrypting messages.
    Let \( M \) be the message we wish to encrypt.
    Using \( n \) and \( e \), we calculate the ciphertext \( c \) by
    evaluating the following:
    \[ c \equiv M^e \pmod{n}. \]
    The encrypted message \( c \) can then be sent to the recipient.

    When the message is received, it can be decrypted using our private
    exponent \( d \).
    Since \( d \) and \( e \) are multiplicative inverses, we have that
    \( de \equiv 1 \pmod{n} \), thus we can obtain \( M \) as follows:
    \[ c^d \equiv (M^e)^d \equiv M^{ed} \equiv M^1 \equiv M \pmod{n}, \]
    thus by calculating \( c^d \), we obtain the original message \( M \),
    and since \( d \) is kept secret, only the recipient can decrypt the
    ciphertext.

    \section*{Methods of Cracking RSA}

    To crack RSA encryption, we must find a way to derive the private key
    \( d \) from the public key values \( n \) and \( e \).
    The most straightforward way of doing this is to find the prime factors
    \( p \) and \( q \) of \( n \), which, as previously noted, can be used to
    easily calculate \( \lambda(n) \), and thus calculate \( d \).
    
    The typical naive approach to factoring numbers is the brute force method.
    A custom implementation of a brute force factoring algorithm in python is
    given as follows:

    First, we define two important utility functions, \verb|divides(d,n)| and
    \verb|isPrime(n)|, which are implemented as
    \begin{verbatim}
    def divides(d, n):
        q = n // d
        return n == d * q
    \end{verbatim}
    and
    \begin{verbatim}
    def isPrime(n):
        for i in range(2, n):
            if divides(i, n):
                return false
        
        return true
    \end{verbatim}
    respectively.
    The \verb|divides| function checks whether or not \( d \) divides \( n \),
    that is, it checks if \( n \) is an integer multiple of \( d \).
    The \verb|isPrime| function simply checks whether \( n \) is prime.
    Using these two utility functions, we can define \verb|primefac(n)|
    as follows:
    \begin{verbatim}
    def primefac(n):
        current = n
        i = 2

        while i <= n:
            if isPrime(i) and
                divides(i, current):
                print(i)
                current = current // i

                if current == 1:
                    return
            else:
                i = i + 1
    \end{verbatim}
    Disallowing pre-computed tables of primes, this algorithm runs roughly in
    \( O(2^n) \).

    A more efficient, albeit far more sophisticated prime factorization
    method, is the General Number Field Sieve (GNFS) algorithm.
    Its full implementation is beyond the scope of this paper, so we will
    only discuss how it compares to our brute force approach.

    The time complexity of the GNFS algorithm is
    \[
        O\parens{e^{
            \parens{(64 / 9)^{1 / 3} + O(1)}
            \parens{\log n}^{1 / 3}(\log\log n)^{2 / 3}
        }}.
    \]
    In other words, the GNFS runs in sub-exponential time.
    ~\cite{TOTS}
    While the brute force algorithm does factor small numbers faster than the
    GNFS algorithm due to the significant overhead incurred by GNFS, the smaller
    runtime complexity of the GNFS guarantees that, past a certain point, all
    numbers can be factored faster by the GNFS.

    \section*{Improving the Brute Force Method}

    Our original implementation of the brute force prime factorization
    algorithm is suboptimal; there are several apparent optimizations that can
    be made to it.

    If \( n \) is prime, which is easy to check, our algorithm will still
    attempt to factor it, in which case it will take the longest amount of
    time to complete, as it will run through every number less than or equal to
    \( n \) before finally finding \( n \) as a factor and terminating.
    Additionally, given \( n \) is composite, we know that no number greater
    than \( n / 2 \) can divide it, or else \( n \) would either have to be a
    non-integer multiple of that number, or prime, contrary to our assumption.
    Consequently, it suffices to only check factors up to
    \( \lfloor n / 2 \rfloor \).
    We can further reduce the number of factors we have to check by realizing
    that no even integer greater than 2 is prime, thus we only have to check
    odd numbers greater than 2 for factors.
    A final small optimization that we can make is not repeatedly checking
    whether factors are prime, and instead repeatedly dividing the same factor
    until it no longer divides the residue.

    \pagebreak
    Implementing these optimizations, we end up with our final brute force
    algorithm:

    \begin{verbatim}
    def primefac(n):
        if isPrime(n):
            print(n)
            return

        current = n

        while divides(2, current):
            print(2)
            current = current // 2

            if current == 1:
                return
        
        i = 3

        while i <= n // 2:
        if isPrime(i):
            while divides(i, current):
                print(i)
                current = current // i

            if current == 1:
                return

            i = i + 2
    \end{verbatim}

    Despite the several optimizations made to this algorithm, it still fails to
    run better than exponential time. This ultimately illustrates that
    different, more clever approaches are needed if we wish to factor large
    numbers faster.

    \section*{Elliptic Curve Encryption}

    Elliptic curve cryptography, henceforth referred to as ECC, describes
    several cryptographic schemes that rely on the intractability of the
    elliptic curve discrete logarithm problem (ECDLP).
    The idea to use elliptic curves in cryptography originated independently
    in 1985 from Neal Koblitz and Victor Miller.
    ~\cite{ECCSource}
    Before we can discuss how public-private key pairs are generated,
    we will briefly introduce
    the mathematics behind elliptic curves and the ECDLP.
    
    An elliptic curve is a cubic polynomial in two variables that takes the
    form of
    \[ y^2 = x^3 + ax^2 + bx + c, \]
    combined with a ``point at infinity'', which we will denote as
    \( \mathcal{O} \).
    For the purposes of ECC, we take this curve over a finite field
    \( \mathbb{F}_p \), where \( p \) is a carefully chosen prime number.
    In abstract algebra, a field simply represents a set where the usual
    notions of addition and multiplication hold true, and in our case, this
    field has finitely many elements, namely \( p \) elements.

    Given the set of points on this curve over \( \mathbb{F}_p \), along with
    \( \mathcal{O} \), we can endow it with a group structure by equipping it
    with a special ``addition'' operation.
    A group is a mathematical structure that describes the symmetries of a
    certain object, in this case, the points on our curve.
    Additionally, it allows us to define the ``discrete logarithm'' of points
    on the curve.
    Given two elements \( a \) and \( b \) in our group, we define the discrete
    logarithm \( \log_b(a) \) as the integer \( k \) such that \( b^k = a \).
    In this case, \( a \) and \( b \) are points on our curve. It should also
    be noted that \( b^k \) does not represent typical exponentiation; in this
    case, it represents
    \[ \underbrace{b + b + \cdots + b}_{k\ \textrm{times}}, \]
    with \( + \) being our special addition operation.
    Despite the notations being similar, the discrete logarithm and regular
    logarithms should not be confused with one another.
    
    With this mathematical primer out of the way, we can now discuss how
    ECC generates public-private key pairs. Given an elliptic curve that
    is suitable for cryptographic purposes, each party chooses an integer
    \( d \), and a point on the curve \( P \) that is equal to
    \( S^d \) for some other carefully chosen point \( S \).
    The integer is kept as the private
    key, while the point \( P \) is distributed as the public key.

    \section*{Methods of Cracking ECC}

    In order to crack ECC, one must be able to derive our chosen integer \( d \)
    from the public point \( P \), which entails evaluating the discrete logarithm
    \( \log_SP \).
    Currently, the most powerful algorithm that we have to solve the discrete
    logarithm problem is the ``baby-step giant-step'' algorithm, invented by
    Daniel Shanks in 1971.
    ~\cite{BSGS}
    This algorithm works not only for our elliptic curve group, but for any
    finite cyclic group in general. While this algorithm has an impressive
    runtime of \( O(\sqrt n) \), the sheer difficulty of solving the discrete
    logarithm \textit{in general} makes it so that it is still infeasible to
    crack ECC schemes that are based off particularly difficult curves.

    \section*{Related Work}

    The GNFS algorithm is itself the result of decades of research and
    improvements on other sieve
    factoring methods like the rational and quadratic sieves.
    ~\cite{TOTS}
    In addition, from 1991 to 2007, RSA Laboratories, now known as RSA
    Security, offered cash prizes for those who could successfully factor
    certain RSA keys.
    ~\cite{RSAChallenge}
    This challenged spurred a lot of research into integer factorization,
    and so far, the largest RSA number that has been factored is RSA-250.
    ~\cite{RSA250}

    As for the general discrete logarithm problem, not much research has been
    conducted
    since Shanks' original paper in 1971. However, other similar algorithms have
    been developed to solve special cases of the discrete logarithm problem,
    for example, the Pohlig-Hellman algorithm.
    ~\cite{PHA}

    \section*{Future Threats to RSA and ECC}

    The invention of the GNFS algorithm instilled doubts in the security of RSA
    encryption.
    While, for the time being, large RSA keys are safe of being
    factored by the GNFS, it is possible that future advancements, both in
    computing power, and in the GNFS algorithm itself, will allow these keys to
    be broken relatively quickly.

    Another threat, completely distinct from the GNFS algorithm, is Shor's
    algorithm.
    Invented by computer scientist Peter Shor, it is a quantum algorithm that
    can be used to factor prime numbers extremely efficiently.
    ~\cite{QC:Shors}
    It is significantly faster than even the GNFS algorithm, having a time
    complexity of
    \[ O\parens{ (\log n)^2(\log\log n)(\log\log\log n) } \]
    using fast multiplication.
    ~\cite{ShorsTime}
    For very large numbers, this runtime can be improved even further to
    \[ O\parens{ (\log n)^2(\log\log n)} \]
    using the asymptotically fastest multiplication algorithm currently known.
    ~\cite{FastMult}
    The major barrier that prevents this algorithm from being used is the lack
    of sufficiently powerful quantum computers to run it.
    In 2001, IBM successfully factored 15 into 3 and 5 using Shor's algorithm.
    ~\cite{2001IBM}
    Additionally, an attempt was made by IBM in 2019 to factor 35 using Shor's
    algorithm, but it was unsuccessful.
    ~\cite{Failed}
    These numbers are a far cry from the often hundreds-of-digits long numbers
    that RSA uses, so Shor's algorithm poses no immediate threat to RSA or ECC.
    However, as quantum
    computing technology continues to improve, it is possible that RSA and ECC
    will one day be made obsolete. This will 
    force us to explore new encryption schemes that can resist quantum computers.
    
    \section*{Future Work}

    As our understanding of mathematics and computer science continues to grow,
    our methods of cracking encryption schemes will become better.
    Consequently, we will always need to be improving the encryption schemes
    that we use. Current research into post-quantum encryption schemes are
    focused on several approaches, including but not limited to lattice-based,
    hash-based, and even symmetric key cryptography.
    ~\cite{XMSS}
    ~\cite{SPHINCS}
    ~\cite{Symmetric}
    Of course, methods to
    crack these encryption schemes will undoubtedly be invented, which will
    in turn spur the creation of yet more encryption schemes, as is the nature
    of computer security.

    \bibliographystyle{unsrt}
    \bibliography{\jobname}
\end{document}
