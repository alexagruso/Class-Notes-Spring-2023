\documentclass[12pt]{article}

\usepackage{fancyhdr}
\usepackage[headheight=1in,margin=1in]{geometry}

\begin{document}
    \pagestyle{fancy}
    \fancyhead[L]{Alex Agruso}
    \fancyhead[R]{Security Homework 4}

    \begin{itemize}
        \item [1.)] A simple way to bypass the restriction would be to use a
        VPN to trick the website into thinking the request came from an
        actual Californian user.
        With a VPN, your request is sent to a third-party server, in this case,
        one in California, which then makes the request to the website on your
        behalf.
        Since the server is in California, its IP address is a Californian one,
        so the request is accepted and thus you can gain access to the website
        outside of California.

        \item [2.)] The bank would not be able to use the old SSL certificate,
        as individual SSL certificates are either tied to a single domain, or
        tied to multiple subdomains of a single domain.

        \item [3.)] The benefit is that public-key authentication uses a
        computer to generate the key, while password authentication relies on
        a human to create the password. Some people are lazy and use super
        insecure passwords like ``password'', but a computer will always
        generate the keys randomly, so there is no chance of a key being easy
        to guess or crack.

        \item [4.)] Using many rounds for a password hashing algorithm ensures
        that any user who wishes to build large a table of hashes, say for a
        brute force attack, requires far more computing power than if the
        passwords were hashed using fewer rounds.

        \item [5.)] An attacker can poison a DNS cache by inserting malicious
        entries. When the DNS server fetches this entry from the cache, then
        the user, rather than being sent to their intended destination, they
        are sent to the IP address stated in the cache, which may be the
        address of a malicious site.

        \item [6.)] DNS amplification is a form of DDoS attack where the
        attacker sends a malicious request to an open DNS server that overloads
        the victim with a very large response from the DNS server. This
        response sends large amounts of traffic to the victim which may slow
        down or disable their systems, thus it is a DDoS attack.

        \scriptsize Source: \verb|https://www.cisa.gov/news-events|
        \verb|/alerts/2013/03/29/dns-amplification-attacks|
    \end{itemize}
\end{document}