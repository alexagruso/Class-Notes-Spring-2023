\documentclass[12pt]{article}
\pagenumbering{gobble}
\linespread{1.15}

\usepackage{amsfonts}
\usepackage{amsmath}
\usepackage{amssymb}
\usepackage{array}
\usepackage{fancyhdr}
\usepackage{textcomp}
\usepackage[margin=1in,headheight=1in]{geometry}

\newcommand{\contradiction}{
    \ensuremath{{\Rightarrow\mspace{-2mu}\Leftarrow}}
}

\newcommand{\angleb}[1]{\left\langle#1\right\rangle}
\newcommand{\vertb}[1]{\left\vert#1\right\vert}
\newcommand{\bracks}[1]{\left[#1\right]}

\newcommand{\derv}[2]{\dfrac{d#1}{d#2}}

\begin{document}
\pagestyle{fancy}
\fancyhead{}
\fancyhead[L]{Alexander Agruso}
\fancyhead[R]{MATH 3323 Homework 4}

\begin{itemize}
    \item [1.)] Establish initial variables:
    \[V_0=200,Q_0=1,f_{in}=f_{out}=2,c_{in}=0,c_{out}=\frac{Q(t)}{V(t)},V(t)=V_0=200\]
    Setup differential equation for salt concentration:
    \[\frac{dQ}{dt}=f_{in}c_{in}-f_{out}c_{out}=0-\frac{2Q}{200}\implies\frac{dQ}{dt}+\frac{Q}{100}=0\]
    Solve for $Q$:
    \[\mu(t)=e^{\int\frac{1}{100}dt}=e^{\frac{t}{100}}\]
    \[Q=\frac{1}{e^{\frac{t}{100}}}\bracks{\int(0)e^{\frac{t}{100}}\ dt}=\frac{C}{e^{\frac{t}{100}}}\]
    Solve for $C$ given $Q(0)=Q_0$:
    \[1=\frac{C}{e^{\frac{0}{100}}}=C\]
    Solve for $t$ given $Q(t)=\dfrac{Q_0}{100}=\dfrac{1}{100}$
    \[\frac{1}{100}=\frac{1}{e^\frac{t}{100}}\implies100=e^\frac{t}{100}\implies t=100\ln(100)\approx460.52\]
    Which is the time (in minutes) before the concentration is $1\%$ of the initial value.

    \item [2.)] Establish initial variables:
    \[V_0=120,Q_0=0,f_{in}=f_{out}=2,c_{in}=\gamma,c_{out}=\frac{Q(t)}{V(t)},V(t)=V_0=120\]
    Setup differential equation for salt concentration:
    \[\frac{dQ}{dt}=2\gamma-\frac{Q}{60}\implies\frac{dQ}{dt}+\frac{Q}{60}=2\gamma\]
    Solve for $Q$:
    \[\mu(t)=e^{\int\frac{1}{60}dt}=e^\frac{t}{60}\]
    \[Q=\frac{1}{e^\frac{t}{60}}\bracks{\int2\gamma e^\frac{t}{60}\ dt}=\frac{1}{e^\frac{t}{60}}\left(120\gamma e^\frac{t}{60}+C\right)=120\gamma+\frac{C}{e^\frac{t}{60}}\]
    Solve for $C$ given $Q(0)=Q_0=0$:
    \[0=120\gamma+\frac{C}{e^\frac{0}{60}}\implies C=-120\gamma\]
    \[\therefore\ Q=120\gamma-\frac{120\gamma}{e^\frac{t}{60}}=120\gamma\left(1-\frac{1}{e^\frac{t}{60}}\right)\]
    Which is the explicit solution. To find the limiting amount of salt, take the limit as $t\rightarrow\infty$:
    \[\lim_{t\rightarrow\infty}\bracks{120\gamma-\frac{120\gamma}{e^\frac{t}{60}}}=120\gamma-0=120\gamma\]
    Which is the limiting amount.

    \pagebreak
    \item [3.)] Establish initial variables:
    \[V_0=100,Q_0=0,f_{in}=f_{out}=2,c_{in}=\frac{1}{2},c_{out}=\frac{Q(t)}{V(t)},V(t)=V_0=100\]
    Setup differential equation:
    \[\frac{dQ}{dt}=1-\frac{Q}{50}\implies\frac{dQ}{dt}+\frac{Q}{50}=1\]
    Solve for $Q$:
    \[\mu(t)=e^{\int\frac{1}{50}dt}=e^\frac{t}{50}\]
    \[Q=\frac{1}{e^\frac{t}{50}}\bracks{\int e^\frac{t}{50}\ dt}=\frac{1}{e^\frac{t}{50}}\left(50e^\frac{t}{50}+C\right)=50+\frac{C}{e^\frac{t}{50}}\]
    Solve for $C$ given $Q(0)=Q_0=0$:
    \[0=50+\frac{C}{e^\frac{0}{50}}\implies C=-50\]
    \[\therefore\ Q=50\left(1-\frac{1}{e^\frac{t}{50}}\right)\]
    $Q(10)\approx9.06$, and $c_{in}=0$. Now, setup the equation with new initial values:
    \[\frac{dQ}{dt}=0-\frac{2Q}{100}\implies\frac{dQ}{dt}+\frac{Q}{50}=0\]
    Solve for $Q$:
    \[\mu(t)=e^{\int\frac{1}{50}dt}=e^{\frac{t}{50}}\]
    \[Q=\frac{1}{e^{\frac{t}{50}}}\bracks{\int(0)e^{\frac{t}{50}}}=\frac{C}{e^\frac{t}{50}}\]
    Solve for $C$ given $Q(10)=9.06$:
    \[9.06=\frac{C}{e^\frac{10}{50}}\implies C=e^\frac{1}{5}9.06\approx11.07\]
    \[\therefore\ Q(20)\approx7.42\]
    Which is the amount of salt at $t=20$.

    \pagebreak
    \item [4.)] Establish initial variables:
    \[V_0=200,V_{max}=500,Q_0=100,f_{in}=3,c_{in}=1,f_{out}=2,c_{out}=\frac{Q(t)}{V(t)}\]
    \[V_(t)=t(f_{in}-f_{out})+V_0=t+200\]
    Setup differential equation:
    \[\frac{dQ}{dt}=3-\frac{2Q}{t+200}\implies\frac{dQ}{dt}+\frac{2Q}{t+200}=3\]
    Solve for $Q$:
    \[\mu(t)=e^{\int\frac{2}{t+200}dt}=e^{2\ln\vert t+200\vert}=(t+200)^2\]
    \[Q=\frac{1}{(t+200)^2}\bracks{\int3(t+200)^2\ dt}=\frac{(t+200)^3}{(t+200)^2}+\frac{C}{(t+200)^2}=t+200+\frac{C}{(t+200)^2}\]
    Solve for $C$ given $Q(0)=Q_0=100$:
    \[100=0+200+\frac{C}{(0+200)^2}\implies-100=\frac{C}{40000}\implies C=-4000000\]
    \[\therefore\ y=t+200-\frac{4000000}{(t+200)^2}\]
    When $t=300$, $V(300)=300+200=500$, so $t=300$ is the point at which the tank overflows.
    \[Q(300)=300+200-\frac{4000000}{(300+200)^2}=500-\frac{4000000}{500^2}\]
    \[=500-\frac{4000000}{250000}=500-16=484\]
    Which is the concentration when the tank overflows. To find the theoretical maximum concentration, take the limit of $Q(t)$ as $t$ approaches infinity:
    \[\lim_{t\to\infty}\bracks{t+200-\frac{4000000}{(t+200)^2}}=\lim_{t\to\infty}\bracks{\infty+200-0}\implies\text{diverges to }\infty\]
    Thus the theoretical maximum concentration of salt in the tank is $\infty$.
\end{itemize}

\end{document}
