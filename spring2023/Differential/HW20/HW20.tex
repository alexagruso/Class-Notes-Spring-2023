\documentclass[12pt]{article}
\pagenumbering{gobble}

\usepackage{amsfonts}
\usepackage{amsmath}
\usepackage{amssymb}
\usepackage{array}
\usepackage{fancyhdr}
\usepackage{mathtools}
\usepackage{textcomp}
\usepackage{tikz}

\usepackage{pgfplots}
\pgfplotsset{compat=1.9}

\usetikzlibrary{math}

\usepackage[margin=1in,headheight=1in]{geometry}

\newcommand{\contradiction}{
    \ensuremath{{\Rightarrow\mspace{-2mu}\Leftarrow}}
}

\newcommand{\angleb}[1]{\left\langle#1\right\rangle}
\newcommand{\vertb}[1]{\left\vert#1\right\vert}
\newcommand{\bracks}[1]{\left[#1\right]}
\newcommand{\parns}[1]{\left(#1\right)}

\newcommand{\lp}{\mathcal{L}}
\newcommand{\lpi}{\mathcal{L}^{-1}}

\newcommand{\derv}[2]{\dfrac{d#1}{d#2}}
\newcommand{\pderv}[2]{\dfrac{\partial#1}{\partial#2}}

\begin{document}
\pagestyle{fancy}
\fancyhead{}
\fancyhead[L]{Alexander Agruso}
\fancyhead[R]{MATH 3323 Homework 20}

\normalsize
\begin{itemize}
    \item [1.)] $g(t)=u_1(t)+2u_3(t)-6u_4(t)$

    \begin{tikzpicture}
        \begin{axis}[
            axis x line=middle,
            axis y line=middle,
            xlabel=$x$,
            xtick={0,...,6},
            ylabel=$y$,
            ytick={-3,...,3},
            xmax=6,
            ymax=4,
            xmin=0,
            ymin=-4,
        ]

        \addplot[very thick,red,domain=0:1] {0};
        \addplot[very thick,red,domain=1:3] {1};
        \addplot[very thick,red,domain=3:4] {3};
        \addplot[very thick,red,domain=4:6] {-3};
        \addplot[fill=white,only marks,mark=*] coordinates{(1, 0)};
        \addplot[only marks,mark=*] coordinates{(1, 1)};
        \addplot[fill=white,only marks,mark=*] coordinates{(3, 1)};
        \addplot[only marks,mark=*] coordinates{(3, 3)};
        \addplot[fill=white,only marks,mark=*] coordinates{(4, 3)};
        \addplot[only marks,mark=*] coordinates{(4, -3)};
            
        \end{axis}
    \end{tikzpicture}

    \item [5.)] $f(t)=\begin{cases}
        0, & 0\leq t<3 \\
        -2, & 3\leq t<5 \\
        2, & 5\leq t<7 \\
        1, & t\geq7
    \end{cases}$

    \begin{tikzpicture}
        \begin{axis}[
            axis x line=middle,
            axis y line=middle,
            xlabel=$x$,
            xtick={0,...,9},
            ylabel=$y$,
            ytick={-3,...,3},
            xmax=9,
            ymax=4,
            xmin=0,
            ymin=-4,
        ]

        \addplot[very thick,red,domain=0:3] {0};
        \addplot[very thick,red,domain=3:5] {-2};
        \addplot[very thick,red,domain=5:7] {2};
        \addplot[very thick,red,domain=7:9] {1};
        \addplot[fill=white,only marks,mark=*] coordinates{(3, 0)};
        \addplot[only marks,mark=*] coordinates{(3, -2)};
        \addplot[fill=white,only marks,mark=*] coordinates{(5, -2)};
        \addplot[only marks,mark=*] coordinates{(5, 2)};
        \addplot[fill=white,only marks,mark=*] coordinates{(7, 2)};
        \addplot[only marks,mark=*] coordinates{(7, 1)};
            
        \end{axis}
    \end{tikzpicture}

    $f(t)=-2u_3(t)+4u_5(t)-u_7(t)$

    \pagebreak
    \item [7.)] $f(t)=\begin{cases}
        1, & 0\leq t<2 \\
        e^{-(t-2)}, & t\geq2
    \end{cases}$

    \begin{tikzpicture}
        \begin{axis}[
            axis x line=middle,
            axis y line=middle,
            xlabel=$x$,
            xtick={0,...,4},
            ylabel=$y$,
            ytick={-3,...,3},
            xmax=4,
            ymax=3,
            xmin=0,
            ymin=0,
        ]

        \addplot[very thick,red,domain=0:2] {1};
        \addplot[very thick,red,domain=2:4] {e^-(x-2)};
        \addplot[only marks,mark=*] coordinates{(2, 1)};
            
        \end{axis}
    \end{tikzpicture}

    $f(t)=1+(e^{-(t-2)}-1)u_2(t)$
    
    \item [9.)] \[\lp\bracks{(t-2)^2u_2(t)}=e^{-2s}\lp\bracks{t^2}=\frac{2e^{-2s}}{s^3}\]

    \item [12.)] \[\lp\bracks{(t-3)u_2(t)-(t-2)u_3(t)}=e^{-2s}\lp\bracks{t-1}-e^{-3s}\lp\bracks{t+1}\]
    \[=e^{-2s}\parns{\frac{1}{s^2}-\frac{1}{s}}-e^{-3s}\parns{\frac{1}{s^2}+\frac{1}{s}}=\frac{(1-s)e^{-2s}-(1+s)e^{-3s}}{s^2}\]
    
    \item [14.)] \[\lpi\bracks{\frac{1}{s^2+s-2}}=\lpi\bracks{\frac{1}{(s-1)(s+2)}}=\frac{1}{3}\lpi\bracks{\frac{1}{s-1}-\frac{1}{s+2}}=\frac{1}{3}\parns{e^{t}-e^{-2t}}\]
    \[\therefore\,\lpi\bracks{\frac{e^{-2s}}{s^2+s-2}}=\frac{1}{3}u_2(t)\parns{e^{t-2}-e^{-2(t-2)}}\]

    \item [15.)] \[\lpi\bracks{\frac{2(s-1)}{s^2-2s+2}}=2\lpi\bracks{\frac{s-1}{(s-1)^2+1}}=2e^{t}\cos t\]
    \[\therefore\,\lpi\bracks{\frac{2(s-1)e^{-2s}}{s^2-2s+2}}=2u_2(t)e^{t-2}\cos(t-2)\]

    \item [16.)] \[\lpi\bracks{\frac{e^{-s}+e^{-2s}-e^{-3s}-e^{-4s}}{s}}=u_1(t)+u_2(t)-u_3(t)-u_4(t)\]

\end{itemize}
\end{document}
