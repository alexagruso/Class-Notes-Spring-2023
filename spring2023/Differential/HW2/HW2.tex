\documentclass[12pt]{article}
\pagenumbering{gobble}

\usepackage{amsfonts}
\usepackage{amsmath}
\usepackage{amssymb}
\usepackage{array}
\usepackage{fancyhdr}
\usepackage{textcomp}
\usepackage[margin=1in,headheight=1in]{geometry}

\newcommand{\contradiction}{
    \ensuremath{{\Rightarrow\mspace{-2mu}\Leftarrow}}
}

\newcommand{\angleb}[1]{\left\langle#1\right\rangle}
\newcommand{\vertb}[1]{\left\vert#1\right\vert}

\begin{document}
\pagestyle{fancy}
\fancyhead{}
\fancyhead[L]{Alexander Agruso}
\fancyhead[R]{MATH 3323 Homework 2}

\normalsize
\begin{itemize}
    \item [1.)] We can manipulate the equation as follows:
    \begin{equation*}
        \dfrac{dy}{dx}=e^{x-y}=\frac{e^x}{e^y}\implies e^ydy=e^xdx\\
    \end{equation*}
    \begin{equation*}
        \int e^ydy=e^y
    \end{equation*}
    \begin{equation*}
        \int e^xdx=e^x+C
    \end{equation*}
    \begin{equation*}
        \therefore\ \ln(e^y)=y=\ln(e^x+C)
    \end{equation*}
    Which is the explicit solution.

    \item [2.)] Manipulate:
    \begin{equation*}
        x^2y^2\dfrac{dy}{dx}=(1+x)\csc(2y)\implies y^2\sin(2y)dy=\frac{1+x}{x^2}dx
    \end{equation*}
    \begin{equation*}
        \int y^2sin(2y)dy=-\frac{y^2\cos(2y)}{2}+\frac{y\sin(2y)}{2}-\frac{cos(2y)}{4}
    \end{equation*}
    \begin{equation*}
        \int \frac{1+x}{x^2}dx=\int \frac{1}{x}+\frac{1}{x^2}dx=\ln\vert x\vert-\frac{1}{x}+C
    \end{equation*}
    \begin{equation*}
        \therefore\ -\frac{y^2\cos(2y)}{2}+\frac{y\sin(2y)}{2}-\frac{cos(2y)}{4}=\ln\vert x\vert-\frac{1}{x}+C
    \end{equation*}
    Which is the implicit solution.

    \item [3.)] Manipulate:
    \begin{equation*}
        (1+e^x)\cos(y)\dfrac{dy}{dx}=e^x\sin(y)\implies\cot(y)dy=\frac{e^x}{1+e^x}dx
    \end{equation*}
    \begin{equation*}
        \int \cot(y)dy=\ln\vert\sin(y)\vert
    \end{equation*}
    \begin{equation*}
        \int\frac{e^x}{1+e^x}dx=\ln\vert 1+e^x\vert+C
    \end{equation*}
    \begin{equation*}
        \therefore\ y=\sin^{-1}(e^x+C)
    \end{equation*}
    Which is the explicit solution.

    \pagebreak
    \item [4.)] Manipulate:
    \begin{equation*}
        (1+x)dy-ydx=0\implies\frac{dy}{y}=\frac{dx}{1+x}
    \end{equation*}
    \begin{equation*}
        \int\frac{dy}{y}=\ln\vert y\vert
    \end{equation*}
    \begin{equation*}
        \int\frac{dx}{1+x}=\ln\vert 1+x\vert+C
    \end{equation*}
    \begin{equation*}
        \therefore\ y=(1+x)e^C
    \end{equation*}
    Solving given the initial values:
    \begin{equation*}
        1=(1+0)e^C=e^C\implies\ln(1)=0=C
    \end{equation*}
    \begin{equation*}
        \therefore\ y=1+x
    \end{equation*}
    Which is the particular solution.

    \item [5.)] Manipulate:
    \begin{equation*}
        \dfrac{dy}{dx}=\frac{x^2}{y}\implies ydy=x^2dx
    \end{equation*}
    \begin{equation*}
        \int ydy=\frac{y^2}{2}
    \end{equation*}
    \begin{equation*}
        \int x^2dx=\frac{x^3}{3}+C
    \end{equation*}
    \begin{equation*}
        \therefore\ y=\pm\sqrt{\frac{2x^3}{3}+C}
    \end{equation*}
    Which is the explicit solution.

    \item [6.)] Manipulate:
    \begin{equation*}
        \dfrac{dy}{dx}=\frac{x-e^{-x}}{y+e^y}\implies y+e^ydy=x-e^{-x}dx
    \end{equation*}
    \begin{equation*}
        \int y+e^ydy=\frac{y^2}{2}+e^y
    \end{equation*}
    \begin{equation*}
        \int x-e^{-x}dx=\frac{x^2}{2}+e^{-x}+C
    \end{equation*}
    \begin{equation*}
        \therefore\ \frac{y^2}{2}+e^y=\frac{x^2}{2}+e^{-x}+C
    \end{equation*}
    Which is the implicit solution.

    \pagebreak
    \item [7.)] Manipulate:
    \begin{equation*}
        \dfrac{dy}{dx}=\frac{(1-2x)}{y}\implies ydy=1-2xdx
    \end{equation*}
    \begin{equation*}
        \int ydy=\frac{y^2}{2}
    \end{equation*}
    \begin{equation*}
        \int 1-2xdx=x-x^2+C
    \end{equation*}
    \begin{equation*}
        \therefore\ y=\pm\sqrt{2(x-x^2+C)}
    \end{equation*}
    Solving given the initial values:
    \begin{equation*}
    -2=\sqrt{2(0-0^2+C)}=\sqrt{2C}\implies C=2i;\ \text{no real solutions}
    \end{equation*}
    \begin{equation*}
        -2=-\sqrt{2(0-0^2+C)}=-\sqrt{2C}\implies C=2
    \end{equation*}
    \begin{equation*}
        y=-\sqrt{2(x-x^2+2)}
    \end{equation*}
    Which is the particular solution given $C\in\mathbb{R}$.

    \item [8.)] Manipulate:
    \begin{equation*}
        \dfrac{dy}{dx}=\frac{3x^2-e^x}{2y-5}\implies2y-5dy=3x^2-e^xdx
    \end{equation*}
    \begin{equation*}
        \int2y-5dy=y^2-5y
    \end{equation*}
    \begin{equation*}
        \int3x^2-e^xdx=x^3-e^x
    \end{equation*}
    \begin{equation*}
        \therefore\ y^2-5y=x^3-e^x+C
    \end{equation*}
    Solving given the initial values:
    \begin{equation*}
        1^2-5(1)=0^3-e^0+C\implies-4=C-1\implies C=3
    \end{equation*}
    \begin{equation*}
        \therefore\ y^2-5y=x^3-e^x+3
    \end{equation*}
    Which is the particular implicit solution.

\end{itemize}

\end{document}
