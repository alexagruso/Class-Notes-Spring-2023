\documentclass[12pt]{article}
\pagenumbering{gobble}
\linespread{1.5}

\usepackage{amsfonts}
\usepackage{amsmath}
\usepackage{amssymb}
\usepackage{array}
\usepackage{fancyhdr}
\usepackage{mathrsfs}
\usepackage{mathtools}
\usepackage{textcomp}
\usepackage[margin=1in,headheight=1in]{geometry}

\newcommand{\contradiction}{
    \ensuremath{{\Rightarrow\mspace{-2mu}\Leftarrow}}
}

\newcommand{\angleb}[1]{\left\langle#1\right\rangle}
\newcommand{\vertb}[1]{\left\vert#1\right\vert}
\newcommand{\bracks}[1]{\left[#1\right]}
\newcommand{\parns}[1]{\left(#1\right)}

\newcommand{\derv}[2]{\dfrac{d#1}{d#2}}
\newcommand{\e}{\varepsilon}
\newcommand{\di}{\,/\,}

\newcommand{\lm}[1]{\displaystyle\lim_{#1}}

\begin{document}
\pagestyle{fancy}
\fancyhead{}
\fancyhead[L]{Alexander Agruso}
\fancyhead[R]{MATH 3380 Homework 13}

\normalsize
\begin{itemize}
    \item [29.)] \begin{itemize}
        \item [a.)] Since $S\ne\varnothing$ and is bounded below, there exists $l\in\mathbb{R}$ where $l\leq x$ for all $x\in S$, thus $l\in\mathscr{L}$, thus $\mathscr{L}\ne\varnothing$. Q.E.D.

        \item [b.)] Let $w=\sup(\mathscr{L})$, thus $w\geq l$ for all $l\in\mathscr{L}$. For the sake of establishing a contradiction, suppose $w>x$ for some $x\in S$. Let $m=\text{mid}(w,x)$, thus $w>m>x$. Since $m>x$ for some $x\in S$, $m$ cannot be a lower bound of $S$, thus $w$ cannot be a lower bound of $S$, thus $w\ne\sup(\mathscr{L})$\contradiction, thus $w\leq x$ for all $x\in S$, thus $w=\sup(\mathscr{L})$ is a lower bound of $S$, thus $w\in\mathscr{L}$. Q.E.D.

        \item [c.)] Since $w=\sup(\mathscr{L})$, $w\geq l$ for all $l\in\mathscr{L}$ and thus all lower bounds $l$ of $S$, thus $w=\inf(S)$. Q.E.D.
    \end{itemize}

    \item [33.)] Consider $\vertb{x}$:
    \[\vertb{x}=\vertb{x-y+y}\leq\vertb{x-y}+\vertb{y}\implies\vertb{x}-\vertb{y}\leq\vertb{x-y}\]
    Now consider $\vertb{y}$:
    \[\vertb{y}=\vertb{y-x+x}\leq\vertb{y-x}+\vertb{x}\implies\vertb{y}-\vertb{x}\leq\vertb{y-x}=\vertb{x-y}\]
    \[\implies\vertb{x}-\vertb{y}\geq-\vertb{x-y}\]
    Since $-\vertb{x-y}\leq\vertb{x}-\vertb{y}\leq\vertb{x-y}$, $\big\vert\vertb{x}-\vertb{y}\big\vert\leq\vertb{x-y}$, thus the inequality holds. Q.E.D.

    \item [37.)] Since $b\in\mathbb{B}\implies b\geq a$ for all $a\in A$, $b$ is an upper bound of $A$ for all $b\in B$. Since $A\ne\varnothing$ and is bounded above, there exists $u\in\mathbb{R}$ where $u=\sup(A)$. For the sake of establishing a contradiction, suppose $u>b$ for some $b\in B$. Let $m=\text{mid}(u,b)$, thus $u>m>b$. Since $m>b$ for some $b\in B$, $m>a$ for all $a\in A$, thus $m$ is an upper bound of $A$, but since $u>m$, $u\neq\sup(A)$\contradiction, thus $u\leq b$ for all $b\in B$, thus $u$ is a lower bound of $B$, and thus $u=\sup(A)\leq\inf(B)$. Q.E.D.

    \pagebreak
    \item [38.)] Suppose $S$ is uniformly discrete and $S\ne\varnothing$. For the sake of establishing a contradiction, suppose $S$ has no maximal element, thus $x\in S\implies(x+\e)\in S$ for some $\e>0$. Let $x\in S$, thus there exists $\e>0$ where $(x+\e)\in S$. Since $x,(x+\e)\in S$ and $x,(x+\e)\in(x-2\e,x+2\e)$, then $\{x,x+\e\}\subseteq S\cap(x-2\e,x+2\e)$. Let $\e_0=2\e$, thus for some $\e_0>0$, $S\cap(x-\e_0,x+\e_0)\ne\{x\}$, thus $S$ is not uniformly discrete\contradiction, thus if $S$ is uniformly discrete, then $S$ has a maximal element $m\in S$, thus $m=\sup(S)$, thus $\sup(S)\in S$. Q.E.D.

    \item [45.)] Since $y_n\to B$, then for all $\e_0>0$, there exists $k_0\in\mathbb{N}$ where
    \[n\geq k_0\implies\vertb{y_n-B}<\e_0\]
    Let $\lambda=\vertb{B}/2$. Since $B\neq0$, there exists $k_1\in\mathbb{N}$ where
    \[n\geq k_1\implies\vertb{y_n}>\lambda\]
    Let $n\geq\max(k_0,k_1)$, then
    \[\vertb{\frac{1}{y_n}-\frac{1}{B}}=\frac{\vertb{y_n-B}}{\vertb{y_n}\vertb{B}}<\frac{\e_0}{\lambda\vertb{B}}\]
    \[\frac{\e_0}{\lambda\vertb{B}}=\e\implies\e_0=\e\vertb{B}\lambda\]
    Let $\e_0=\e\vertb{B}\lambda$ and $n\geq\max(k_0,k_1)$, then
    \[\vertb{y_n-B}<\e_0\implies\vertb{y_n-B}<\e\vertb{B}\lambda\implies \frac{\vertb{y_n-B}}{\vertb{y_n}\vertb{B}}<\frac{\vertb{y_n-B}}{\lambda\vertb{B}}<\e\]
    \[\implies\frac{\vertb{y_n-B}}{\vertb{y_n}\vertb{B}}=\vertb{\frac{1}{y_n}-\frac{1}{B}}<\e\]
    Thus if $y_n\to B$ and $B\neq0$, then $1/y_n\to1/B$. Q.E.D.

    \pagebreak
    \item [47.)] Since $y_n\to0$, then for all $\e_0>0$, there exists $k\in\mathbb{R}$ where
    \[n\geq k\implies\vertb{y_n-0}<\e_0\]
    Since $x_n$ is bounded, there exists $M\in\mathbb{R}$ where $M\geq\vertb{x_n}$ for all $n\in\mathbb{N}$. Let $\e_0=\e/M$ and $n\geq k$:
    \[\vertb{y_n-0}<\e_0\implies\vertb{y_n-0}<\frac{\e}{M}\leq\frac{\e}{\vertb{x_n}}\implies\vertb{y_n-0}<\frac{\e}{\vertb{x_n}}\]
    \[\implies\vertb{x_n}\vertb{y_n-0}=\vertb{x_ny_n-0}<\e\]
    Thus if $x_n$ is bounded and $y_n\to0$, then $x_ny_n\to0$. Q.E.D.

    \item [48.)] Let $x_n$ and $y_n$ be sequences defined as follows:
    \[x_n=\begin{cases}
        1 & n\text{ is even}\\
        0 & n\text{ is odd}
    \end{cases}\ \ \ y_n=\begin{cases}
        0 & n\text{ is even}\\
        1 & n\text{ is odd}
    \end{cases}\]
    Since $x_n,y_n<2$ for all $n\in\mathbb{N}$, $x_n$ and $y_n$ are bounded. In addition, $x_n$ and $y_n$ do not converge. However, $x_ny_n=0$ for all $n\in\mathbb N$, thus $x_ny_n\to0$. Q.E.D.

    \item [49.)]
    \[\lm{n\to\infty}\frac{1-5n^2+40x^3+2n^{-2}}{4-12n-2n^3}=\lm{n\to\infty}\frac{n^{-3}}{n^{-3}}\cdot\frac{1-5n^2+40x^3+2n^{-2}}{4-12n-2n^3}\]
    \[=\lm{n\to\infty}\frac{x^{-3}-5n^{-1}+40+2n^{-5}}{4n^{-3}-12n^{-2}-2}=\frac{\lm{n\to\infty}\frac{1}{n^3}-\frac{5}{n}+40+\frac{2}{n^5}}{\lm{n\to\infty}\frac{4}{n^3}-\frac{12}{n^2}-2}=\frac{0-0+40+0}{0-0-2}\]
    \[=\frac{40}{-2}=-20\]

    \item [50.)] Since $x_n\to A$, then for all $\e_0>0$, there exists $k\in\mathbb{N}$ where
    \[n\geq k\implies\vertb{x_n-A}<\e_0\]
    Let $\e_0=\e/\vertb{c}$, then
    \[\vertb{x_n-A}<\e_0\implies\vertb{x_n-A}<\frac{\e}{\vertb{c}}\implies\vertb{c}\vertb{x_n-A}=\vertb{cx_n-cA}<\e\]
    Thus if $x_n\to A$, then $cx_n\to cA$. Q.E.D.
    
    \item [63.)] Consider the cases of $S$:

    \textbf{Case:} $S$ has a minimal element: then $\inf(S)\in S$. Let $x_n=\inf(S)$, thus $x_n\in S$ for all $n\in\mathbb{N}$ and $x_n\to\inf(S)$.

    \textbf{Case:} $S$ has no minimal element, then for all $x\in S$, there exists $\e>0$ where $(x-\e)\in S$. Let $x\in S$ be given and $x_n$ be a sequence defined as follows:
    \[x_1=x,\ \ x_n=x_{n-1}-\e\ \text{ where }\e>0\text{ and }(x_{n-1}-\e)\in S\]
    Since $S$ has no minimal element, $x_n$ is well defined. Since $x_n\in S$ for all $n\in\mathbb N$, then $\inf(S)\leq x_n$ for all $n\in\mathbb{N}$, thus $x_n$ is bounded below. Since $x_n=x_{n-1}-\e$, $x_{n-1}-x_n=\e>0$, thus $x_{n-1}>x_n$, thus $x_n$ is strictly decreasing and thus monotone. Finally, according to the monotone convergence theorem, $x_n\to\inf(S)$.

    Thus for all $S\subseteq\mathbb R$ where $S$ is bounded below, there exists a sequence $x_n$ where $x_n\in S$ for all $x\in\mathbb N$ and $x_n\to\inf(S)$. Q.E.D.

    \item [86.)] \begin{itemize}
        \item [a.)] True; Let $n_k=k$, thus $x_k=x_{n_k}$, thus $x\preceq x$.

        \item [b.)] False; Let $x_n=(-1)^n$ and $y_n=(-1)^{n+1}$. 

        \item [c.)] True

        \item [d.)] False; Let $x_n=1/2n$, $y_n=1/n$, $z_n=1/3n$, and $w_n=1/n$. $x_n+z_n=1/2n+1/3n=5/6n$ and $y_n+w_n=2/n$, but $2/n\ne5/6$ for all $n\in\mathbb{N}$, thus $x_n+z_n\npreceq y_n+w_n$.

        \item [e.)] False; Let $x_n=1$ and $y_n=2$.
    \end{itemize}

    \item [87.)] Let $x_n$ and $y_n$ be defined as follows:
    \[x_n=\begin{cases}
        -1 & n=1\\
        1 & n=2\\
        2 & n>2
    \end{cases}\ \ \ y_n=(-1)^n\]
    By definition, $X=\{-1,1,2\}$, and $Y=\{-1,1\}$. In this case, $Y\subset X$, however $y_n\npreceq x_n$, thus the implication does not hold for all $x_n,y_n$. Q.E.D.
    
    \item [105.)] Suppose $y_n=1/n$. $y_n\ne0$ for all $n\in\mathbb{N}$, yet $y_n\to0$. Since $y_n$ converges, it is also cauchy. Because $y_n\to0$, $x_n/y_n$ does not converge, thus $z_n=x_n/y_n$ does not converge, thus $z_n$ is not cauchy. Q.E.D.

    \item [109.)] \begin{itemize}
        \item [a.)] A cauchy sequence is a sequence whose terms, past a certain point, get arbitrarily close to eachother. A type-C sequence is a sequence whose terms, past a certain point, remain constant.

        \item [b.)] Since $n\neq m\implies1/n\ne1/m$, there exists no $N$ such that $n,m\geq N\implies\vertb{x_n-x_m}<\e$ for all $\e>0$, thus $x_n=1/n$ is not type-C.

        \item [c.)] Let $n\in\mathbb{N}$ be fixed, and consider $y_n,y_{n+1}$, and $y_{n+2}$:
        \[\vertb{y_n-y_{n+1}}=2,\ \vertb{y_n-y_{n+2}}=0\]
        Since the distance between any two terms of $y_n$ is either $0$ or $2$, it cannot be less than all $\e>0$ given $m,n>N$ for some $N\in\mathbb N$, thus $y_n$ is not type-C.

        \item [d.)] Since $\sin(n)$ is an oscillating function, $\sin(n)$ never becomes a constant sequence, thus $2^{-n}\sin(n)$ cannot be type-C.

        \item [e.)] Since any type-C sequence eventually reaches a point where its terms remain constant, we know that every type-C sequence converges to this constant. Since it converges, it is also cauchy.

        \item [f.)] $1/n$ is cauchy, but not a type-C sequence, thus not every cauchy sequence is type-C.
    \end{itemize}

    \item [121.)] $E=\{1\}$

    \item [122.)] $E=\{x_n\}\cup\{y_n\}\cup\{z_n\}$ where $x_n=1/n$, $y_n=(n+1)/n$, and $z_n=(2n+1)/n$.

    \item [123.)] $E=\displaystyle\bigcup_{k\in\mathbb{N}}\left\{\frac{n}{kn+1}\right\}_{n\in\mathbb{N}}$ where $\displaystyle\left\{\frac{n}{kn+1}\right\}_{n\in\mathbb{N}}$ is a sequence given $k\in\mathbb{N}$.

    \pagebreak
    \item [130.)] Let $x_n$ be sequence where $x_n\in E$ and $x_n\ne a$ for all $n\in\mathbb{N}$ and where $x_n\to c$. By theorem 25, since $\lm{x\to c}f(x)=\lm{x\to c}h(x)=L$, then $f(x_n)\to L$ and $h(x_n)\to L$ as $n\to\infty$. Consider $g(x)$. Since $f(x)\leq g(x)\leq h(x)$ for all $x\ne c$, $f(x_n)\leq g(x_n)\leq h(x_n)$ for all $n\in\mathbb{N}$. Consider $g(x_n)$ as $n\to\infty$. Since $f(x_n)\to L$ and $h(x_n)\to L$, and since $f(x_n)\leq g(x_n)\leq h(x_n)$, then by the squeeze theorem, $g(x_n)\to L$, thus $\lm{x\to c}g(x)=L$. Q.E.D.

    \item [132.)] Intuitively, $f(x)$ being continuous at $c$ means the limit of $f(x)$ as $x\to c$ exists, and is equal to the value of the function at that point.

    \item [135.)] For $\vertb{x}$ to be continuous over all $c\in\mathbb{R}$, then for all $\e>0$, there must exist $\delta>0$ where
    \[\vertb{x-c}<\delta\implies\big\vert{\vertb{x}-\vertb{c}}\big\vert<\e\]
    By the reverse triangle inequality, we know that $\big\vert{\vertb{x}-\vertb{c}}\big\vert\leq\vertb{x-c}$, thus
    \[\big\vert{\vertb{x}-\vertb{c}}\big\vert\leq\vertb{x-c}<\delta\]
    Let $\delta=\e$, then
    \[\big\vert{\vertb{x}-\vertb{c}}\big\vert<\e\]
    Thus $\vertb{x}$ is continuous for all $c\in\mathbb{R}$. Q.E.D.

    \item [136.)] For $g(x)$ to be continuous at $c\in\mathbb R$, then for all $\e>0$, there exists $\delta>0$ where
    \[\vertb{x-c}<\delta\implies\vertb{g(x)-g(c)}<\e\]
    Consider $\vertb{g(x)-g(c)}$:
    \[\vertb{g(x)-g(c)}=\vertb{(x-c)f(x)-(c-c)f(x)}=\vertb{(x-c)f(x)}=\vertb{x-c}\vertb{f(x)}\]
    \[\leq M\vertb{x-c}<\e\implies\vertb{x-c}<\frac{\e}{M}\]
    Let $\delta=\e/M$:
    \[\vertb{x-c}<\delta\implies\vertb{x-c}<\frac{\e}{M}\implies M\vertb{x-c}<\e\implies \vertb{f(x)}\vertb{x-c}<\e\]
    \[\implies\vertb{(x-c)f(x)-0}<\e\implies\vertb{g(x)-g(c)}<\e\]
    Thus if $f(x)$ is bounded and $g(x)=(x-c)f(x)$, then $g(x)$ is continuous at $c$. Q.E.D.

    \item [137.)] $g(x)=\lfloor x\rfloor$ is continuous over all $x\in\mathbb{R}-\mathbb{Z}$.
 
    \item [139.)] Let $f:\mathbb R\to\mathbb R$ be $1/2$-H\"older. For $f$ to be continuous over all $\mathbb R$, then for all $\e>0$, there exists $\delta>0$ where
    \[\vertb{x-c}<\delta\implies\vertb{f(x)-f(c)}<\e\]
    Let $c=y$ and $\delta=(\e/C)^2$:
    \[\vertb{x-y}<\delta\implies\vertb{x-y}<\parns{\frac{\e}{C}}^2\implies\vertb{x-y}^{1/2}<\frac{\e}{C}\implies C\vertb{x-y}^{1/2}<\e\]
    Since $f$ is $1/2$-H\"older, $\vertb{f(x)-f(y)}<C\vertb{x-y}^{1/2}$, thus $\vertb{f(x)-f(y)}<\e$, thus $f$ is continuous over all $\mathbb{R}$. Q.E.D.

    \item [145.)] Without loss of generality, let $f:(0,1)\to(0,1)$ where $f(x)=x^2$. For $f(x)$ to have a fixed point, then $f(x)$ has to intersect the line $y=x$ at some $x\in(0,1)$. However, $0<x<1\implies0<x^2<x$, thus $f(x)<x$ for all $x\in(0,1)$, thus $f(x)\ne x$ for all $x\in(0,1)$, thus $f(x)$ does not have a fixed point, thus $f:(a,b)\to(a,b)$ being continuous does not imply that $f(x)$ has a fixed point. Q.E.D.

    \item [150.)] \begin{itemize}
        \item [a.)] For $f(x)$ to be uniformly continuous over $(3,5)$, then for all $\e>0$, there exists $\delta>0$ where
        \[x,y\in(3,5)\land\vertb{x-y}<\delta\implies\vertb{\frac{1}{x}-\frac{1}{y}}<\e\]
        Let $\delta=25\e$:
        \[\vertb{x-y}<\delta\implies\vertb{x-y}<25\e\implies\frac{\vertb{x-y}}{25}<\e\implies\frac{\vertb{x-y}}{\vertb{xy}}<\e\]
        \[\implies\vertb{\frac{x}{xy}-\frac{y}{xy}}=\vertb{\frac{1}{y}-\frac{1}{x}}=\vertb{\frac{1}{x}-\frac{1}{y}}<\e\]
        Thus $f(x)$ is uniformly continuous over $(3,5)$. Q.E.D.
        
        \pagebreak
        \item [b.)] Suppose $f(x)$ is uniformly continuous over $(0,2)$, then for all $\e>0$, there exists $\delta>0$ where
        \[x,y\in(0,2)\land\vertb{x-y}<\delta\implies\vertb{\frac{1}{x}-\frac{1}{y}}<\e\]
        Let $x<\delta$ and $y=x/2$:
        \[\vertb{x-y}=\vertb{x-\frac{x}{2}}=\vertb{\frac{x}{2}}=\frac{x}{2}<\delta\]
        Thus $\vertb{x-y}<\delta$. Let $\e=1$:
        \[\vertb{x-y}<\delta\implies\vertb{\frac{1}{x}-\frac{1}{y}}=\vertb{\frac{1}{x}-\frac{2}{x}}=\vertb{-\frac{1}{x}}=\frac{2}{x}<\e=1\]
        \[\implies2<x\]
        But since $x\in(0,2)$, $x<2$\contradiction, thus $f(x)$ is not uniformly continuous over $(0,2)$. Q.E.D.
     \end{itemize}

\end{itemize}

\end{document}
