\documentclass[12pt]{article}
\pagenumbering{gobble}
\linespread{1.1}

\usepackage{amsfonts}
\usepackage{amsmath}
\usepackage{amssymb}
\usepackage{array}
\usepackage{fancyhdr}
\usepackage{mathrsfs}
\usepackage{textcomp}
\usepackage[margin=1in,headheight=1in]{geometry}

\newcommand{\contradiction}{
    \ensuremath{{\Rightarrow\mspace{-2mu}\Leftarrow}}
}

\newcommand{\angleb}[1]{\left\langle#1\right\rangle}
\newcommand{\vertb}[1]{\left\vert#1\right\vert}
\newcommand{\bracks}[1]{\left[#1\right]}
\newcommand{\parns}[1]{\left(#1\right)}

\newcommand{\derv}[2]{\dfrac{d#1}{d#2}}
\newcommand{\e}{\varepsilon}

\newcommand{\lm}[1]{\displaystyle\lim_{#1}}

\begin{document}
\pagestyle{fancy}
\fancyhead{}
\fancyhead[L]{Alexander Agruso}
\fancyhead[R]{MATH 3380 Homework 10}

\normalsize
\begin{itemize}
    % \item [12.)] \begin{itemize}
    %     \item [a.)] Since $0<\lambda<1$, $0<(1-\lambda)<1$ and $(1-\lambda)y < y$.
    % \end{itemize}

    % \item [29.)] \begin{itemize}
    %     \item [a.)] Since $S$ is bounded below, $x\in S\implies x\geq M$ for some $M\in\mathbb{R}$. Let $l<M$, thus $l<M\leq x$, thus $l<x$ for all $x\in S$, thus $l$ is a lower bound of $S$, thus $\mathscr{L}\ne\varnothing$. Next, let $x\in S$, thus $x\geq l$ for all lower bounds $l$ of $S$, thus $x$ is an upper bound of $\mathscr{L}$, thus $\mathscr{L}$ is bounded above. Q.E.D.

    %     \item [b.)] Let $w=\sup(\mathscr{L})$, thus $w\geq l$ for all $l\in\mathscr{L}$.
    % \end{itemize}

    % \item [33.)] Let $x,y\in\mathbb{R}$, and consider $x$:
    %     \[x=x+0=x-y+y\]
    %     And thus by the triangle inequality:
    %     \[\vertb{x-y+y}\leq\vertb{x-y}+\vertb{y}\]
    %     Manipulating:
    %     \[\vert
    %     x-y+y\vert\leq\vertb{x-y}+\vertb{y}\implies\vertb{x}-\vertb{y}\leq\vertb{x-y}\]
    %     \[\implies\vert x\vert-\vert x\vert-\vert y\vert=-\vert y\vert\leq\vert x\vert-\vert y\vert\leq\vert x-y\vert\]
    %     Thus the inequality holds. Q.E.D.***

    % \item [37.)] ***

    % \item [38.)] Let $u$ be an upper bound of $S$, thus $u\geq x$ for all $x\in S$.

    % \item [45.)] ***

    % \item [47.)] Let $y_n\to0$, thus for all $\e_1>0$, there exists $k\in\mathbb{N}$ where
    % \[n\geq k\implies\vertb{y_n-0}<\e_1\]
    % Now suppose $x_n$ is a bounded sequence, and $x_ny_n\to0$, then for all $\e>0$,
    % \[n\geq k\implies\vertb{x_ny_n-0}<\e\]
    % Manipulating the inequality:
    % \[\vertb{x_ny_n-0}=\vertb{x_n(y_n-0)}=\vertb{x_n}\vertb{y_n-0}<\e\implies\vertb{y_n-0}<\frac{\e}{\vertb{x_n}}\]
    % Let $\e_1=\dfrac{\e}{\vertb{x_n}}$, then
    % \[\vertb{y_n-0}<\frac{\e}{\vertb{x_n}}\implies\vertb{x_n}\vertb{y_n-0}<\e\implies\vertb{x_ny_n-0}<\e\]
    % Thus $n\geq k\implies\vertb{x_ny_n-0}<\e$, thus $x_ny_n\to0$. Q.E.D.

    % \item [48.)] ***

    % \item [49.)] ***

    % \item [50.)] Since $x_n\to A$, then for all $\e_1>0$, there exists $k\in\mathbb{N}$ where
    % \[n\geq k\implies\vertb{x_n-A}<\e_1\]
    % Now suppose $cx_n\to cA$, then for all $\e>0$,
    % \[n\geq k\implies\vertb{cx_n-cA}<\e\]
    % Manipulating the inequality:
    % \[\vertb{cx_n-cA}<\vertb{c(x_n-A)}=\vertb{c}\vertb{x_n-A}<\e\]
    % \[\implies\vertb{x_n-A}<\frac{\e}{\vertb c}\]
    % Let $\e_1=\dfrac{\e}{\vertb c}$, then
    % \[\vertb{x_n-A}<\frac{\e}{\vertb c}\implies\vertb{c}\vertb{x_n-A}<\e\implies\vertb{cx_n-cA}<\e\]
    % Thus $n\geq k\implies\vertb{cx_n-cA}<\e$, thus $cx_n\to cA$. Q.E.D.

    % \item [52.)] Let $M,N\in\mathbb{R}$ where $M\geq x_n$ and $N\geq y_n$ for all $n\in\mathbb{N}$. Consider $z_n=x_n+y_n$. $x_n+y_n\leq M+N$, thus $z_n\leq M+N$ for all $n\in\mathbb{N}$, thus $z_n$ is bounded. Q.E.D.

    % \item [57.)] ***

    % \item [58.)] ***

    % \item [59.)] ***

    % \item [60.)] ***

    % \item [62.)] Assume $x_n\to A$, thus for all $\varepsilon>0$, there exists $k$ such that
    % \[n\geq k\implies\vertb{x_n-A}<\varepsilon\]
    % Since $x_n<M$, we know that
    % \[\vertb{x_n-A}<\vertb{M-A}\]
    % ***

    % \item [63.)] Suppose $\inf(S)\in S$, then when $x_n=\inf(S)$, $x_n\to\inf(S)$. Now, suppose $\inf(S)\notin S$.

    % \item [64.)] ***

    \item [65.)] So far, the topics I have had the most trouble understanding are limits and continuity. I know the definitions, and can write proofs, but I lack an intuitive understanding of what the definitions are saying. I plan on supplementing with online resources in order to build a better understanding.

    \item [66.)] \begin{itemize}
        \item [a.)] $x_n=n$

        \item [b.)] DNE; a sequence cannot converge to a value while also having terms arbitrarily far from that value.
    \end{itemize}

    \item [68.)] Let $P(n)$ propose that $1-y_1-y_2-\cdots-y_n=(1-x_1)(1-x_2)\cdots(1-x_n)$. For the base case, consider $P(1)$:
    \[y_1=x_1\implies1-y_1=1-x_1\]
    Thus $P(1)$ holds. For the induction step, assume $P(n)$ and consider $P(n+1)$:
    \[1-y_1-y_2-\cdots-y_n=(1-x_1)(1-x_2)\cdots(1-x_n)\]
    \[\implies1-y_1-y_2-\cdots-y_n-y_{n+1}=(1-x_1)(1-x_2)\cdots(1-x_n)-y_{n+1}\]
    \[=(1-x_1)(1-x_2)\cdots(1-x_n)-x_{n+1}(1-x_1)(1-x_2)\cdots(1-x_n)\]
    \[=(1-x_1)(1-x_2)\cdots(1-x_n)(1-x_{n+1})\]
    Thus $P(n)$ holds for all $n\in\mathbb{N}$. Q.E.D.

    \item [69.)] \begin{itemize}
        \item [a.)] $\{x_n\}_{n\in\mathbb{N}}$ is nondecreasing if $x_n\leq x_{n+1}$ for all $n\in\mathbb{N}$.

        \item [b.)] $\{x_n\}_{n\in\mathbb{N}}$ is strictly decreasing if $x_n>x_{n+1}$ for all $n\in\mathbb{N}$.
    \end{itemize}

    \item [70.)] Consider $x_n$ and $x_{n+1}$:
    \[x_{n+1}-x_n=2(n+1)+6-(2n+6)=2n+2+6-2n-6=2\]
    \[\implies x_{n+1}-x_n=2>0\implies x_{n+1}>x_n\]
    Thus $x_{n+1}>x_n$ for all $n\in\mathbb{N}$, thus $x_n$ is strictly increasing. Q.E.D.

    \item [71.)] Consider $x_n$ and $x_{n+1}$:
    \[x_{n+1}-x_n=5^{-(n+1)}-5^{-n}=\frac{1}{5^{n+1}}-\frac{1}{5^n}=\frac{1}{5^{n+1}}-\frac{5}{5^{n+1}}=-\frac{4}{5^{n+1}}<0\]
    \[\implies x_{n+1}-x_n=-\frac{4}{5^{n+1}}<0\implies x_{n+1}<x_n\]
    Thus $x_{n+1}<x_n$ for all $n\in\mathbb{N}$, thus $x_n$ is strictly decreasing. Q.E.D.

    \item [74.)] 
    \[
    \begin{tabular}{c|c|c|c|c|c|c|c}
        $a_n$ & X & X & X & X & & & X \\
        \hline
        $b_n$ & X & X & & X & & X & X \\
        \hline
        $c_n$ & X & & & & & & \\
        \hline
        $d_n$ & X & X & & & & & \\
    \end{tabular}
    \]

    \item [75.)] 
    \[
    \begin{tabular}{c|c|c|c|c|c|c|c}
        $e_n$ & X & X & X & & X & & X \\
        \hline
        $f_n$ & & & X & & X & & X \\
        \hline
        $g_n$ & & & & & & & \\
        \hline
        $h_n$ & & & X & & & & X\\
    \end{tabular}
    \]

    \item [76.)]
    \[
    \begin{tabular}{c|c|c|c|c|c|c|c}
        $i_n$ & & & & & & & \\
        \hline
        $j_n$ & & & X & & X & X \\
        \hline
        $k_n$ & & & & & & & \\
        \hline
        $l_n$ & & & X & & & & X \\
    \end{tabular}
    \]

    \item [85.)] \begin{itemize}
        \item [a.)] True; Let $x_n$ be a bounded sequence, thus $x_n\leq M$ for some $M\in\mathbb{R}$. Let $y_k\preceq x_n$, thus for all $k\in\mathbb{N}$, there exists $n\in\mathbb{N}$ where $y_k=x_n$, thus $y_k\leq M$ for all $k\in\mathbb{N}$, thus $y_k$ is bounded.

        \item [b.)] True; Since $x_n$ is monotonic, all terms $x_n$ maintain monotonicity with all $x_m$ where $m>n$, thus if $y_k\preceq x_n$, then $y_k$ is monotonic.
    \end{itemize}

    % \item [86.)] \begin{itemize}
    %     \item [a.)] True; Let $n_k=k$, thus $x_k=x_{n_k}$, thus $x\preceq x$.

    %     \item [b.)] False; Let $x_n=(-1)^n$ and $y_n=(-1)^{n+1}$. 

    %     \item [c.)] True; Let ***
    % \end{itemize}

    % \item [87.)] ***

    % \item [88.)] ***

    % \item [89.)] ***

    \item [91.)] \begin{itemize}
        \item [a.)] $S=\mathbb{N}$ are the friends of $x_n$.

    \item [b.)] $S=\{2n\}$ are the friends of $y_n$.
    \end{itemize}

    \item [92.)] $S=\{n\in\mathbb{N}:1\leq n\leq36\}$ are the friends of $z_n$.

    % \item [93.)] Since $x_n$ is bounded, $x_n\leq M$ for some $M\in\mathbb{R}$. Consider $\overline{x}_n$: ***

    % \item [94.)] ***

    % \item [95.)] ***

    % \item [96.)] ***

    % \item [97.)] ***

    \item [99.)] Let $x_n=n-(-1)^nn$. $x_n$ is unbounded, but $y_k=x_{2k}=2k-(-1)^{2k}2k=2k-2k=0$, thus $y_k\preceq x_n$ and $y_k\to0$.

    \item [100.)] Every cauchy sequence is convergent according to theorem 23, and no convergent sequence can be unbounded.

    \item [101.)] Every cauchy sequence is convergent according to theorem 23.

    \item [102.)] $x_n=\dfrac{(-1)^n}{n}$ is not monotonic, but $x_n\to0$, thus $x_n$ is convergent and thus cauchy.

    \item [103.)] Let $x_n=\dfrac{1}{n^2}$. Since $x_n\to\dfrac{\pi^2}{6}$, $x_n$ is convergent and thus cauchy. Q.E.D.

    \item [104.)] Let $x_n$ and $y_n$ be cauchy sequences, thus $x_n\to A$ and $y_n\to B$ for some $A,B\in\mathbb{R}$. Let $z_n=x_ny_n$, thus $z_n\to AB$, thus $z_n$ is convergent and thus cauchy. Q.E.D.

    \item [105.)] Similarly, let $z_n=\dfrac{x_n}{y_n}$ where $y_n\neq0$, thus $z_n\to\dfrac{A}{B}$, thus $z_n$ is convergent and thus cauchy. Q.E.D.

    % \item [106.)] ***

    % \item [107.)] ***

    % \item [108.)] ***

    % \item [109.)] ***

    % \item [110.)] ***

    % \item [111.)] ***

    % \item [112.)] ***

    \item [114.)] Let $S=\{y\in\mathbb{R}:\vertb{x-y}<r\}$,
    \[y\in S\implies\vertb{x-y}<r\implies-r<x-y<r\implies -r-x<-y<r-x\]
    \[\implies x-r<y<x+r\implies y\in(x-r,x+r)\]
    Thus $S\subseteq(x-r,x+r)$. Next, consider $(x-r,x+r)$:
    \[y\in(x-r,x+r)\implies x-r<y<x+r\implies r-x>-y>-r-x\]
    \[\implies -r<x-y<r\implies\vertb{x-y}<r\implies y\in S\]
    Thus $(x-r,x+r)\subseteq S$, thus $S=(x-r,x+r)$. Q.E.D.

    % \item [118.)] ***

    % \item [119.)] ***

    % \item [120.)] ***

    % \item [121.)] ***

    % \item [122.)] ***

    % \item [123.)] ***

    \item [126.)] Let $\e>0$ be given, then there exists $\delta>0$ where
    \[\vertb{x-2}<\delta\implies\vertb{3x+1-7}<\e\]
    Let $\delta=\dfrac{\e}{3}$.
    \[\vertb{x-2}<\frac{\e}{3}\implies\vertb{3}\vertb{x-2}<\e\implies\vertb{3x-6}<\e\implies\vertb{3x+1-7}<\e\]
    Thus $\vertb{x-2}<\delta\implies\vertb{3x+1-7}<\e$, thus $\displaystyle\lim_{x\to2}3x+1=7$. Q.E.D.

    % \item [127.)] Suppose $\lm{x\to5}x^2=25$, then for all $\e>0$, there exists $\delta>0$ where
    % \[\vertb{x-5}<\delta\implies\vertb{x^2-25}<\e\]
    % We can manipulate the inequality to find a sufficient value for $\delta$:
    % \[\vertb{x^2-25}=\vertb{(x-5)(x+5)}=\vertb{x-5}\vertb{x+5}<\e\]
    % \[\implies\vertb{x-5}<\frac{\e}{\vertb{x+5}}\]
    % Let $\delta=\dfrac{\e}{\vertb{x+5}}$. Manipulating the inequality:
    % \[\vertb{x-5}<\frac{\varepsilon}{\vertb{x+5}}\implies\vertb{x-5}\vertb{x+5}=\vertb{(x-5)(x+5)}=\vertb{x^2-25}<\e\]
    % Thus $\vertb{x-5}<\delta\implies\vertb{x^2-25}<\e$, thus $\lm{x\to5}x^2=25$. Q.E.D.

    % \item [128.)] Suppose $\lm{x\to\frac{1}{2}}\dfrac{1}{x}=2$, then for all $\e>0$ there exists $\delta>0$ where
    % \[\vertb{x-\frac{1}{2}}<\delta\implies\vertb{\frac{1}{x}-2}<\e\]
    % We can manipulate the inequality to find a sufficient value for $\delta$:
    % \[\vertb{\frac{1}{x}-2}=***\]

    % \item [129.)] For $\displaystyle\lim_{x\to0}f(x)$ to exist, then for all $\e>0$, there must exist $\delta>0$ where
    % \[\vertb{x-0}<\delta\implies\vertb{f(x)-f(0)}<\e\]

    % \item [130.)] Since $\lm{x\to c}f(x)=\lm{x\to c}h(x)=L$, there exist 

    \item [131.)] Let $f:D\subseteq\mathbb R\to\mathbb R$ and $c\in D$. $f$ is continuous at $c$ if for all $\e>0$, there exists $\delta>0$ where
    \[x\in D\,\land \vertb{x-c}<\delta\implies\vertb{f(x)-f(c)}<\e\]

    % \item [132.)] When a function $f$ is continuous at $c$, you can visualize that 

    % \item [133.)] ***

    % \item [134.)] ***

    % \item [135.)] ***

    % \item [136.)] ***

    % \item [137.)] ***

    % \item [138.)] ***

\end{itemize}

\end{document}
