\documentclass[12pt]{article}
\pagenumbering{gobble}
\linespread{1}

\usepackage{amsfonts}
\usepackage{amsmath}
\usepackage{amssymb}
\usepackage{array}
\usepackage{fancyhdr}
\usepackage{mathtools}
\usepackage{textcomp}
\usepackage[margin=1in,headheight=10mm]{geometry}

\newcommand{\contradiction}{%
    \ensuremath{{\Rightarrow\mspace{-2mu}\Leftarrow}}%
}

\newcommand{\angleb}[1]{\left\langle#1\right\rangle}
\newcommand{\vertb}[1]{\left\vert#1\right\vert}
\newcommand{\bracks}[1]{\left[#1\right]}
\newcommand{\parns}[1]{\left(#1\right)}
\newcommand{\dvert}[1]{\left\Vert#1\right\Vert}

\begin{document}
\pagestyle{fancy}
\fancyhead{}
\fancyhead[L]{Alexander Agruso}
\fancyhead[R]{MATH 2393 Final Project}

\normalsize
\begin{itemize}
    \item [4.)] This rule does not hold for the dot product of two vectors. Let $\vec u=\angleb{1,1}$, $\vec{v_1}=\angleb{2,3}$, and $\vec{v_2}=\angleb{3,2}$, and consider $\vec u\cdot\vec{v_1}$ and $\vec u\cdot\vec{v_2}$:
    \[\vec u\cdot\vec{v_1}=\angleb{1,1}\cdot\angleb{2,3}=1\cdot2+1\cdot3=5\]
    \[\vec u\cdot\vec{v_2}=\angleb{1,1}\cdot\angleb{3,2}=1\cdot3+1\cdot2=5\]
    Thus $\vec u\cdot\vec{v_1}=\vec u\cdot\vec{v_2}$, but $\vec{v_1}\ne\vec{v_2}$, thus the rule does not hold for all vectors. Q.E.D.
\end{itemize}
\end{document}
