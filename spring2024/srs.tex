\documentclass[12pt]{article}

\pagenumbering{gobble}
\linespread{1.2}

\usepackage{fancyhdr}
\usepackage[headheight=0.5in,margin=1in]{geometry}

\setlength{\parindent}{0in}

\begin{document}
    \pagestyle{fancy}

    \fancyhead[L]{Software Requirements Specification \\ Group 1}
    \fancyhead[R]{Alex Agruso, Mauro Alvizo Jr. \\ Daniel Arreguin, Jake Bram}

    \section{Introduction}

    Blah blah blah talk about the program

    \section{Functional Requirements}

    \subsection{User Interface}

    \subsection{Particles and Elements}
    
    Physics interactions occur between particles, with each particle being of a
    certain element.
    A particle's element determines how it interacts with other particles.
    
    \subsubsection{Element Classes}

    There are four primary classes of elements: solids, liquids, gasses, and powders.
    Each is described as follows:

    \textbf{Solid:} Static and rigid; is not affected by gravity.

    \textbf{Liquid:} Dynamic; flows freely but is affected by gravity and
    collides with other particles.
    
    \textbf{Gas:} Dynamic; diffuses evenly, is not affected by gravity but still
    collides with other particles.

    \textbf{Powder:} Dynamic, awd

    We currently plan on implementing 20 different elements

    \subsubsection{Solid Elements}

    \textbf{Wall:}

    \textbf{Metal:}

    \textbf{Wood:}

    \textbf{Glass:}
    
    \textbf{TNT:}

    \textbf{Plant:}

    \textbf{Rock:}

    \subsubsection{Liquid Elements}

    \textbf{Water:}

    \textbf{Lava:}

    \textbf{Poison:}

    \textbf{Oil:}

    \subsubsection{Gas Elements}

    \textbf{Smoke:}

    \textbf{Fire:}

    \textbf{Steam:}

    \textbf{Gas:}

    \subsubsection{Powder Elements}

    \textbf{Stone:}

    \textbf{Sand:}

    \textbf{Gunpowder:}

    \textbf{Ice:}

    \subsubsection{Miscellaneous Elements}

    \textbf{Spark:}

    \section{Non-Functional Requirements}

    \subsection{Event System}

    Our program will be built on an event system.
    Any interaction between different components of the program will be handled
    as an event.
    There will be many different types of events for our application, many of
    which will have no common functionality with each other.
    In order to limit unnecessary abstraction, we will implement the events as
    an algebraic data type using \verb|std::variant|.
    We will also use pattern matching with \verb|std::visit|.

    \subsection{Physics Engine}

    For stability, the physics engine will be limited to running at 60 ticks per
    second.
    Each particle will be represented as an instance of \verb|std::optional|
    wrapping a particle physics instance. If the particle is empty, or air, it
    takes the type of \verb|std::nullopt|, while if it has a 

    \section{Sequence Diagram}

    \section{UML Class Diagram}
    
\end{document}