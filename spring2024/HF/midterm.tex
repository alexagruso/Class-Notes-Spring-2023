\documentclass{article}

\usepackage[final]{neurips_2019}
\usepackage[utf8]{inputenc}
\usepackage[T1]{fontenc}   
\usepackage{hyperref}      
\usepackage{url}           
\usepackage{booktabs}      
\usepackage{amsfonts}      
\usepackage{nicefrac}      
\usepackage{microtype}     
\usepackage{graphicx}
\usepackage{geometry}


\title{A Survey of Accessibility in Web Design}

\author{%
  Alex Agruso\\
  \texttt{aja243@txstate.edu} \\
}


\begin{document}


\noindent\begin{minipage}{0.15\textwidth}%
	\includegraphics[width=2.0cm]{images/txst-black.jpg}
\end{minipage}%
\hfill%
\begin{minipage}{1\textwidth}\raggedright
	Department of Computer Science\\
	Texas State University\\
\end{minipage}


\maketitle


\begin{abstract}
	Abstract here
\end{abstract}


\paragraph{ACM Category} Human-centered Computing > Accessibility > Accessibility design and evaluation methods
\paragraph{Keywords} Accessibility, Web Design


\section{Introduction}
In 2020, about 60\% of the global population used the internet, with that number growing to nearly 90\% in developed countries.
\cite{WorldInternetUsage}
It is easy to see that the internet has proliferated to nearly every aspect of our lives.
Applications like Canvas manage our classes, we use social media to keep in touch with others, even shopping can be done almost entirely online.
Because of this, it is important that these online applications are built to be as accessible as possible.
Otherwise, we risk compromising efficient and effective use of our applications, and in certain cases risk rendering our applications unusable by certain groups of people.

Though there are many types of applications which we could discuss, the scope of this paper will be limited to web design.
We will first exhibit evaluations of the accessibility of various websites.
Then we will discuss existing research on how to design accessible websites.
Finally, we will perform an accessibility case study that applies the existing research on an example website, in this case, \verb|digitalocean.com|.


\section{Website Evaluations}
In this section, we will o


\subsection*{Library Websites}
The first study we will examine was performed by Paul Khawaja in 2022.
\cite{Library}
In it, Khawaja seeks to


\section{How to Design Accessible Websites}


\section{Case Study}


\section{Conclusion}


\small
\bibliographystyle{plain}
\bibliography{references}


\end{document}

